%Kannan,

%Here is some text that describes the TrafficGenerator and derived classes,
%as well as how they are used with UDP_Agent objects.  I have no idea
%if the level of treatment is appropriate, and I was particularly unsure
%how to deal with C++ vs. OTcl stuff.  Let me know if you would like
%me to make some changes.

%I'm also going to try and write some stuff about the Source class.

%Also, in a subsequent message I'll forward some stuff I had written
%about RandomVariable objects.  I had sent this to Kevin, but don't
%see it in the current documentation.  I have no idea whether it slipped
%off his queue or whether he didn't like it.

%Lee

\chapter{Traffic Generation}
\label{chap:trafgen}

There are currently two methods of traffic generation in \ns.
One method uses the abstract
\clsref{TrafficGenerator}{../ns-2/trafgen.h}
to generate inter-packet intervals and packet sizes.
Currently, classes derived
from TrafficGenerator are used in conjunction with the UDP\_Agent
objects, which are responsible for actually allocating and
transmitting the generated packets (Section~\ref{sec:trafgenclass}).
The second method of traffic generation uses the Source class.
Source objects generate traffic that is transported by TCPAgent objects
(Section~\ref{sec:sourceobjects}).

\section{The Class TrafficGenerator}
\label{sec:trafgenclass}

TrafficGenerator is an abstract C++ class defined as follows:
\begin{program}
        class TrafficGenerator : public NsObject \{
        public:
                TrafficGenerator() \{\}
                virtual double next_interval(int &) = 0;
                virtual void init() \{\}
        protected:
                void recv(Packet*, Handler*) \{ abort(); \}
                int size_;
        \};
\end{program}
The pure virtual function \fcn[]{next\_interval} returns the time until the
next packet is created and also sets the size in bytes of the next
packet.
The member function \fcn[]{init} is called by the UDP\_Agent with
which the TrafficGenerator is associated (see below) when the agent is
started.
Necessary initializations specific to a traffic generation
process are done in \fcn[]{init}.

Currently, there are three C++ classes derived from the
class TrafficGenerator:
\begin{enumerate}
\item \code{EXPOO_Source}---generates traffic according to an
  Exponential On/Off distribution.
  Packets are sent at a fixed rate during on periods, and
  no packets are sent during off periods.
  Both on and off periods are taken from an exponential distribution.
  Packets are constant size.
\item \code{POO_Source}---generates traffic
  according to a Pareto On/Off distribution.
  This is identical to the Exponential On/Off distribution,
  except the on and off periods are taken from a pareto distribution.
  These sources can be used to generate aggregate traffic
  that exhibits long range dependency.
\item \code{TrafficTrace}---generates traffic according to a trace file.
  Each record in the trace file consists of 2 32-bit fields.
  The first contains the time in microseconds
  until the next packet is generated.
  The second contains the length in bytes of the next packet.
\end{enumerate}
These classes can be created from OTcl.  The OTcl classes names and
associated parameters are given below:

\paragraph{Exponential On/Off}
An Exponential On/Off object is embodied in the OTcl class
Traffic/Expoo.  The member variables that parameterize this
object are:
\begin{alist}
\code{packet-size} & the constant size of the packets generated\\
\code{burst-time} & the average "on" time for the source\\
\code{idle-time} & the average "off" time for the source\\
\code{rate} & the sending rate during "on" times\\
\end{alist}
Hence a new Exponential On/Off source can be created an parameterized
as follows:
\begin{program}
        set e [new Traffic/Expoo]
        $e set packet-size 210
        $e set burst-time 500ms
        $e set idle-time 500ms
        $e set rate 100k
\end{program}

\paragraph{Pareto On/Off}
A Pareto On/Off object is embodied in the OTcl class Traffic/Pareto.
The member variables that parameterize this object are:
\begin{alist}
\code{packet-size} & the constant size of the packets generated\\
\code{burst-time} & the average "on" time for the source\\
\code{idle-time} & the average "off" time for the source\\
\code{rate} & the sending rate during "on" times\\
\code{shape} & the "shape" parameter used by the pareto distribution\\
\end{alist}
A new Pareto On/Off source can be created as follows:
\begin{program}
        set p [new Traffic/Pareto]
        $p set packet-size 210
        $p set burst-time 500ms
        $p set idle-time 500ms
        $p set rate 200k
        $p set shape 1.5
\end{program}

\paragraph{Traffic Trace}
A Traffic Trace object is instantiated by the OTcl class Traffic/Trace.
The associated class Tracefile is used to enable multiple 
Traffic/Trace objects to be associated with a single trace file.
The Traffic/Trace class uses the method attach-tracefile to associate
a Traffic/Trace object with a particular Tracefile object.
The method filename of the Tracefile class associates a trace file
with the Tracefile object.
The following example shows how to create two Traffic/Trace objects,
each associated with the same trace file
(called "example-trace" in this example).
To avoid synchronization of the traffic generated,
random starting places within the trace file are chosen for
each Traffic/Trace object.
\begin{program}
        set tfile [new Tracefile]
        $tfile filename example-trace

        set t1 [new Traffic/Trace]
        $t1 attach-tracefile $tfile
        set t2 [new Traffic/Trace]
        $t2 attach-tracefile $tfile
\end{program}

\section{The Class UDP\_Agent}

TrafficGenerator objects merely generate inter-packet times and sizes.
They do not actually allocate packets, fill in header fields and
transmit packets.  Hence, each TrafficGenerator object must be
associated with another object that performs these functions.  This
functionality is currently implemented in the C++ class UDP\_Agent.
This class is defined as follows:
\begin{program}
        class UDP_Agent : public CBR_Agent \{
        public:
                UDP_Agent();
                int command(int, const char*const*);
                virtual void timeout(int);
        protected:
                void start();
                void stop();
                TrafficGenerator *trafgen_;
                virtual void sendpkt();
        \};
\end{program}
This class is derived from the class CBR\_Agent.
It differs only in
the manner in which inter-packet times and packet sizes are
determined.
Whereas the CBR\_Agent uses fixed size packets and
constant inter-arrival times (with optional randomization added to the
inter-packet times), UDP\_Agent objects invoke the \fcn[]{next\_interval}
method on an associated TrafficGenerator object to determine these
values.  

UDP\_Agent objects are created from OTcl using the OTcl class
Agent/CBR/UDP.  The method attach-traffic is used to associate the
UDP\_Agent with a corresponding TrafficGenerator object.  For example,
the following OTcl commands create a UDP\_Agent object and an
Exponential On/Off source (with default parameters), and uses the
Exponential On/Off source to generate traffic for the UDP\_Agent:
\begin{program}
        set u [new Agent/CBR/UDP]
        set e [new Traffic/Expoo]
        $u attach-traffic $e
\end{program}


\section{Source class}
\label{sec:sourceobjects}
 
Classes derived from the Source class are used to generate traffic for TCP.
There are currently two classes derived from Source:
Source/FTP and Source/Telnet.
These classes work by advancing the
count of packets available to be sent by a TCPAgent.
The actual transmission of available packets
is still controlled by TCP's flow control algorithm.
 
\paragraph{Source/FTP} 
Source/FTP represents bulk data transfer.
When a Source/FTP object is started,
it makes \code{maxpkts_} packets available to the associated TCP.
\code{maxpkts_} is set to $2^31 - 1 = 268435456$ by default.
The following are methods of the Source/FTP class:
\begin{alist}
\code{attach agent} & attaches a Source/FTP object to an agent.\\ 
\code{start} & start the Source/FTP by setting the number of packets
        available for sending to \code{maxpkts\_}.\\ 
\code{stop} & stop sending packets.\\ 
\code{produce n} &  set the counter of packets to be sent to $n$.\\ 
\code{producemore n} &  increase the counter of packets to be sent by $n$.
\end{alist} 

\paragraph{Source/Telnet} 
Source/Telnet objects generate packets in one of two ways.
If the member variable \code{interval_} is non-zero,
then inter-packet times are chosen
from an exponential distribution with average equal to \code{interval_}.
If \code{interval_} is zero, then inter-arrival times are chosen
according to the tcplib distribution (see tcplib-telnet.cc).
The start method starts the packet generation process.
 
\paragraph{Creation of Source Objects} 
Creation and configuration of Source objects is accomplished by the
Agent instance procedure, \proc[sourceType]{attach-source},
which creates a new object of type Source/stype and returns a handle to it.
The returned object may be started at any point in the simulation using
the handle.
 
\endinput
