\chapter{UDP Agents}
\label{sec:udpAgents}

\section{UDP Agents}
This section describes the operation of the UDP agents in \ns.
UDP agents are implemented in \code{udp.\{cc, h\}}.  A UDP agent accepts
data in variable size chunks from an application, and segments the data 
if needed.

The default segment size (MSS) for UDP agents is 1000 bytes:
\begin{program}
Agent/UDP set packetSize_   1000              \; max segment size;
\end{program}
This OTcl instvar is bound to the C++ agent variable \code{size_}.  If an 
application submits more than the MSS, the UDP agent segments the 
\fcn[]{sendmsg} request.

Applications can access UDP agents via the \fcn[]{sendmsg} function in C++,
or via the \code{send} or \code{sendmsg} methods in OTcl, as described in
section \ref{sec:systemcalls}.  

The following is a simple example of how a UDP agent may be used in a program.  
In the example, the CBR traffic agent is started at time 1.0, at which time
the traffic agent begins to periodically call the UDP agent \fcn[]{sendmsg}
function (according to the \ns defaults).
\begin{program}
        set ns [new Simulator]
        set n0 [$ns node]
        set n1 [$ns node]
        $ns duplex-link $n0 $n1 5Mb 2ms DropTail

        set udp0 [new Agent/UDP]
        $ns attach-agent $n0 $udp0
        set cbr0 [new Application/Traffic/CBR]
        $cbr0 attach-agent $udp0
        $udp0 set packetSize_ 536	\; set MSS to 536 bytes;

        set null0 [new Agent/Null]
        $ns attach-agent $n1 $null0
        $ns connect $udp0 $null0
        $ns at 1.0 "$cbr0 start"
\end{program}

\section{CBR Agents}
CBR agents are similar to UDP agents with the following two additions:
\begin{enumerate}
	\item CBR agents implement a ``send'' timer for generating packet
sending events periodically, and 
	\item CBR agents include a packet sequence number.
\end{enumerate}

There are therefore two ways to generate CBR-like traffic in \ns.  The first
way is to create a CBR application and hook it to a UDP agent, as follows:
\begin{program}
        set n0 [$ns node]
        set udp0 [new Agent/UDP]
        $ns attach-agent $n0 $udp0
        set cbr0 [new Application/Traffic/CBR]
        $cbr0 attach-agent $udp0
        $ns at 1.0 "$cbr0 start"
\end{program}
The second way is to use the CBR agent with no application:
\begin{program}
        set n0 [$ns node]
        set cbr0 [new Agent/CBR]
        $ns attach-agent $n0 $cbr0
        $ns at 1.0 "$cbr0 start"
\end{program}
The behavior in \ns of these two examples is identical; however, the trace
files for these code fragments will be different.  In particular, the 
packet type will be ``cbr'' in the latter case, rather than ``udp,'' and
the sequence number for the UDP packets will be ``-1,'' rather than 
monotonically increasing for the CBR agent.

Historically, the class \code{Agent/CBR} predates the class \code{Agent/UDP}
in the simulator.  CBR agents were used to define UDP-like agents via the
class \code{Agent/CBR/UDP}.  This type of UDP agent was used to support 
traffic generators before the \ns API was developed.
\endinput
