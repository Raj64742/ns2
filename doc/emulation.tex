\chapter{Emulation}
\label{sec:emulation}

This chapter describes the {\em emulation} facility
of \ns.
Emulation refers to the ability to introduce the
simulator into a live network.
Special objects within the simulator are capable
of introducing live traffic into the simulator and
injecting traffic from the simulator into the
live network.

\section{Introduction}

The emulation facility can be subdivided into
two modes:
\begin{enumerate}
\item {\sf opaque mode} -- live data treated as opaque data packets
\item {\sf protocol mode} -- live data may interpreted/generated by simulator
\end{enumerate}
In opaque mode, the simulator
treats network data as uninterpreted packets.
In particular, real-world protocol fields
are not directly manipulated by the simulator.
In opaque mode, live data packets may be dropped, delayed, re-ordered, or
duplicated, but because no protocol processing is performed,
protocol-specific traffic manipulation scenarios (e.g. ``drop the TCP segment
containing a retransmission of sequence number 23045'') may not be performed.
In protocol mode, the simulator is able to interpret and/or generate
live network traffic containing arbitrary field assignments.
{\bf To date (Mar 1998), only Opaque Mode is currently implemented}.

The interface between the simulator and live network is provided by
a collection of objects including {\em tap agents} and {\em network objects}.
Tap agents embed live network data into simulated packets and
vice-versa.
Network objects are installed in tap agents and provide an entrypoint
for the sending and receipt of live data.
Both objects are described in the following sections.

\section{Tap Agents}

The class {\tt TapAgent} is a simple class derived from the base
{\tt Agent} class.
As such, it is able to generate simulator packets containing
arbitrarily-assigned values within the \ns common header.
The tap agent handles the setting of the common header packet
size field and the type field.  
It uses the packet type {\tt PT\_LIVE} for packets injected
into the simulator.
Each tap agent can have at most one associated network object, although
more than one tap agent may be instantiated on a single simulator node.

\paragraph{Configuration}
Tap agents are able to send and receive packets to/from an
associated {\tt Network} object.
Assuming a network object {\tt \$netobj} refers to a network
object, a tap agent is configured using the {\tt network} method:
\begin{verbatim}
        set a0 [new Agent/Tap]
	$a0 network $netobj
	$a0 set fid_ 26
	$a0 set prio_ 2
	$ns connect $a0 $a1
\end{verbatim}
Note that the configuration of the flow ID and priority are
handled through the {\tt Agent} base class.
The purpose of setting the flow id field in the common header
is to label packets belonging to particular flows of live data.
Such packets can be differentially treated with respect
to drops, reorderings, etc.
The {\tt connect} method instructs agent {\tt \$a0} to send
its live traffic to the {\tt \$a1} agent via the current
route through the simulated topology.

\section{Network Objects}

\endinput
