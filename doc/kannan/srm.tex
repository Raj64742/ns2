\documentclass{article}

%\usepackage{times}
%\usepackage[T1]{fontenc}

\PassOptionsToPackage{draft}{MyPreamble}
\usepackage[widen-page,skrunch-figures]{MyPreamble}
\usepackage{nsDoc}

\begin{document}

\title{\nsTcl\ internals documentation}
\author{%
  Kevin Fall \tup{kfall@ee.lbl.gov}\\
  Kannan Varadhan \tup{kannan@catarina.usc.edu}}
\date{\today}

\def\c#1{\ensuremath{C_{#1}}}
\def\d#1{\ensuremath{D_{#1}}}

% \maketitle

\section{Agent/SRM}
\label{sec:agent/srm}

This section describes the internals of the SRM implementation in \ns.
The section is in three parts:
the first part is an overview of a minimal SRM configuration,
and a ``complete'' description of the configuration parameters 
of the base SRM agent.
The second part describes the architecture, internals, and the code path
of the base SRM agent.
The last part of the section is a description of the extensions
for other types of SRM agents that have been attempted to date.

\subsection{Configuration}
\label{sec:srm-config}

Running an SRM simulation requires
creating and configuring the agent,
attaching an application-level data source (a traffic generator), and
starting the agent and the traffic generator.

\subsubsection{Trivial Configuration}

\paragraph{Creating the Agent}
\begin{program}
        set ns [new Simulator]          \; preamble initialization;
        $ns enableMcast
        set node [$ns node]             \; agent to reside on this node;
        set group [$ns allocaddr]       \; multicast group for this agent;

        {\bfseries{}set srm [new Agent/SRM]}
        $srm  set dst_ $group           \; configure the SRM agent;
        {\bfseries{}$ns attach-agent $node $srm}

        $srm set fid_ 1                \; optional configuration;
        $srm log [open srmStats.tr w]  \; log statistics in this file;
        $srm trace [open srmEvents.tr w] \; trace events for this agent;
\end{program}
The key steps in configuring a virgin SRM agent are to assign
its multicast group, and attach it to a node.

Other useful configuration parameters are
to assign a separate flow id to traffic originating from this agent,
to open a log file for statistics, and
a trace file for trace data%
\footnote{%
Note that the trace data can also be used
to gather certain kinds of trace data.
We will illustrate this later.}.

The file
\fcnref{\|tcl/mcast/srm-nam.tcl|}{../ns-2/srm-nam.tcl}{Agent/SRM::send}
contains definitions that overload the agent's \|send| methods;
this separates control traffic originating from the agent by type.
Each type is allocated a separate flowID.
The traffic is separated into session messages (flowid = 40),
requests (flowid = 41), and repair messages (flowid = 42).
The base flowid can be changed by setting global variable \|ctrlFid|
to one less than the desired flowid before sourcing \|srm-nam.tcl|.
To do this, the simulation script must source \|srm-nam.tcl|
before creating any SRM agents.
This is useful for analysis of traffic traces, or
for visualization in nam.

\paragraph{Application Data Handling}
The agent does not generate any application data on its own;
instead, the simulation user can connect any traffic generation
module to any SRM agent to generate data.
The following code demonstrates
how a traffic generation agent can be attached to an SRM agent:
\begin{program}
        set packetSize 210
        set exp0 [new Traffic/Expoo]          \; configure traffic generator;
        $exp0 set packet-size $packetSize
        $exp0 set burst-time 500ms 
        $exp0 set idle-time 500ms
        $exp0 set rate 100k 

        set s0 [new Agent/CBR/UDP]  \; attach traffic generator to application;
        $s0 set fid_ 0
        $s0 attach-traffic $exp0

        {\bfseries{}$srm(0) traffic-source $s0} \; attach application to SRM agent;
        {\bfseries{}$srm(0) set packetSize_ $packetSize} \; to generate repair packets of appropriate size;
\end{program}
The instproc \texttt{\textbf{traffic-source}} specifies the application agent
that will produce data for the SRM agent.
The user can attach any agent;
the only distinguishing criteria is that the destination address must be zero.
The SRM agent will add the SRM headers, 
set the destination address to the multicast group, and
deliver the packet to its target.
The SRM header contains the type of the message,
the identity of the sender,
the sequence number of the message,
and (for control messages), the round for which this message is being sent.
Each data unit in SRM is identified as
\tup{sender's id, message sequence number}.

The SRM agent does not generate its own data;
it does not also keep track of the data sent,
except to record the sequence numbers of messages received
in the event that it has to do error recovery.
Since the agent has no actual record of past data,
it needs to know what packet size to use for each repair message.
Hence, the instance variable \|packetSize_| specifies the size
of repair messages generated by the agent.

\paragraph{Starting the Agent and Traffic Generator}
The agent and the traffic generator must be started separately.
\begin{program}
        {\bfseries{}\fcnref{$srm start}{../ns-2/srm.tcl}{Agent/SRM::start}}
        {\bfseries{}\fcnref{$srm start-source}{../ns-2/srm.tcl}{Agent/SRM::start-source}}
\end{program}
At \|start|, the agent joins the multicast group, and 
starts generating session messages.
The \|start-source| triggers the traffic generator to start sending
data.

\subsubsection{Other Configuration Parameters}
\label{sec:config-param}

In addition to the above parameters,
the SRM agent supports additional configuration variables.
Each of the variables described in this subsection is
both an OTcl class variable and an OTcl object's instance variable.
Changing the class variable changes the default value
for all agents that are created subsequently.
Changing the instance variable of a particular agent
only affects the values used by that agent.
For example,
\begin{program}
                Agent/SRM set D1_ 2.0   \; Changes the class variable;
                $srm set D1_ 2.0        \; Changes D1_ for the $srm object only;
\end{program}

The default request and repair timer parameters \cite{Floy95:Reliable}
for each SRM agent are:
\begin{program}
        Agent/SRM set C1_       2.0             \; request parameters;
        Agent/SRM set C2_       2.0
        Agent/SRM set D1_       1.0             \; repair parameters;
        Agent/SRM set D2_       1.0
\end{program}
It is thus possible to trivially obtain two flavours of SRM agents
based on whether the agents use probabilistic or deterministic
suppression by using the following definitions:
\begin{program}
        Class Agent/SRM/Deterministic -superclass Agent/SRM
        Agent/SRM/Deterministic set C2_ 0.0
        Agent/SRM/Deterministic set D2_ 0.0

        Class Agent/SRM/Probabilistic -superclass Agent/SRM
        Agent/SRM/Probabilistic set C1_ 0.0
        Agent/SRM/Probabilistic set D1_ 0.0
\end{program}
In \href{a later section}{Section}{sec:extensions},
we will discuss other ways of extending the SRM agent.

Timer related functions are handled by separate objects
belonging to the class  SRM.
Timers are required for loss recovery and sending periodic session messages.
There are loss recovery objects to send request and repair messages.
The agent creates a separate request or repair object to handle each loss.
In contrast, the agent only creates one session object to send
periodic session messages.
The default classes the express each of these functions are:
\begin{program}
        Agent/SRM set requestFunction_  "SRM/request"
        Agent/SRM set repairFunction_   "SRM/repair"
        Agent/SRM set sessionFunction_  "SRM/session"

        Agent/SRM set requestBackoffLimit_      5       \; parameter to requestFunction_;
        Agent/SRM set sessionDelay_             1.0     \; parameter to sessionFunction_;
\end{program}
The instance procedures
\fcnref{\proc[]{requestFunction}}{../ns-/srm.tcl}{Agent/SRM::requestFunction},
\fcnref{\proc[]{repairFunction}}{../ns-/srm.tcl}{Agent/SRM::repairFunction},
and
\fcnref{\proc[]{sessionFunction}}{../ns-/srm.tcl}{Agent/SRM::sessionFunction}
can be used to change the default function for individual agents.
The last two lines are specific parameters used by the request 
and session objects.
The \href{following section}{Section}{sec:architecture}
describes the implementation of theses objects in greater detail.

\subsubsection{Statistics}
Each agent tracks two sets of statistics:
statistics to measure the response to data loss,
and overall statistics for each request/repair.
In addition, there are methods to access other
information from the agent.

\paragraph{Data Loss}
The statistics to measure the response to data losses
tracks the duplicate requests (and repairs),
and the average request (and repair) delay.
The algorithm used is documented in Floyd \etal \cite{Floy95:Reliable}.
In this algorithm,
each new request (or repair) starts a new request (or repair) period.
During the request (or repair) period, the agent measures
the number of first round duplicate requests (or repairs)
until the round terminates either due to receiving a request (or
repair), or due to the agent sending one.
% These statistics are used by the adaptive timer algorithms;
% we will describe our implementation of these algorithms
% in the following subsections.
The following code illustrates how the user can simple retrieve the
current values in an agent:
\begin{program}
        set statsList [$srm array get statistics_]
        array set statsArray [$srm array get statistics_]
\end{program}
The first form returns a list of key-value pairs.
The second form loads the list into the \|statsArray| for further manipulation.
The keys of the array are
\|dup-req|, \|ave-dup-req|, \|req-delay|, \|ave-req-delay|,
\|dup-rep|, \|ave-dup-rep|, \|rep-delay|, and \|ave-rep-delay|.

\paragraph{Overall Statistics}
In addition, each loss recovery and session object keeps track of
times and statistics.
In particular, each object records its
\|startTime|, \|serviceTime|, \|distance|, as are relevant to that object;
startTime is the time that this object was created,
serviceTime is the time for this object to complete its task, and the
distance is the one-way time to reach the remote peer.

For request objects, startTime is the time a packet loss is detected,
serviceTime is the time to finally receive that packet,
and distance is the distance to the original sender of the packet.
For repair objects, startTime is the time that a request for
retransmission is received, serviceTime is the time send a repair,
and the distance is the distance to the original requester.
For both types of objects, the serviceTime is normalised by the
distance.
For  the session object,
startTime is the time that the agent joins the multicast group.
serviceTime and distance are not relevant.

Each object also maintains statistics particular to that type of object.
Request objects track the number of duplicate requests and repairs received,
the number of requests sent, and the number of times this object
had to backoff before finally receiving the data.
Repair objects track the number of duplicate requests and repairs,
as well as whether or not this object for this agent sent the repair.
Session objects simply record the number of session messages sent.

The values of the timers and the statistics for each object are written
to the log file every time an object completes the error recovery function
it was tasked to do.
The format of this trace file is:
\begin{program}
                \tup{prefix} \tup{id} \tup{times} \tup{stats}
{\itshape{}where}
\tup{prefix} is         \tup{time} n \tup{node id} m \tup{msg id} r \tup{round}
                \tup{msg id} is expressed as \tup{source id:sequence number}
\tup{id} is             type \tup{of object}
\tup{times} is          list of key-value pairs of startTime, serviceTime, distance
\tup{stats} is          list of key-value pairs of per object statistics
                \|dupRQST|, \|dupREPR|, \|#sent|, \|backoff|    {\itshape for request objects}
                \|dupRQST|, \|dupREPR|, \|#sent|                {\itshape for repair objects}
                \|#sent|                                        {\itshape for session objects}
\end{program}
The following sample output illustrates the output file format (the lines
have been folded to fit on the page):
\begin{verbatim}
 3.6274 n 0 m <1:1> r 1 type repair serviceTime 0.500222 startTime 3.5853553333333332 \
        distance 0.0105 #sent 1 dupREPR 0 dupRQST 0
 3.6417 n 1 m <1:1> r 2 type request serviceTime 2.66406 startTime 3.5542666666666665 \
        distance 0.0105 backoff 1 #sent 1 dupREPR 0 dupRQST 0
 3.6876 n 2 m <1:1> r 2 type request serviceTime 1.33406 startTime 3.5685333333333333 \
        distance 0.021 backoff 1 #sent 0 dupREPR 0 dupRQST 0
 3.7349 n 3 m <1:1> r 2 type request serviceTime 0.876812 startTime 3.5828000000000002 \
        distance 0.032 backoff 1 #sent 0 dupREPR 0 dupRQST 0
 3.7793 n 5 m <1:1> r 2 type request serviceTime 0.669063 startTime 3.5970666666666671 \
        distance 0.042 backoff 1 #sent 0 dupREPR 0 dupRQST 0
 3.7808 n 4 m <1:1> r 2 type request serviceTime 0.661192 startTime 3.5970666666666671 \
        distance 0.0425 backoff 1 #sent 0 dupREPR 0 dupRQST 0
\end{verbatim}

\paragraph{Miscellaneous Information}
Finally, the user can use the following methods to gather
additional information about the agent:
\begin{list}{\textbullet}{}
\item
  \fcnref{\proc[]{groupSize?}}{../ns-2/srm.tcl.html}{Agent/SRM::groupSize?} 
  returns the agent's current estimate of the multicast group size.
\item
  \fcnref{\proc[]{distances?}}{../ns-2/srm.cc.html}{SRMAgent::command}
  returns a list of key-value pairs of distances;
  the key is the address of the agent, 
  the value is the estimate of the distance to that agent.
  The first element is the address of this agent, and the distance of 0.
\item
  \fcnref{\proc[]{distance?}}{../ns-2/srm.cc.html}{SRMAgent::command}
  returns the distance to the particular agent specified as argument.
\end{list}
\begin{program}
        $srm(i) groupSize?    \; returns $srm(i)'s estimate of the group size;
        $srm(i) distances?    \; returns list of \tup{address, distance} tuples;
        $srm(i) distance? 257 \; returns the distance to agent at address 257;
\end{program}

\subsubsection{Tracing}
Each object writes out trace information that can be used to track the
progress of the object in its error recovery.
Each trace entry is of the form:
\begin{program}
\tup{prefix} \tup{tag} \tup{type of entry} \tup{values}
\end{program}
The prefix is as describe in the previous subsection for statistics.
The tag is {\bf Q} for request objects, {\bf P} for repair objects, and
{\bf S} for session objects.
The following types of trace entries and parameters are written by each
object:

\centerline{\small\renewcommand{\arraystretch}{1.3}
\begin{tabular}{rclp{2in}}\hline
      & Type of &              & \\
  Tag & Object  & Other values & Comments\\ \hline
  Q & DETECT & & \\
  Q & INTERVALS & C1 \tup{C1\_} C2 \tup{C2\_} dist \tup{distance} i \<backoff\_> & \\
  Q & NTIMER & at \tup{time} & Time the request timer will fire \\
  Q & SENDNACK & & \\
  Q & NACK & IGNORE-BACKOFF \tup{time} & Receive NACK, ignore other NACKs until
  \tup{time} \\
  Q & REPAIR & IGNORES \tup{time} & Receive REPAIR, ignore NACKs until \tup{time}  \\
  Q & DATA & & Agent receives data instead of repair.  Possibly indicates out of order arrival of data. \\ \hline
  P & NACK & from \tup{requester} & Receive NACK, initiate repair \\
  P & INTERVALS & D1 \tup{D1\_} D2 \tup{D2\_} dist \tup{distance} & \\
  P & RTIMER & at \tup{time} & Time the repair timer will fire \\
  P & SENDREP & \\
  P & REPAIR & IGNORES \tup{time} & Receive REPAIR, ignore NACKs until \tup{time} \\
  P & DATA & & Agent receives data instead of repair.  Indicates premature request by an agent. \\ \hline
  S & SESSION & & logs session message sent \\ \hline
\end{tabular}}
The following illustrates a typical trace for a single loss and recovery.
\begin{verbatim}
 3.5543 n 1 m <1:1> r 0 Q DETECT
 3.5543 n 1 m <1:1> r 1 Q INTERVALS C1 2.0 C2 0.0 d 0.0105 i 1
 3.5543 n 1 m <1:1> r 1 Q NTIMER at 3.57527
 3.5685 n 2 m <1:1> r 0 Q DETECT
 3.5685 n 2 m <1:1> r 1 Q INTERVALS C1 2.0 C2 0.0 d 0.021 i 1
 3.5685 n 2 m <1:1> r 1 Q NTIMER at 3.61053
 3.5753 n 1 m <1:1> r 1 Q SENDNACK
 3.5753 n 1 m <1:1> r 2 Q INTERVALS C1 2.0 C2 0.0 d 0.0105 i 2
 3.5753 n 1 m <1:1> r 2 Q NTIMER at 3.61727
 3.5753 n 1 m <1:1> r 2 Q NACK IGNORE-BACKOFF 3.59627
 3.5828 n 3 m <1:1> r 0 Q DETECT
 3.5828 n 3 m <1:1> r 1 Q INTERVALS C1 2.0 C2 0.0 d 0.032 i 1
 3.5828 n 3 m <1:1> r 1 Q NTIMER at 3.6468
 3.5854 n 0 m <1:1> r 0 P NACK from 257
 3.5854 n 0 m <1:1> r 1 P INTERVALS D1 1.0 D2 0.0 d 0.0105
 3.5854 n 0 m <1:1> r 1 P RTIMER at 3.59586
 3.5886 n 2 m <1:1> r 2 Q INTERVALS C1 2.0 C2 0.0 d 0.021 i 2
 3.5886 n 2 m <1:1> r 2 Q NTIMER at 3.67262
 3.5886 n 2 m <1:1> r 2 Q NACK IGNORE-BACKOFF 3.63062
 3.5959 n 0 m <1:1> r 1 P SENDREP
 3.5959 n 0 m <1:1> r 1 P REPAIR IGNORES 3.62736
 3.5971 n 4 m <1:1> r 0 Q DETECT
 3.5971 n 4 m <1:1> r 1 Q INTERVALS C1 2.0 C2 0.0 d 0.0425 i 1
 3.5971 n 4 m <1:1> r 1 Q NTIMER at 3.68207
 3.5971 n 5 m <1:1> r 0 Q DETECT
 3.5971 n 5 m <1:1> r 1 Q INTERVALS C1 2.0 C2 0.0 d 0.042 i 1
 3.5971 n 5 m <1:1> r 1 Q NTIMER at 3.68107
 3.6029 n 3 m <1:1> r 2 Q INTERVALS C1 2.0 C2 0.0 d 0.032 i 2
 3.6029 n 3 m <1:1> r 2 Q NTIMER at 3.73089
 3.6029 n 3 m <1:1> r 2 Q NACK IGNORE-BACKOFF 3.66689
 3.6102 n 1 m <1:1> r 2 Q REPAIR IGNORES 3.64171
 3.6172 n 4 m <1:1> r 2 Q INTERVALS C1 2.0 C2 0.0 d 0.0425 i 2
 3.6172 n 4 m <1:1> r 2 Q NTIMER at 3.78715
 3.6172 n 4 m <1:1> r 2 Q NACK IGNORE-BACKOFF 3.70215
 3.6172 n 5 m <1:1> r 2 Q INTERVALS C1 2.0 C2 0.0 d 0.042 i 2
 3.6172 n 5 m <1:1> r 2 Q NTIMER at 3.78515
 3.6172 n 5 m <1:1> r 2 Q NACK IGNORE-BACKOFF 3.70115
 3.6246 n 2 m <1:1> r 2 Q REPAIR IGNORES 3.68756
 3.6389 n 3 m <1:1> r 2 Q REPAIR IGNORES 3.73492
 3.6533 n 4 m <1:1> r 2 Q REPAIR IGNORES 3.78077
 3.6533 n 5 m <1:1> r 2 Q REPAIR IGNORES 3.77927
\end{verbatim}
The logging of request and repair traces is done by
\fcnref{\proc[]{SRM::evTrace}}{../ns-2/srm.tcl}{SRM::evTrace}.
However, the routine
\fcnref{\proc[]{SRM/Session::evTrace}}{../ns-2/srm.tcl}{SRM/Session::evTrace},
overrides the base class definition of \proc[]{srm::evTrace},
and writes out nothing.
Individual simulation scripts can override these methods
for greater flexibility in logging options.
One possible reason to override these methods might to
reduce the amount of data generated;
the new procedure could then generate compressed and processed output.

Notice that the trace filoe contains sufficient information and details
to derive most of the statistics written out in the log file, or
is stored in the statistics arrays.

\subsection{Architecture and Internals}
\label{sec:architecture}

The SRM agent implementation splits the protocol functions
into packet handling, loss recovery, and session message activity.
\begin{list}{}{}
\item  Packet handling consists of forwarding application data messages,
  sending and receipt of control messages.
  These activities are executed by C++ methods.
\item  Error detection is done in C++ due to receipt of messages.
  However, the loss recovery is entirely done through 
  instance procedures in OTcl.
\item  The sending and processing of messages is accomplished in C++;
  the policy about when these messages should be sent is decided
  by instance procedures in OTcl.
\end{list}
We first describe the C++
\href{processing due to receipt of messages}{Section}{sec:reciept}.
Loss recovery and the sending of session messages involves
timer based processing.
The agent uses a separate \clsref{SRM}{../ns-2/srm.tcl}
to perform the timer based functions.
For each loss, an agent may do either request or repair processing.
Each agent will instantiate a separate loss recovery object
for every loss, as is appropriate for the processing that it has to do.
In the following section
\href{we describe the basic timer based functions and
the loss recovery mechanisms}{Section}{sec:recovery}.
Finally, each agent uses one timer based function
for \href{sending periodic session messages}{Section}{sec:session}.

\subsection{Packet Handling: Processing received messages}
\label{sec:reciept}

The
\fcnref{\fcn[]{recv}}{../ns-2/srm.cc}{SRMAgent::recv}
method can receive four type of messages:
data, request, repair, and session messages.

\paragraph{Data Packets}
The agent does not generate any data messages.
The user has to specify an external agent to generate traffic.
The \fcn[]{recv} method must distinguish between
locally originated data that must be sent to the multicast group,
and data received from multicast group that must be processed.
Therefore, the application agent must
set the packet's destination address to zero.

For locally originated data, 
the agent adds the appropriate SRM headers,
sets the destination address to the multicast group, 
and forwards the packet to its target.

On receiving a data message from the group,
\fcnref{\fcn[sender, msgid]{recv\_data}}{../ns-2/srm.cc}{SRMAgent::recv\_data}
will update its state marking message \tup{sender, msgid} received,
and possibly trigger requests if it detects losses.
In addition, if the message was an older message received out of order,
then there must be a pending request or repair that must be cleared.
In that case, the compiled object invokes the OTcl instance procedure,
\fcnref{\proc[sender, msgid]{recv-data}}{%
  ../ns-2/srm.tcl}{Agent/SRM::recv-data}%
\footnote{Technically,
  \fcn[]{recv\_data} invokes the instance procedure
  \|recv data \tup{sender} \tup{msgid}|,
  that then invokes \proc[]{recv-data}.}.

Currently, there is no provision for the receivers
to actually receive any application data.
The agent does not also store any of the user data.
It only generates repair messages of the appropriate size,
defined by the instance variable \|packetSize_|.
However, the agent assumes that any application data
is placed in the data portion of the packet,
pointed to by \|packet->accessdata()|.

\paragraph{Request Packets}
On receiving a request, 
\fcnref{\fcn[sender, msgid]{recv\_rqst}}{../ns-2/srm.cc}{SRMAgent::recv\_rqst}
will check whether it needs to schedule requests for other missing data.
If it has received this request
before it was aware that the source had generated this data message
(\ie, the sequence number of the request is higher than 
the last known sequence number of data from this source),
then the agent can infer that it is missing this, as well as data
from the last known sequence number onwards;
it schedules requests for all of the missing data and returns.
On the other hand, if the sequence number of the request is less
than the last known sequence number from the source,
then the agent can be in one of three states:
(1) it does not have this data, and has a request pending for it,
(2) it has the data, and has seen an earlier request,
    upon which it has a repair pending for it, or
(3) it has the data, and it should instantiate a repair.
All of these error recovery mechanisms are done in OTcl;
\fcn[]{recv\_rqst} invokes the instance procedure
\fcnref{\proc[sender, msgid,
  requester]{recv-rqst}}{../ns-2/srm.tcl}{Agent/SRM::recv-rqst}
for further processing.

\paragraph{Repair Packets}
On receiving a repair, 
\fcnref{\fcn[sender, msgid]{recv\_repr}}{../ns-2/srm.cc}{SRMAgent::recv\_repr}
will check whether it needs to schedule requests for other missing data.
If it has received this repair
before it was aware that the source had generated this data message
(\ie, the sequence number of the repair is higher than 
the last known sequence number of data from this source),
then the agent can infer that it is missing all
data between the last known sequence number and that on the repair;
it schedules requests for all of this data,
 marks this message as received, and returns.
On the other hand, if the sequence number of the request is less
than the last known sequence number from the source,
then the agent can be in one of three states:
(1) it does not have this data, and has a request pending for it,
(2) it has the data, and has seen an earlier request,
    upon which it has a repair pending for it, or
(3) it has the data, and probably scheduled a repair for it at some time;
    after error recovery, its holddown timer (equal to three times its
    distance to some requestor) expired, at which time the pending object
    was cleared.  In this last situation, the agent will simply ignore
    the repair, for lack of being able to do anything meaningful.
All of these error recovery mechanisms are done in OTcl;
\fcn[]{recv\_repr} invokes the instance procedure
\fcnref{\proc[sender, msgid]{recv-repr}}{%
  ../ns-2/srm.tcl}{Agent/SRM::recv-rqst}
to complete the loss recovery phase for the particular message.
  
\paragraph{Session Packets}
On receiving a session message,
the agent updates its sequence numbers for all active sources,
and computes its instantaneous distance to the sending agent if possible.
The agent will ignore earlier session messages from a group member,
if it has received a later one out of order.
  
Session message processing is done in
\fcnref{\fcn[]{recv\_sess}}{../ns-2/srm.cc}{SRMAgent::recv\_sess}.
The format of the session message is:
\tup{count of tuples in this message, list of tuples},
where each tuple indicates the
\tup{sender id, last sequence number from the source, time the last
  session message was received from this sender, time that that message
  was sent}.
The first tuple is the information about the local agent%
\footnote{Note that this implementation of session message handling
  is subtly different from that used in \emph{wb} or described in
  \cite{Floy95:Reliable}.
  In principle, an agent disseminates a list of the data it has
  actually received.
  Our implementation, on the other hand, only disseminates
  a count of the last message sequence number per source that the
  agent knows that that the source has sent.
  This is a constraint when studying aspects of loss recovery
  during partition and healing.
  It is reasonable to expect that the maintainer of this code will fix
  this problem during one of his numerous intervals of copious spare time.}.

\subsection{Loss Recovery Objects}
\label{sec:recovery}

In the last section,
we described the agent behaviour when it receives a message.
Timers are used to control when any particular control message is to be sent.
The SRM agent uses a separate
\clsref{SRM}{../ns-2/srm.tcl}
to do the timer based processing.
In this section, we describe the basecs if the class SRM,
and the loss recovery objects.
The following section will describe how the class SRM is used 
for sending periodic session messages.
An SRM agent will instantiate one object to recover from one lost data packet.
Agents that detect the loss will instantiate an object in the
\clsref{SRM/request}{../ns-2/srm.tcl};
agents that receive a request and have the required data will
instantiate an object in the \clsref{SRM/repair}{../ns-2/srm.tcl}.

\paragraph{Request Mechanisms}
SRM agents detect loss when they receive a message, and
infer the loss based on the sequence number on the message received.
Since packet reception is handled entirely by the compiled object,
loss detection occurs in the C++ methods.
Loss recovery, however, is handled entirely by instance procedures
of the corresponding interpreted object in OTcl.

When any of the methods detects new losses, it invokes
\fcnref{\proc[]{Agent/SRM::request}}{../ns-2/srm.tcl}{Agent/SRM::request}
with a list of the message sequence numbers that are missing.
\proc[]{request} will create a new \|requestFunction_|
object for each message that is missing.
The agent stores the object handle in its array of \|pending_| objects.
The key to the array is the message identifier \tup{sender}:\tup{msgid}.
\begin{list}{}{}
\item 
  The default \|requestFunction_| is \clsref{SRM/request}.
  The constructor for the class SRM/request
  calls the base class constructor to initialise 
  the simulator instance (\|ns_|), the SRM agent (\|agent_|),
  trace file (\|trace_|), and the \|times_| array.
  It then initialises its \|statistics_| array with the pertinent elements.

\item
  A separate call to
  \fcnref{\proc[]{set-params}}{../ns-2/srm.tcl}{SRM::set-params}
  sets the \|sender_|, \|msgid_|, \|round_| instance variables for
  the request object.
  The object determines \|C1_| and \|C2_| by querying its \|agent_|.
  It sets its distance to the sender (\|times_(distance)|)
  and fixes other scheduling parameters:
  the backoff constant (\|backoff_|),
  the current number of backoffs (\|backoffCtr_|),
  and the limit (\|backoffLimit_|) fixed by the agent.
  \proc[]{set-params} writes the trace entry ``\textsc{q detect}''.

\item
  The final step in \proc[]{request} is to schedule the timer
  to send the actual request at the appropriate moment.
  \fcnref{\proc[]{SRM/request::schedule}}{../ns-2/srm.tcl}{%
    SRM/request::schedule}
  uses 
  \fcnref{\proc[]{compute-delay}}{%
    ../ns-2/srm.tcl}{SRM/request::compute-delay}
  and its current backoff constant to determine the delay.
  The object schedules
  \fcnref{\proc[]{send-request}}{../ns-2/srm.tcl}{SRM/request::send-request}
  to be executed after \|delay_| seconds.
  The instance variable \|eventID_| stores a handle to the scheduled event.
  The default \proc[]{compute-delay} function returns a value
  uniformly distributed in the interval $[C_1 d_s, (C_1 + C_2) d_s]$,
  where $d_s$ is twice \|$times_(distance)|.
  The \proc[]{schedule} schedules an event to send a request
  after the computed delay. 
  The routine writes a trace entry ``\textsc{q ntimer } at \tup{time}''.
\end{list}

When the scheduled timer fires, the routine
\fcnref{\proc[]{send-request}}{../ns-2/srm.tcl}{SRM/request::send-request}
sends the appropriate message.
It invokes ``\|$agent_| send request \tup{args}'' to send the request.
Note that \proc[]{send} is an instproc-like,
executed by the \fcn[]{command} method of the compiled object.
However, it is possible to overload the instproc-like
with a specific instance procedure \proc[]{send}
for specific configurations.
As an example, recall that the file \|tcl/mcast/srm-nam.tcl|
overloads the \proc[]{send} command
to set the flowid based on type of message that is sent.
\proc[]{send-request} updates the statistics, and writes the trace entry
``\textsc{q sendnack}''.

When the agent receives a control message for a packet
for which a pending object exists,
the agent will hand the message off to the object for processing.
\begin{list}{}{}
\item When a 
  \fcnref{request for a particular packet is received}{../ns-2/srm.tcl}{%
        SRM/request::recv-request},
  the request object can be in one of two states:
  it is ignoring requests, considering them to be duplicates, or
  it will cancel its send event and re-schedule another one,
  after having backed off its timer.
  If ignoring requests it will update its statistics,
  and write the trace entry ``\textsc{q nack } dup''.
  Otherwise, set a time based on its current estimate of the \|delay_|,
  until which to ignore further requests.
  This interval is marked by the instance variable \|ignore_|.
  If the object reschedules its timer, it will write the trace entry
  ``\textsc{ q nack ignore-backoff } \tup{ignore}''.
  Note that this re-scheduling relies on the fact that
  the agent has joined the multicast group, and will therefore
  receive a copy of every message it sends out.
  
\item When the
  \fcnref{request object receives a repair for the particular packet}{%
    ../ns-2/srm.tcl}{SRM/request::recv-repair},
  it can be in one of two states:
  either it is still waiting for the repair,
  or it has already received an earlier repair.
  If it is the former, there will be an event pending
  to send a request, and \|eventID_| will point to that event.
  The object will compute its serviceTime, cancel that event,
  and set a holddown period during which it will ignore 
  other requests.
  At the end of the holddown period, the object will ask its
  agent to clear it.
  It will write the trace entry ``\textsc{q repair ignores } \tup{ignore}''.
  On the other hand, if this is a duplicate repair,
  the object will update its statistics, and write the trace entry
  ``\textsc{q repair } dup''.
\end{list}

When the loss recovery phase is completed by the object,
\fcnref{\proc[]{Agent/SRM::clear}}{../ns-2/srm.tcl}{Agent/SRM::clear}
will remove the object from its array of \|pending_| objects,
and place it in its list of \|done_| objects.
Periodically, the agent will cleanup and delete the \|done_| objects.

\paragraph{Repair Mechanisms}
The agent will initiate a repair if it receives a request for a packet,
and it does not have a request object \|pending_| for that packet.
The default repair object belongs to the
\clsref{SRM/repair}{../ns-2/srm.tcl}.
Barring minor differences,
the sequence of events and the instance procedures in this class
are identical to those for SRM/request.
Rather than outline every single procedure, we only outline
the differences from those described earlier for a request object.

The repair object uses the repair parameters, \|D1_|, \|D2_|.
A repair object does not repeatedly reschedule is timers;
therefore, it does not use any of the backoff variables
such as that used by a request object.
The repair object ignores all requests for the same packet.
The repair objet does not use the \|ignore_| variable that
request objects use.
The trace entries written by repair objects are marginally different;
they are ``\textsc{p nack } from \tup{requester}'',
``\textsc{p rtimer } at \tup{fireTime}'',
``\textsc{p sendrep}'', ``\textsc{p repair ignores } \tup{holddown}''.

Apart from these differences,
the calling sequence for events in a repair object is similar to that
of a request object.

\paragraph{Mechanisms for Statistics}
The agent, in concert with the request and repair objects, 
collect statistics about their response to data loss \cite{Floy95:Reliable}.
Each call to the agent \proc[]{request} procedure marks a new period.
At the start of a new period,
\fcnref{\proc[]{mark-period}}{../ns-2/srm.tcl}{Agent/SRM::mark-period}
computes the moving average of the number of duplicates in the last period.
Whenever the agent receives a first round request from another agent,
and it had sent a request in that round, then it considers the request
as a duplicate request, and increments the appropriate counters.
A request object does not consider duplicate requests if it did not
itself send a request in the first round. 
If the agent has a repair object pending, then it does not consider
the arrival of duplicate requests for that packet.
The object methods
\fcnref{\proc[]{SRM/request::dup-request?}}{../ns-2/srm.tcl}{%
        SRM/request::dup-request?} and
\fcnref{\proc[]{SRM/repair::dup-request?}}{../ns-2/srm.tcl}{%
        SRM/repair::dup-request?} 
encode these policies, and return 0 or 1 as required.

A request object also computes the elapsed time between 
when the loss is detected to when it receives the first request.
The agent computes a moving average of this elapsed time.
The object computes the elapsed time (or delay) when it
\fcnref{cancels}{../ns-2/srm.tcl}{SRM/request::cancel}
its scheduled event for the first round.
The object invokes
\fcnref{Agent/SRM::update-ave}{../ns-2/srm.tcl}{Agent/SRM::update-ave}
to compute the moving average of the delay.

The agent keeps similar statistics of the duplicate repairs,
and the repair delay.

The agent stores the number of rounds taken for one loss recovery,
to ensure that subsequent loss recovery phases for that packet
that are not definitely not due to data loss
do not account for these statistics.
The agent stores the number of routes taken for a phase in
the array \|old_|.
When a new loss recovery object is instantiated,
the object will use the agent's instance procedure
\fcnref{\proc[]{round?}}{../ns-2/srm.tcl}{Agent/SRM::round?}
to determine the number of rounds in a previous loss recovery phase
for that packet.

\subsection{Session Objects}
\label{sec:session}

Session objects,
\href{like the loss recovery objects}{Section}{sec:recovery},
are derived from the base \clsref{SRM}.
Unlike the loss recovery objects though,
the agent only creates one session object for the lifetime of the agent.
The constructor invokes the base class constructor as before;
it then sets its instance variable \|sessionDelay_|.
The agent creates the session object when it \proc[]{start}s.
At that time, it also invokes
\fcnref{SRM/session::schedule}{../ns-2/srm.tcl}{SRM/session::schedule},
to send a session message after \|sessionDelay_| seconds.

When the object sends a session message,
it will schedule to send the next one after some interval.
It will also update its statistics.
\fcnref{\proc[]{send-session}}{../ns-2/srm.tcl}{SRM/session::send-session}
writes out the trace entry ``\textsc{s session}''.

The class overrides the
\proc[]{evTrace} routine that writes out the trace entries.
\fcnref{SRM/session::evTrace}{../ns-2/srm.tcl}{SRM/sesion::evTrace}
disable writing out the trace entry for session messages.

Two types of session message scheduling strategies are currently
available:
The function in the base class schedules sending session messages at
fixed intervals of \|sessionDelay_| jittered around a small value
to avoid synchronization among all the agents at all the nodes.
\clsref{SRM/session/logScaled} schedules sending messages
at intervals of \|sessionDelay| times $\log_2$(\|groupSize_|)
so that the frequency of session messages is inversely proportional to 
the size of the group.

The base class that sends messages at fixed intervals
is the default \|sessionFunction_| for the agent.

\subsection{Extending the Base Class Agent}
\label{sec:extensions}

In
\href{the earlier section on configuration parameters}{Section}{sec:config-param},
we had shown how to trivially extend the agent to
get deterministic and probabilistic protocol behaviour.
In this section, we describe how to derive more complex
extensions to the protocol for fixed and adaptive timer mechanisms.

\subsubsection{Fixed Timers}

The fixed timer mechanism are done in
the derived \clsref{Agent/SRM/Fixed}.
The main difference with fixed timers is that
the repair parameters are set to $\log$(\|groupSize_|).
Therefore, 
\fcnref{the repair procedure of a fixed timer agent}{../ns-2/srm.tcl}{%
        Agent/SRM/Fixed::repair}
will set \d1 and \d2 to be proportional to the group size
before scheduling the repair object.

\subsubsection{Adaptive Timers}

Agents using adaptive timer mechanisms
modify their request and repair parameters under three conditions
(1) every time a new loss object is created;
(2) when sending a message; and
(3) when they receive a duplicate, if their relative distance to the loss
    is less than that of the agent that sends the duplicate.
All three changes require extensions to the agent and the loss objects.
The \clsref{Agent/SRM/Adaptive}{../ns-2/srm-adaptive.tcl}
uses \clsref{SRM/request/Adaptive}{../ns-2/srm-adaptive.tcl} and
\clsref{SRM/repair/Adaptive}{../ns-2/srm-adaptive.tcl}
as the request and repair functions respectively.
In addition, the last item requires extending the packet headers,
to advertise their distances to the loss.
The corresponding compiled class for the agent is the
\clsref{ASRMAgent}{../ns-2/srm.h}.

\paragraph{Recompute for Each New Loss Object}
Each time a new request object is created,
\fcnref{SRM/request/Adaptive::set-params}{../ns-2/srm-adaptive.tcl}{%
        SRM/request/Adaptive::set-params}
invokes \|$agent_ recompute-request-params|.
The agent method
\fcnref{\fcn[]{recompute-request-params}}{../ns-2/srm-adaptive.tcl}{%
        Agent/SRM/Adaptive::recompute-request-params}.
uses the statistics about duplicates and delay
to modify \c1 and \c2 for the current and future requests.

Similarly,
\fcnref{SRM/request/Adaptive::set-params}{../ns-2/srm-adaptive.tcl}{%
        SRM/request/Adaptive::set-params}
for a new repair object
invokes \|$agent_ recompute-repair-params|.
The agent method
\fcnref{\fcn[]{recompute-repair-params}}{../ns-2/srm-adaptive.tcl}{%
        Agent/SRM/Adaptive::recompute-repair-params}.
uses the statistics objects to modify \d1 and \d2
for the current and future repairs.

\paragraph{Sending a Message}
If a loss object 
\fcnref{sends a request}{../ns-2/srm-adaptive.tcl}{%
        SRM/request/Adaptive::send-request}
in its first \|round_|,
then the agent, in the instance procedure
\fcnref{\proc[]{sending-request}}{../ns-2/srm-adaptive.tcl}{%
        Agent/SRM/Adaptive::sending-request},
will lower \c1,
and set its instance variable \|closest_(requestor)| to 1.

Similarly,
a loss object that
\fcnref{sends a repair}{../ns-2/srm-adaptive.tcl}{%
        SRM/repair/Adaptive::send-repair}
in its first \|round_|
will invoke the agent's instance procedure,
\fcnref{\proc[]{sending-repair}}{../ns-2/srm-adaptive.tcl}{%
        Agent/SRM/Adaptive::sending-repair},
to lower \d1 and set \|closest_(repairor)| to 1.

\paragraph{Advertising the Distance}
Each agent must add additional information to each request/repair
that it sends out.
The base \clsref{SRMAgent}{../ns-2/srm.cc}
invokes the virtual method
\fcnref{\fcn[]{addExtendedHeaders}}{../ns-2/srm.h}{%
        SRMAgent::addExtendedHeaders}
for each SRM packet that it sends out.
The method is invoked after adding the SRM packet headers, and
before the packet is transmitted.
The adaptive SRM agent overloads the method
\fcnref{\fcn[]{addExtendedHeaders}}{../ns-2/srm.h}{%
        ASRMAgent::addExtendedHeaders}
to specify its distances in the additional headers.
When sending a request, that agent unequivocally knows the
identity of the sender 

Sinilarly, the method
\fcnref{\fcn[]{parseExtendedHeaders}}{../ns-2/srm.h}{%
        ASRMAgent::parseExtendedHeaders}
is invoked everytime an SRM paket is received.
It sets the agent member variable \|pdistance_|
to the distance advertised by the peer that sent the message.
The member variable is bound to an instance variable of the same name,
so that the peer distance can be accessed
by the appropriate instance procedures.

Finally, the adaptive SRM agent's extended headers are defined as
\structref{hdr\_asrm}{../ns-2/srm.h}.
Unlike most other packet headers, 
these are not automatically available in the packet.
The
\fcnref{interpreted constructor}{../ns-2/srm-adaptive.tcl}{%
        Agent/SRM/Adaptive::init}
for the first adaptive agent
will add the header to the packet format.

\end{document}

### Local Variables:
### mode: latex
### comment-column: 60
### backup-by-copying-when-linked: t
### file-precious-flag: nil
### End:
