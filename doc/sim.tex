%
% wee haw
% personal commentary:
%        DRAFT DRAFT DRAFT
%        - KFALL
%
\chapter{The Class Simulator}
\label{chap:sim}

The overall simulator is described by a Tcl
\clsref{Simulator}{../ns-2/ns-lib.h}.
It provides a set of interfaces for configuring a simulation
and for choosing the type of event scheduler used to drive
the simulation.
A simulation script generally begins by creating an instance
of this class and calling various methods to create nodes, topologies,
and configure other aspects of the simulation.
A subclass of Simulator called \code{OldSim} is used to support
\ns~v1 backward compatibility.

The procedures and functions described in this chapter can be found in
\nsf{tcl/lib/ns-lib.tcl}, \nsf{scheduler.\{cc,h\}}, and, \nsf{heap.h}.

\section{Simulator Initialization}
\label{sec:siminit}

When a new simulation object is created in tcl,
\fcnref{the initialization procedure}{../ns-2/ns-lib.h}{Simulator::init}
performs the following operations:
\begin{itemize}
        \item initialize the packet format (calls {\tt create\_packetformat})
        \item create a scheduler (defaults to a calendar scheduler)
        \item create a ``null agent'' (a discard sink used in various places)
\end{itemize}
The packet format initialization sets up field offsets within packets
used by the entire simulation.  It is described in more detail
in \href{the following chapter on packets}{Chapter}{chap:pformat}.
The scheduler runs the simulation in an event-driven manner and may
be replaced by alternative schedulers which provide somewhat
different semantics (see the following section for more detail).
The null agent is created with the following call:
\begin{program}
        set nullAgent_ [new Agent/Null]
\end{program}
This agent is generally useful as a sink for dropped packets or
as a destination for packets that are not counted or recorded.

\section{Schedulers and Events}
\label{sec:sched}

The simulator is an event-driven simulator.
There are presently four schedulers available in the simulator, each
of which is implemented using a different data structure:
a simple linked-list (default), heap, calendar queue, and a special
type called ``real-time''.  Each of these are described below.
The scheduler runs by selecting the next earliest event, executing
it to completion, and returning to execute the next event.
Presently, the simulator is single-threaded, and only one event
in execution at any given time.
If more than one event are scheduled to execute at the same time,
their execution is performed on the first scheduled -- first
dispatched manner.  Simultaneous events are not re-ordered anymore by
schedulers (as it was in earlier versions) and all schedulers should
yeild the same order of dispatching given the same input.
 
No partial execution of events or pre-emption is supported.

An {\em event} generally comprises a ``firing time'' and a handler function.
The actual definition of an event is found in \nsf{scheduler.h}:
\begin{program}
        class Event \{ 
        public: 
                Event* next_;           \* event list */ 
                Handler* handler_;      \* handler to call when event ready */
                double time_;           \* time at which event is ready */
                int uid_;               \* unique ID */
                Event() : time_(0), uid_(0) \{\}
        \};   
        /*   
         * {\cf The base class for all event handlers.  When an event's scheduled}
         * {\cf time arrives, it is passed to handle which must consume it.}
         * {\ie, if it needs to be freed it, it must be freed by the handler.}
         */  
        class Handler \{
         public: 
                virtual void handle(Event* event) = 0;
        \};   
\end{program}
Two types of objects are derived from the base
\clsref{Event}{../ns-2/scheduler.cc}: packets and ``at-events''.
Packets are described in detail 
\href{in the next chapter}{Chapter}{sec:packetclass}.
An at-event is a tcl procedure execution scheduled to occur at
a particular time. 
This is frequently used in simulation scripts.
A simple example of how it is used is as follows:
\begin{program}
        \ldots
        set ns_ [new Simulator]
        $ns_ use-scheduler Heap
        $ns_ at 300.5 "$self complete_sim"
        \ldots
\end{program}
This tcl code fragment first creates a simulation object,
then changes the default scheduler implementation to be heap-based
(see below), and finally schedules the function {\tt \$self complete\_sim}
to be executed at time 300.5 (seconds)%
\footnote{Note that this particular code fragment expects to be encapsulated 
in an object instance procedure, where the appropriate
reference to \code{\$self} is correctly defined.}.
At-events are implemented as events where the handler is
effectively an execution of the tcl interpreter.

\subsection{The List Scheduler}
\label{sec:listsched}

The list scheduler (\clsref{Scheduler/List}{../ns-2/scheduler.cc})
implements the scheduler using a simple linked-list structure.
The list is kept in time-order (earliest to latest), so event
insertion and deletion require scanning the list to find the
appropriate entry.
Choosing the next event for execution requires trimming the first
entry off the head of the list.
This implementation preserves event execution in a FIFO manner
for simultaneous events.

\subsection{the heap scheduler}
\label{sec:heapsched}

The heap scheduler (\clsref{Scheduler/Heap}{../ns-2/scheduler.cc})
implements the scheduler using a heap structure.
This structure is superior to the list structure for a large number
of events, as insertion and deletion times are in $O(\log n)$
for $n$ events.
This implementation in \ns~v2 is borrowed from the
MaRS-2.0 simulator \cite{Alae94:Design};
it is believed that MaRS itself borrowed the code
from NetSim \cite{Heyb89:Netsim},
although this lineage has not been completely verified.

\subsection{The Calendar Queue Scheduler}
\label{sec:cqsched}

The calendar queue scheduler
(\clsref{Scheduler/Calendar}{../ns-2/scheduler.cc})
uses a data structure analogous to a one-year desk calendar,
in which events on the same month/day of multiple years can be recorded in
one day.
It is formally described in \cite{Brow88:Calendar}, and informally described
in Jain (p. 410) \cite{Jain91:Art}.
The implementation of Calendar queues in \ns~v2
was contributed by David Wetherall (presently at MIT/LCS).

\subsection{The Real-Time Scheduler}
\label{sec:rtsched}

The real-time scheduler (\clsref{Scheduler/RealTime}{../ns-2/scheduler.cc})
attempts to synchronize the execution of events with real-time.
It is currently implemented as a subclass of the list scheduler.
The real-time capability in ns is still under development, but is used
to introduce an \ns\ simulated network into a real-world topology
to experiment with easily-configured network topologies, cross-traffic, etc. 
This only works for relatively slow network traffic data rates, as the
simulator must be able to keep pace with the real-world packet arrival
rate, and this synchronization is not presently enforced.

\section{Other Methods}
\label{sec:other}

The {\tt Simulator} class provides a number of methods used
to set up the simulation.
They generally fall into three categories:
methods to create and manage the topology 
(which in turn consists of
\href{managing the nodes}{Chapter}{chap:nodes} and
\href{managing the links}{Chapter}{chap:links}),
\href{methods to perform tracing}{Chapter}{chap:trace},
and helper functions to deal with the scheduler.
The following is a list of the non-topology related simulator methods:
\begin{program}
Simulator instproc now {} \; return scheduler's notion of current time;
Simulator instproc at args \; schedule execution of code at specified time;
Simulator instproc cancel args \; cancel event;
Simulator instproc run args \; start scheduler;
Simulator instproc halt {} \; stop (pause) the scheduler;
Simulator instproc flush-trace {} \; flush all trace object write buffers;
Simulator instproc create-trace { type files src dst } \; create trace object;
Simulator instproc create_packetformat \; set up the simulator's packet format;
\end{program}

\endinput
