\chapter{Multicast Routing}
\label{chap:multicast}

This section describes the usage and the internals of multicast
routing implementation in \ns.
We first describe 
\href{the user interface to enable multicast routing}{Section}{sec:mcast-api},
specify the multicast routing protocol to be used and the
various methods and configuration parameters specific to the
protocols currently supported in \ns.
We then describe in detail 
\href{the internals and the architecture of the
multicast routing implementation in \ns}{Section}{sec:mcast-internals}.

The procedures and functions described in this chapter can be found in
various files in the directories \nsf{tcl/mcast}, \nsf{tcl/ctr-mcast};
additional support routines
are in \nsf{mcast\_ctrl.\{cc,h\}},
\nsf{tcl/lib/ns-lib.tcl}, and \nsf{tcl/lib/ns-node.tcl}.

\section{Multicast API}
\label{sec:mcast-api}

Multicast forwarding requires enhancements
to the nodes and links in the topology.
Therefore, the user must specify multicast requirements
to the Simulator class before creating the topology.
This is done in one of two ways:
\begin{program}
        set ns [new Simulator -multicast ignored_flag]
    {\rm or}
        set ns [new Simulator]
        $ns multicast
\end{program}			%$
With this, nodes contain additional classifiers and replicators
for multicast forwarding, and links contain elements to assign
incoming interface labels to all packets entering a node.

A multicast routing strategy is the mechanism by which
the multicast distribution tree is computed in the simulation.
\ns\ supports two multiast route computation strategies:
	centralised or detailed.
The method \proc[]{mrtproto} in the Class Simulator specifies
either the route computation strategy, for centralised multicast routing,
or the specific detailed multicast routing protocol that should be used.
For detailed multicast routing, \proc[]{mrtproto} will accept,
as additional arguments, a list of nodes
that will run an instance of that routing protocol.
Thus, the following are examples of
valid invocations of multicast routing in \ns:
\begin{program}
        set cmc [$ns mrtproto CtrMcast \{\}]    \; specify centralized multicast for all nodes;
        \; cmc is the handle for multicast protocol object;
        $ns mrtproto DM $n1 $n2 $n3 \; specify dense mode multicast for nodes n1, n2 and n3;
        $ns mrtproto detailedDM                  \; specify dynamic dense mode to run on all nodes;
\end{program}
Notice in the above examples that CtrMcast returns a handle that
can be used for additional configuration of centralised multicast routing.
The other routing protocols will return a null string.
If no additional nodes are specified for detailed multicast routing,
all the nodes in the topology will run instances of the same protocol.

New/unused multicast address are allocated using the procedure
\proc[]{allocaddr}.
The default configuration in \ns\ only provides for
128 node topologies when multicast routing is enabled.
The procedure \proc[]{expandaddr}
expands the address space to support $2^{30} - 1$ node topologies.
\proc[]{allocaddr} and \proc[]{expandaddr} are class
procedures in the class Node.

The agents use the instance procedures
\proc[]{join-group} and \proc[]{leave-group}, in
the class Node to join and leave multicast groups. These procedures
take two mandatory arguments. The first argument identifies the
corresponding agent and second argument specifies the group address.

An example of a relatively simple multicast configuration is:
\begin{program}
        set ns [new Simulator {\bfseries{}-multicast on}] \; enable multicast routing;
        Node expandaddr  \; expand address space, required if more than 128 nodes;
        set group [{\bfseries{}Node allocaddr}]   \; allocate a multicast address;
        set node0 [$ns node]         \; create multicast capable nodes;
        set node1 [$ns node]
        $ns duplex-link $n0 $n1 1.5Mb 10ms DropTail

        set mproto detailedDM          \; configure multicast protocol;
        set mrthandle [{\bfseries{}$ns mrtproto $mproto \{\}}] \; if an empty list is give,;
                \; all nodes will contain multicast protocol agents;
        set udp [new Agent/UDP]		\; create a source agent at node 0;
        $ns attach-agent $node0 $udp 
        set src [new Application/Traffic/CBR]        
        $src attach-agent $udp
        {\bfseries{}$udp set dst_ $group}

        set rcvr [new Agent/LossMonitor]  \; create a receiver agent at node 1;
        $ns attach-agent $node1 $rcvr
        $ns at 0.3 "{\bfseries{}$node1 join-group $rcvr $group}" \; join the group at simulation time 0.3 (sec);
\end{program}

\subsection{Multicast Behavior Monitor Configuration}
\ns\ supports a multicast monitor module that can trace
some types of packet activity about a specific source group pair.
The module counts the number of packets in transit periodically
and prints the results to stdout. \proc[]{trace-topo} counts
all links.  \proc[]{trace-tree} counts only links on the tree.
Currently, it only traces prune, join, register, and data packets.

\begin{program}
        set mcastmonitor [$ns McastMonitor]
        $mcastmonitor set period_ 0.02         \; default 0.03 (sec);
        $mcastmonitor trace-topo $src $group   \; trace the entire topology;
\end{program}

% SAMPLE OUTPUT?
The following sample output illustrates the output file format (time, prune, join, register, data, source, group):
{\small
\begin{verbatim}
0.14999999999999999 0 0 4 0 0 32770
0.17999999999999999 0 0 4 0 0 32770
0.20999999999999999 0 0 4 2 0 32770
0.23999999999999999 0 0 4 7 0 32770
0.27000000000000002 0 0 4 8 0 32770
\end{verbatim}
}

\subsection{Protocol Specific configuration}

In this section, we briefly illustrate the
protocol specific configuration mechanisms
for all the protocols implemented in \ns.

\paragraph{Centralized Multicast}
The centralized multicast is a sparse mode implementation of multicast
similar to PIM-SM \cite{Deer94a:Architecture}.
A Rendezvous Point (RP) rooted shared tree is built
for a multicast group.  The actual sending of prune, join messages
etc. to set up state at the nodes is not simulated.  A centralized
computation agent is used to compute the forwarding trees and set up
multicast forwarding state, \tup{S, G} at the relevant nodes as new
receivers join a group.  Data packets from the senders to a group are
unicast to the RP.  Note that data packets from the senders are
unicast to the RP even if there are no receivers for the group.

The method of enabling centralised multicast routing in a simulation is:
\begin{program}
        set mproto CtrMcast    \; set multicast protocol;
        set mrthandle [$ns mrtproto $mproto \{\}]
\end{program}
The command procedure \proc[]{mrtproto}
returns a handle to the multicast protocol object.
This handle can be used to control the RP and the boot-strap-router (BSR),
switch tree-types for a particular group,
from shared trees to source specific trees or vice-versa, and
recompute multicast routes.
\begin{program}
        $mrthandle set_c_rp \{$node0 $node1\}      \; set the RPs;
        $mrthandle set_c_bsr \{$node0:0 $node1:1\} \; set the BSR, specified as list of node:priority;

        $mrthandle get_c_rp $node0 $group          \; get the current RP ???;
        $mrthandle get_c_bsr $node0                \; get the current BSR;

        $mrthandle switch-treetype $group         \; to source specific or shared tree;

        $mrthandle compute-mroutes       \; recompute routes. usually invoked automatically as needed;
\end{program}

Note that whenever network dynamics occur or unicast routing changes,
\proc[]{compute-mroutes} could be invoked to recompute the multicast routes.
The instantaneous re-computation feature of centralised algorithms
may result in causality violations during the transient
periods.

\paragraph{Static Dense Mode}
The Static Dense Mode protocol is loosely based on DVMRP \cite{rfc1075}.
It computes parent-child relationships as in DVMRP
to reduce the number of links over which data packets are broadcast.
At each node,
the implementation
 infers poison-reverse information
and
 computes the parent-child relationships 
 for that node by looking
at the unicast routing tables at each of the adjacent nodes.
The implementation only works on point-to-point links, \ie, no LANs;
it does not also adapt to network dynamics.

Any node that receives data for a particular group for which
it has no downstream receivers, send a prune upstream.
A prune message causes the upstream node to initiate prune state 
at that node.
The prune state prevents that node from sending data for that group
downstream to the node that sent the original prune message
while the state is active.
The time duration for which a prune state is active is
configured through the DM class variable, \code{PruneTimeout}.
A typical DM configuration is shown below:
\begin{program}
        DM set PruneTimeout 0.3           \; default 0.5 (sec);
        set mproto DM
        set mrthandle [$ns mrtproto $mproto \{\}]
\end{program}

\paragraph{Detailed dense mode}
This protocol implements a basic dense mode protocol,
similar to DM described earlier.
However, it does not compute parent-child relationships;
this implementation also works over LANs, 
and is capable of adapting to changes in the topology.
The timers have been implemented in a separate class,
with more fine grained control.
The three timers are:
\begin{alist}
Timer/GraftRtx & \textbf{Graft Retransmit Timer} \hfill default = 0.05s.\\
		 & set by a downstream node
		when it has active receivers downstream and 
		needs to remain grafted onto the multicast tree 
		at all times, even in the face of topology or
		group membership dynamics. \\[2ex]
Timer/Iface/Prune & \textbf{Prune Timer} \hfill  default = 0.5s.\\
		& set by an upstream node on specific outgoing links to avoid
		sending data on those links. \\[2ex]
Timer/Iface/Deletion & \textbf{Interface Deletion Timer} \hfill default = 0.1s.\\
		& set by a node on a LAN interface
		to mark that interface as non-forwarding,
		possibly because another node on that LAN has 
		``assert''ed. \\[2ex]
\end{alist}
Each timer has a class variable, \code{timeout_},
that controls the firing interval of that timer,
and hence can be individually set.
A typical detailedDM script might be:
\begin{program}
        Timer/Iface/Prune set timeout_ 0.3s.	\; default = 0.5s. ;
        set mproto dynamicDM
        set mrthandle [$ns mrtproto $mproto \{\}]
\end{program}

\section{Internals of Multicast Routing}
\label{sec:mcast-internals}

We describe the internals in three parts:
first the classes to implement and support multicast routing;
second, the specific protocol implementation details; and
finally, provide a list of the variables that are used in the implementations.

\subsection{The classes}
The main classes in the implementation are
the \clsref{mrtObject}{../ns-2/tcl/mcast/McastProto.tcl} and
the \clsref{McastProtocol}{../ns-2/tcl/mcast/McastProto.tcl}.
There are also extensions to the base classes:
Simulator, Node, Classifier, \etc.
We describe these classes and extensions in this subsection.
The specific protocol implementations also use adjunct data structures
for specific tasks, such as timer mechanisms by detailed dense mode,
encapsulation/decapsulation agents for centralised multicast \etc.;
we defer the description of these objects to the section on 
the description of the particular protocol itself.

\paragraph{mrtObject class}
There is one mrtObject object per multicast capable node.
Each  mrtObject object maintain a list of multicast protocols.
This arbiter supports the ability for a node to run multiple multicast
routing protocols.
The node uses the arbiter to perform protocol actions,
either to a specific protocol instance active at that node,
or to all protocol instances at that node.
\begin{alist}
\proc[instance]{addproto} &
	adds the handle for a protocol instance to its list of protocols. \\
\proc[protocol]{getType} &
	returns the handle to the protocol instance active at that node,
	and the NULL string if none exists. \\
\proc[op, args]{all-mprotos} &
	internal routine to execute ``\code{op}'' with ``\code{args}''
	on all protocol instances active at that node. \\
\proc[]{start} &
	starts execution of all protocols. \\
\proc[]{stop} &
	stop execution of all protocols. \\
\proc[]{notify} &
	force route recompute of all protocols following a topology change. \\
\proc[file-handle]{dump-mroutes} &
	dump multicast routes to specified file-handle. \\
{\let\[=[\let\]=]
\proc[G, \[S\]]{join-group}} &
	signals all protocol instances to join \tup{S, G}. \\
{\let\[=[\let\]=]
\proc[G, \[S\]]{leave-group}} &
	signals all protocol instances to leave \tup{S, G}. \\
\proc[code, s, g, iif]{upcall} &
	signalled by node on forwarding errors in classifier;
	this routine in turn signals all protocol instances that will
	then invoke the appropriate handle function for that particular code.\\
\proc[rep, s, g, iif]{drop} &
	Called on packet drop, possibly to prune an interface. \\
\end{alist}

In addition, the mrtObject class supports the concept of well known
groups, \ie, those groups that do not require explicit protocol support.
Two well known groups, \code{ALL_ROUTERS} and \code{ALL_PIM_ROUTERS}
are predefined in \ns.
The \clsref{mrtObject}{../ns-2/tcl/mcast/McastProto.tcl} defines
two class procedures to set and get information about these well known groups.
\begin{alist}
\proc[name]{registerWellKnownGroups} & 
	assigns \code{name} a well known group address. \\
\proc[name]{getWellKnownGroup} &
	returns the address allocated to well known group, \code{name}.
	If \code{name} is not registered as a well known group,
	then it returns the address for \code{ALL\_ROUTERS}.
\end{alist}

\paragraph{McastProtocol class}
This is the base class for the implementation of all the multicast protocols.
It contains basic multicast functions:
\begin{alist}
\proc[]{start}, \proc[]{stop} &
	Set the \code{status\_} of execution of this protocol instance. \\
\proc[]{getStatus} &
	return the status of execution of this protocol instance. \\
\proc[]{getType} &
	return the type of protocol executed by this instance. \\
\proc[code args]{upcall} &
	invoked when the node classifier signals an error, either due to 
	a cache-miss or a wrong-iif for incoming packet.  This routine
	invokes the protocol specific handle, \proc{handle-\tup{code}} with
	specified \code{args} to handle the signal. \\
\end{alist}
The class also defines wrappers to access interface related information.
These directly translate into calls to the node or link;
however, these wrappers also perform some additional minimal checking
as noted below:
\begin{alist}
\proc[src]{rpf-nbr} &
	return the RPF neighbour towards the \code{src}
	at that node. \\
\proc[ifid]{iface2link} &
	return the link that tags the specified \tup{if id}.\\
\proc[link]{link2iif} &
	return the label affixed by the specified link.
	Check if the destination of the link is this node itself.
	(However, at this time, do nothing if this is not true.) \\
\proc[link]{link2oif} &
	return the output object at the head of the specifed link.
	Check if the link is incident to it.
	(However, at this time, do nothing if this is not true.) \\
\end{alist}

\subsection{Extensions to other classes in \ns}
In 
\href{the earlier chapter describing nodes in \ns}{Chapter}{chap:nodes},
we described the internal structure of the node in \ns.
To briefly recap that description, the node entry for a multicast node is
the \code{switch_}.  It looks at the highest bit to decide
 if the destination is a multicast or unicast packet.
 Multicast packets are forwarded to a multicast
classifier which maintains a list of replicators;
there is one replicator per \tup{source, group, incoming interface} tuple.
Replicators copy the incoming packet and forward to all outgoing interfaces.

\paragraph{Class Node}
Node support for multicast in two primary ways:
as a focal point for access to the multicast protocols,
in the areas of address allocation, control and management, and
group membership dynamics;
and secondly, primitives to access and control interfaces on links
incident on that node.
\begin{alist}
\proc[]{expandaddr}, & \\
\proc[]{allocaddr} &
	Class procedures for address management.
	\proc[]{expandaddr} increases the address space from 128
	multicast capable nodes to $2^{30} - 1$.
	\proc[]{allocaddr} allocates the next available multicast
	address.\\[2ex]
\proc[]{start-mcast}, & \\
\proc[]{stop-mcast} &
	To start and stop multicast routing at that node. \\
\proc[]{notify-mcast} &
	\proc[]{notify-mcast} signals the mrtObject at that node to
	recompute multicastroutes following a topology change or unicast 
	route update from a neighbour.
	\\[2ex]
\proc[]{getArbiter} &
	returns a handle to mrtObject operating at that node. \\
\proc[file-handle]{dump-routes} &
	to dump the multicast forwarding tables at that node. \\[2ex]
\proc[s g iif code]{new-group} &
	When a multicast data packet is received,
	and the multicast classifier cannot find the slot
	corresponding to that data packet,
	it invokes \proc[]{Node::new-group}
	to establish the appropriate entry.
	The code indicates the reason for not finding the slot.
	Currently there are two possibilities, cache-miss and wrong-iif.
	This procedure notifies the arbiter instance
	to establish the new group. \\
\proc[a g]{join-group} &
	An \tup{agent} at a node that joins a particular group
	invokes ``\code{\$node join-group \tup{agent} \tup{group}}''.	%$
	The node signals the mrtObject to join  the particular \tup{group},
	and adds \tup{agent} to its list of agents at that \tup{group}.
	It then adds \tup{agent} to all replicators
	associated with \tup{group}. \\
\proc[a g]{leave-group} &
	\proc[]{Node::leave-group} reverses the process described earlier.
	It disables the outgoing interfaces to the receiver agents
	for all the replicators of the group,
	deletes the receiver agents from the local \code{Agents\_} list;
	it then invokes the arbiter instance's \proc[]{leave-group}.\\[2ex]
\proc[s g iif oiflist]{add-mfc} &
	\proc[]{Node::add-mfc} adds a multicast forwarding cache entry for
	a particular \tup{source, group, iif}.
	The mechanism is:
	(1) create a new replicator (if one does not already exist),
	(2) update the \code{replicator\_} and \code{repByGroup\_}
	 instance variable arrays at the node,
	(3) adds all outgoing interfaces and local agents
	to the appropriate replicator,
	and finally,
	(4) invoke the multicast classifier's \proc[]{add-rep}
	 to create a slot for the replicator in the multicast classifier. \\
\proc[s g oiflist]{del-mfc} &
	disables each oif in \tup{oiflist} from the replicator for \tup{s g}.\\
\end{alist}
The list of primitives accessible at the node to control its interfaces are listed below.
\begin{alist}
\proc[ifid link]{add-iif}, & \\
\proc[link if]{add-oif} &
	Invoked during link creation to prep the node about its 
	incoming interface label and outgoing interface object. \\
\proc[link]{get-oif}, & \\
\proc[]{get-all-oifs} &
	Return the oif for a specific link, or the oifs for all links
	incident on this node. \\
\proc[link]{link2oif} &
	Returns the nsObject on link that is incident to the node.\\
\proc[src]{rpf-nbr} &
	Returns a handle to the neighbour node that is its next hop to the 
	specified \tup{src}.\\[2ex]
\proc[s g]{getReps} &
	Returns a handle to the replicator that matches \tup{s, g}.
	If either is a wildcard, then the match is also a wildcard,
	returning the list of handles to such replicators
	that match the wildcard. \\
\proc[s g]{getReps-raw} &
	As above, but returns a list of \tup{key, handle} pairs. \\
\proc[s g]{clearReps} &
	Removes all replicators associated with \tup{s, g}. \\[2ex]
\end{alist}

\paragraph{Class Link and SimpleLink}
This class provides methods to check the type of link, and the label it 
affixes on individual packets that traverse it.
There is one additional method to actually place the interface objects on this link.
These methods are:
\begin{alist}
\proc[]{isLan?} & \\
\proc[]{if-label?} & returns the interface label affixed by this link
	to packets that traverse it. \\
\proc[]{enable-mcast} & Internal procedure
	called by the SimpleLink constructor to add appropriate objects
	and state for multicast.
	By default, (and for the point-to-point link case)
	it places a networkInterface object on the link,
	and signals the nodes on incident on the link about this link.\\
\end{alist}

\paragraph{Class networkInterface}
This is a simple connector that is placed on each link.
It affixes its label id to each packet that traverses it.  
The packet id is used by the destination node incident on that link
to identify the link by which the packet reached it.
The label id is configured by the Link constructor.
This object is an internal object, and is not designed to be manipulated
by user level simulation scripts.
The object only supports two methods:
\begin{alist}
\proc[ifid]{label} & assigns \tup{if id} that this object will affix to each packet. \\
\proc[]{label} & returns the label that this object affixes to each packet.\\
\end{alist}
The global class variable, \code{ifnum_}, specifies the next available 
\tup{ifid} number.

\paragraph{Class Classifier}
There is one multicast classifier per node.
The node stores a reference to this classifier in its instance variable
\code{multiclassifier_}.
When this classifier receives a packet,
it looks at the \tup{source, group} information in the packet headers,
and the interface that the packet arrived from (the incoming interface or iif);
using that information, the classifier will identify the slot
that should be used to forward that packet.  The slot will point
to the appropriate replicator.

However, if the slot is invalid, or the classifier does not have an
entry for this \tup{source, group, iif},
then it will invoke \proc[]{new-group} for the classifier,
with one of two codes to identify the problem:
1) \code{cache-miss} indicates that the classifier did not find any
\tup{source, group} entries;
2) \code{wrong-iif} indicates that the classifier found \tup{source, group}
entries, but none matching the interface that this packet arrived over.
In this latter situation, this particular data is dropped.

\proc[]{add-rep} creates a slot in the classifier
and adds a replicator for \tup{source, group, iif} to that slot.

\paragraph{Class Replicator}
When a replicator receives a packet,
it copies the packet to all of its slots.
Each slot points to an outgoing interface for a particular 
\tup{source, group, iif}.
If no slot is found, the C++ replicator invokes the class 
instance procedure \proc[]{drop} to
trigger protocol specific actions.
We will describe the protocol specific actions in the next section,
when we describe the internal procedures of each of the 
multicast routing protocols.

There are instance procedures to control the elements in each slot:
\begin{alist}
\proc[oif]{insert} & inserting a new outgoing interface
                        to the next available slot.\\
\proc[oif]{disable} & disable the slot pointing to the specified oif.\\
\proc[oif]{enable} &  enable the slot pointing to the specified oif.\\
\proc[]{is-active} & returns true if the replicator has at least one active slot.\\
\proc[oif]{exists} & returns true if the slot pointing to the specified oif is active.\\
\proc[source, group, oldiif, newiif]{change-iface} & modified the iif entry for the particular replicator.\\
\end{alist}

\subsection{Protocol Internals}
\label{sec:mcastproto-internals}

We now describe the implementation of
the different multicast routing protocol agents.

\subsubsection{Centralized Multicast}
\code{CtrMcast} is inherits from \code{McastProtocol}.
One CtrMcast agent needs to be created for each node.
There is a central CtrMcastComp agent to compute and install
multicast routes for the entire topology.
Each CtrMcast agent processes membership dynamic commands, 
and redirects the CtrMcastComp agent to recompute the appropriate routes.
\begin{alist}
\proc[]{join-group} &
	registers the new member with the \code{CtrMCastComp} agent, and
	invokes that agent to recompute routes for that member. \\
\proc[]{leave-group} & is the inverse of \proc[]{join-group}. \\
\proc[]{handle-cache-miss} &
	 called when no proper forwarding entry is found
	 for a particular packet source.
	In case of centralized multicast,
	it means a new source has started sending data packets.
	Thus, the CtrMcast agent registers this new source with the
	\code{CtrMcastComp} agent.
	If there are any members in that group, compute the new multicast tree.
	If the group is in RPT (shared tree) mode, then
	1) create an encapsulation agent at the source;
	2) a corresponding decapsulation agent is created at the RP;
	3) the two agents are connected by unicast; and
	4) the \tup{S,G} entry points its outgoing interface
	to the encapsulation agent.
\end{alist}

\code{CtrMcastComp} is the centralised multicast route computation agent.
\begin{alist}
\proc[]{reset-mroutes} & resets all multicast forwarding entries.\\
\proc[]{compute-mroutes} & (re)computes all multicast forwarding entries.\\
\proc[source, group]{compute-tree} & computes a multicast tree for one source to reach 
                all the receivers in a specific group.\\
\proc[source, group, member]{compute-branch} & is executed when a receiver joins a multicast group.
        It could also be invoked by \proc[]{compute-tree} when it itself
        is recomputing the multicast tree, and has to reparent
        all receivers.
        The algorithm starts at the receiver, recursively
        finding successive next hops,
        until it either reaches the source or RP,
        or it reaches a node that is already 
        a part of the relevant multicast tree.
         During the process, several new replicators and an
        outgoing interface will be installed.\\
\proc[source, group, member]{prune-branch} & is similar to \proc[]{compute-branch} except the
        outgoing interface is disabled;
        if the outgoing interface list is empty at that node,
        it will walk up the multicast tree, pruning at each of the
        intermediate nodes, until it reaches a node that has a
        non-empty outgoing interface list for the particular multicast tree.
\end{alist}

\subsubsection{Dense Mode}
\begin{alist}
\proc[group]{join-group} & sends graft messages upstream if \tup{S,G} does not
        contain any active outgoing slots (\ie, no downstream receivers).\\
\proc[group]{leave-group} & does not do anything.\\
\proc[group]{handle-cache-miss} & creates \tup{S,G} with only
        the outgoing interfaces to each of the child nodes.
        For basic dense mode,
        parent-child relationship is computed only once at the start-up.\\
 &      However, for dynamic dense mode,
        the incoming interface must be set properly so packets from the
        wrong incoming interfaces will be dropped. \\
 &      Finally, in the case of PIM dense mode, 
        \proc[]{handle-cache-miss} is the same as in dynamic dense mode,
        except that the outgoing interface list is set to all neighbors
        excluding the incoming interface.\\
\proc[replicator, source, group]{drop} & sends prune messages upstream.
        Recall that the packet is only dropped when there are
        no downstream receivers for the \tup{S, G} tuple.\\
\proc[from, source, group]{recv-prune} & resets the prune timer
         if the interface had been pruned previously;
        otherwise, it starts the prune timer and disables the interface;
        furthermore,  if the outgoing interface list becomes empty,
        it forwards the prune message upstream.\\
\proc[from, source, group]{recv-graft} & cancels any existing prune timer, and
        re-enables the pruned interface.
        If the outgoing interface list was previously empty,
        it forwards the graft upstream.\\
\proc[]{periodic-check} & This procedure is only used in dynamic dense mode.
        Each node periodically updates its parent-child relationships
        with respect to each of its neighbors.\\
\proc[]{handle-wrong-iif} & This procedure is only used by PIM dense mode.
        This is invoked when the multicast classifier drops a packet
        because it arrived on the wrong interface, and
        invoked \proc[]{new-group}.
        This routine is invoked by \proc[]{mrtObject::new-group}.
        When invoked, the agent checks if the incoming interface in the
        forwarding entry has become outdated.
        If so, this procedure will update the forwarding entry to the
        new incoming interface;
        otherwise, it will send a prune message back towards the
        wrong incoming interface to stop packets
        reaching from the wrong path.\\
\proc[file-handle]{dump-routes} &
	Dumps the state of replicators installed at the node
	on the specified file handle. \\
\end{alist}

\subsection{The internal variables}
\begin{alist}
\textbf{Class mrtObject}\hfill & \\
\code{protocols\_} &
	A list of handles of protocol instances active at the node 
	at which this protocol operates. \\
\code{mask-wkgroups} &
	Class variable---defines the mask used to identify well known groups. \\
\code{wkgroups} &
	Class array variable---array of allocated well known groups addresses,
	indexed by the group name.
	\code{wkgroups}(Allocd) is a special variable indicating the highest
	currently allocated well known group. \\[3ex]

\textbf{McastProtocol}\hfill & \\
\code{status\_} &
	takes values ``up'' or ``down'', to indicate the status of
	execution of the protocol instance. \\
\code{type\_} &
	returns the type of protocol executed by this instance,
	\eg, DM, or detailedDM. \\[3ex]

\textbf{Simulator}\hfill & \\
\code{multiSim\_} &
	1 if multicast simulation is enabled, 0 otherwise.\\
\code{MrtHandle\_} &
	handle to the centralised multicast simulation object.\\[3ex]

\textbf{Node}\hfill & \\
\code{switch\_} & 
	handle for classifier that looks at the high bit of the destination 
	address in each packet to determine
	whether it is a multicast packet (bit = 1)
	or a unicast packet (bit = 0).\\
\code{multiclassifier\_} & 
	handle to classifier that performs the \tup{s, g, iif} match. \\
\code{replicator\_} & 
	array indexed by \tup{s:g} of handles that replicate a multicast packet on to the required links. \\
\code{Agents\_} & array indexed by multicast group of the list of agents
	 at the local node that listen to the specific group. \\
\code{outLink\_} & 
	Cached list of outgoing interfaces at this node.\\[3ex]

\textbf{Link} and \textbf{SimpleLink}\hfill & \\
\code{iif\_} & handle for the networkInterface object placed on this link.\\
\code{head\_} & first object on the link, a no-op constructor.
	However, this object contains the instance variable, \code{link\_},
	that points to the container Link object.\\[3ex]

\textbf{networkInterface}\hfill & \\
\code{ifnum\_} & Class variable---holds the next available interface number.\\[3ex]
\end{alist}

\endinput
