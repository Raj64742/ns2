%
% personal commentary:
%        handlers and how they are used are confusing
%        Connector::send is needed, but so is just send()... confusing
%        default handler in Connector::recv is confusing
%        this is a DRAFT DRAFT DRAFT
%        - KFALL
%
\section{\shdr{Agents}{agent.h}{sec:agents}}

Agents represent endpoints where network-layer
packets are constructed or consumed, and provide
some functions helpful in developing transport-layer and other
protocols.
The C++ agent class supports an array of
timers which are often
useful in constructing protocol sources or sinks.
Generally, a user wishing to create a new
source or sink for network-layer packets
will create a class derived from {\tt Agent}.
The class \code{Agent} has an implementation partly in
OTcl and partly in C++.
The C++ implementation is contained in \code{agent.cc} and
\code{agent.h}, and the OTcl support is in
\code{tcl/lib/ns-agent.tcl}.

\subsection{\shdr{Agent state}{agent.h}{sec:agentstate}}

The C++ class \code{Agent} includes enough internal state
to assign various fields to a simulated packet before
it is sent.
This state includes the following:
\begin{quote}
\begin{itemize}
        \item[addr\_] - node address of myself
        \item[dst\_] - where I am sending packets to
        \item[size\_] - size placed into the common packet header
        \item[type\_] - type of packet (in the common header, used by the trace facility)
        \item[fid\_] - the IP flow identifier (formerly class in ns-1)
        \item[prio\_] - the IP priority field
\end{itemize}
\end{quote}
The class also keeps an array of timers which may be used
by classes derived from \code{Agent} as needed.
These variables may be modified by any class derived from \code{Agent},
although not all of them may be needed by any particular agent.

\subsection{\shdr{Agent methods}{agent.h}{sec:agentmethods}}

The \code{Agent} class supports packet generation, reception,
and timer management.
The following member functions are implemented by the C++ Agent class, and are
generally {\em not} over-ridden by derived classes:
\begin{tt}
\begin{quote}
\begin{itemize}
        \item[void sched(delay, timeout number)] - schedule a time out
        \item[void cancel(timeout number)] - cancel a time out
        \item[void handle(Event*)] - dispatcher for time outs
        \item[Packet* allocpkt()] - allocate new packet and assign its fields
\end{itemize}
\end{quote}
\end{tt}
The following member functions are also defined by the \code{Agent} class,
but {\em are} intended to be over-ridden by classes deriving from Agent:
\begin{tt}
\begin{quote}
\begin{itemize}
        \item[void timeout(timeout number)] - subclass-specific time out handler
        \item[void recv(Packet*, Handler*)] - receiving agent main receive path
\end{itemize}
\end{quote}
\end{tt}
The \code{sched} function causes the \code{handle} function
of the same agent to be invoked after \code{delay} simulated
seconds have elapsed.
The \code{timeout number} parameter is an index into the
timeout arrays \code{pending\_} and \code{timer\_} included in the
C++ \code{Agent} base class.
The \code{cancel} function cancels a scheduled timeout.
Generally, the \code{handle} function is invoked either
because the agent has scheduled an event using \code{sched}, or
because a packet has arrived from an upstream node (packets are events).
If the handler has \emph{not} been invoked due to a packet arrival,
\code{handle} invokes the \code{timeout} function.
The \code{timeout} function is intended to be used in whatever fashion
a specific type of agent requires it
(e.g. the TCP agent might generate a retransmission).

The \code{allocpkt} function is used by derived classes to create
packets to send.
The function fills in the following fields in the common packet
header (see \ref{sec:pformat}): {\tt uid, ptype, size}, and the
following fields in the IP header: {\tt src, dst, flowid, prio ttl}.
It also zero-fills in the following fields of the Flags header:
{\tt ecn, pri, usr1, usr2}.
Any packet header information not included in these lists must
be must be handled in the classes derived from \code{Agent}.

The \code{recv} function is the main entrypoint for an
Agent which receives packets, and
is invoked by upstream nodes when sending a packet.
In most cases, Agents make no use of the second argument (the handler
defined by upstream nodes).

\subsection{\shdr{Protocol Agents}{cbr.h}{sec:protoagents}}

There are several agents supported in the simulator.
These are their names in OTcl:
\begin{quote}
\begin{itemize}
        \item[TCP] - a ``Tahoe'' TCP sender (cwnd = 1 on any loss)
        \item[TCP/Reno] - a ``Reno'' TCP sender  (with fast recovery)
        \item[TCP/NewReno] - a modified Reno TCP sender (changes fast recovery)
        \item[TCP/Sack1] - a SACK TCP sender
        \item[TCPSink] - a Reno or Tahoe TCP receiver
        \item[TCPSink/DelAck] - a TCP delayed-ACK receiver
        \item[TCPSink/Sack1] - a SACK TCP receiver
	\item[TCPSink/Sack1/DelAck] - a delayed-ACK SACK TCP receiver

        \item[CBR] - connectioness protocol with sequence numbers
        \item[CBR/RTP] - an RTP sender and receiver
	\item[CBR/UDP] - UDP with sequence numbers and traffic sources
        \item[RTCP] - an RTCP sender and receiver

	\item[IVS/Source] - an IVS source
	\item[IVS/Receiver] - an IVS receiver

	\item[Message] - a protocol to carry textual messages
	\item[Message/Prune] - processes multicast routing prune messages

	\item[Null] - a degenerate agent

	\item[rtProto/DV] - distance-vector routing protocol agent
\end{itemize}
\end{quote}
Agents are used in the implementation of protocols at various layers.
Thus, for some transport protocols (e.g.~UDP) the distribution
of packet sizes and/or inter-departure times
may be dictated by some separate
object representing the demands of an application.
For agents used in the implementation of lower-layer protocols
(e.g. routing agents), size and departure timing is generally dictated
by the agent's own processing of protocol messages.

\subsection{\shdr{OTcl Linkage}{../ns/ns-default.tcl}{sec:agentotcl}}

Agents may be created within OTcl and an agent's internal
state can be modified by use of Tcl's \code{set} function and
any Tcl functions an Agent (or its base classes) implements.
Note that some of an Agent's internal state may exist
only within OTcl, and is thus is not directly accessible from C++.

\subsubsection{\shdr{creating and manipulating agents}{../ns/ns-lib.tcl}{sec:agentcreateotcl}}

The following example illustrates the creation and modification
of an Agent in OTcl:
\begin{small}
\begin{verbatim}
        set newtcp [new Agent/TCP] ; # create new object (and C++ shadow object)
        $newtcp set window_ 20 ; # sets the tcp agent's window to 20
        $newtcp target $dest ; # target is implemented in Connector class
        $newtcp set portID_ 1 ; # exists only in OTcl, not in C++
\end{verbatim}
\end{small}

\subsubsection{\shdr{default values}{../ns/ns-default.tcl}{sec:agentdefaults}}

Default values for member variables, those visible in OTcl only and those
linked between OTcl and C++ with \code{bind} are initialized
in the \code{tcl/lib/ns-default.tcl} file.  For example,
\code{Agent} and \code{CBR\_Agent}
are initialized as follows:
\begin{small}
\begin{verbatim}
        Agent set fid_ 0
        Agent set prio_ 0
        Agent set addr_ 0
        Agent set dst_ 0
        Agent set flags_ 0

        Agent/CBR set interval_ 3.75ms
        Agent/CBR set random_ 0
        Agent/CBR set packetSize_ 210
	...
\end{verbatim}
\end{small}

Generally these initializations are placed in the OTcl namespace
before any objects of these types are created.
Thus, when an \code{Agent} or \code{Agent/CBR} object
is created, the calls to \code{bind}
in the objects' constructors will causes the corresponding member variables
to be set to these specified defaults.

\subsubsection{\shdr{OTcl methods}{../ns/ns-agent.tcl}{sec:agentmethodsotcl}}

The instance procedures defined for the OTcl \code{Agent} class are
currently found in \code{tcl/lib/ns-agent.tcl}.
They are as follows:
\begin{small}
\begin{verbatim}
	port - the agent's port identifier
	dst-port - the destination's port identifier
	attach-source <stype> - create and attach a Source object to an agent
\end{verbatim}
\end{small}

\subsection{\shdr{Examples: Tcp, TCP Sink Agents}{tcp-simple.cc}{sec:agentexample}}

The class \code{TCP} represents a simplified TCP sender.
It sends data to a \code{TCPSink} agent and processes its acknowledgments.
It does not implement the TCP slow-start or congestion avoidance
algorithms, nor timeouts.
It does have a separate object associated with it which represents
an application's demand.
By looking at the \code{TCPAgent} and \code{TCPSinkAgent} classes
we may see how relatively complex agents are constructed.
An example from the Tahoe TCP agent \code{TCPAgent} is also given
to illustrate the use of timers.

\subsubsection{\shdr{creating the agent}{tcp-simple.cc}{sec:createtcpsimple}}

The following OTcl code fragment creates a \code{TCP} agent
and sets it up:
\begin{small}
\begin{verbatim}
        set tcp [new Agent/TCP] ; # create sender agent
        $tcp set fid_ 2 ; # set IP-layer flow ID
        set sink [new Agent/TCPSink] ; # create receiver agent
        $ns attach-agent $n0 $tcp ; # put sender on node $n0
        $ns attach-agent $n3 $sink ; # put receiver on node $n3
        $ns connect $tcp $sink ; # establish TCP connection
        set ftp [new Source/FTP] ; # create an FTP source "application"
        $ftp set agent_ $tcp ; # associate FTP with the TCP sender
        $ns at 1.2 "$ftp start" ; # arrange for FTP to start at time 1.2 secs
\end{verbatim}
\end{small}
The OTcl instruction \code{new Agent/TCP} results in the
creation of a C++ \code{TcpAgent} class object.
It's constructor performs first invokes the constructor of the
\code{Agent} base class and then performs its own bindings.
These two constructors appear as follows:
\begin{small}
\begin{verbatim}

The TcpSimpleAgent constructor (tcp.cc):

        TcpAgent::TcpAgent() : Agent(PT_TCP), rtt_active_(0), rtt_seq_(-1)
        {
                bind("window_", &wnd_);
                bind("windowInit_", &wnd_init_);
                bind("windowOption_", &wnd_option_);
                bind("windowConstant_", &wnd_const_);
                ...
                bind("off_ip_", &off_ip_);
                bind("off_tcp_", &off_tcp_);
                ...

The Agent constructor (agent.cc):

        Agent::Agent(int pkttype) : 
                addr_(-1), dst_(-1), size_(0), type_(pkttype), fid_(-1),
                prio_(-1), flags_(0)
        {
                memset(pending_, 0, sizeof(pending_)); // timers
                // this is really an IP agent, so set up
                // for generating the appropriate IP fields...
                bind("addr_", (int*)&addr_);
                bind("dst_", (int*)&dst_);
                bind("fid_", (int*)&fid_);
                bind("prio_", (int*)&prio_);
                bind("flags_", (int*)&flags_);
                ...
\end{verbatim}
\end{small}
These code fragments illustrate the common case where an agent's
constructor passes a packet type identifier to the \code{Agent}
constructor.
The values for the various packet types are used by the packet tracing
facility (see \ref{sec:trace}) and are defined in \code{trace.h}.
The variables which are bound in the \code{TcpAgent} constructor
are ordinary instance/member variables for the class
with the exception of the special integer values \code{off\_tcp\_}
and \code{off\_ip\_}.
These are needed in order to access a TCP header and IP header,
respectively.
For details on how to access packet headers, see \ref{sec:ppackethdr}.

\subsubsection{\shdr{starting the agent}{tcp.cc}{sec:starttcp}}

The \code{TcpAgent} agent is started in the example when its
FTP source receives the \code{start} directive at time 1.2.
The \code{start} operation is an instance procedure defined on the
\code{Source/FTP} class (and described in \ref{sec:sources}).
It is defined in \code{tcl/lib/ns-source.tcl} as follows:
\begin{small}
\begin{verbatim}
        Source/FTP instproc start {} {
                $self instvar agent_ maxpkts_
                $agent_ advance $maxpkts_
        }
\end{verbatim}
\end{small}
In this case, \code{agent\_} refers to our simple TCP agent and
\code{maxpkts\_} defaults to a large value (2147483647).

The call to \code{advance} eventually results in the simple TCP sender
generating some packets.  The following function \code{output}
performs this:
\begin{small}
\begin{verbatim}
        void TcpAgent::output(int seqno, int reason)
        {
                Packet* p = allocpkt();
                hdr_tcp *tcph = (hdr_tcp*)p->access(off_tcp_);
                double now = Scheduler::instance().clock();
                tcph->seqno() = seqno;
                tcph->ts() = now;
                tcph->reason() = reason;
                Connector::send(p, 0);
                ...
                if (!pending_[TCP_TIMER_RTX])
                        /* No timer pending.  Schedule one. */
                        set_rtx_timer();
        }
\end{verbatim}
\end{small}
Here we see an illustration of the use of the \code{Agent::allocpkt()}
function.
This output routine first allocates a new packet
(with its common and IP headers already filled in), but then must fill
in the appropriate TCP-layer header fields.
To find the TCP header in a packet (assuming it has been enabled, see
\ref{sec:packethdrmgr}) the \code{off\_tcp\_} must be properly initialized,
as illustrated in the constructor.
The packet \code{access()} function returns a pointer to the TCP
header, its sequence number and time stamp fields are filled in,
and the \code{send} function of the \code{Connector} class is called
to send the packet downstream one hop.
Note that the C++ \code{::} scoping operator is used here to avoid
calling \code{TcpSimpleAgent::send()} (which is also defined).
The check for a pending timer uses the array \code{pending\_} which
is defined in the \code{Agent} base class.
It is used here to set a retransmission timer if one is not already set
(a TCP sender only sets one timer per window of packets on each connection).

\subsubsection{\shdr{processing input at receiver}{tcp-sink.cc}{sec:tcpsink}}

Many of the TCP agents can be used with the \code{TCPSink} class
as the peer.
This class defines the \code{recv} and \code{ack} functions as follows:
\begin{small}
\begin{verbatim}
        void TcpSink::recv(Packet* pkt, Handler*)
        {
                hdr_tcp *th = (hdr_tcp*)pkt->access(off_tcp_);
                acker_->update(th->seqno());
                ack(pkt);
                Packet::free(pkt);
        }
        void TcpSink::ack(Packet* opkt)
        {
                Packet* npkt = allocpkt();
        
                hdr_tcp *otcp = (hdr_tcp*)opkt->access(off_tcp_);
                hdr_tcp *ntcp = (hdr_tcp*)npkt->access(off_tcp_);
                ntcp->seqno() = acker_->Seqno();
                ntcp->ts() = otcp->ts();
        
                hdr_ip* oip = (hdr_ip*)opkt->access(off_ip_);
                hdr_ip* nip = (hdr_ip*)npkt->access(off_ip_);
                nip->flowid() = oip->flowid();
        
                hdr_flags* of = (hdr_flags*)opkt->access(off_flags_);
                hdr_flags* nf = (hdr_flags*)npkt->access(off_flags_);
                nf->ecn_ = of->ecn_;
        
                acker_->append_ack((hdr_cmn*)npkt->access(off_cmn_),
                                   ntcp, otcp->seqno());
                send(npkt, 0);
        }
\end{verbatim}
\end{small}
The \code{recv} function overrides the \code{Agent::recv} function
(which merely discards the received packet).
It updates some internal state with the sequence number of the
received packet (and therefore requires the \code{off\_tcp\_} variable
to be properly initialized.
It then generates an acknowledgment for the received packet.
The \code{ack} function makes liberal use of access to packet header
fields including separate accesses to the TCP header, IP header,
Flags header, and common header.
The call to \code{send} invokes the \code{Connector::send} function.

\subsubsection{\shdr{processing responses at the sender}{tcp-simple.cc}{sec:tcpsimpleack}}

Once the simple TCP's peer receives data and generates an ACK, the
sender must (usually) process the ACK.
In the \code{TcpAgent} agent, this is done as follows:
\begin{small}
\begin{verbatim}
        /*
         * main reception path - should only see acks, otherwise the
         * network connections are misconfigured
         */
        void TcpAgent::recv(Packet *pkt, Handler*)
        {
                hdr_tcp *tcph = (hdr_tcp*)pkt->access(off_tcp_);
                hdr_ip* iph = (hdr_ip*)pkt->access(off_ip_);
#ifdef notdef
        if (pkt->type_ != PT_ACK) {
                Tcl::instance().evalf("%s error \"received non-ack\"",
                                      name());
                Packet::free(pkt);
                return;
        }
#endif
                if (((hdr_flags*)pkt->access(off_flags_))->ecn_)
                        quench(1);
                if (tcph->seqno() > last_ack_) {
                        newack(pkt);
                        opencwnd();
                } else if (tcph->seqno() == last_ack_) {
                        if (++dupacks_ == NUMDUPACKS) {
                                ...
                        }
                }
                Packet::free(pkt);
                send(0, 0, maxburst_);
       }
\end{verbatim}
\end{small}
This routine is invoked when an ACK arrives at the sender.
In this case, once the information in the ACK is processed (by \code{newack})
the packet is no longer needed and is returned to the packet memory
allocator.
In addition, the receipt of the ACK indicates the possibility of sending
additional data, so the \code{TcpSimpleAgent::send} routine is
invoked which attempts to send more data if the TCP window allows.

\subsubsection{\shdr{implementing timers}{tcp.cc}{sec:tcptimer}}

As mentioned above in section \ref{sec:agentmethods}, an agent
may wish to override the \code{Agent::timeout} function (which does
nothing).
In the case of the Tahoe TCP agent, two timers are used: a delayed
send timer and a retransmission timer.
The retransmission timer is the more interesting of the two.
It is set with the following function:
\begin{small}
\begin{verbatim}
from tcp.h:
        #define TCP_TIMER_RTX           0
        #define TCP_TIMER_DELSND        1

from tcp.cc:

        /*
         * Set retransmit timer.  If one is already pending, cancel it
         * and set a new one based on the current rtt estimate.
         */
        void TcpAgent::set_rtx_timer()
        {
                if (pending_[TCP_TIMER_RTX])
                        cancel(TCP_TIMER_RTX);
                sched(rtt_timeout(), TCP_TIMER_RTX);
        }
        void TcpAgent::timeout(int tno)
        {
                /* retransmit timer */
                if (tno == TCP_TIMER_RTX) {
                        ...
                        closecwnd(0);
                        reset_rtx_timer(0);
                        send(0, TCP_REASON_TIMEOUT);
                } else {
                        /*
                         * delayed-send timer, with random overhead
                         * to avoid phase effects
                         */
                        send(1, TCP_REASON_TIMEOUT);
                }
        }

\end{verbatim}
\end{small}

\subsection{\shdr{Creating a New Agent}{agent.cc}{sec:createagent}}

To create a new agent, one has to do the following:
\begin{enumerate}
        \item decide its inheritance structure (section \ref{sec:pingexample})
        \item create the class, \code{recv}, and \code{timeout} functions (section \ref{}
        \item define OTcl linkage functions (section \ref{})
        \item write the necessary OTcl code to access your agent (section \ref{})
\end{enumerate}

The action required to create and agent can be illustrated
by means of a very simple example.
Suppose we wish to construct an agent which performs
the ICMP ECHO REQUEST/REPLY (or ``ping'') operations.

\subsubsection{\shdr{Example: A ``ping'' requestor (Inheritance Structure)}{agent.h}{sec:pingexample}}

Deciding on the inheritance structure is a matter of personal choice, but is
likely to be related to the layer at which the agent will operate
and its assumptions on lower layer functionality.
For protocol endpoints wishing to use a connectionless
datagram-oriented transport layer (like UDP) with sequencing,
the \code{CBR\_Agent} base class is likely to be appropriate.
For protocols wishing to use a connection-oriented stream transport
(like TCP), the various TCP Agents could be used.
Finally, if a new transport or ``sub-transport'' protocol
is to be developed, using \code{Agent}
as the base class would likely be the best choice.
In our example, we'll use Agent as the base class, given that
we are constructing an agent logically belonging to the IP layer
(or just above it).

We may use the following class definition:
\begin{small}
\begin{verbatim}
        class ECHO_Agent : public Agent {
         public:
                ECHO_Agent();
                int command(int argc, const char*const* argv);
         protected:
                void timeout(int);
                void sendit();
                double interval_;
        };
\end{verbatim}
\end{small}

The implementation of the member functions will look similar to
the \code{CBR\_Agent} described above.

\subsubsection{The \code{recv} and \code{timeout} functions}

The \code{recv} function is not defined here, as this agent
represents a request function and will generally not be receiving
events or packets\footnote{This is perhaps unrealistically simple.
An ICMP ECHO REQUEST agent would likely wish to process
ECHO REPLY messages.}
By not defining the \code{recv} function, the base class version
of \code{recv} (i.e. \code{Connector::recv}) is used.
The \code{timeout} function is used to periodically send request packets.
The following \code{timeout} function is used, along with a helper
function \code{sendit}:

\begin{small}
\begin{verbatim}
        void ECHO_Agent::timeout(int)
        {
                sendit();
                sched(interval_, 0);
        }
        void ECHO_Agent::sendit()
        {
                Packet* p = allocpkt();
                ECHOHeader *eh = ECHOHeader::access(p->bits());
                eh->timestamp() = Scheduler::instance().clock();
                send(p, 0);     // Connector::send()
        }
\end{verbatim}
\end{small}

The \code{timeout} function simply arranges for \code{sendit} to be
executed every \code{interval\_} seconds.
The \code{sendit} function create a new packet with most of its
header fields already set up by the \code{allocpkt} function.
The packet is only lacking the current time stamp. 
The call to \code{access} provides for a structured interface to the
packet header fields, and is used to set the timestamp field.
Note that this agent uses its own special header (``ECHOHeader'').
The creation and use of packet headers is described in (see \ref{pheader});
To send the packet to the next downstream node, \code{Connector::send()}
is invoked without a handler.

\subsubsection{Linking the ``ping'' agent with OTcl}

There are three items we must handle to properly link our agent
with Otcl.
First we need to establish a mapping between the OTcl name
for our class and the actual object created when an
instantiation of the class is requested in OTcl.
This is done as follows:
\begin{small}
\begin{verbatim}
        static class ECHOClass : public TclClass {
        public:
                ECHOClass() : TclClass("Agent/ECHO") {}
                TclObject* create(int argc, const char*const* argv) {
                        return (new ECHO\_Agent());
                }
        } class_echo;
\end{verbatim}
\end{small}

Here, a {\em static} object ``class\_echo'' is created. It's constructor
(executed immediately when the simulator is executed) places the class name
``Agent/ECHO'' into the OTcl name space.  The mixing of case is
by convention, although the ``/'' character is interpreted by the
\code{Tcl} library as a hierarchy delimiter and is required.
The definition of the \code{create} function specifies how a C++
shadow object should be created when
the OTcl interpreter is instructed to create an
object of class ``Agent/ECHO''.  In this case, a dynamically-allocated
object is returned.  This is the normal way new C++ shadow objects
are created.
Note that arguments could have been passed to our constructor
via OTcl through the conventional \code{argc/argv} pairs of the
\code{create} function, although this is rare.

Once we have the object creation set up, we will want to link
C++ member variables with corresponding variables in the OTcl
name space, so that accesses to OTcl variables are actually
backed by member variables in C++.
Assume we would like OTcl to be able to adjust the sending
interval and the packet size.
This is accomplished in the class's constructor:
\begin{small}
\begin{verbatim}

        ECHO_Agent::ECHO_Agent() : Agent(PT_ECHO)
        {
                bind_time("interval_", &interval_);
                bind("packetSize_", &size_);
        }
\end{verbatim}
\end{small}

Here, the C++ variables \code{interval\_} and \code{size\_} are
linked to the OTcl instance variables \code{interval\_} and
\code{packetSize\_}, respectively.
Any read or modify operation to the Otcl variables will result
in a corresponding access to the underlying C++ variables.
See \ref{c++linkage} for more detail on how the various \code{bind}
functions work.
The defined constant \code{PT\_ECHO} is passed to the \code{Agent}
constuctor so that the \code{Agent::allocpkt} function may set
the packet type field used by the trace support
(see section \ref{traceptype}).  In this case, \code{PT\_ECHO} represents
a new packet type and must be defined in \code{trace.h}.
(see section \ref{sec:traceformat}).

Once object creation and variable binding is set up, we may
want to create methods implemented in C++ but which can
be invoked from OTcl.
These are often control functions that initiate, terminate or
modify behavior.
In our present example, we may wish to be able to start the
ping query agent from OTcl using a ``start'' directive.
This may be implemented as follows (very similar to the CBR Agent):
\begin{small}
\begin{verbatim}

        int ECHO_Agent::command(int argc, const char*const* argv)
        {
                if (argc == 2) {
                        if (strcmp(argv[1], "start") == 0) {
                                timeout(0);
                                return (TCL_OK);
                        }
                }
                return (Agent::command(argc, argv));
        }
\end{verbatim}
\end{small}
Here, the \code{start} method available to OTcl simply calls
the C++ member function \code{timeout} which initiates the
first packet generation and schedules the next.
Note this class is so simple it does not even include a
way to be stopped.

\subsection{Using the agent through OTcl}

The agent we have created will have to be instantiated and attached
to a node.
Note that a node and simulator object is assumed to have
already been
created (Section \ref{tcllink} describes how this is done).
The following OTcl code performs these functions:
\begin{small}
\begin{verbatim}
        set echoagent [new Agent/ECHO]
        $simulator attach-agent $node $echoagent
\end{verbatim}
\end{small}

To set the interval and packet size, and start packet generation,
the following OTcl code is executed:
\begin{small}
\begin{verbatim}

        $echoagent set dst_ $dest
        $echoagent set fid_ 0
        $echoagent set prio_ 0
        $echoagent set flags_ 0
        $echoagent set interval_ 1.5
        $echoagent set packetSize_ 1024
        $echoagent start
\end{verbatim}
\end{small}

This will cause our agent to generate one 1024-byte packet destined for
node \code{\$dest} every 1.5 seconds.
