\chapter{Nam Trace}
\label{chap:namtrace}

Nam is a Tcl/Tk based animation tool that is used to visualize the ns
simulations and real world packet trace data. The first step to use nam is
to produce a nam trace file. The nam trace file should contain topology
information like nodes, links, queues, node connectivity etc as well as
packet trace information. In this chapter we shall describe the nam trace
format and simple ns commands/APIs that can be used to produce topology 
configurations and control animation in nam.

\section{Nam Trace format}
\label{sec:namtraceformat}
The C++ class Trace used for ns tracing is used for nam tracing as
well. Description of this class may be found under section
\ref{sec:tracemoncplus}. The method Trace::format() defines nam 
format used in nam trace files which are used by nam for
visualization of ns simulations. Trace class method Trace::format() is
described in section \ref{sec:traceformat} of chapter \ref{chap:trace}. If
the macro NAM\_TRACE has been defined (by default it is defined in
trace.h), then the following code is executed as part of the
Trace::format() function:

\begin{program}
#ifdef NAM_TRACE
        if (namChan_ != 0)
                sprintf(nwrk_,
                        "%c -t "TIME_FORMAT" -s %d -d %d -p %s -e %d -c %d
-i %d -a %d -x {%s.%s %s.%s %d %s %s}",
                        tt,
                        Scheduler::instance().clock(),
                        s,
                        d,
                        name,
                        th->size(),
                        iph->flowid(),
                        th->uid(),
                        iph->flowid(),
                        src_nodeaddr,
                        src_portaddr,
                        dst_nodeaddr,
                        dst_portaddr,
                        seqno,flags,sname);
#endif  
\end{program}

Thus we see the nam format may be represented as follows:
\code{<event-type> -t <time> -s <source> -d <dest> -p <pkt-type> -e <pkt-size> 
-c <flow-id> -i <unique-id> -a <pkt-attribute> -x <ns-trace info>}\\

The nam trace format can be divided into 6 types, each defining a given
event or object. Incase of a trace
describing a packet event the first field signifies the type of
event/operation taking place in the node and it maybe + (enqueue), -
(dequeue), r (receive) d (drop) or h (hop).
\begin{description}
\item['h'] Hop: The packet started to be transmitted on the link from
src\_addr to dst\_addr and is forwarded to the next\_hop towards its
dst\_addr.

\item['r'] Receive: The packet finished transmission and started to be
received at the destination.

\item['d] Drop: The packet was dropped from queue or link from src\_addr
to dst\_addr. Drop here doesn't distinguish between dropping from queue or
link. This is decided by the drop time.  

\item['+'] Enter queue: The packet entered the queue from src\_addr to
dst\_addr.

\item['-'] Leave queue: The packet left the queue from src\_addr to
dst\_addr.  
\end{description}

The other flags have the following meaning:
\begin{description}
\item[-t <time>] is the time the event occurred.
\item[-s <src>] is the originating node.
\item[-d <dst>] is the destination node.
\item[-p <pkt-type>] is the descriptive name of the type of packet seen.
See section \ref{sec:traceptype} for the different types of packets 
supported in \ns.
\item[-e <extent>] is the size (in bytes) of the packet.
\item[-c <conv>] is the conversation id or flow-id of that session.
\item[-i <id>] is the packet id in the conversation.
\item[-a <attr>] is the packet attribute, which is currently used as color
id. 
\item[-x <src-na.pa> <dst-sa.na> <seq> <flags> <sname>] is taken from
ns-traces and it gives the source and destination node and port
addresses, sequence number, flags (if any) and the type of message.
For example \code{ -x {0.1 -2147483648.0 -1 ------- SRM_SESS} } denotes an
SRM message being sent from node 0 (port 1).
\end{description}

In addition to the above nam format for packet events there are nam traces
that provide information about nam version, hierarchical
addressing structure, node/link/queue states, node-marks, protocol states,
color and annotations. These nam trace outputs typically have the
following letters (or tags) as their first field and they represent the
following trace types: 
n(node state), m (node marking), l(link state), q (queue), a 
(agent/protocol state), f (protocol variables),
V(nam version), A(hierarchy information), c(nam color) and v(annotations).

\textsc{Node state}\\
The nam trace format defining node state is:\\
\code{n -t <time> -a <src-addr> -s <src-id> -S <state> -v <shape> -c 
<color> -i <l-color> -o <color>}\\
"n" denotes the node state. Flags "-t" indicates time and "-a" and "-s"
denotes the node address and id. "-S" gives the node state transition.
The possible values: UP, DOWN indicates node recovery and
failure, COLOR indicates node color change, DLABEL indicates addition of
label to node. If COLOR is given, a following
-c <color> is expected which gives the new color value. If DLABEL is
given, a following -l <old-label> -L <new-label> is expected that gives
the old-label, if any (for backtracing) and current label. Shape gives
the node shape. Flag `-o'
is used in backtracing to restore old colors of a node. The line\\
\code{n -t * -a 4 -s 4 -S UP -v circle -c tan -i tan }\\
defines a node with address and id of 4 that has the shape of a
circle, and color of tan and label-color (-i) of tan.

\textsc{Node marking}\\
Node marks are colored boundaries/markings around nodes. They are created
by:\\
\code{m -t <time> -n <mark name> -s <node> -c <color> -h <shape> -o 
<color>}\\
and can be deleted by:\\
\code{m -t <time> -n <mark name> -s <node> -X}\\
Note that once created, a node mark cannot change its shape. The possible
choices for shapes are, circle, square, and hexagon. They are defined as 
lower-case strings exactly as above. A nam trace showing node mark is:\\
\code{m -t 4 -s 0 -n m1 -c blue -h circle}\\
indicating node 0 is marked with a blue circle at time 4.0. The name of
the mark is m1.


\textsc{Link/Queue state}\\
The nam trace for link and queue states are given by:\\
\code{l -t <time> -s <src> -d <dst> -S <state> -c <color> -o orientation 
-r <bw> -D <delay>}\\
\code{q -t <time> -s <src> -d <dst> -a <attr> }\\
State and color indicates the same attributes as described for node state
traces. "-o" gives the link orientation (angle between link and
horizontal). "-r" and "-D" gives the bandwidth (in Mb) and delay (in ms).
Example of a link trace is:\\
\code{l -t * -s 0 -d 1 -S UP -r 1500000 -D 0.01 -c black -o right }\\
For queue trace event, "-a" signifies the queue-position (angle between
queue line and horizontal). Example queue trace is given as:\\
\code{q -t * -s 1 -d 0 -a 0.5}\\
\code{q -t * -s 0 -d 1 -a 0.5 }\\


\textsc{Agent tracing}\\
The protocol state for each node is constructed by the following nam trace
format:\\
\code{a -t <time> -n <agent name> -s <src> -d <dst>}\\
They can be destructed by:\\
\code{a -t <time> -n <agent name> -s <src> -d <dst> -X}\\
Example of an agent (protocol:SRM) trace:\\
\code{a -t 0.00000000000000000 -s 5 -d -2147483648 -n srm(5)}\\


\textsc{Variable tracing}\\
To visualize protocol state variables associated with an agent, we use the
name `feature'. Currently we allow three types of features: timers,
lists and simple variables. But only the last one is
implemented in \ns tracing APIs. Features may be added or modified at any
time after agent creation using:\\
\code{f -t <time> -s <src> -d <dst> -a <agent name> -T <type> -n <var
name> -v <value> -o <prev value>}\\
<type> is `l' for a list, `v' for a simple variable, `s' for a stopped
timer, `u' for an up-counting timer, `d' for a down-counting timer.
-v <value> gives the new value of the variable. Variable values are
simple ASCII strings obeying the TCL string quoting conventions.
List values obey the TCL list conventions. Timer values are ASCII
numeric. -o <prev value> gives the previous value of the variable. This is
to allow backward play of animation. Example of a simple variable trace is
given by: \\
\code{f -t 0.00000000000000000 -s 1 -d -2147483648 -n C1_ -a srm(1) -v
-T v}\\

Features may be deleted using:\\
\code{f -t <time> -a <agent name> -n <var name> -o <prev value> -X }\\


\textsc{Misc. tracing}
Other than the trace formats described so far there aresome other
miscellaneous traces that defines the nam version or hierarchy model or
provide annotations. They are listed as below:
 \begin{description}
\item[Annotation] The generic trace format starts with the tag "v". It is
used for annotation. The format is:\\ \code{v -t <time> <TCL script
string>}\\ This trace may include an arbitrary tcl script to be executed
at a given
time, as long as the script is in one line (no more than 256 characters).
The order of flag and the string is important.  Example: \\
\code{v -t 4 sim_annotation 4 3 node 0 added one mark }\\

\item[Color] Color trace is given by:\\
\code{c -t <time> -i <color id> -n <color name>}\\
This trace defines a color. The color name should be one of the names
listed in color database in X11 (/usr/X11/lib/rgb.txt). After this
definition, the color can be referenced using its id. 

\item[Version]
The nam version is given by:\\
\code{V -t <time> -v <version> -a <attr>}\\
The line \code{V -t * -v 1.0a5 -a 0 } denotes the nam version being used
as 1.0a5.

\item[Hierarchy] Hierarchical address information is given by:\\
\code{A -t <time> -n <levels> -o <address-space size> -c <mcastshift> -a 
<mcastmask> -h <nth level> -m <mask in nth level> -s <shift in nth
level>}\\
This trace gives the details of hierarchy, if hierarchical addressing is
being used for simulation. <levels> indicate the total number of
hierarchical tiers, which is 1 for flat addressing, 2 for a 2-level
hierarchy etc. Address-space size denotes the total number of bits used
for addressing. For a given level of hierachy, indicated by -h <nth
level>, the
address mask and shift values for that level are given by tags -m and -s
respectively.
Example of a trace for topology with 3 level hierachy is given by:
\begin{program}
A -t * -n 3 -p 0 -o 0xffffffff -c 31 -a 1
A -t * -h 1 -m 1023 -s 22
A -t * -h 2 -m 2047 -s 11
A -t * -h 3 -m 2047 -s 0 
\end{program}
\end{description}

The functions that implement the different nam trace formats described
above may be found in the following files: \ns/trace.cc, 
\ns/trace.h, \ns/tcl/lib/ns-namsupp.tcl and
\ns/tcl/lib/ns-nam.tcl.


\section{Ns commands for creating and controlling nam animations}
\label{sec:namcommands}

This section describes different APIs in \ns that may be used to
manipulate nam animations for objects like nodes, links, queues and
agents. 

\textsc{Node}\\
Nodes are created from "n" trace event in trace file. It represents a
source/destination/router. Nam terminates if there are duplicate
definitions of the same node. 
Attributes specific to node are color, shape, label, label-color, position
of label and adding/deleting mark on the node.
Nodes may have many shapes like circle (default), square or hexagon, but
once created the shape of a node cannot be changed during the simulation.
Nodes have many colors and the node color may be changed during animation. 
The following node procedures are used to set node attributes:

\begin{program}
$node color <color>      ;# sets color of node
$node shape <shape>      ;# sets shape of node
$node label <label>      ;# sets label on node
$node label-color <lcolor>  ;# sets color of label
$node label-at <ldirection> ;# sets position of label
$node add-mark <name> <color> <shape>   ;# adds a mark to node;
$node delete-mark <name>    ;# deletes mark from node
\end{program}

\textsc{Link/Queue}\\
Links are created between nodes to form a network topology. nam links are
internally simplex, but it is invisible to the users. The trace event 'l'
creates two simplex links and other necessary setups, hence it looks
to users identical to a duplex link. Link may have many colors and it can
change its color during animation. Queues are constructed in nam
between two nodes. Unlike link, nam queue is associated to a simplex link.
The trace event 'q' only creates a queue for a simplex link. In nam,
queues are visualized as stacked packets. Packets are stacked along a
line, and the angle between the line and the horizontal line can be
specified in the trace event 'q'.
Commands to setup different animation attributes of a link are as
follows:\\

\code{$ns duplex-link-op <attribute> <value>}\\

The <attribute> may be one of the following: orient, color, queuePos.
Orient or the link orientation defines the angle between the link and
horizontal. The optional orientation values may be difined in
degrees or by text like right (0), right-up (45), right-down (-45), left
(180), left-up (135), left-down (-135), up (90), down (-90). The queuePos
or position of queue is defined as the angle of the queue line with
horizontal. 
Examples for each attribute are given as following : 
\begin{program}
$ns duplex-link-op orient right      ;# orientation is set as right. The order
                                     ;# in which links are created in nam
                                     ;# depends on calling order of this function.
$ns duplex-link-op color "green"
$ns duplex-link-op queuePos 0.5
\end{program}


\textsc{Agent}\\
Agents are used to separate protocol states from nodes. They are always
associated with nodes. An agent has a name, which is a unique identifier
of the agent. It is shown as a square with its name inside, and a line
link the square to its associated node. The following are commands that
support agent tracing:
\begin{program}
$ns add-agent-trace <agent> <name> <optional:tracefile>
$ns delete-agent-trace <agent>
$ns monitor-agent-trace <agent>
\end{program}


\textsc{Some generic commands}\\

\code{$ns color <color-id>}\\
The above command defines color index for nam. Once specified, the
color-id can be used in place of the color name in nam traces.

\code{$ns trace-annotate <annotation>}\\
An example for the above API would be\\
\code{$ns at $time "$ns trace-annotate \"Event A happened\""}\\
This annotation appears in the nam window and is used to control playing of
nam by events. Demonstration of nam APIs may be found in
\ns/tcl/ex/nam-example.tcl.
