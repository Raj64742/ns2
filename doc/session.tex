\chapter{Session-level Packet Distribution}
\label{chap:session}

This section describes the internals of the Session-level Packet Distribution
implementation in \ns.
The section is in two parts:
the first part is an overview of 
a basic Session configuration,
and a ``complete'' description of the configuration parameters 
of a Session.
The second part describes the architecture, internals, and the code path
of the Session-level Packet distribution.

Session-level Packet Distribution enables simulations with large-scale 
topologies.  A 2048 node and 8 connectivity degree topology takes roughly 
40 MB in memory, 2049-4096 node topology takes about 167 MB, and 4097-
8194 node topology takes about 671 MB.  However, the queuing delays that
may occur in routers are ignored.  Therefore, if simulations are involved 
with high source rate or multiple sources merging at some point resulting
a high aggregated rate, please avoid using Session-level Packet Distribution.

\section{Configuration}

\subsection{Basic Configuration}
\label{sec:basic-config}

Each Session (i.e., a multicast tree) must be configured strictly in
this order:
creating(obtaining) the session source,
assigning the destination address,
creating the session helper, 
attaching to session source, and
the session members joining the group.


\begin{program}
        set ns [new SessionSim]          \; preamble initialization;
        set node [$ns node]              \; source and receiver to reside on this node;
        set group [$ns allocaddr]        \; multicast group for this session;

        set src [new Agent/CBR]
        $src set dst_ $group            \; configure the source;
        $ns attach-agent $node $src

        $ns create-session $node $src   \; creating the session helper and attaching to the source;

        set rcvr [new Agent/NULL]        \; configure the receiver;
        $ns attach-agent $node $rcvr
        $ns at 0.0 "$node join-group $rcvr $group" \; joining the session;

        $ns at 0.1 "$src start"          \; start the source;

\end{program}

\subsection{Inserting a Loss Module}
\label{sec:loss-config}

When simulating mechanism robustness(e.g., SRM error recovery mechanism), 
modules like lossy links are desired to create error senarios.  This 
subsection is describe how to create a lossy link, meaning inserting 
a loss module for a 'virtual link' (a link directly connecting source
and receiver with accumulative bandwidth and delay).

Please note that packets dropped at a particular link in a
multicast tree will not be received by
the receivers in the particular downstream subtree.  To
avoid abnormal behaviors, you will have to find
all the affected receivers and place your loss modules for them as well.
We are currently working on solving this error dependency problem.
In the future, the loss modules for the downstream receivers will be 
installed automatically when a lossy link is created.


\paragraph{Creating a Loss Module}
Before we can insert a loss module in between a source-receiver pair,
we have to create the loss module.  Basically,
a loss module compares two values to decide whether to drop a packet.
The first value is obtained every time when the loss module receives 
a packet from a random variable.  The second value
is fixed and configured when the loss module is created.

The following code gives an example to create a uniform 
0.1 loss rate.

\begin{program}
        # creating the uniform distribution random variable
        set loss_random_variable [new RandomVariable/Uniform] 
        # setting the range of random variable
        $loss_random_variable set min_ 0
        $loss_random_variable set max_ 100

        # creating an error module;
        set loss_module [new ErrorModel]
        # set target for dropped packets;
        $loss_module drop-target [new Agent/Null]
        # setting error rate to 0.1, 10/(100-0);
        $loss_module set rate_ 10
        # attaching the random variable to the loss module;
        $loss_module ranvar $loss_random_variable 

\end{program}

Several random variable distributions are available.
%%% Need xref to ranvar pages
Please refer to tcl/ex/ranvar.tcl.

\paragraph{Inserting a Loss Module}

One loss module is required for each pair source-receiver pair. If it is
intended to insert a loss module for a receiver, keep a handle to the 
loss module when created.  Loss modules can only be inserted after the
corresponding receivers finish joining the group.

\begin{program}
        # keep a handle to the loss module;
        set sessionhelper [$ns create-session $node $src] 
        # insert the loss module;
        $ns at 0.1 "$sessionhelper insert-loss $loss_module $rcvr" 
\end{program}

\section{Architecture}
\label{sec:session-arch}
The purpose of Session-level packet distribution is to
speed up simulations and reduce memory consumption while 
maintaining reasonable accuracy(if no queuing involved).  The first
bottleneck observed is the memory consumption by heavy-weight
links and nodes.  Therefore, in SessionSim (Simulator for Session-level
packet distribution), we keep only minimal amount of 
states for links and nodes, and connect the higher level source and 
receiver applications with appropriate delay and loss modules.  When
a connection is a multicast group, we attach a replicator 
to the source application, so the replicator replicates packets
to all loss or delay modules attached to the receiver applications.

In short, almost the entire network layer(routing and queuing)
is abstract out.  Packets in SessionSim do not get routed.  
They only follow the established Session.

\section{Internals}
In this section, we explain the internals of Session-level Packet 
Distribution.  The implementation is split into two parts:
\begin{list}{}{}
\item  Linkage of objects to make a Session in OTcl 
\item  Packet forwarding activities are executed by C++ methods.  
\end{list}

\subsection{Object Linkage}
\label{sec:session-objlink}

\begin{list}{}{}
\item  Simplified links and nodes.
\item  Replicator
\item  Delay and loss modules
\end{list}

\paragraph{Nodes and Links}
A link only contains the values of
its bandwidth and delay, and a node contains only its id and port number
for next agent.

\begin{program}
SessionSim instproc simplex-link \{ n1 n2 bw delay type \} \{
    $self instvar bw_ delay_
    set sid [$n1 id]
    set did [$n2 id]

    set bw_($sid:$did) [expr [string trimright $bw Mb] * 1000000]
    set delay_($sid:$did) [expr [string trimright $delay ms] * 0.001]
\}

SessionNode instproc init \{\} \{
    $self instvar id_ np_
    set id_ [Node getid]
    set np_ 0
\}
\end{program}

\paragraph{Replicator}
One replicator is required per source.  While the source is configured,
a replicator (session helper) need to be attached to the source.  By
calling \proc[]{create-session}, a replicator is:
created,
attached to the source application, and 
kept in a SessionSim instance variable \code{session_} array with 
its source and destination addresses as the index.

Note that the destination of source agent must be set before
calling \proc[]{create-session}.

\begin{program}
SessionSim instproc create-session \{ node agent \} \{
    $self instvar session_

    set nid [$node id]                           \; get source address;
    set dst [$agent set dst_]                    \; get destination address;
    set session_($nid:$dst) [new Classifier/Replicator/Demuxer]  \; creating the replicator;
    $agent target $session_($nid:$dst)           \; attach the replicator to the source;
    return $session_($nid:$dst) \; keep the replicator in the SessionSim instance variable session_ array;
\}
\end{program}

\paragraph{Delay and Loss Modules}

At least one delay module is required per receiver.
See Section~\ref{sec:loss-config} for inserting a loss module for a receiver.
When a receiver joins a group, 
the \proc[]{join-group} method goes through
all replicators (session helpers) maintained in \code{session_}.
If the destination index matches the group address
the receiver are joining, then the following actions are performed.

1. A new slot of the replicator (session helper) is created and assigned to the receiver.

2. An accumulated bandwidth and delay between the source and receiver are obtained by SessionSim instance procedure \proc[]{get-bw} and \proc[]{get-delay}.

3. A constant random variable is created and assigned with the
accumulative delay.

4. A delay module is created and assigned with the constant random 
variable and the accumulative bandwidth.

5. The delay module in inserted into the replicator slot in
front of the receiver.

\begin{program}
SessionSim instproc join-group \{ agent group \} \{
    $self instvar session_

    foreach index [array names session_] \{
        set pair [split $index :]
        if \{[lindex $pair 1] == $group\} \{
            # Note: must insert the chain of loss, delay, 
            # and destination agent in this order:

            #1. insert destination agent into session replicator
            $session_($index) insert $agent

            #2. find accumulative bandwidth and delay
            set src [lindex $pair 0]
            set dst [[$agent set node_] id]
            set accu_bw [$self get-bw $dst $src]
            set delay [$self get-delay $dst $src]

            #3. set up a constant delay random variable
            set random_variable [new RandomVariable/Constant]
            $random_variable set avg_ $delay

            #4. set up the delay module
            set delay_module [new DelayModel]
            $delay_module bandwidth $accu_bw
            $delay_module ranvar $random_variable

            #5. insert the delay module in front of the dest agent
            $session_($index) insert-module $delay_module $agent
        \}
    \}
\}
\end{program}


\subsection{Packet Forwarding}
\label{sec:session-pktforward}
Packet forwarding activities are executed in C++.  A source application 
generates a packet and forwards to its target which must be a replicator 
(session helper).  The replicator copies the packet and forwards 
to targets in the active slots which are either delay modules or loss modules. If loss modules, a decision is made whether to drop the packet.
If yes, the packet is forwarded to the loss modules drop target.  If not,
the loss module forwards it to its target which must be a delay module.
The delay module will forward the packet with a delay to its target which
must be a receiver application.

%% PH: not sure this will come out right
%% PH: make .eps picture but not sure how to import that

\begin{program}
                    / Loss module - Delay module - Receiver 1
Source - Replicator --------------- Delay module - Receiver 2
    (Session Helper)\bs Loss module - Delay module - Receiver 3

\end{program}

\endinput

### Local Variables:
### mode: latex
### comment-column: 60
### backup-by-copying-when-linked: t
### file-precious-flag: nil
### End:
