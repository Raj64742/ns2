\chapter{TCP Agents}
\label{sec:tcpAgents}

This section describes the operation of the TCP agents in \ns.
There are two major types of TCP agents: one-way agents
and a two-way agent.
One-way agents are further subdivided into a set of TCP senders
(which obey different congestion and error control techniques)
and receivers (``sinks'').
The two-way agent is symmetric in the sense that it represents
both a sender and receiver.
It is still under development.

The files described in this section are too numerous to enumerate here.
Basically it covers most files matching the regular expression
\nsf{tcp*.\{cc, h\}}.

The one-way TCP sending agents currently supported are:
\begin{itemize}\itemsep0pt
        \item Agent/TCP - a ``tahoe'' TCP sender
        \item Agent/TCP/Reno - a ``Reno'' TCP sender
        \item Agent/TCP/NewReno - Reno with a modification
        \item Agent/TCP/Sack1 - TCP with selective repeat (follows RFC2018)
        \item Agent/TCP/Vegas - TCP Vegas
        \item Agent/TCP/Fack - Reno TCP with ``forward acknowledgment''
\end{itemize}
The one-way TCP receiving agents currently supported are:
\begin{itemize}\itemsep0pt
        \item Agent/TCPSink - TCP sink with one ACK per packet
        \item Agent/TCPSink/DelAck - TCP sink with configurable delay per ACK
        \item Agent/TCPSink/Sack1 - selective ACK sink (follows RFC2018)
        \item Agent/TCPSink/Sack1/DelAck - Sack1 with DelAck
\end{itemize}
The two-way experimental sender currently supports only a Reno form of TCP:
\begin{itemize}
        \item Agent/TCP/FullTcp
\end{itemize}

The section comprises three parts:
the first part is a simple overview and example of configuring
the base TCP send/sink agents (the sink requires no configuration).
The second part describes the internals of the base send agent,
and last part is a description of the extensions
for the other types of agents that have been included in the
simulator.

\section{One-Way TCP Senders}
\label{sec:oneWayTcp}

The simulator supports several versions of an abstracted TCP sender.
These objects attempt to capture the essence of the TCP congestion
and error control behaviors, but are not intended to be faithful
replicas of real-world TCP implementations.
They do not contain a dynamic window advertisement, they do segment
number and ACK number computations entirely in packet units,
there is no SYN/FIN connection establishment/teardown, and no
data is ever transferred (e.g. no checksums or urgent data).

\subsection{The Base TCP Sender (Tahoe TCP)}
\label{sec:tahoetcp}

The ``Tahoe'' TCP agent \code{Agent/TCP} performs congestion
control and round-trip-time estimation
in a way similar to the version of TCP released with the
4.3BSD ``Tahoe'' UN'X system release from UC Berkeley.
The congestion window is increased by one packet per new ACK received
during slow-start (when $cwnd\_ < ssthresh\_$) and is increased
by $\frac{1}{cwnd\_}$ for each new ACK received during congestion avoidance
(when $cwnd\_ \geq ssthresh\_$).

\paragraph{Responses to Congestion}
Tahoe TCP assumes a packet has been lost (due to congestion)
when it observes {\tt NUMDUPACKS} (defined in \code{tcp.h}, currently 3)
duplicate ACKs, or when a retransmission timer expires.
In either case, Tahoe TCP reacts by setting {\tt ssthresh\_} to half
of the current window size (the minimum of {\tt cwnd\_} and {\tt window\_})
or 2, whichever is larger.
It then initializes {\tt cwnd\_} back to the value of
{\tt windowInit\_}.  This will typically cause the TCP to
enter slow-start.

\paragraph{Round-Trip Time Estimation and RTO Timeout Selection}
Four variables are used to estimate the round-trip time and
set the retransmission timer: {\tt rtt\_, srtt\_, rttvar\_, tcpTick\_,
and backoff\_}.
TCP initializes rttvar to $3/tcpTick\_$ and backoff to 1.
When any future retransmission timer is set, it's timeout
is set to the current time plus $\max(bt(a+4v+1), 64)$ seconds,
where $b$ is the current backoff value, $t$ is the value of tcpTick,
$a$ is the value of srtt, and $v$ is the value of rttvar.

Round-trip time samples arrive with new ACKs.
The RTT sample is computed as the difference between the current
time and a ``time echo'' field in the ACK packet.
When the first sample is taken, its value is used as the initial
value for {\tt srtt\_}.  Half the first sample is used as the initial
value for {\tt rttvar\_}.
For subsequent samples, the values are updated as follows:

\[ srtt = \frac{7}{8} \times srtt + \frac{1}{8} \times sample \]
\[ rttvar = \frac{3}{4} \times rttvar + \frac{1}{4} \times |sample-srtt| \]

\subsection{Configuration}
\label{sec:tcp-config}

Running an TCP simulation requires
creating and configuring the agent,
attaching an application-level data source (a traffic generator), and
starting the agent and the traffic generator.

\subsection{Simple Configuration}

\paragraph{Creating the Agent}
\begin{program}
set ns [new Simulator]                  \; preamble initialization;
set node1 [$ns node]                     \; agent to reside on this node;
set node2 [$ns node]                     \; agent to reside on this node;

{\bfseries{}set tcp1 [$ns create-connection TCP $node1 TCPSink $node2 42]}
$tcp  set window_ 50                   \; configure the TCP agent;

{\bfseries{}set ftp1 [$tcp1 attach-source FTP]}

$ns at 0.0 "$ftp start"
\end{program}
This example illustrates the use of the simulator built-in
function {\tt create-connection}.
The arguments to this function are: the source agent to create,
the source node, the target agent to create, the target node, and
the flow ID to be used on the connection.
The function operates by creating the two agents, setting the
flow ID fields in the agents, attaching the source and target agents
to their respective nodes, and finally connecting the agents
(i.e. setting appropriate source and destination addresses and ports).
The return value of the function is the name of the source agent created.

\paragraph{TCP Data Source}
The TCP agent does not generate any application data on its own;
instead, the simulation user can connect any traffic generation
module to the TCP agent to generate data.
Two sources are commonly used for TCP: FTP and Telnet.
FTP represents a bulk data transfer of large size, and telnet chooses
its transfer sizes randomly from tcplib (see the file
\code{tcplib-telnet.cc}.
Creation and configuration of the source
is accomplished by the {\tt Agent attach-source} {\em stype} function,
which creates a new object of type \code{Source/}{\em stype} and
returns its name.
The returned object may be started at a later time.
Additional details on these source objects are in Section~\ref{sec:sourceobjects}.

\subsection{Other Configuration Parameters}

In addition to the \code{window_} parameter listed above,
the TCP agent supports additional configuration variables.
Each of the variables described in this subsection is
both a class variable and an instance variable.
Changing the class variable changes the default value
for all agents that are created subsequently.
Changing the instance variable of a particular agent
only affects the values used by that agent.
For example,
\begin{program}
  Agent/TCP set window_ 100     \; Changes the class variable;
  $tcp set window_ 2.0          \; Changes window_ for the $tcp object only;
\end{program}

The default parameters for each TCP agent are:
\begin{program}
Agent/TCP set window_   20              \; max bound on window size;
Agent/TCP set windowInit_ 1             \; initial/reset value of cwnd;
Agent/TCP set windowOption_ 1           \; cong avoid algorithm (1: standard);
Agent/TCP set windowConstant_ 4         \; used only when windowOption != 1;
Agent/TCP set windowThresh_ 0.002       \; used in computing averaged window;
Agent/TCP set overhead_ 0               \; !=0 adds random time between sends;
Agent/TCP set ecn_ 0                    \; TCP should react to ecn bit ;
Agent/TCP set packetSize_ 1000          \; packet size used by sender (bytes);
Agent/TCP set bugFix_ true              \; see explanation;
Agent/TCP set slow_start_restart_ true  \; see explanation;
Agent/TCP set tcpTick_ 0.1              \; timer granulatiry in sec (.1 is NONSTANDARD);
Agent/TCP set maxrto_ 64                \; bound on RTO (seconds);
Agent/TCP set dupacks_ 0                \; duplicate ACK counter;
Agent/TCP set ack_ 0                    \; highest ACK received;
Agent/TCP set cwnd_ 0                   \; congestion window (packets);
Agent/TCP set awnd_ 0                   \; averaged cwnd (experimental);
Agent/TCP set ssthresh_ 0               \; slow-stat threshold (packets);
Agent/TCP set rtt_ 0                    \; rtt sample;
Agent/TCP set srtt_ 0                   \; smoothed (averaged) rtt;
Agent/TCP set rttvar_ 0                 \; mean deviation of rtt samples;
Agent/TCP set backoff_ 0                \; current RTO backoff factor;
Agent/TCP set maxseq_ 0                 \; max (packet) seq number sent;
\end{program}

For many simulations, few of the configuration parameters are likely
to require modification.
The more commonly modified parameters include: {\tt window\_} and
{\tt packetSize\_}.
The first of these bounds the window TCP uses, and is considered
to play the role of the receiver's advertised window in real-world
TCP (although it remains constant).
The packet size essentially functions like the MSS size in real-world
TCP.
Changes to these parameters can have a profound effect on the behavior
of TCP.
Generally, those TCPs with larger packet sizes, bigger windows, and
smaller round trip times (a result of the topology and congestion) are
more agressive in acquiring network bandwidth.

\subsection{Other One-Way TCP Senders}

\paragraph{Reno TCP}
The Reno TCP agent is very similar to the Tahoe TCP agent,
except it also includes {\em fast recovery}, where the current
congestion window is ``inflated'' by the number of duplicate ACKs
the TCP sender has received before receiving a new ACK.
A ``new ACK'' refers to any ACK with a value higher than the higest
seen so far.
In addition, the Reno TCP agent does not return to slow-start during
a fast retransmit.
Rather, it reduces sets the congestion window to half the current
window and resets {\tt ssthresh\_} to match this value.

\paragraph{NewReno TCP}
This agent is based on the Reno TCP agent, but which modifies the
action taken when receiving new ACKS.
In order to exit fast recovery, the sender must receive an ACK for the
highest sequence number sent.
Thus, new ``partial ACKs'' (those which represent new ACKs but do not
represent an ACK for all outstanding data) do not deflate the window
(and possibly lead to a stall, characteristic of Reno).

\paragraph{Vegas TCP}
This agent implements ``Vegas'' TCP (\cite{Brak94:TCP,Brak94a:TCP}).
It was contributed by Ted Kuo.

\paragraph{Sack TCP}
This agent implements selective repeat, based on selective ACKs provided
by the receiver.
It follows the ACK scheme described in \cite{rfc2018}, and was developed
with Matt Mathis and Jamshid Mahdavi.

\paragraph{Fack TCP}
This agent implements ``forward ACK'' TCP, a modification of Sack
TCP described in \cite{Math96:Forward}.

\section{TCP Receivers (sinks)}

The TCP senders described above represent one-way data senders.
They must peer with a ``TCP sink'' object.

\subsection{The Base TCP Sink}

The base TCP sink object ({\tt Agent/TCPSink})
is responsible for returning ACKs to
a peer TCP source object.
It generates one ACK per packet received.
The size of the ACKs may be configured.
The creation and configuration of the TCP sink object
is generally performed automatically by a library
call (see {\tt create-connection} above).

\paragraph{configuration parameters}
\begin{program}
        Agent/TCPSink set packetSize_ 40
\end{program}

\subsection{Delayed-ACK TCP Sink}

A delayed-ACK sink object ({\tt Agent/Agent/TCPSink/DelAck}) is available
for simulating a TCP receiver that ACKs less than once per packet received.
This object contains a bound variable {\tt interval\_} which gives the
number of seconds to wait between ACKs.
The delayed ACK sink implements an agressive ACK policy whereby
only ACKs for in-order packets are delayed.
Out-of-order packets cause immediate ACK generation.

\paragraph{configuration parameters}
\begin{program}
        Agent/TCPSink/DelAck set interval_ 100ms
\end{program}

\subsection{Sack TCP Sink}

The selective-acknowledgment TCP sink ({\tt Agent/TCPSink/Sack1}) implements
SACK generation modeled after the description of SACK in RFC 2018.
This object includes a bound variable {\tt maxSackBlocks\_} which gives
the maximum number of blocks of information in an ACK available for
holding SACK information.
The default value for this variable is 3, in accordance with the expected
use of SACK with RTTM (see RFC 2018, section 3).
Delayed and selective ACKs together are implemented by
an object of type {\tt Agent/TCPSink/Sack1/DelAck}.

\paragraph{configuration parameters}
\begin{program}
        Agent/TCPSink set maxSackBlocks_ 3
\end{program}

\section{Two-Way TCP Agents (FullTcp)}
\label{sec:fulltcp}

The {\tt Agent/TCP/FullTcp} object is a new addition to the suite of
TCP agents supported in the simulator and is still under development.
It is different from (and incompatible with) the other agents, but
does use some of the same architecture.
It differs from these agents in the following ways:
following ways:
\begin{itemize}\itemsep0pt
\item connections may be establised and town down
(SYN/FIN packets are exchanged)
\item bidirectional data transfer is supported
\item sequence numbers are in bytes rather than packets
\end{itemize}

The generation of SYN packets (and their ACKs) can be
of critical importance in trying to model real-world behavior
when using many very short data transfers.
This version of TCP currently defaults to sending
data on the 3rd segment of an initial 3-way handshake, a behavior
somewhat different than common real-world TCP implementations.
A ``typical'' TCP connection proceeds with an active opener
sending a SYN, the passive opener responding with a SYN+ACK,
the active opener responding with an ACK, and then some time later
sending the first segment with data (corresponding to the first
application write).
Thus, this version of TCP sends data at a time somewhat earlier
than typical implementations.
This TCP can also be configured to send data on the initial SYN
segment.
Future changes to FullTCP may include a modification to send the
first data segment later, and possibly to implement T/TCP functionality.

Currently FullTCP is only implemented with Reno congestion control,
but ultimately it should be available with the full range of
congestion control algorithms (e.g., Tahoe, SACK, Vegas, etc.).


\subsection{Simple Configuration}
Running an Full TCP simulation requires
creating and configuring the agent,
attaching an application-level data source (a traffic generator), and
starting the agent and the traffic generator.

\paragraph{Creating the Agent}
\begin{program}
# {\cf set up connection (do not use "create-connection" method because }
# {\cf we need a handle on the sink object)}
set src [new Agent/TCP/FullTcp] \; create agent;
set sink [new Agent/TCP/FullTcp] \; create agent;
$ns_ attach-agent $node_(s1) $src \; bind src to node;
$ns_ attach-agent $node_(k1) $sink \; bind sink to node;
$src set fid_ 0   \; set flow ID field;
$sink set fid_ 0  \; set flow ID field;
$ns_ connect $src $sink \; active connection src to sink;

# {\cf set up TCP-level connections}
$sink listen \; will figure out who its peer is;
$src set window_ 100;
\end{program}

The creation of the FullTcp agent is similar to the other agents,
but the sink is placed in a listening state by the {\tt listen} method.
Because a handle to the receiving side is required in order to make
this call, the {\tt create-connection} call used above cannot be used.

\paragraph{Configuration Parameters}
The following configuration parameters are available through Tcl
for the FullTcp agent:
\begin{program}
Agent/TCP/FullTcp set segsperack_ 1 \; segs received before generating ACK;
Agent/TCP/FullTcp set segsize_ 536  \; segment size (MSS size for bulk xfers);
Agent/TCP/FullTcp set tcprexmtthresh_ 3 \; dupACKs thresh to trigger fast rexmt;
Agent/TCP/FullTcp set iss_ 0 \; initial send sequence number;
Agent/TCP/FullTcp set nodelay_ false \; disable sender-side Nagle algorithm;
Agent/TCP/FullTcp set data_on_syn_ false \; send data on initial SYN?;
Agent/TCP/FullTcp set dupseg_fix_ true \; avoid fast rxt due to dup segs+acks;
Agent/TCP/FullTcp set dupack_reset_ false \; reset dupACK ctr on !0 len data segs containing dup ACKs;
Agent/TCP/FullTcp set interval_ 0.1 \; as in TCP above, (100ms is non-std);
\end{program}

\section{Architecture and Internals}
\label{sec:tcparchitecture}

The base TCP agent (class {\tt Agent/TCP}) is constructed
as a collection of routines for sending packets, processing ACKs,
managing the send window, and handling timeouts.
Generally, each of these routines may be over-ridden by a
function with the same name in a derived class (this is how
many of the TCP sender variants are implemented).

\paragraph{The TCP header}
The TCP header is defined by the {\tt hdr\_tcp} structure
in the file \nsf{tcp.h}.
The base agent only makes use of the following subset of the fields:
\begin{program}
ts_     \* current time packet was sent from source */
ts_echo_ \* for ACKs: timestamp field from packet associated with this ACK */
seqno_ \* sequence number for this data segment or ACK (Note: overloading!) */
reason_ \* set by sender when (re)transmitting to trace reason for send */
\end{program}

\paragraph{Functions for Sending Data}
Note that generally the sending TCP never actually sends
data (it only sets the packet size).

{\bf send\_much(force, reason, maxburst)} - this function
attempts to send as many packets as the current sent window allows.
It also keeps track of how many packets it has sent, and limits to the
total to {\em maxburst}. \\
The function {\tt output(seqno, reason)} sends one packet
with the given sequence number and updates the maximum sent sequence
number variable ({\tt maxseq\_}) to hold the given sequence number if
it is the greatest sent so far.
This function also assigns the various fields in the TCP
header (sequence number, timestamp, reason for transmission).
This function also sets a retransmission timer if one is not already
pending.

\paragraph{Functions for Window Management}

The usable send window at any time is given by the function {\bf window()}.
It returns the minimum of the congestion window and the variable {\tt wnd\_},
which represents the receiver's advertised window.

{\bf opencwnd()} - this function opens the congestion window.  It is invoked
when a new ACK arrives.
When in slow-start, the function merely increments {\tt cwnd\_} by each
ACK received.
When in congestion avoidance, the standard configuration increments {\tt cwnd\_}
by its reciprocal.
Other window growth options are supported during congestion avoidance,
but they are experimental (and not documented; contact Sally Floyd for
details).

{\bf closecwnd(int how)} - this function reduces the congestion window. It
may be invoked in several ways: when entering fast retransmit, due to
a timer expiration, or due to a congestion notification (ECN bit set).
Its argument {\tt how} indicates how the congestion window should
be reduced.  The value {\bf 0} is used for retransmission timeouts and
fast retransmit in Tahoe TCP.  It typically causes the TCP to enter
slow-start and reduce {\tt ssthresh\_} to half the current window.
The value {\bf 1} is used by Reno TCP for implementing fast recovery
(which avoids returning to slow-start).
The value {\bf 2} is used for reducing the window due to an ECN indication.
It resets the congestion window to its initial value (usually causing
slow-start), but does not alter {\tt ssthresh\_}.

\paragraph{Functions for Processing ACKs}

{\bf recv()} - this function is the main reception path for ACKs.
Note that because only one direction of data flow is in use, this function
should only ever be invoked with a pure ACK packet (i.e. no data).
The function stores the timestamp from the ACK in {\tt ts\_peer\_}, and
checks for the presence of the ECN bit (reducing the send window if
appropriate).
If the ACK is a new ACK, it calls {\bf newack()}, and otherwise
checks to see if it is a duplicate of the last ACK seen.
If so, it enters fast retransmit by closing the window, resetting the
retransmission timer, and sending a packet by calling {\tt send\_much}.

{\bf newack()} - this function processes a ``new'' ACK (one that contains
an ACK number higher than any seen so far).
The function sets a new retransmission timer by calling {\bf newtimer()},
updates the RTT estimation by calling {\bf rtt\_update}, and updates
the highest and last ACK variables.

\paragraph{Functions for Managing the Retransmission Timer}

These functions serve two purposes: estimating the round-trip time
and setting the actual retransmission timer.
{\bf rtt\_init} - this function initializes {\tt srtt\_} and {\tt rtt\_}
to zero, sets {\tt rttvar\_} to $3/tcp\_tick\_$, and sets the backoff
multiplier to one.

{\bf rtt\_timeout} - this function gives the timeout value in seconds that
should be used to schedule the next retransmission timer.
It computes this based on the current estimates of the mean and deviation
of the round-trip time.  In addition, it implements Karn's
exponential timer backoff for multiple consecutive retransmission timeouts.

{\bf rtt\_update} - this function takes as argument the measured RTT
and averages it in to the running mean and deviation estimators
according to the description above.
Note that {\tt t\_srtt\_} and {\tt t\_rttvar} are both
stored in fixed-point (integers).
They have 3 and 2 bits, respectively, to the right of the binary
point.

{\bf reset\_rtx\_timer} -  This function is invoked during fast retransmit
or during a timeout.
It sets a retransmission timer
by calling {\tt set\_rtx\_timer} and if invoked by a timeout also calls
{\tt rtt\_backoff}.

{\bf rtt\_backoff} - this function backs off the retransmission timer
(by doubling it).

{\bf newtimer} - this function called only when a new ACK arrives.
If the sender's left window edge is beyond the ACK, then
{\tt set\_rtx\_timer} is called, otherwise if a retransmission timer
is pending it is cancelled.

\section{Tracing TCP Dynamics}
\label{sec:traceTcpdyn}

The behavior of TCP is often observed by constructing a
sequence number-vs-time plot.
Typically, a trace is performed by enabling tracing on a link
over which the TCP packets will pass.
Two trace methods are supported: the default one (used for tracing
TCP agents), and an extension used only for FullTcP.

\section{One-Way Trace TCP Trace Dynamics}
\label{sec:trace1WayTcpdyn}

TCP packets generated by one of the one-way TCP agents and destined for
a TCP sink agent
passing over a traced link (see section~\ref{chap:trace})
will generate a trace file lines of the form:
\begin{verbatim}
+ 0.94176 2 3 tcp 1000 ------ 0 0.0 3.0 25 40
+ 0.94276 2 3 tcp 1000 ------ 0 0.0 3.0 26 41
d 0.94276 2 3 tcp 1000 ------ 0 0.0 3.0 26 41
+ 0.95072 2 0 ack 40 ------ 0 3.0 0.0 14 29
- 0.95072 2 0 ack 40 ------ 0 3.0 0.0 14 29
- 0.95176 2 3 tcp 1000 ------ 0 0.0 3.0 21 36
+ 0.95176 2 3 tcp 1000 ------ 0 0.0 3.0 27 42
\end{verbatim}
The exact format of this trace file is given in section~\ref{sec:traceformat}.
When tracing TCP, packets of type {\sf tcp} or {\sf ack} are relevant.
Their type, size, sequence number (ack number for ack packets),
and arrival/depart/drop time are given by field positions
5, 6, 11, and 2, respectively.
The {\sf +} indicates a packet arrival, {\sf d} a drop, and {\sf -} a
departure.
A number of scripts process this file to produce graphical output or
statistical summaries (see,  for example, \nsf{test-suite.tcl}, the
{\tt finish} procedure.

\section{One-Way Trace TCP Trace Dynamics}
\label{sec:tcpdyn}

TCP packets generated by FullTcp and
passing over a traced link contain additional information not displayed
by default using the regular trace object.
By enabling the flag {\tt show\_tcphdr\_} on the trace object
(see section~ref{sec:traceformat}), 3 additional header fields are
written to the trace file: ack number, tcp-specific flags, and header length.

\endinput
