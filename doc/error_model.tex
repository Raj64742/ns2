%
% personal commentary:
%        DRAFT DRAFT DRAFT
%        - GNGUYEN
%
\chapter{Error Model}
\label{chap:error_model}

The procedures and functions described in this chapter can be found in
\nsf{errmodel.\{cc, h\}}.

Error model simulates link-level errors or loss by either marking the
packet's error flag or dumping the packet to a drop target.  In
simulations, errors can be generated from a simple model such as the
packet error rate, or from more complicated statistical and empirical models.
To support a wide variety of models, the unit of error can be specified
in term of packet, bits, or time-based.

The \code{ErrorModel} class is derived from the \code{Connector} base
class.  As the result, it inherits some methods for hooking up objects
such as \code{target} and \code{drop-target}.  If the drop target
exists, it will received corrupted packets from \code{ErrorModel}.
Otherwise, \code{ErrorModel} just marks the \code{error_} flag of the
packet's common header, thereby, allowing agents to handle the loss.
The \code{ErrorModel} also defines additional Tcl method \code{unit} to
specify the unit of error and \code{ranvar} to specify the random
variable for generating errors.  If not specified, the unit of error
will be in packets, and the random variable will be uniform distributed
from 0 to 1.  Below is a simple example of creating an error model with
the packet error rate of 1 percent (0.01):
\begin{program}
        # {]cf create a loss_module and set its packet error rate to 1 percent}
        set loss_module [new ErrorModel]
        $loss_module set rate_ 0.01

        # {\cf optional:  set the unit and random variable}
        $em unit pkt            \; error unit: packets (the default);
        $em ranvar [new RandomVariable/Uniform]

        # {\cf set target for dropped packets}
        $loss_module drop-target [new Agent/Null]
\end{program}

In C++, the \code{ErrorModel} contains both the mechanism and policy for
dropping packets.  The packet dropping mechanism is handled by the
\code{recv} method, and packet corrupting policy is handled by the
\code{corrupt} method.
\begin{program}
	enum ErrorUnit \{ EU_PKT=0, EU_BIT, EU_TIME \};

	class ErrorModel : public Connector \{
	public:
	       	ErrorModel();
	        void recv(Packet*, Handler*);
        	virtual int corrupt(Packet*);
	        inline double rate() \{ return rate_; \bs{}n \}
	protected:
        	int command(int argc, const char*const* argv);
	        ErrorUnit eu_;		\* error unit in pkt, bit, or time */
	        RandomVariable* ranvar_;
        	double rate_;
	\};
\end{program}
The \code{ErrorModel} only implements a simple policy based on a single
error rate, either in packets of bits.  More sophisticated dropping
policy can be implemented in C++ by deriving from \code{ErrorModel} and
redefining its \code{corrupt} method.
