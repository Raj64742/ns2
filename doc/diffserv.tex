%
% Most parts are ported from the Manual of diffserv from Nortel
% -xuanc (Nov 16, 2000)        
%
% DRAFT!!!
%
\chapter{Differentiated Services Module in ns}
\label{chap:diffserv}


\textbf{Note: This chapter is incomplete and subject to change.}

This chapter describes the Differentiated Services Module that was originally 
ported as Nortel's Differentiated Services extension to ns \cite{Diffserv}.

Differentiated Services, or Diffserv, is an IP QoS architecture based on 
packet-marking that allows packets to be prioritized according to user 
requirements.  During the time of congestion, more low priority packets are 
discarded than high priority packets.

\section{Overview}
\label{sec:diffservoverview}

The procedures and functions described in this section can be found in
\nsf{diffserv/dsred, dsredq, dsEdge, dsCore, dsPolicy.\{cc, h\}}.

The Diffserv architecture provides QoS by dividing traffic into different 
categories, marking each packet with a code point that indicates its category, 
and scheduling packets according to their code points. The Diffserv model in ns 
currently defines four classes of traffic, each of which has three drop 
precedences.  Those drop precedences enable differential treatment of traffic 
within a single class. A single class of traffic is enqueued into one 
corresponding physical RED queue, which contains three virtual queues (one for
each drop precedence).

Different RED parameters are used for the virtual queues, causing packets from 
one virtual queue to be dropped more frequently than packets from another.  A 
packet with a lower drop precedence is given better treatment in times of 
congestion because it is assigned a code point that corresponds to a virtual 
queue with relatively lenient RED parameters.  

The Diffserv model in ns has three major components:
\begin{description}

\item [policy]
Policy is specified by network administrator about the level of service a class
of traffic shouldreceive in the network.  

\item [edge routers]
Edge routers marks packets with a code point according to the policy specified.

\item [core routers]
Core routers examine packets' code point marking and forward them accordingly.

\end{description}
Diffserv attempts to restrict complexity to only the edge routers.


\section{Implementation}
\label{sec:diffservimplement}
The procedures and functions described in this section can be found in
\nsf{diffserv/dsred, dsredq, dsEdge, dsCore, dsPolicy.\{cc, h\}}.

\subsection{RED queue for Diffserv}
\label{sec:dsredq}

In ns, the Diffserv functionality is captured in a Queue object, which is 
implemented as an alternative to other queue types such as DropTail, CBQ, and 
RED.  A Diffserv queue (\code{dsREDQueue} derived from the base class 
\code{Queue}, see \code{dsred.\{h,cc\}}) contains the abilities:
\begin{itemize}
\item
to implement multiple physical RED queues along a single link;
\item
to implement multiple virtual queues within a physical queue, with 
individual set of parameters for each virtual queue;
\item
to determine in which physical and virtual queue a packet is enqueued, according
to policy specified.
\end{itemize}

The \code{dsREDQueue} consists of four (defined in \code{numQueues_}) physical 
RED queues,
each containing three (defined in \code{numPrec}) virtual queues, 
referred to as precedence levels. Each physical queue corresponds to a class of
traffic; and each combination of a queue and precedence number is associated 
with a code point (or a drop preference), which specifies a certain level of 
service.

Each physical RED queue is defined in \code{class redQueue} 
(see \code{dsredq.\{h,cc\}}). 
The \code{redQueue class} enables that traffic differentiation by defining 
virtual RED queues, each of which has independent configuration and state 
parameters. 
For example, the length of each virtual queue is calculated only on packets 
mapped to that queue.  Thus, packet dropping decisions can be applied based on 
the state and configuration parameters of the virtual queues.

The redQueue class is not equivalent to the REDQueue class, which was already 
present in ns.  Instead, it is a modified copy of that class that includes the 
notion of virtual queues.

Instances of the redQueue class only exist inside instances of the dsREDQueue class.  All user interaction with the redQueue class is handled through the command interface of the dsREDQueue class.

The \code{dsREDQueue} class contains a data structure known as the PHB Table 
(per hop behaviour table).  Edge devices handle marking packets with code points
 and core devices simply respond to existing code points.  However, both devices
 need to determine how to map a code point to a particular queue and precedence 
level.

The PHB Table handles this mapping by defining an array with three fields:
\begin{itemize}
\item
Code point
\item
Class (Physical Queue)
\item
Precedence (Virtual Queue)
\end{itemize}


\subsection{Core and edge routers}
\label{sec:dsedge}
The Diffserv edge and core routers are defined in class \code{edgeQueue} and 
class \code{coreQueue}, which are both derived from class \code{dsREDQueue},
see \code{dsEdge, dsCore.\{h,cc\}}.

\subsection{Policy}
\label{sec:dspolicy}
The class \code{Policy} (defined in \code{dsPolicy.\{cc, h\}}) is used by the 
class \code{edgeQueue} to handle all policy functionality.

A policy is established between a source and destination node.  All flows 
matching that source-destination pair are treated as a single traffic aggregate.
Each policy defines a policer type, a target rate, and other policer-specific 
parameters.  As a minimum, each policy defines two code points; and the choice 
of code point depends on a comparison between the aggregate's target rate and 
current sending rate.

Each traffic aggregate has an associated policer type, meter type, and initial 
code point.  The meter type specifies the method for measuring the state 
variables needed by the policer.  For example, the TSW Tagger is a meter that 
measures the average traffic rate, using a specified time window.

When a packet arrives at an edge router, it is examined to determine to which 
aggregate it belongs.  The meter specified for that aggregate is invoked to 
update all state variables.  Then the policer is invoked to determine how to 
mark the packet.  Depending on the aggregate's state variables, either the 
specified initial code point is used or a downgraded code point is used; and 
the packet is enqueued accordingly.

The \code{Policy} class uses a Policy Table to store the policies of each 
traffic aggregate.  This table is an array that includes fields for the 
source and destination nodes, a policer type, a meter type, an initial code 
point, and various state information.

The fields of the Policy Table are:
\begin{itemize}
\item
Source node ID
\item
Destination node ID
\item
Policer type
\item
Meter type
\item
Initial code point
\item
CIR (committed information rate)
\item
CBS (committed burst size)
\item
C bucket (current size of the committed bucket)
\item
EBS (excess burst size)
\item
E bucket (current size of the excess bucket)
\item
PIR (peak information rate)
\item
PBS (peak burst size)
\item
P bucket (current size of the peak bucket)
\item
Arrival time of last packet
\item
Average sending rate
\item
TSW window length
\end{itemize}

The \code{Policy} class currently supports six policer types:
\begin{enumerate}
\item
TSW2CM (TSW2CMPolicer): uses a CIR and two drop precedences.  The lower precedence is used probabilistically when the CIR is exceeded.
\item
TSW3CM (TSW3CMPolicer): uses a CIR, a PIR, and three drop precedences.  The medium drop precedence is used probabilistically when the CIR is exceeded and the lowest drop precedence is used probabilistically when the PIR is exceeded.
\item
Token Bucket (tokenBucketPolicer): uses a CIR and a CBS and two drop precedences.  An arriving packet is marked with the lower precedence if and only if it is larger than the token bucket.
\item
Single Rate Three Color Marker (srTCMPolicer): uses a CIR, CBS, and an EBS to choose from three drop precedences, as explained in [12].
\item
Two Rate Three Color Marker (trTCMPolicer): uses a CIR, CBS, PIR, and a PBS to choose from three drop precdences, as explained in [13].

\end{enumerate}

Policer Table:

The Policy class uses a Policer Table to store the mappings from a policy type and initial code point pair to its associated downgraded code point(s).  This table is an array with four fields:
\begin{itemize}
\item
Policer type
\item
Initial code point
\item
Downgraded code point 1
\item
Downgraded code point 2
\end{itemize}
The last field is not used for policer types with only two drop precedences; and it should be set to the worst code point for policer types with three drop precedences.

\section{Configuration}
\label{sec:diffservconfig}

The number of physical and virtual queues can be configured as:

\code{$dsredq set numQueues_ 1}
\code{$dsredq setNumPrec 2}

\code{$dsredq configQ 0 1 10 20 0.10}

This command configures the RED parameters for one virtual queue.  The above example specifies that physical queue 0/virtual queue 1 has a minth value of 10 packets, a maxth value of 20 packets, and a maxp value of 0.10.  [9] contains an explanation of these RED parameters. For DropTail queues, only the first three parameters are required and the second is disregarded because there is no notion of precedence level for a DropTail queue.

\code{$dsredq addPHBEntry 11 0 1}

The addPHBEntry command adds an entry to the PHB Table; in this example, code point 11 is mapped to physical queue 0 and virtual queue 1. In ns, the packets are defaulted to a code point of zero. Therefore, to handle best effort traffic one must add a PHB entry for the zero code point.

\code{$dsredq meanPktSize 1500}

This command specifies the mean packet size (in bytes), which is used for RED calculations.

In addition, commands are available which allow the user to choose the scheduling mode between queues.

\code{$dsredq setSchedularMode WRR}\\
\code{$dsredq addQueueWeights 1 5}\\

The above pair of commands sets the scheduling mode to Weighted Round Robin and then sets the queue weight for queue 1 to 5. Other scheduling modes supported are Weighted Interleaved Round Robin (WIRR), Round Robin (RR), and Priority (PRI). The default scheduling mode is Round Robin.

For Priority scheduling, priority is arranged in sequential order with queue 0 having the highest priority. Also, one can set the a limit on the maximum bandwidth a particular queue can get using the addQueueRate command.

\code{$dsredq setSchedularMode PRI}
\code{$dsredq addQueueRate 0 5000000}


The values of the bound variables may be checked from a script; and the 
\code{dsREDQueue} Tcl interface also interprets three additional queries:

\code{$dsredq printPHBTable}

This command prints the entire PHB Table, one line at a time.

\code{$dsredq printStats}

This command is meant to be a debugging tool that can be altered as needed.  Currently, it prints the defined number of physical and virtual queues.

\code{$dsredq getAverage 0}

This query returns the RED weighted average size of the specified physical queue.

The addPolicyEntry command is used to add an entry to the Policy Table.  It takes different parameters depending on what policer type is used.  The first two parameters after the command name are always the source and destination node IDs, and the next parameter is the policer type. 

Following the policer type are the parameters needed by that policer as summarized below:

\begin{quote}
\begin{tabular}{llllll}
TSW2CM&{Initial code point}&CIR\\
TSW3CM&{Initial code point}&CIR&PIR\\
TokenBucket&{Initial code point}&CIR&CBS\\
srTCM&{Initial code point}&CIR&CBS&EBS\\
trTCM&{Initial code point}&CIR&CBS&PIR&PBS
\end{tabular}
\end{quote}

The rates CIR and PIR are specified in bits per second; the buckets CBS, EBS, and PBS are specified in bytes.
 
Consider a Tcl script for which \code{$q} is a variable for an edge queue, 
and \code{$s} and \code{$d} are source and destination nodes.   
The following command adds a TSW2CM policer for traffic going from the source to the destination:

\code{$q addPolicyEntry [$s id] [$d id] TSW2CM 10 2000000}

The following parameters could be used in place of "TSW2CM . . ." to use a different policer:
\begin{quote}
\begin{tabular}{llllll}
TSW3CM&10&2000000&3000000\\
TokenBucket&10&2000000&10000\\
srTCM&10&2000000&10000&20000\\
trTCM&10&2000000&10000&3000000&10000
\end{tabular}
\end{quote}

Note, however, that only one policy can be applied to any source-destination pair.

The following command adds an entry to the Policer Table, specifying that the trTCM initial (green) code point 10 should be downgraded to yellow code point 11 and red code point 12:

\code{$dsredq addPolicerEntry trTCM 10 11 12}

There must be a Policer Table entry in place for every policer type/initial code point pair.

Four queries are interpreted by an edgeQueue class instance:

\code{$dsredq printPolicyTable}

This command prints the entire Policy Table, one line at a time.

\code{$dsredq printPolicerTable}

This command prints the entire Policer Table, one line at a time.

\code{$dsredq getCBucket}

This query returns the current size of the C Bucket, in bytes.

\section{Adding new policy}
\label{sec:diffservaddnewpolicy}

\section{Commands at a glance}
\label{sec:diffservcommand}

The following is a list of error-model related commands commonly used in
simulation scripts:

\begin{program}
$ns simplex-link $edge $core 10Mb 5ms dsRED/edge
$ns simplex-link $core $edge 10Mb 5ms dsRED/core
\end{program}

These two commands create the queues along the link between an edge router and 
a core router.

\begin{program}
set qEC [[$ns link $edge $core] queue]

# Set DS RED parameters from Edge to Core:
$qEC meanPktSize $packetSize
$qEC set numQueues_ 1
$qEC setNumPrec 2
$qEC addPolicyEntry [$s1 id] [$dest id] TokenBucket 10 $cir0 $cbs0
$qEC addPolicyEntry [$s2 id] [$dest id] TokenBucket 10 $cir1 $cbs1
$qEC addPolicerEntry TokenBucket 10 11
$qEC addPHBEntry 10 0 0
$qEC addPHBEntry 11 0 1
$qEC configQ 0 0 20 40 0.02
$qEC configQ 0 1 10 20 0.10
\end{program}

This block of code obtains handle to the Diffserv queue from an edge router to
a core router and configures all of the parameters for it.

The meanPktSize command is required for the RED state variables to be calculated
accurately.  Setting the number of queues and precedence levels is optional, but
it aids efficiency. Because neither the scheduling or MRED mode type are set, 
they default to Round Robin scheduling and RIO-C Active Queue Management.

The addPolicyEntry commands establish two policies at the edge queue: one 
between nodes S1 and Dest and one between nodes S2 and Dest.  Note that the 
\code{[$s1 id]} command returns the ID value needed by addPolicyEntry.  The CIR and CBS values used in the policies are the ones set at the beginning of the script.

The addPolicerEntry line is required because each policer type/initial code point pair requires an entry in the Policer Table.  Each of the policies uses the same policer and initial code point, so only one entry is needed.

The addPHBEntry commands map each code point to a queue/precedence pair.  Although each code point in this example maps to a unique queue/precedence pair, that need not be the case; multiple code points could receive identical treatment.

Finally, the configQ commands set the RED parameters for each virtual queue.  Note that as the precedence value increases, the RED parameters become harsher, which follows [8].

\begin{program}

set qCE [[$ns link $core $e1] queue]

# Set DS RED parameters from Core to Edge:
$qCE meanPktSize $packetSize
$qCE set numQueues_ 1
$qCE setNumPrec 2
$qCE addPHBEntry 10 0 0
$qCE addPHBEntry 11 0 1
$qCE configQ 0 0 20 40 0.02
$qCE configQ 0 1 10 20 0.10

\end{program}

Note that the configuration of a core queue matches that of an edge queue, except that there is no Policy Table or Policer Table to configure at a core router.  A core router's chief requirement is that it has a PHB entry for all code points that it will see.

\begin{program}
$qE1C printPolicyTable
$qCE2 printCoreStats
\end{program}

These methods output the policy or policer tables on link and different statistics.  

For further information, please refer to the example scripts under \ns/tcl/ex/diffserv.
