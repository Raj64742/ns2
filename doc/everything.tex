\documentclass{report}

\usepackage{times}
\usepackage[T1]{fontenc}

\PassOptionsToPackage{draft}{nsdoc}
\usepackage{nsdoc}

\title{\ns\ Notes and Documentation}
\author{Multiple Members of the VINT project\\
  Kevin Fall \tup{kfall@ee.lbl.gov}, Editor\\
  Kannan Varadhan \tup{kannan@catarina.usc.edu}, Editor}
\date{\today}

\begin{document}
\maketitle
\thispagestyle{empty}

\chapter{Preamble}

\xref{\ns}{http://www-nrg.ee.lbl.gov/ns/}
\xref{\copyright}{../copyright.html}
is LBNL's \underline{N}etwork \underline{S}imulator \cite{ns}.
The simulator is written in C++;
it uses OTcl as a command and configuration interface.
\ns~v2 has three substantial changes from \ns~v1:
(1) the more complex objects in \ns~v1
    have been decomposed into simpler components
    for greater flexibility and composability;
(2) the configuration interface is now OTcl, 
    an object oriented version of Tcl; and
(3) the interface code to the OTcl interpreter is
    separate from the main simulator.

In this document,
we use the term ``interpreter''
to be synonymous with the OTcl interpreter.
The code to interface with the interpreter resides
in a separate directory, \code{Tcl}.
The rest of the simulator code resides in the directory, \code{ns-2}.
We will use the notation \Tclf{\tup{file}}\
to refer to a particular \tup{file}\ in the
\code{Tcl}\ directory.
Similarly, we will use the notation, \nsf{\tup{file}}
to refer to a particular \tup{file}\ in the \code{ns-2} directory.

\tableofcontents

\part{Interface to the Interpreter}
\chapter{OTcl Linkage}
\label{chap:otcl:intro}

\ns\ is an object oriented simulator,
written in C++, with an OTcl interpreter as a frontend.
The simulator supports a class hierarchy in C++
(also called the compiled hierarchy in this document),
and a similar class hierarchy within the OTcl interpreter
(also called the interpreted hierarchy in this document).
The two hierarchies are closely related to each other;
from the user's perspective,
there is a one-to-one correspondence
between a class in the interpreted hierarchy
and one in the compiled hierarchy.
The root of this hierarchy is the class TclObject.
Users create new simulator objects through the interpreter;
these objects are instantiated within the interpreter, 
and are closesly mirrored by a corresponding object
in the compiled hierarchy.
The interpreted class hierachy is automatically established through
methods defined in the class TclClass.
user instantiated objects are mirrored through methods
defined in the class TclObject.
There are other hierarchies in the C++ code and OTcl scripts;
these other hierarchies are not mirrored in the manner of TclObject.

In this document,
we use the term ``interpreter''
to be synonymous with the OTcl interpreter.
The code to interface with the interpreter resides
in a separate directory, \code{Tcl}.
The rest of the simulator code resides in the directory, \code{ns-2}.
We will use the notation \Tclf{\tup{file}}\
to refer to a particular \tup{file}\ in the
\code{Tcl}\ directory.
Similarly, we will use the notation, \nsf{\tup{file}}
to refer to a particular \tup{file}\ in the \code{ns-2} directory.

There are a number of classes defined in \Tclf{}.
We only focus on the six that are used in \ns:
The \href{Class Tcl}{Section}{sec:Tcl} contains the methods that
C++ code will use to access the interpreter.
The \href{class TclObject}{Section}{sec:TclObject}
is the base class for all simulator objects that are also mirrored 
in the compiled hierarchy.
The \href{class TclClass}{Section}{sec:TclClass} defines
the interpreted class hierarchy, and 
the methods to permit the user to instantiate TclObjects.
The \href{class TclCommand}{Section}{sec:TclCommand}
is used to define simple global interpreter commands.
The \href{class EmbeddedTcl}{Section}{sec:EmbeddedTcl}
contains the methods to load higher level builtin commands
that make configuring simulations easier.
Finally, the \href{class InstVar}{Section}{sec:InstVar}
contains methods to access C++ member variables
as OTcl instance variables.

The procedures and functions described in this chapter can be found in
\Tclf{Tcl.\{cc, h\}}, \Tclf{Tcl2.cc}, \Tclf{tcl-object.tcl}, and,
\Tclf{tracedvar.\{cc, h\}}.
The file \Tclf{tcl2c++.c} is used in building \ns, and is mentioned
briefly in this chapter.

\section{Class Tcl}
\label{sec:Tcl}

The \clsref{Tcl}{../Tcl/Tcl.h} encapsulates the actual instance of
the OTcl interpreter, and provides the methods
to access and communicate with that interpreter.
The methods described in this section are relevant to the
\ns\ programmer who is writing C++ code.
The class provides methods for the following operations:
\begin{list}{\textbullet}{}\itemsep0pt
\item obtain a reference to the Tcl instance;
%  {\tt
%    \begin{list}{}{}
%    \item \fcn[]{Tc::instance}
%    \end{list}
%  }
\item invoke OTcl procedures through the interpreter;
%  {\tt
%    \begin{list}{}{}
%    \item \fcn[char* $s$]{Tcl::eval}
%    \item \fcn[const char* $s$]{Tcl::evalc}
%    \item \fcn[]{Tcl::eval}
%    \item \fcn[const char* $\mathit{fmt}$, \ldots]{Tcl::evalf}
%    \end{list}
%  }
\item retrieve, or pass back results to the interpreter;
%  {\tt
%    \begin{list}{}{}
%    \item \fcn[const char* $s$]{Tcl::result}
%    \item \fcn[const char* $\mathit{fmt}$, \ldots]{Tcl::resultf}
%    \item \fcn[]{Tcl::result}
%    \end{list}
%  }
\item report error situations and exit in an uniform manner; and
%  {\tt
%    \begin{list}{}{}
%    \item \fcn[const char* $s$]{Tcl::error}
%    \end{list}
%  }
\item store and lookup ``TclObjects''.
%  {\tt
%    \begin{list}{}{}
%    \item \fcn[const char* $s$]{Tcl::lookup}
%    \item \fcn[TclObject* $o$]{Tcl::enter}
%    \item \fcn[TclObject* $o$]{Tcl::remove}
%    \end{list}
%  }
\item acquire direct access to the interpreter.
%  {\tt
%    \begin{list}{}{}
%    \item \fcn[]{Tcl::interp}
%    \end{list}
%  }
\end{list}
We describe each of the methods in the following subsections.

\subsection{Obtain a Reference to the class Tcl instance}
\label{sec:instance}

A single instance of the class is declared in \Tclf{Tcl.cc}
as a static member variable;
the programmer must obtain a reference to this instance
to access other methods described in this section.
The statement required to access this instance is:
\begin{program}
        Tcl& tcl = Tcl::instance();
\end{program}

\subsection{Invoking OTcl Procedures}
\label{sec:Invoke}
There are four different methods to invoke an OTcl command
through the instance, \code{tcl}.
They differ essentially in their calling arguments.
Each function passes a string to the interpreter,
that then evaluates the string in a global context.
These methods will return to the caller if the interpreter returns TCL\_OK.
On the other hand, if the interpreter returns TCL\_ERROR,
the methods will call \proc{tkerror}.
The user can overload this procedure to selectively disregard
certain types of errors.
Such intricacies of OTcl programming are outside the
scope of this document.
\href{The next section}{Section}{sec:Result}
describes methods to access the result returned by the interpreter.
\begin{itemize}\itemsep0pt
\item \fcnref{\fcn[char* $s$]{tcl.eval}}{../Tcl/Tcl.cc}{Tcl::eval}
  invokes \fcn[]{Tcl\_GlobalEval} to execute $s$ through the interpreter.

\item \fcnref{\fcn[const char* $s$]{tcl.evalc}}{../Tcl/Tcl.cc}{Tcl::evalc}
  preserves the argument string $s$.
  It copies the string $s$ into its internal buffer;
  it then invokes the previous \fcn[char* $s$]{eval} on the internal buffer.

\item \fcnref{\fcn[]{tcl.eval}}{../Tcl/Tcl.cc}{Tcl::eval}
  assumes that the command is already stored in the class' internal
  \code{bp_}; it directly invokes \fcn[char* bp\_]{tcl.eval}.
  A handle to the buffer itself is available through the method
  \fcnref{\fcn{tcl.buffer}}{../Tcl/Tcl.h}{Tcl::buffer}.

\item
  \fcnref{\fcn[const char* $s$, \ldots]{tcl.evalf}}{../Tcl/Tcl2.cc}{Tcl::evalf}
  is a \code{Printf}(3) like equivalent.
  It uses \code{vsprintf}(3) internally to create the input string.
\end{itemize}
As an example, here are some of the ways of using the above methods:
\begin{program}
        Tcl& tcl = {\bfseries{}Tcl::instance}();
        char wrk[128];
        strcpy(wrk, "Simulator set NumberInterfaces_ 1");
        {\bfseries{}tcl.eval}(wrk);

        sprintf({\bfseries{}tcl.buffer}(), "Agent/SRM set requestFunction_ %s", "Fixed");
        {\bfseries{}tcl.eval}();

        {\bfseries{}tcl.evalc}("puts stdout {hello world}");
        {\bfseries{}tcl.evalf}("%s request %d %d", name_, sender, msgid);
\end{program}

\subsection{Passing Results to/fro the Interpreter}
\label{sec:Result}

When the interpreter invokes a C++ method,
it expects the result back in the private member variable,
\code{tcl_->result}.
Two methods are available to set this variable.
\begin{list}{\textbullet}{}
\item \fcnref{\fcn[const char* $s$]{tcl.result}}{../Tcl/Tcl.h}{Tcl::result}

        Pass the result string $s$ back to the interpreter.
\item
  \fcnref{\fcn[const char* fmt, \ldots]{tcl.resultf}}{../Tcl/Tcl2.cc}{Tcl::resultf}

        \code{varargs}(3) variant of above
        to format the result using \code{vsprintf}(3),
        pass the result string back to the interpreter.
\end{list}
\begin{program}
        if (strcmp(argv[1], "now") == 0) \{
                {\bfseries{}tcl.resultf}("%.17g", clock());
                return TCL_OK;
        \}
        {\bfseries{}tcl.result}("Invalid operation specified");
        return TCL_ERROR;
\end{program}

Likewise, when a C++ method invokes an OTcl command,
the interpreter returns the result in \code{tcl_->result}.
\begin{list}{\textbullet}{}
\item \fcnref{\fcn{tcl.result}}{../Tcl/Tcl.h}{Tcl::result}
      must be used to retrieve the result.
      Note that the result is a string, that must be converted
      into an internal format appropriate to the type of result.
\end{list}
\begin{program}
        tcl.evalc("Simulator set NumberInterfaces_");
        char* ni = {\bfseries{}tcl.result}();
        if (atoi(ni) != 1)
                tcl.evalc("Simulator set NumberInterfaces_ 1");
\end{program}
        
\subsection{Error Reporting and Exit}
\label{sec:ErrorReporting}

This method provides a uniform way to report errors in the compiled code.
\begin{list}{\textbullet}{}
\item \fcnref{\fcn[const char* $s$]{tcl.error}}{../Tcl/Tcl.cc}{Tcl::error}
performs the following functions:
write $s$ to stdout; write \code{tcl_->result} to stdout;
exit with error code 1.
\end{list}
\begin{program}
        {\bfseries{}tcl.resultf}("cmd = %s", cmd);
        {\bfseries{}tcl.error}("invalid command specified");
        /*{\cf{}NOTREACHED}*/
\end{program}

Note that
there are minor differences between returning TCL\_ERROR
\href{as we did in the previous subsection}{Section}{sec:Result},
and calling \fcn[]{Tcl::error}.
The former generates an exception within the interpreter;
the user can trap the exception and possibly recover from the error.
If the user has not specified any traps, 
the interpreter will print a stack trace and exit.
However, if the code invokes \fcn[]{error},
then the simulation user cannot trap the error;
in addition, \ns\ will not print any stack trace.

\subsection{Hash Functions within the Interpreter}
\label{sec:HashTables}

\ns\ stores a reference to every TclObject in the compiled hierarchy
in a hash table;
this permits quick access to the objects.
The hash table is internal to the interpreter.
\ns\ uses the name of the \code{TclObject} as the key
to enter, lookup, or delete the TclObject in the hash table.
\begin{list}{\textbullet}{}
\item \fcnref{\fcn[TclObject* $o$]{tcl.enter}}{../Tcl/Tcl.cc}{Tcl::enter}
  will insert a pointer to the TclObject $o$ into the hash table.

  It is used by
  \fcnref{\fcn[]{TclClass::create\_shadow}}{../Tcl/Tcl.cc}{TclClass::create\_shadow}
  to insert an object into the table, when that object is created.

\item \fcnref{\fcn[char* $s$]{tcl.lookup}}{../Tcl/Tcl.h}{Tcl::lookup}
  will retrieve the TclObject with the name $s$.

  It is used by
  \fcnref{\fcn[]{TclObject::lookup}}{../Tcl/Tcl.h}{TclObject::lookup}.
\item \fcnref{\fcn[TclObject* $o$]{tcl.remove}}{../Tcl/Tcl.cc}{Tcl::remove}
  will delete references to the TclObject $o$ from the hash table.

  It is used by
  \fcnref{\fcn[]{TclClass::delete\_shadow}}{../Tcl/Tcl.cc}{TclClass::delete\_shadow}
  to remove an existing entry from the hash table,
  when that object is deleted.
\end{list}
These functions are used internally by
the class TclObject and class TclClass.

\subsection{Other Operations on the Interpreter}
\label{sec:otcl:other}

If the above methods are not sufficient,
then we must acquire the handle to the interpreter,
and write our own functions.
\begin{list}{\textbullet}{}
\item \fcnref{\fcn{tcl.interp}}{../Tcl/Tcl.h}{Tcl::interp}
        returns the handle to the interpreter that is stored
        within the class Tcl.
\end{list}

\section{Class TclObject}
\label{sec:TclObject}

\clsref{TclObject}{../Tcl/Tcl.h}
is the base class for most of the other classes
in the interpreted and compiled hierarchies.
Every object in the class TclObject is created by the user
from within the interpreter.
An equivalent shadow object is created in the compiled hierarchy.
The two objects are closely associated with each other.
The class TclClass, described in the next section,
contains the mechanisms that perform this shadowing.

In the rest of this document, we often refer to an object as a TclObject%
\footnote{In the latest release of \ns\ and \nsTcl,
  this object has been renamed to \code{SplitObjefct},
  which more accurately reflects its nature of existence.
  However, for the moment,
  we will continue to use the term TclObject
  to refer to these objects and this class.}.
By this, we refer to a particular object that is either in the class
TclObject, or in a class that is derived from the class TclObject.
If it is necessary, we will explicitly qualify whether that object is
an object within the interpreter, or an object within the compiled code.
In such cases,
we will use the abbreviations ``interpreted object'', and
``compiled object'' to distinguish the two.
and within the compiled code respectively.

\paragraph{Differences from \ns~v1}
Unlike \ns~v1, the class TclObject
subsumes the earlier functions of the NsObject class.
It therefore stores the
\href{interface variable bindings}{Section}{sec:VarBinds}
that tie OTcl instance variables in the interpreted object
to corresponding C++ member variables in the compiled object.
The binding is stronger than in \ns~v1 in that
any changes to the OTcl variables are trapped,
and the current C++ and OTcl values
are made consistent after each access through the interpreter.
The consistency is done through the
\href{class InstVar}{Section}{sec:InstVar}.
Also unlike \ns~v1, objects in the class TclObject
are no longer stored as a global link list.
Instead, they are stored in a hash table in the
\href{class Tcl}{Section}{sec:HashTables}.

\paragraph{Example configuration of a TclObject}
The following example illustrates the configuration of
an SRM agent (\clsref{Agent/SRM/Adaptive}{../ns-2/srm-adaptive.tcl}).
\begin{program}
        set srm [new Agent/SRM/Adaptive]
        $srm set packetSize_ 1024
        $srm traffic-source $s0
\end{program}
By convention in \ns,
the class Agent/SRM/Adaptive is a subclass of Agent/SRM,
is a subclass of Agent, is a subclass of TclObject.
The corresponding compiled class hierarchy is
the ASRMAgent, derived from SRMAgent, derived from Agent,
derived from TclObject respectively.
The first line of the above example shows how a TclObject is 
\href{created (or destroyed)}{Section}{sec:Creation};
the next line configures
\href{a bound variable}{Section}{sec:VarBinds};
and finally, the last line illustrates
the interpreted object invoking a C++ method
\href{as if they were an instance procedure}{Section}{sec:Commands}.

\subsection{Creating and Destroying TclObjects}
\label{sec:Creation}

When the user creates a new TclObject,
using the procedures \proc[]{new} and \proc[]{delete};
these procedures are defined in \Tclf{tcl-object.tcl}.
They can be used to create and destroy objects in all classes,
including TclObjects.%
\footnote{As an example, the classes Simulator, Node, Link, or rtObject,
are classes that are \emph{not} derived from the class TclObject.
Objects in these classes  are not, therefore, TclObjects.
However, a Simulator, Node, Link, or route Object is also instantiated
using the \code{new} procedure in \ns.}.
In this section,
we describe the internal actions executed when a TclObject
is created.

\paragraph{Creating TclObjects}
By using \proc[]{new}, the user creates an interpreted TclObject.
the interpreter will execute the constructor for that object, \proc[]{init},
passing it any arguments provided by the user.
\ns\ is responsible for automatically  creating the compiled object.
The shadow object gets created by the base class TclObject's constructor.
Therefore, the constructor for the new TclObject
must call the parent class constructor first.
\proc[]{new} returns a handle to the object, that can then be used
for further operations upon that object.

The following example illustrates the Agent/SRM/Adaptive constructor:
\begin{program}
        Agent/SRM/Adaptive instproc init args \{
                eval $self next $args
                $self array set closest_ "requestor 0 repairor 0"
                $self set eps_    [$class set eps_]
        \}
\end{program}

The following sequence of actions are performed by the interpreter
as part of instantiating a new TclObject.
For ease of exposition, we describe the steps that are executed
to create an Agent/SRM/Adaptive object.
The steps are:
\begin{enumerate}
\item
  Obtain an unique handle for the new object   from the TclObject name space.
  The handle is returned to the user.
  Most handles in \ns\ have the form \code{_o\tup{NNN}}, where \tup{NNN}
  is an integer.  This handle is created by
  \fcnref{\proc{getid}}{../Tcl/tcl-object.tcl}{TclObject::getid}.
\item Execute the constructor for the new object.
  Any user-specified arguments are passed as arguments to the constructor.
  This constructor must invoke the constructor
  associated with its parent class.

  In our example above, the Agent/SRM/Adaptive calls its parent class
  in the very first line.  

  Note that each constructor,
  in turn invokes its parent class' constructor \textit{ad nauseum}.
  The last constructor in \ns\ is
  \fcnref{the TclObject constructor}{../Tcl/tcl-object.tcl}{TclObject::init}.
  This constructor is responsible for setting up the shadow object, and 
  performing other initialisations and bindings, as we explain below.
  \emph{It is preferable to call the parent constructors first before
    performing the initialisations required in this class.}
  This allows the shadow objects to be set up,
  and the variable bindings established.
\item The TclObject constructor invokes the instance procedure
  \proc[]{create-shadow} for the class Agent/SRM/Adaptive.
\item When the shadow object is created,
  \ns\ calls all of the constructors for the compiled object,
  each of which may establish variable bindings for objects in that class,
  and perform other necessary initialisations.
  Hence our earlier injunction that it is preferable to invoke the parent
  constructors prior to performing the class initialisations.
\item After the shadow object is successfully created,
  \fcnref{\fcn{create\_shadow}}{../Tcl/Tcl.cc}{TclClass::create\_shadow}
  \begin{enumerate}
  \item adds the new object to hash table of TclObjects
        \href{described earlier}{Section}{sec:HashTables}.
  \item makes \proc[]{cmd} an instance procedure of the newly created
    interpreted object.
    This instance procedure
    invokes the \fcn[]{command} method of the compiled object.
    In \href{a later subsection}{Section}{sec:Commands},
    we describe how the \code{command} method is defined, and invoked.
  \end{enumerate}
\end{enumerate}
Note that all of the above shadowing mechanisms only work when
the user creates a new TclObject through the interpreter.
It will not work if the programmer creates a compiled TclObject unilaterally.
Therefore, the programmer is enjoined not to use the C++ new method
to create compiled objects directly.

\paragraph{Deletion of TclObjects}
The \code{delete} operation
destroys the interpreted object, and the corresponding shadow object.
For example,
\fcnref{\proc[\tup{scheduler}]{use-scheduler}}{%
  ../ns-2/ns-lib.tcl}{Simulator::use-scheduler}
uses the \code{delete} procedure to remove the default list scheduler,
and instantiate an alternate scheduler in its place.
\begin{program}
        Simulator instproc use-scheduler type \{
                $self instvar scheduler_

                delete scheduler_ \; first delete the existing list scheduler;
                set scheduler_ [new Scheduler/$type]
        \}
\end{program}

As with the constructor, the object destructor must call the destructor
for the parent class explicitly as the very last statement of the destructor.
The TclObject destructor
will invoke the instance procedure \code{delete-shadow},
that in turn invokes \fcnref{the equivalent compiled method}{%
  ../Tcl/Tcl.cc}{TclClass::delete\_shadow}
to destroy the shadow object.
The interpreter itself will destroy the interpreted object.

\subsection{Variable Bindings}
\label{sec:VarBinds}

In most cases,
access to compiled member variables is restricted to compiled code,
and access to interpreted member variables is likewise
confined to access via interpreted code;
however, it is possible to establish bi-directional bindings
such that both the interpreted member variable
and the compiled member variable access the same data, 
and changing the value of either variable
changes the value of the corresponding paired variable to same value.

The binding is established by the compiled constructor
when that object is instantiated;
it is automatically accessible by the interpreted object as 
an instance variable.
\ns\ supports five different data types: reals, bandwidth valued variables, 
time valued variables, integers, and booleans.
The syntax of how these values can be specified in OTcl is different
for each variable type.
\begin{itemize}\itemsep0pt
\item Real and Integer valued variables are specified in the ``normal'' form.
        For example,
        \begin{program}
        $object set realvar 1.2e3
        $object set intvar  12
        \end{program}
\item Bandwidth is specified as a real value, optionally
  suffixed by a `k' or `K' to mean kilo-quantities, or `m' or `M' to
  mean mega-quantities.
  A final optional suffix of `B' indicates that the quantity expressed
  is in Bytes per second.
  The default is bandwidth expressed in bits per second.
        For example, all of the following are equivalent:
        \begin{program}
        $object set bwvar 1.5m
        $object set bwvar 1.5mb
        $object set bwvar 1500k
        $object set bwvar 1500kb
        $object set bwvar .1875MB
        $object set bwvar 187.5kB
        $object set bwvar 1.5e6
        \end{program}

\item Time is specified as a real value, optionally suffixed by a
  `m' to express time in milli-seconds, `n' to express time in
  nano-seconds, or `p' to express time in pico-seconds.
  The default is time expressed in seconds.
        For example, all of the following are equivalent:
        \begin{program}
        $object set timevar 1500m
        $object set timevar 1.5
        $object set timevar 1.5e9n
        $object set timevar 1500e9p
        \end{program}
  Note that we can also safely add a $s$ to reflect the time unit of seconds.
  \ns\ will ignore anything other than a valid real number specification,
  or a trailing `m', `n', or `p'.

\item Booleans can be expressed either as an integer, or as `T' or `t'
  for true.  Subsequent characters after the first letter are ignored.
  If the value is neither an integer, nor a true value,
  then it is assumed to be false.
        For example,
        \begin{program}
        $object set boolvar t           \; set to true;
        $object set boolvar true
        $object set boolvar 1   \; or any non-zero value;

        $object set boolvar false       \; set to false;
        $object set boolvar junk        
        $object set boolvar 0
        \end{program}

\end{itemize}

The following example shows the constructor for the ASRMAgent%
\footnote{Note that this constructor is embellished to illustrate
        the features of the variable binding mechanism.}.
\begin{program}
        ASRMAgent::ASRMAgent() \{
                bind("pdistance_", &pdistance_);      \* real variable */
                bind("requestor_", &requestor_);      \* integer variable */
                bind_time("lastSent_", &lastSessSent_); \* time variable */
                bind_bw("ctrlLimit_", &ctrlBWLimit_); \* bandwidth variable */
                bind_bool("running_", &running_);     \* boolean variable */
        \}
\end{program}
Note that all of the functions above take two arguments,
the name of an OTcl variable,
and the address of the corresponding compiled member variable
that is linked.
While it is often the case that these bindings are established
by the constructor of the object, 
it need not always be done in this manner.
We will discuss such alternate methods
when we describe \href{the class InstVar}{Section}{sec:InstVar}
in detail later.

Each of the variables that is bound is automatically initialised
with default values when the object is created.
The default values are specified as interpreted class variables.
This initialisation is done by the routing \proc[]{init-instvar},
invoked by methods in the class Instvar,
\href{described later}{Section}{sec:InstVar}.
\proc[]{init-instvar} checks the class of the interpreted object,
and all of the parent class of that object, to find the first
class in which the variable is defined.
It uses the value of the variable in that class to initialise the object.
Most of the bind initialisation values are defined in
\nsf{tcl/lib/ns-default.tcl}.

For example, if the following class variables are defined for the ASRMAgent:
\begin{program}
        Agent/SRM/Adaptive set pdistance_ 15.0
        Agent/SRM set pdistance_ 10.0
        Agent/SRM set lastSent_ 8.345m
        Agent set ctrlLimit_    1.44M
        Agent/SRM/Adaptive set running_ f
\end{program}
Therefore, every new Agent/SRM/Adaptive object will have
\code{pdistance_} set to 15.0;
\code{lastSent_} is set to 8.345m
from the setting of the class variable of the parent class;
\code{ctrlLimit_} is set to 1.44M
using the class variable of the parent class twice removed;
\code{running} is set to false;
the instance variable \code{pdistance_} is not initialised,
because no class variable
exists in any of the class hierarchy of the interpreted object.
In such instance, \proc[]{init-instvar} will invoke 
\proc[]{warn-instvar}, to print out a warning about such a variable.
The user can selectively override this procedure
in their simulation scripts, to elide this warning.

Note that the actual binding
is done by instantiating objects in the class InstVar.
Each object in the class InstVar binds 
one compiled member variable to one interpreted member variable.
A TclObject stores a list of InstVar objects corresponding
to each of its member variable that is bound in this fashion.
The head of this list is stored in its member variable
\code{instvar_} of the TclObject.

One last point to consider is that
\ns\ will guarantee that the actual values
of the variable, both in the interpreted object and the compiled object,
will be identical at all times.
However, if there are methods and other variables
of the compiled object that track the value of this variable,
they must be explicitly invoked or changed whenever the
value of this variable is changed.
This usually requires additional primitives that the user should invoke.
One way of providing such primitives in \ns\ is through
the \fcn[]{command} method described in the next section.


\subsection{Variable Tracing}
\label{sec:VarTrace}

In addition to variable bindings, TclObject also supports tracing of
both C++ and Tcl instance variables.  A traced variable can be created
and configured either in C++ or Tcl.  To establish variable tracing at
the Tcl level, the variable must be visible in Tcl, which means that it
must be a bounded C++/Tcl or a pure Tcl instance variable.  In addition,
the object that owns the traced variable is also required to establish
tracing using the Tcl \code{trace} method of TclObject.  The first
argument to the \code{trace} method must be the name of the variable.
The optional second argument specifies the trace object that is
responsible for tracing that variable.  If the trace object is not
specified, the object that own the variable is responsible for tracing
it.

For a TclObject to trace variables, it must extend the C++
\code{trace} method that is virtually defined in TclObject.  The Trace
class implements a simple \code{trace} method, thereby, it can act as a
generic tracer for variables.

\begin{verbatim}
class Trace : public Connector {
        ...
        virtual void trace(TracedVar*);
};
\end{verbatim}

Below is a simple example for setting up variable tracing in Tcl:

\begin{small}
\begin{verbatim}
        # $tcp tracing its own variable cwnd_
        $tcp trace cwnd_

        # the variable ssthresh_ of $tcp is traced by a generic $tracer
        set tracer [new Trace/Var]
        $tcp trace ssthresh_ $tracer
\end{verbatim}
\end{small}

For a C++ variable to be traceable, it must belong to a class that
derives from TracedVar.  The virtual base class TracedVar keeps track of
the variable's name, owner, and tracer.  Classes that derives from
TracedVar must implement the virtual method \code{value}, that takes a
character buffer as an argument and writes the value of the variable
into that buffer.

\begin{small}
\begin{verbatim}
class TracedVar {
        ...
        virtual char* value(char* buf) = 0;
protected:
        TracedVar(const char* name);
        const char* name_;      // name of the variable
        TclObject* owner_;      // the object that owns this variable
        TclObject* tracer_;     // callback when the variable is changed
        ...
};
\end{verbatim}
\end{small}

The TclCL library exports two classes of TracedVar:  \code{TracedInt} and
\code{TracedDouble}.  These classes can be used in place of the basic
type int and double respectively.  Both TracedInt and TracedDouble
overload all the operators that can change the value of the variable
such as assignment, increment, and decrement.  These overloaded
operators use the \code{assign} method to assign the new value to the
variable and call the tracer if the new value is different from the old
one.  TracedInt and TracedDouble also implement their \code{value}
methods that output the value of the variable into string.  The width
and precision of the output can be pre-specified.

\subsection{\code{command} Methods: Definition and Invocation}
\label{sec:Commands}

For every TclObject that is created, \ns\ establishes
the instance procedure, \proc[]{cmd},
as a hook to executing methods through the compiled shadow object.
The procedure \proc[]{cmd} invokes the method \fcn[]{command}
of the shadow object automatically, passing the arguments to \proc[]{cmd}
as an argument vector to the \fcn[]{command} method.

The user can invoke the \proc[]{cmd} method in one of two ways:
by explicitly invoking the procedure, specifying the desired
operation as the first argument, or
implicitly, as if there were an instance procedure of the same name as the
desired operation.
Most simulation scripts will use the latter form, hence, we will
describe that mode of invocation first.

Consider the that the distance computation in SRM is done by
the compiled object; however, it is often used by the interpreted object.
It is usually invoked as:
\begin{program}
        $srmObject distance? \tup{agentAddress}
\end{program}
If there is no instance procedure called \code{distance?},
the interpreter will invoke the instance procedure
\proc[]{unknown}, defined in the base class TclObject.
The unknown procedure then invokes
\begin{program}
        $srmObject cmd distance? \tup{agentAddress}
\end{program}
to execute the operation through the compiled object's
\fcn[]{command} procedure.

Ofcourse, the user could explicitly invoke the operation directly.
One reason for this might be to overload the operation by using
an instance procedure of the same name.
For example,
\begin{program}
        Agent/SRM/Adaptive instproc distance? addr \{
                $self instvar distanceCache_
                if ![info exists distanceCache_($addr)] \{
                        set distanceCache_($addr) [{\bfseries{}$self cmd distance? $addr}]
                \}
                set distanceCache_($addr)
        \}
\end{program}

We now illustrate how the \fcn[]{command} method using
\fcn[]{ASRMAgent::command} as an example.
\begin{program}
        int ASRMAgent::command(int argc, const char*const*argv) \{
                Tcl& tcl = Tcl::instance();
                if (argc == 3) \{
                        if (strcmp(argv[1], "distance?") == 0) \{
                                int sender = atoi(argv[2]);
                                SRMinfo* sp = get_state(sender);
                                tcl.tesultf("%f", sp->distance_);
                                return TCL_OK;
                        \}
                \}
                return (SRMAgent::command(argc, argv));
        \}
\end{program}
We can make the following observations from this piece of code:
\begin{itemize}
\item The function is called with two arguments:
  
  The first argument (\code{argc}) indicates
  the number of arguments specified in the command line to the interpreter.

  The command line arguments vector (\code{argv}) consists of
  
  --- \code{argv[0]} contains the name of the method, ``\code{cmd}''.

  --- \code{argv[1]} specifies the desired operation.

  --- If the user specified any arguments, then they are placed in
  \code{argv[2\ldots(argc - 1)]}.

  The arguments are passed as strings;
  they must be converted to the appropriate data type.
\item If the operation is successfully matched,
  the match should return the result of the operation
  using methods \href{described earlier}{Section}{sec:Result}.
\item \fcn[]{command} itself must return either \code{TCL_OK} or \code{TCL_ERROR}
  to indicate success or failure as its return code.
\item If the operation is not matched in this method, it must
  invoke its parent's command method, and return the corresponding result.

  This permits the user to concieve of operations as having the same
  inheritance properties as instance procedures or compiled methods.

  In the event that this \code{command} method 
  is defined for a class with multiple inheritance,
  the programmer has the liberty to choose one of two implementations:

  1) Either they can invoke one of the parent's \code{command} method,
  and return the result of that invocation, or

  2) They can each of the parent's \code{command} methods in some sequence,
  and return the result of the first invocation that is successful.
  If none of them are successful, then they should return an error.
\end{itemize}
In our document, we call operations executed through the 
\fcn[]{command} \emph{instproc-like}s.
This reflects the usage of these operations as if they were
OTcl instance procedures of an object,
but can be very subtly different in their realisation and usage.


\section{Class TclClass}
\label{sec:TclClass}

This compiled class (\clsref{TclClass}{../Tcl/Tcl.h})
is a pure virtual class.
Classes derived from this base class provide two functions:
construct the interpreted class hierarchy
to mirror the compiled class hierarchy; and
provide methods to instantiate new TclObjects.
Each such derived class is associated with a particular compiled class
in the compiled class hierarchy, and can instantiate new objects in the
associated class.

As an example, consider a class such as the
class \code{RenoTcpClass}.
It is derived from class \code{TclClass}, and
is associated with the class \code{RenoTcpAgent}.
It will instantiate new objects in the class \code{RenoTcpAgent}.
The compiled class hierarchy for \code{RenoTcpAgent} is that
it derives from \code{TcpAgent}, that in turn derives from \code{Agent},
that in turn derives (roughly) from \code{TclObject}.
\code{RenoTcpClass} is defined as
\begin{program}
        static class RenoTcpClass: public TclClass \{
        public:
                RenoTcpClass() : TclClass("Agent/TCP/Reno") \{\}
                TclObject* create(int argc, const char*const* argv) \{
                        return (new RenoTcpAgent());
                \}
        \} class_reno;
\end{program}
We can make the following observations from this definition:
\begin{enumerate}
\item The class defines only the constructor, and one additional method,
  to \code{create} instances of the associated TclObject.
\item \ns\ will execute the \code{RenoTcpClass} constructor
  for the static variable \code{class_reno}, when it is first started.
  This sets up the appropriate methods and the interpreted class hierarchy.
\item The constructor specifies the interpreted class explicitly as
  \code{Agent/TCP/Reno}.  This also specifies the interpreted class
  hierarchy implicitly.

  Recall that the convention in \ns\ is to use
  the character slash ('/') is a separator.
  For any given class \code{A/B/C/D},
  the class \code{A/B/C/D} is a sub-class of \code{A/B/C},
  that is itself a sub-class of \code{A/B},
  that, in turn, is a sub-class of \code{A}.
  \code{A} itself is a sub-class of \code{TclObject}.

  In our case above, the TclClass constructor creates three classes,
  \code{Agent/TCP/Reno} sub-class of \code{Agent/TCP}
  sub-class of \code{Agent} sub-class of \code{TclObject}.
\item This class is associated with the class \code{RenoTcpAgent};
  it creats new objects in this associated class.
\item The \code{RenoTcpClass::create} method returns TclObjects in the
  class \code{RenoTcpAgent}.
\item When the user specifies \code{new Agent/TCP/Reno},
  the routine \code{RenoTcpClass::create} is invoked.
\item The arguments vector (\code{argv}) consists of

  --- \code{argv[0]} contains the name of the object.

  --- \code{argv[2\ldots4]} contain
  \code{$self}, \code{$class}, and \code{$proc}.
  Since \code{create} is called
  through the instance procedure \code{create-shadow},
  \code{argv[4]} contains \code{create-shadow}.

  --- \code{argv[5\ldots]}
  contain any additional arguments provided by the user.
\end{enumerate}
The \clsref{Trace}{../ns-2/trace.cc} illustrates
argument handling by TclClass methods.
\begin{program}
        class TraceClass : public TclClass \{
        public:
                TraceClass() : TclClass("Trace") \{\}
                TclObject* create(int args, const char*const* argv) \{
                        if (args >= 5)
                                return (new Trace(*argv[4]));
                        else
                                return NULL;
                \}
        \} trace_class;
\end{program}
A new Trace object is created as
\begin{program}
        new Trace "X"
\end{program}
Finally, the nitty-gritty details of how the 
interpreted class hierarchy is constructed:
\begin{enumerate}
\item The object constructor is executed when \ns\ first starts.
\item This constructor calls the TclClass constructor
  with the name of the interpreted class as its argument.
\item The TclClass constructor stores the name of the class,
  and inserts this object into a linked list of the TclClass objects.
\item During initialisation of the simulator,
  \fcnref{\fcn{Tcl\_AppInit}}{../ns-2/ns_tclsh.cc}{::Tcl\_AppInit}
  invokes 
  \fcnref{\fcn{TclClass::bind}}{../Tcl/Tcl.cc}{TclClass::bind}
\item For each object in the list of TclClass objects,
  \fcn[]{bind} invokes 
  \fcnref{\proc[]{register}}{../Tcl/tcl-object.tcl}{TclObject::register},
  specifying the name of the interpreted class as its argument.
\item \proc[]{register} establishes the class hierarchy,
  creating the classes that are required, and not yet created.
\item Finally, \fcn[]{bind} defines instance procedures
  \code{create-shadow} and \code{delete-shadow} for this new class.
\end{enumerate}

\section{Class TclCommand}
\label{sec:TclCommand}

This class (\clsref{TclCommand}{../Tcl/Tcl.h})
provides just the mechanism for \ns\ to export
simple commands to the interpreter, 
that can then be executed within a global context by the interpreter.
There are two functions defined in \nsf{misc.cc}:
\code{ns-random} and \code{ns-version}.
These two functions are initialised by the function
\fcnref{\fcn{init\_misc}}{../ns-2/misc.cc}{::init\_misc},
defined in \nsf{misc.cc};
\code{init_misc} is invoked by
\fcnref{\fcn{Tcl\_AppInit}}{../ns-2/ns_tclsh.cc}{::Tcl\_AppInit}
during startup.
\begin{itemize}\itemsep0pt
\item \clsref{VersionCommand}{../ns-2/misc.cc}
  defines the command \code{ns-version}.
  It takes no argument, and returns the current \ns\ version string.
\begin{program}
            % ns-version                \; get the current version;
            2.0a12
\end{program}

\item \clsref{RandomCommand}{../ns-2/misc.cc}
  defines the command \code{ns-random}.
  With no argument, \code{ns-random} returns an integer,
  uniformly distributed in the interval $[0, 2^{31}-1]$.

  When specified an argument, it takes that argument as the seed.
  If this seed value is 0, the command uses a hueristic seed value;
  otherwise, it sets the seed for the random number generator to the
  specified value.
\begin{program}
            % ns-random                 \; return a random number;
            2078917053
            % ns-random 0               \;set the seed hueristically;
            858190129
            % ns-random 23786           \;set seed to specified value;
            23786
\end{program}
\end{itemize}

\emph{Note that, it is generally not advisable to construct
  top-level commands that are available to the user.}
We now describe how to define a new command
using the example \code{class say_hello}.
The example defines the command \code{hi},
to print the string ``hello world'',
followed by any command line arguments specified by the user.
For example,
\begin{program}
            % hi this is ns [ns-version]
            hello world, this is ns 2.0a12
\end{program}
\begin{enumerate}
\item The command must be defined within a class
  derived from the \clsref{TclCommand}{../Tcl/Tcl.h}.
  The class definition is:
  \begin{program}
        class say_hello : public TclCommand \{
        public:
                say_hello();
                int command(int argc, const char*const* argv);
        \};
  \end{program}
\item The constructor for the class must invoke the
  \fcnref{TclCommand constructor}{../Tcl/Tcl.cc}{TclCommand::TclCommand}
  with the command as argument; \ie,
  \begin{program}
        say_hello() : TclCommand("hi") \{\}
  \end{program}
  The \code{TclCommand} constructor sets up "hi"
  as a global procedure that invokes
  \fcnref{\fcn[]{TclCommand::dispatch\_cmd}}{../ns-2/Tcl.cc}{TclCommand::dispatch\_cmd}.
\item  The method \fcn[]{command} must perform the desired action.

  The method is passed two arguments.  The first argument, \code{argc},
  contains the number of actual arguments passed by the user.

  The actual arguments passed by the user are passed as an
  argument vector (\code{argv}) and contains the following:
  
  --- \code{argv[0]} contains the name of the command (\code{hi}).

  --- \code{argv[1\ldots(argc - 1)]} contains additional arguments
  specified on the command line by the user.

  \fcn[]{command} is invoked by \fcn[]{dispatch\_cmd}.
\begin{program}
        #include <streams.h>        \* because we are using stream I/O */
        
        int say_hello::command(int argc, const char*const* argv) \{
                cout << "hello world:";
                for (int i = 1; i < argc; i++)
                        cout << ' ' << argv[i];
                cout << '\bs n';
                return TCL_OK;
        \}
\end{program}
\item Finally, we require an instance of this class.
  \code{TclCommand} instances are created in the routine
  \fcnref{\fcn{init\_misc}}{../ns-2/misc.cc}{::init\_misc}.
  \begin{program}
        new say_hello;
  \end{program}
\end{enumerate}
Note that there used to be more functions such as \code{ns-at}\ and
\code{ns-now}\ that were accessible in this manner.
Most of these functions have been subsumed into existing classes.
In particular, \code{ns-at}\ and \code{ns-now}\ are accessible
through the
\fcnref{scheduler TclObject}{../ns-2/scheduler.cc}{Scheduler::command}.
These functions are defined in \nsf{tcl/lib/ns-lib.tcl}.
\begin{program}
            % set ns [new Simulator]    \; get new instance of simulator;
            _o1
            % $ns now                   \; query simulator for current time;
            0
            % $ns at \ldots             \; specify at operations for simulator;
            \ldots
\end{program}
          

\section{Class EmbeddedTcl}
\label{sec:EmbeddedTcl}

\ns\ permits the development of functionality in either compiled code,
or through interpreter code, that is evaluated at initialisation.
For example, the scripts \Tclf{tcl-object.tcl} or the scripts in
\nsf{tcl/lib}.
Such loading and evaluation of scripts is done through objects in the
\clsref{EmbeddedTcl}{../Tcl/Tcl.h}.

The easiest way to extend \ns\ is to add OTcl code
to either \Tclf{tcl-object.tcl} or through scripts
in the \nsf{tcl/lib} directory.
Note that, in the latter case, \ns\ sources
\nsf{tcl/lib/ns-lib.tcl} automatically, and hence
the programmer must add a couple of lines to this file
so that their script will also get automatically sourced by \ns\
at startup.
As an example,
the file \nsf{tcl/mcast/srm.tcl} defines some of the instance procedures
to run SRM.
In \nsf{tcl/lib/ns-lib.tcl}, we have the lines:
\begin{program}
	source tcl/mcast/srm.tcl
\end{program}
to automatically get srm.tcl sourced by \ns\ at startup.

Two points to note with EmbeddedTcl code are that
firstly, if the code has an error that is caught during the eval,
then \ns\ will not run.
Secondly, the user can explicitly override any of the code in the scripts.
In particular, they can re-source the entire script after making their own
changes. 

The rest of this subsection illustrate
how to integrate individual scripts directly into \ns.
The first step is convert the script into an EmbeddedTcl object.
The lines below expand ns-lib.tcl and create the EmbeddedTcl object
instance called \code{et_ns_lib}:
\begin{program}
        tclsh bin/tcl-expand.tcl tcl/lib/ns-lib.tcl | \bs
                               ../Tcl/tcl2c++ et_ns_lib > gen/ns_tcl.cc
\end{program}
The script, \xref{\nsf{bin/tcl-expand.tcl}}{../ns-2/tcl-expand.tcl}
expands \code{ns-lib.tcl} by replacing all \code{source} lines
with the corresponding source files.
The program, \xref{\Tclf{tcl2cc.c}}{../Tcl/tcl2c++.c.html},
converts the OTcl code into an equivalent EmbeddedTcl object, \code{et_ns_lib}.

During initialisation, invoking the method \code{EmebddedTcl::load}
explicitly evaluates the array.
\begin{list}{---}{}
\item
  \xref{\Tclf{tcl-object.tcl}}{../Tcl/tcl-object.tcl}
  is evaluated by the method
  \fcnref{\fcn{Tcl::init}}{../Tcl/Tcl.cc}{Tcl::init};
  \fcnref{\fcn[]{Tcl\_AppInit}}{../ns-2/tclAppInit.cc}{::Tcl\_AppInit}
  invokes \fcn[]{Tcl::Init}.
  The exact command syntax for the load is:
  \begin{program}
	et_tclobject.load();
  \end{program}
\item
  Similarly,
  \xref{\nsf{tcl/lib/ns-lib.tcl}}{../ns-2/tcl/lib/ns-lib.tcl}
  is evaluated directly by \code{Tcl_AppInit} in \nsf{ns\_tclsh.cc}.
  \begin{program}
	et_ns_lib.load();
  \end{program}
\end{list}

\section{Class InstVar}
\label{sec:InstVar}

This section describes the internals of the \clsref{InstVar}{../Tcl/Tcl.cc}.
This class defines the methods and mechanisms to bind
a C++ member variable in the compiled shadow object
to a specified OTcl instance variable in the equivalent interpreted object.
The binding is set up such that the value of the variable can be
set or accessed either from within the interpreter, or from
within the compiled code at all times.

There are five instance variable classes:
\clsref{InstVarReal}{../Tcl/Tcl.cc},
\clsref{InstVarTime}{../Tcl/Tcl.cc},
\clsref{InstVarBandwidth}{../Tcl/Tcl.cc},
\clsref{InstVarInt}{../Tcl/Tcl.cc},
and \clsref{InstVarBool}{../Tcl/Tcl.cc},
corresponding to bindings for real, time, bandwidth, integer, and
boolean valued variables respectively.

We now describe the mechanism by which instance variables are set up.
We use the \clsref{InstVarReal}{../Tcl/Tcl.cc}
to illustrate the concept.
However, this mechanism is applicable to all five types of instance variables.

When setting up an interpreted variable to access a member variable,
the member functions of the class InstVar assume that they are executing
in the appropriate method execution context;
therefore, they do not query the interpreter to determine the context in
which this variable must exist.

In order to guarantee the correct method execution context,
a variable must only be bound if its class is already established within
the interpreter, and
the interpreter is currently operating on an object in that class.
Note that the former requires that when a method in a given class is
going to make its variables accessible via the interpreter,
there must be an associated 
\href{class TclClass}{Section}{sec:TclClass}
defined that identifies the appropriate class hierarchy to the interpreter.
The appropriate method execution context can therefore be created in one
of two ways.

An implicit solution occurs whenever a new TclObject is created within
the interpreter.
This sets up the method execution context within the interpreter.
When the compiled shadow object of the interpreted TclObject is created,
the constructor for that compiled object can bind its member variables
of that object
to interpreted instance variables in the context of the newly created
interpreted object.

An explicit solution is to define a \code{bind-variables} operation
within a \code{command} function, that can then be invoked via the
\code{cmd} method.
The correct method execution context is established in order to execute
the \code{cmd} method.
Likewise, the compiled code is now operating on the appropriate
shadow object, and can therefore safely bind the required member variables.

An instance variable is created by specifying the name of the
interpreted variable, and the address of the member variable in the
compiled object.
The
\fcnref{constructor}{../Tcl/Tcl.cc}{InstVar::InstVar}
for the base class InstVar 
creates an instance of the variable in the interpreter,
and then sets up a
\fcnref{trap routine}{../Tcl/Tcl.cc}{InstVar::catch_var}
to  catch all accesses to the variable through the interpreter.

Whenever the variable is read through the interpreter, the
\fcnref{trap routine}{../Tcl/Tcl.cc}{InstVar::catch_read}
is invoked just prior to the occurrence of the read.
The routine invokes the appropriate
\fcnref{\code{get} function}{../Tcl/Tcl.cc}{InstVarReal::get}
that returns the current value of the variable.
This value is then used to set the value of the interpreted variable
that is then read by the interpreter.

Likewise,
whenever the variable is set through the interpreter, the
\fcnref{trap routine}{../Tcl/Tcl.cc}{InstVar::catch_write}
is invoked just after to the write is completed.
The routine gets the current value set by the interpreter, 
and invokes the appropriate
\fcnref{\code{set} function}{../Tcl/Tcl.cc}{InstVarReal::set}
that sets the value of the compiled member to the current value set
within the interpreter.

\endinput

### Local Variables:
### mode: latex
### comment-column: 60
### backup-by-copying-when-linked: t
### file-precious-flag: nil
### End:


\part{Simulator Basics}
\include{simulator}
\include{nodeslinks}
\include{packets}
%       This draft written by Tom Henderson (8/29/97) based on John Heidemann's
%   code comments.
%
%
% If you get conflicts, here's what you need to keep:  The chapter heading
% in the first entry is essential.  The \endinput at end is useful.
% Other mods are to promote each sub*section one level up.
%
\chapter{\shdr{Timers}{timer-handler.h}{sec:timers}}

Timers may be implemented in C++ or OTcl.  In C++, timers are based on an 
abstract base class defined in \code{timer-handler.h}.  They are most often 
used in agents, but the 
framework is general enough to be used by other objects.  The discussion
below is oriented towards the use of timers in agents.

In OTcl, a simple timer class is defined in \code{tcl/ex/timer.tcl}.  
Subclasses can be derived to provide a simple mechanism for scheduling events 
at the OTcl level.

\section{\shdr{C++ abstract base class TimerHandler}{timer-handler.h}{sec:abstractbaseclass}}

The abstract base class \code{TimerHandler} contains the following public member functions:
\begin{tt}
\begin{quote}
\begin{itemize}
\item[void sched(double delay)] - schedule a timer to expire delay seconds in the future
\item[void resched(double delay)] - reschedule a timer (similar to sched(), but
timer may be pending)
\item[void cancel()] - cancel a pending timer
\item[int status()] - returns timer status (either IDLE, PENDING, or HANDLING)
\end{itemize}
\end{quote}
\end{tt}

The abstract base class \code{TimerHandler} contains the following protected members:
\begin{tt}
\begin{quote}
\begin{itemize}
\item[virtual void expire(Event *e) = 0] - this method must be filled in by the timer client
\item[virtual void handle(Event *e) = 0] - consumes an event 
\item[int status\_] - keeps track of the current timer status
\item[Event event\_] - event to be consumed upon timer expiry 
\end{itemize}
\end{quote}
\end{tt}

The pure virtual functions must be defined by the timer classes deriving
from this abstract base class.

Finally, two private inline functions are defined:
\begin{small}
\begin{verbatim}
        inline void _sched(double delay) {
            (void)Scheduler::instance().schedule(this, &event_, delay);
        }
        inline void _cancel() {
            (void)Scheduler::instance().cancel(&event_);
        }
\end{verbatim}
\end{small}

From this code we can see that timers make use of methods of the 
\code{Scheduler} class.

\subsection{\shdr{Definition of a new timer}{timer-handler.h}{sec:definition}}

To define a new timer, subclass this function and define handle() if needed 
(handle() is not always required):

\begin{small}
\begin{verbatim}

        class MyTimer : public TimerHandler {
        public:
          MyTimer(MyAgentClass *a) : TimerHandler() { a_ = a; }
          virtual double expire(Event *e);
        protected:
          MyAgentClass *a_;
        };

\end{verbatim}
\end{small}

Then define expire:

\begin{small}
\begin{verbatim}

        double
        MyTimer::expire(Event *e)
        {
          // do the work
          // return TIMER_HANDLED;    // => do not reschedule timer
          // return delay;            // => reschedule timer after delay
        }

\end{verbatim}
\end{small}

Note that \code{expire()} can return either the flag TIMER\_HANDLED or a
delay value, depending on the requirements for this timer.

Often \code{MyTimer} will be a friend of \code{MyAgentClass}, or 
\code{expire()} will only call a public function of \code{MyAgentClass}.

Timers are not directly accessible from the OTcl level, although users are
free to establish method bindings if they so desire.

\subsection{\shdr{Example: Tcp retransmission timer}{tcp.cc}{sec:timerexample}}

TCP is an example of an agent which requires timers.  There are three timers
defined in the basic Tahoe TCP agent defined in \code{tcp.cc}:

\begin{small}
\begin{verbatim}
        rtx_timer_;      //  Retransmission timer
        delsnd_timer_;   //  Delays sending of packets by a small random
                             amount of time, to avoid phase effects
        burstsnd_timer_;   // Helps TCP to stagger the transmission of a large
                              window into several smaller bursts
\end{verbatim}
\end{small}

In \code{tcp.h}, three classes are derived from the base class 
\code{TimerHandler}:
\begin{small}
\begin{verbatim}

class RtxTimer : public TimerHandler {
public:
    RtxTimer(TcpAgent *a) : TimerHandler() { a_ = a; }
protected:                   
    virtual void expire(Event *e);
    TcpAgent *a_;
};  
    
class DelSndTimer : public TimerHandler {
public:
    DelSndTimer(TcpAgent *a) : TimerHandler() { a_ = a; }
protected:
    virtual void expire(Event *e);
    TcpAgent *a_;
};      
    
class BurstSndTimer : public TimerHandler {
public: 
    BurstSndTimer(TcpAgent *a) : TimerHandler() { a_ = a; }
protected:
    virtual void expire(Event *e); 
    TcpAgent *a_;
};  

\end{verbatim}
\end{small}

In the constructor for \code{TcpAgent} in \code{tcp.cc}, each of these timers
is initialized with the \code{this} pointer, which is assigned to the pointer
\code{a_}.

\begin{small}
\begin{verbatim}

TcpAgent::TcpAgent() : Agent(PT_TCP), rtt_active_(0), rtt_seq_(-1), 
    ...
    rtx_timer_(this), delsnd_timer_(this), burstsnd_timer_(this)
{
    ...
}

\end{verbatim}
\end{small}

In the following, we will focus only on the retransmission timer.  Various
helper methods may be defined to schedule timer events; \eg,

\begin{small}
\begin{verbatim}

/*
 * Set retransmit timer using current rtt estimate.  By calling resched(),
 * it does not matter whether the timer was already running.
 */
void TcpAgent::set_rtx_timer()
{
    rtx_timer_.resched(rtt_timeout());
}

/*
 * Set new retransmission timer if not all outstanding
 * data has been acked.  Otherwise, if a timer is still
 * outstanding, cancel it.
 */
void TcpAgent::newtimer(Packet* pkt)
{
    hdr_tcp *tcph = (hdr_tcp*)pkt->access(off_tcp_);
    if (t_seqno_ > tcph->seqno())
        set_rtx_timer();
    else if (rtx_timer_.status() == TIMER_PENDING)
        rtx_timer_.cancel();
}

\end{verbatim}
\end{small}

In the above code, the \code{set_rtx_timer()} method reschedules the 
retransmission timer by calling \code{rtx_timer_.resched()}.  Note that if
it is unclear whether or not the timer is already running, calling
\code{resched()} eliminates the need to explicitly cancel the timer.  In
the second function, examples are given of the use of the \code{status()}
and \code{cancel()} methods.

Finally, the \code{expire()} method for class \code{RtxTimer} must be 
defined.  In this case, \code{expire()} calls the \code{timeout()} method
for \code{TcpAgent}.  This is possible because \code{timeout()} is a 
public member function; if it were not, then \code{RtxTimer} would have
had to have been declared a friend class of \code{TcpAgent}.

\begin{small}
\begin{verbatim}

void TcpAgent::timeout(int tno)
{                     
    /* retransmit timer */
    if (tno == TCP_TIMER_RTX) {
        if (highest_ack_ == maxseq_ && !slow_start_restart_) {
            /*
             * TCP option:
             * If no outstanding data, then don't do anything.
             */
            return;  
        };
        recover_ = maxseq_;
        recover_cause_ = 2;
        closecwnd(0);
        reset_rtx_timer(0,1);
        send_much(0, TCP_REASON_TIMEOUT, maxburst_); 
    }       
    else {  
        /*  
         * delayed-send timer, with random overhead
         * to avoid phase effects  
         */     
        send_much(1, TCP_REASON_TIMEOUT, maxburst_);
    }           
}           
            
void RtxTimer::expire(Event *e) {
    a_->timeout(TCP_TIMER_RTX);
}

\end{verbatim}
\end{small}

The various TCP agents contain additional examples of timers.

\section{\shdr{OTcl Timer class}{timer.tcl}{sec:otcltimer}}

A simple timer class is defined in \code{tcl/ex/timer.tcl}.  Subclasses of
\code{Timer} can be defined as needed.  Unlike the C++ timer API, where a 
\code{sched()} aborts if the timer is already set, \code{sched()} and
\code{resched()} are the same; i.e., no state is kept for the OTcl timers.
The following methods are defined in the \code{Timer} base class:
\begin{program}

    $self sched $delay   \; causes "$self timeout" to be called $delay seconds in the future;
    $self resched $delay \; same as "$self sched $delay" ;
    $self cancel         \; cancels any pending scheduled callback;
    $self destroy        \; same as "$self cancel";
    $self expire         \; calls "$self timeout" immediately;

\end{program}

\endinput

%
% personal commentary:
%        handlers and how they are used are confusing
%        Connector::send is needed, but so is just send()... confusing
%        default handler in Connector::recv is confusing
%        this is a DRAFT DRAFT DRAFT
%        - KFALL
%
\chapter{\shdr{Agents}{agent.h}{sec:agents}}

Agents represent endpoints where network-layer
packets are constructed or consumed, and provide
some functions helpful in developing transport-layer and other
protocols.
Generally, a user wishing to create a new
source or sink for network-layer packets
will create a class derived from {\tt Agent}.
The class \code{Agent} has an implementation partly in
OTcl and partly in C++.
The C++ implementation is contained in \code{agent.cc} and
\code{agent.h}, and the OTcl support is in
\code{tcl/lib/ns-agent.tcl}.

\section{\shdr{Agent state}{agent.h}{sec:agentstate}}

The C++ class \code{Agent} includes enough internal state
to assign various fields to a simulated packet before
it is sent.
This state includes the following:

\begin{tabularx}{\linewidth}{rX}
\code{addr\_} & node address of myself (source address in packets) \\
\code{dst\_} & where I am sending packets to \\
\code{size\_} & packet size in bytes (placed into the common packet header) \\
\code{type\_} & type of packet (in the common header, see packet.h) \\
\code{fid\_} & the IP flow identifier (formerly {\em class} in ns-1) \\
\code{prio\_} & the IP priority field \\
\code{flags\_} & packet flags (similar to ns-1) \\
\code{defttl\_} & default IP ttl value \\
\end{tabularx}

These variables may be modified by any class derived from \code{Agent},
although not all of them may be needed by any particular agent.

\section{\shdr{Agent methods}{agent.h}{sec:agentmethods}}

The \code{Agent} class supports packet generation and reception.
The following member functions are implemented by the C++ Agent class, and are
generally {\em not} over-ridden by derived classes:

\begin{tabularx}{\linewidth}{rX}
\fcn[]{Packet* allocpkt} & allocate new packet and assign its fields \\
\fcn[int]{Packet* allocpkt} & allocate new packet with a data payload of n bytes and assign its fields \\
\end{tabularx}

The following member functions are also defined by the \code{Agent} class,
but {\em are} intended to be over-ridden by classes deriving from Agent:

\begin{tabularx}{\linewidth}{rX}
  \fcn[timeout number]{void timeout} & subclass-specific time out method \\
  \fcn[Packet*, Handler*]{void recv} & receiving agent main receive path \\
\end{tabularx}

The \code{allocpkt} function is used by derived classes to create
packets to send.
The function fills in the following fields in the common packet
header (see \ref{sec:pformat}): {\tt uid, ptype, size}, and the
following fields in the IP header: {\tt src, dst, flowid, prio, ttl}.
It also zero-fills in the following fields of the Flags header:
{\tt ecn, pri, usr1, usr2}.
Any packet header information not included in these lists must
be must be handled in the classes derived from \code{Agent}.

The \code{recv} function is the main entrypoint for an
Agent which receives packets, and
is invoked by upstream nodes when sending a packet.
In most cases, Agents make no use of the second argument (the handler
defined by upstream nodes).

\section{\shdr{Protocol Agents}{cbr.h}{sec:protoagents}}

There are several agents supported in the simulator.
These are their names in OTcl:

\begin{tabularx}{\linewidth}{rX}
  TCP & a ``Tahoe'' TCP sender (cwnd = 1 on any loss)	\\
  TCP/Reno & a ``Reno'' TCP sender  (with fast recovery)	\\
  TCP/NewReno & a modified Reno TCP sender (changes fast recovery)	\\
  TCP/Sack1 & a SACK TCP sender	\\
  TCP/Fack & a ``forward'' SACK sender TCP 	\\
  TCP/FullTcp & a more full-functioned TCP with 2-way traffic	\\
  TCP/Vegas & a ``Vegas'' TCP sender	\\
  TCP/Vegas/RBP & a Vegas TCP with ``rate based pacing''	\\
  TCP/Vegas/RBP & a Reno TCP with ``rate based pacing''	\\
  TCP/Asym & an experimental Tahoe TCP for asymmetric links	\\
  TCP/Reno/Asym & an experimental Reno TCP for asymmetric links	\\
  TCP/Newreno/Asym & an experimental NewReno TCP for asymmetric links	\\
  TCPSink & a Reno or Tahoe TCP receiver (not used for FullTcp)	\\
  TCPSink/DelAck & a TCP delayed-ACK receiver	\\
  TCPSink/Asym & an experimental  TCP sink for asymmetric links	\\
  TCPSink/Sack1 & a SACK TCP receiver	\\
  TCPSink/Sack1/DelAck & a delayed-ACK SACK TCP receiver	\\
	\\
  CBR & connectioness protocol with sequence numbers	\\
  CBR/RTP & an RTP sender and receiver	\\
  CBR/UDP & UDP with sequence numbers and traffic sources	\\
  RTCP & an RTCP sender and receiver	\\
	\\
  LossMonitor & a packet sink which checks for losses	\\
	\\
  IVS/Source & an IVS source	\\
  IVS/Receiver & an IVS receiver	\\
	\\
  CtrMcast/Encap & a ``centralised multicast'' encapsulator	\\
  CtrMcast/Decap & a ``centralised multicast'' de-encapsulator	\\
  Message & a protocol to carry textual messages	\\
  Message/Prune & processes multicast routing prune messages	\\
	\\
  SRM & an SRM agent with non-adaptive timers	\\
  SRM/Adaptive & an SRM agent with adaptive timers	\\
	\\
  Tap & interfaces the simulator to a live network	\\
	\\
  Null & a degenerate agent which discards packets	\\
	\\
  rtProto/DV & distance-vector routing protocol agent	\\
\end{tabularx}

Agents are used in the implementation of protocols at various layers.
Thus, for some transport protocols (e.g.~UDP) the distribution
of packet sizes and/or inter-departure times
may be dictated by some separate
object representing the demands of an application.
For agents used in the implementation of lower-layer protocols
(e.g. routing agents), size and departure timing is generally dictated
by the agent's own processing of protocol messages.

\section{\shdr{OTcl Linkage}{../ns/ns-default.tcl}{sec:agentotcl}}

Agents may be created within OTcl and an agent's internal
state can be modified by use of Tcl's \code{set} function and
any Tcl functions an Agent (or its base classes) implements.
Note that some of an Agent's internal state may exist
only within OTcl, and is thus is not directly accessible from C++.

\subsection{\shdr{creating and manipulating agents}{../ns/ns-lib.tcl}{sec:agentcreateotcl}}

The following example illustrates the creation and modification
of an Agent in OTcl:
\begin{program}
        set newtcp [new Agent/TCP] \; create new object (and C++ shadow object);
        $newtcp set window_ 20 \; sets the tcp agent's window to 20;
        $newtcp target $dest \; target is implemented in Connector class;
        $newtcp set portID_ 1 \; exists only in OTcl, not in C++;
\end{program}

\subsection{\shdr{default values}{../ns/ns-default.tcl}{sec:agentdefaults}}

Default values for member variables, those visible in OTcl only and those
linked between OTcl and C++ with \code{bind} are initialized
in the \nsf{tcl/lib/ns-default.tcl} file.  For example,
\code{Agent} and \code{CBR_Agent}
are initialized as follows:
\begin{program}
        Agent set fid_ 0
        Agent set prio_ 0
        Agent set addr_ 0
        Agent set dst_ 0
        Agent set flags_ 0

        Agent/CBR set interval_ 3.75ms
        Agent/CBR set random_ 0
        Agent/CBR set packetSize_ 210
	\ldots
\end{program}

Generally these initializations are placed in the OTcl namespace
before any objects of these types are created.
Thus, when an \code{Agent} or \code{Agent/CBR} object
is created, the calls to \code{bind}
in the objects' constructors will causes the corresponding member variables
to be set to these specified defaults.

\subsection{\shdr{OTcl methods}{../ns/ns-agent.tcl}{sec:agentmethodsotcl}}

The instance procedures defined for the OTcl \code{Agent} class are
currently found in \nsf{tcl/lib/ns-agent.tcl}.
They are as follows:
\begin{tabularx}{\linewidth}{rX}
\code{port} & the agent's port identifier \\
\code{dst-port} & the destination's port identifier \\
\code{attach-source \tup{stype}} & create and attach a Source object to an agent \\
\end{tabularx}

\section{\shdr{Examples: Tcp, TCP Sink Agents}{tcp-simple.cc}{sec:agentexample}}

The class \code{TCP} represents a simplified TCP sender.
It sends data to a \code{TCPSink} agent and processes its acknowledgments.
It has a separate object associated with it which represents
an application's demand.
By looking at the \code{TCPAgent} and \code{TCPSinkAgent} classes
we may see how relatively complex agents are constructed.
An example from the Tahoe TCP agent \code{TCPAgent} is also given
to illustrate the use of timers.

\subsection{\shdr{creating the agent}{tcp-simple.cc}{sec:createtcpsimple}}

The following OTcl code fragment creates a \code{TCP} agent
and sets it up:
\begin{small}
\begin{verbatim}
        set tcp [new Agent/TCP] ; # create sender agent
        $tcp set fid_ 2 ; # set IP-layer flow ID
        set sink [new Agent/TCPSink] ; # create receiver agent
        $ns attach-agent $n0 $tcp ; # put sender on node $n0
        $ns attach-agent $n3 $sink ; # put receiver on node $n3
        $ns connect $tcp $sink ; # establish TCP connection
        set ftp [new Source/FTP] ; # create an FTP source "application"
        $ftp set agent_ $tcp ; # associate FTP with the TCP sender
        $ns at 1.2 "$ftp start" ; # arrange for FTP to start at time 1.2 secs
\end{verbatim}
\end{small}
The OTcl instruction \code{new Agent/TCP} results in the
creation of a C++ \code{TcpAgent} class object.
It's constructor performs first invokes the constructor of the
\code{Agent} base class and then performs its own bindings.
These two constructors appear as follows:
\begin{small}
\begin{verbatim}

The TcpSimpleAgent constructor (tcp.cc):

        TcpAgent::TcpAgent() : Agent(PT_TCP), rtt_active_(0), rtt_seq_(-1),
			rtx_timer_(this), delsnd_timer_(this)
        {
                bind("window_", &wnd_);
                bind("windowInit_", &wnd_init_);
                bind("windowOption_", &wnd_option_);
                bind("windowConstant_", &wnd_const_);
                ...
                bind("off_ip_", &off_ip_);
                bind("off_tcp_", &off_tcp_);
                ...

The Agent constructor (agent.cc):

        Agent::Agent(int pkttype) : 
                addr_(-1), dst_(-1), size_(0), type_(pkttype), fid_(-1),
                prio_(-1), flags_(0)
        {
                memset(pending_, 0, sizeof(pending_)); // timers
                // this is really an IP agent, so set up
                // for generating the appropriate IP fields...
                bind("addr_", (int*)&addr_);
                bind("dst_", (int*)&dst_);
                bind("fid_", (int*)&fid_);
                bind("prio_", (int*)&prio_);
                bind("flags_", (int*)&flags_);
                ...
\end{verbatim}
\end{small}
These code fragments illustrate the common case where an agent's
constructor passes a packet type identifier to the \code{Agent}
constructor.
The values for the various packet types are used by the packet tracing
facility (see \ref{sec:trace}) and are defined in \code{trace.h}.
The variables which are bound in the \code{TcpAgent} constructor
are ordinary instance/member variables for the class
with the exception of the special integer values \code{off_tcp_}
and \code{off_ip_}.
These are needed in order to access a TCP header and IP header,
respectively.
For details on how to access packet headers, see \ref{sec:ppackethdr}.

Note that the \code{TcpAgent} constructor contains initializations for
two timers, \code{rtx_timer_} and \code{delsnd_timer_}.  \code{TimerHandler} 
objects are initialized by providing a pointer (the \code{this} pointer) to
the relevant agent.

\subsection{\shdr{starting the agent}{tcp.cc}{sec:starttcp}}

The \code{TcpAgent} agent is started in the example when its
FTP source receives the \code{start} directive at time 1.2.
The \code{start} operation is an instance procedure defined on the
\code{Source/FTP} class (and described in \ref{sec:sources}).
It is defined in \code{tcl/lib/ns-source.tcl} as follows:
\begin{small}
\begin{verbatim}
        Source/FTP instproc start {} {
                $self instvar agent_ maxpkts_
                $agent_ advance $maxpkts_
        }
\end{verbatim}
\end{small}
In this case, \code{agent_} refers to our simple TCP agent and
\code{maxpkts_} defaults to a large value (2147483647).

The call to \code{advance} eventually results in the simple TCP sender
generating some packets.  The following function \code{output}
performs this:
\begin{small}
\begin{verbatim}
        void TcpAgent::output(int seqno, int reason)
        {
                Packet* p = allocpkt();
                hdr_tcp *tcph = (hdr_tcp*)p->access(off_tcp_);
                double now = Scheduler::instance().clock();
                tcph->seqno() = seqno;
                tcph->ts() = now;
                tcph->reason() = reason;
                Connector::send(p, 0);
                ...
                if (!(rtx_timer_.status() == TIMER_PENDING))
                        /* No timer pending.  Schedule one. */
                        set_rtx_timer();
        }
\end{verbatim}
\end{small}
Here we see an illustration of the use of the \code{Agent::allocpkt()}
function.
This output routine first allocates a new packet
(with its common and IP headers already filled in), but then must fill
in the appropriate TCP-layer header fields.
To find the TCP header in a packet (assuming it has been enabled, see
\ref{sec:packethdrmgr}) the \code{off_tcp_} must be properly initialized,
as illustrated in the constructor.
The packet \code{access()} function returns a pointer to the TCP
header, its sequence number and time stamp fields are filled in,
and the \code{send} function of the \code{Connector} class is called
to send the packet downstream one hop.
Note that the C++ \code{::} scoping operator is used here to avoid
calling \code{TcpSimpleAgent::send()} (which is also defined).
The check for a pending timer uses the timer method \code{status()} which
is defined in the \code{TimerHandler} base class.
It is used here to set a retransmission timer if one is not already set
(a TCP sender only sets one timer per window of packets on each connection).

\subsection{\shdr{processing input at receiver}{tcp-sink.cc}{sec:tcpsink}}

Many of the TCP agents can be used with the \code{TCPSink} class
as the peer.
This class defines the \code{recv} and \code{ack} functions as follows:
\begin{small}
\begin{verbatim}
        void TcpSink::recv(Packet* pkt, Handler*)
        {
                hdr_tcp *th = (hdr_tcp*)pkt->access(off_tcp_);
                acker_->update(th->seqno());
                ack(pkt);
                Packet::free(pkt);
        }
        void TcpSink::ack(Packet* opkt)
        {
                Packet* npkt = allocpkt();
        
                hdr_tcp *otcp = (hdr_tcp*)opkt->access(off_tcp_);
                hdr_tcp *ntcp = (hdr_tcp*)npkt->access(off_tcp_);
                ntcp->seqno() = acker_->Seqno();
                ntcp->ts() = otcp->ts();
        
                hdr_ip* oip = (hdr_ip*)opkt->access(off_ip_);
                hdr_ip* nip = (hdr_ip*)npkt->access(off_ip_);
                nip->flowid() = oip->flowid();
        
                hdr_flags* of = (hdr_flags*)opkt->access(off_flags_);
                hdr_flags* nf = (hdr_flags*)npkt->access(off_flags_);
                nf->ecn_ = of->ecn_;
        
                acker_->append_ack((hdr_cmn*)npkt->access(off_cmn_),
                                   ntcp, otcp->seqno());
                send(npkt, 0);
        }
\end{verbatim}
\end{small}
The \code{recv} function overrides the \code{Agent::recv} function
(which merely discards the received packet).
It updates some internal state with the sequence number of the
received packet (and therefore requires the \code{off_tcp_} variable
to be properly initialized.
It then generates an acknowledgment for the received packet.
The \code{ack} function makes liberal use of access to packet header
fields including separate accesses to the TCP header, IP header,
Flags header, and common header.
The call to \code{send} invokes the \code{Connector::send} function.

\subsection{\shdr{processing responses at the sender}{tcp-simple.cc}{sec:tcpsimpleack}}

Once the simple TCP's peer receives data and generates an ACK, the
sender must (usually) process the ACK.
In the \code{TcpAgent} agent, this is done as follows:
\begin{small}
\begin{verbatim}
        /*
         * main reception path - should only see acks, otherwise the
         * network connections are misconfigured
         */
        void TcpAgent::recv(Packet *pkt, Handler*)
        {
                hdr_tcp *tcph = (hdr_tcp*)pkt->access(off_tcp_);
                hdr_ip* iph = (hdr_ip*)pkt->access(off_ip_);
                ...
                if (((hdr_flags*)pkt->access(off_flags_))->ecn_)
                        quench(1);
                if (tcph->seqno() > last_ack_) {
                        newack(pkt);
                        opencwnd();
                } else if (tcph->seqno() == last_ack_) {
                        if (++dupacks_ == NUMDUPACKS) {
                                ...
                        }
                }
                Packet::free(pkt);
                send(0, 0, maxburst_);
       }
\end{verbatim}
\end{small}
This routine is invoked when an ACK arrives at the sender.
In this case, once the information in the ACK is processed (by \code{newack})
the packet is no longer needed and is returned to the packet memory
allocator.
In addition, the receipt of the ACK indicates the possibility of sending
additional data, so the \code{TcpSimpleAgent::send} routine is
invoked which attempts to send more data if the TCP window allows.

\subsection{\shdr{implementing timers}{tcp.cc}{sec:tcptimer}}

As described in the following section (Section \ref{sec:timers}), specific
timer classes must be derived from an abstract base class 
\code{TimerHandler} defined in \code{timer-handler.h}.  Instances of these
subclasses can then be used as various agent timers.
An agent may wish to override the \code{Agent::timeout} function (which does 
nothing).  In the case of the Tahoe TCP agent, two timers are used: a delayed
send timer \code{delsnd_timer_} and a retransmission timer \code{rtx_timer_}.
Section \ref{sec:timerexample} describes the retransmission timer in TCP
as an example of timer usage.  

\section{\shdr{Creating a New Agent}{agent.cc}{sec:createagent}}

To create a new agent, one has to do the following:
\begin{enumerate}
        \item decide its inheritance structure (section \ref{sec:pingexample})
        \item create the class, \code{recv}, and \code{timeout} functions (section \ref{})
		\item define any necessary timer classes,
        \item define OTcl linkage functions (section \ref{})
        \item write the necessary OTcl code to access your agent (section \ref{})
\end{enumerate}

The action required to create and agent can be illustrated
by means of a very simple example.
Suppose we wish to construct an agent which performs
the ICMP ECHO REQUEST/REPLY (or ``ping'') operations.

\subsection{\shdr{Example: A ``ping'' requestor (Inheritance Structure)}{agent.h}{sec:pingexample}}

Deciding on the inheritance structure is a matter of personal choice, but is
likely to be related to the layer at which the agent will operate
and its assumptions on lower layer functionality.
For protocol endpoints wishing to use a connectionless
datagram-oriented transport layer (like UDP) with sequencing,
the \code{CBR_Agent} base class is likely to be appropriate.
For protocols wishing to use a connection-oriented stream transport
(like TCP), the various TCP Agents could be used.
Finally, if a new transport or ``sub-transport'' protocol
is to be developed, using \code{Agent}
as the base class would likely be the best choice.
In our example, we'll use Agent as the base class, given that
we are constructing an agent logically belonging to the IP layer
(or just above it).

We may use the following class definitions:
\begin{small}
\begin{verbatim}
	
        class ECHO_Timer;
 
        class ECHO_Agent : public Agent {
         public:
                ECHO_Agent();
                int command(int argc, const char*const* argv);
         protected:
                void timeout(int);
                void sendit();
                double interval_;
                ECHO_Timer echo_timer_;
        };

        class ECHO_Timer : public TimerHandler {
        public:
                ECHO_Timer(ECHO_Agent *a) : TimerHandler() {a_ = a; }
        protected:
                virtual void expire(Event *e);
                ECHO_Agent *a_;
        }; 
\end{verbatim}
\end{small}

The implementation of the member functions will look similar to
the \code{CBR_Agent} described above.

\subsection{The \code{recv} and \code{timeout} functions}

The \code{recv} function is not defined here, as this agent
represents a request function and will generally not be receiving
events or packets\footnote{This is perhaps unrealistically simple.
An ICMP ECHO REQUEST agent would likely wish to process
ECHO REPLY messages.}
By not defining the \code{recv} function, the base class version
of \code{recv} (i.e. \code{Connector::recv}) is used.
The \code{timeout} function is used to periodically send request packets.
The following \code{timeout} function is used, along with a helper
function \code{sendit}:

\begin{small}
\begin{verbatim}
        void ECHO_Agent::timeout(int)
        {
                sendit();
                echo_timer_.resched(interval_);
        }
        void ECHO_Agent::sendit()
        {
                Packet* p = allocpkt();
                ECHOHeader *eh = ECHOHeader::access(p->bits());
                eh->timestamp() = Scheduler::instance().clock();
                send(p, 0);     // Connector::send()
        }

        void ECHO_Timer::expire(Event *e) {
                a_->timeout(0);
        }
\end{verbatim}
\end{small}

The \code{timeout} function simply arranges for \code{sendit} to be
executed every \code{interval_} seconds.
The \code{sendit} function create a new packet with most of its
header fields already set up by the \code{allocpkt} function.
The packet is only lacking the current time stamp. 
The call to \code{access} provides for a structured interface to the
packet header fields, and is used to set the timestamp field.
Note that this agent uses its own special header (``ECHOHeader'').
The creation and use of packet headers is described in (see \ref{pheader});
To send the packet to the next downstream node, \code{Connector::send()}
is invoked without a handler.

\subsection{Linking the ``ping'' agent with OTcl}

There are three items we must handle to properly link our agent
with Otcl.
First we need to establish a mapping between the OTcl name
for our class and the actual object created when an
instantiation of the class is requested in OTcl.
This is done as follows:
\begin{small}
\begin{verbatim}
        static class ECHOClass : public TclClass {
        public:
                ECHOClass() : TclClass("Agent/ECHO") {}
                TclObject* create(int argc, const char*const* argv) {
                        return (new ECHO_Agent());
                }
        } class_echo;
\end{verbatim}
\end{small}

Here, a {\em static} object ``class\_echo'' is created. It's constructor
(executed immediately when the simulator is executed) places the class name
``Agent/ECHO'' into the OTcl name space.  The mixing of case is
by convention, although the ``/'' character is interpreted by the
\code{Tcl} library as a hierarchy delimiter and is required.
The definition of the \code{create} function specifies how a C++
shadow object should be created when
the OTcl interpreter is instructed to create an
object of class ``Agent/ECHO''.  In this case, a dynamically-allocated
object is returned.  This is the normal way new C++ shadow objects
are created.
Note that arguments could have been passed to our constructor
via OTcl through the conventional \code{argc/argv} pairs of the
\code{create} function, although this is rare.

Once we have the object creation set up, we will want to link
C++ member variables with corresponding variables in the OTcl
name space, so that accesses to OTcl variables are actually
backed by member variables in C++.
Assume we would like OTcl to be able to adjust the sending
interval and the packet size.
This is accomplished in the class's constructor:
\begin{small}
\begin{verbatim}

        ECHO_Agent::ECHO_Agent() : Agent(PT_ECHO)
        {
                bind_time("interval_", &interval_);
                bind("packetSize_", &size_);
        }
\end{verbatim}
\end{small}

Here, the C++ variables \code{interval_} and \code{size_} are
linked to the OTcl instance variables \code{interval_} and
\code{packetSize_}, respectively.
Any read or modify operation to the Otcl variables will result
in a corresponding access to the underlying C++ variables.
See \ref{c++linkage} for more detail on how the various \code{bind}
functions work.
The defined constant \code{PT_ECHO} is passed to the \code{Agent}
constuctor so that the \code{Agent::allocpkt} function may set
the packet type field used by the trace support
(see section \ref{traceptype}).  In this case, \code{PT_ECHO} represents
a new packet type and must be defined in \code{trace.h}.
(see section \ref{sec:traceformat}).

Once object creation and variable binding is set up, we may
want to create methods implemented in C++ but which can
be invoked from OTcl.
These are often control functions that initiate, terminate or
modify behavior.
In our present example, we may wish to be able to start the
ping query agent from OTcl using a ``start'' directive.
This may be implemented as follows (very similar to the CBR Agent):
\begin{small}
\begin{verbatim}

        int ECHO_Agent::command(int argc, const char*const* argv)
        {
                if (argc == 2) {
                        if (strcmp(argv[1], "start") == 0) {
                                timeout(0);
                                return (TCL_OK);
                        }
                }
                return (Agent::command(argc, argv));
        }
\end{verbatim}
\end{small}
Here, the \code{start} method available to OTcl simply calls
the C++ member function \code{timeout} which initiates the
first packet generation and schedules the next.
Note this class is so simple it does not even include a
way to be stopped.

\section{Using the agent through OTcl}

The agent we have created will have to be instantiated and attached
to a node.
Note that a node and simulator object is assumed to have
already been
created (Section \ref{tcllink} describes how this is done).
The following OTcl code performs these functions:
\begin{small}
\begin{verbatim}
        set echoagent [new Agent/ECHO]
        $simulator attach-agent $node $echoagent
\end{verbatim}
\end{small}

To set the interval and packet size, and start packet generation,
the following OTcl code is executed:
\begin{small}
\begin{verbatim}

        $echoagent set dst_ $dest
        $echoagent set fid_ 0
        $echoagent set prio_ 0
        $echoagent set flags_ 0
        $echoagent set interval_ 1.5
        $echoagent set packetSize_ 1024
        $echoagent start
\end{verbatim}
\end{small}

This will cause our agent to generate one 1024-byte packet destined for
node \code{$dest} every 1.5 seconds.

\endinput


\part{Support}
%
% personal commentary:
%        DRAFT DRAFT DRAFT
%        - KFALL
%
\section{\shdr{Mathematical Support}{random.h}{sec:math}}

The simulator includes a small collection of mathematical
functions used to implement random variate generation and integration.
This area of the simulator is currently undergoing some
changes.

\subsection{\shdr{Random Number Generation}{rng.h}{sec:random}}

The \code{RNG} class contains an implementation of the minimal standard
multiplicative linear congruential generator of Park, S.K. and
Miller, K.W., "Random Number Generators: Good Ones are Hard to Find,"
CACM 31:10, Oct. 88, pp. 1192-1201.

Multiple instances of the RNG class can be created to allow a
simulation to draw random numbers from independent random number
streams.  For instance, a user who wants to generate the same traffic
(based on some random process) in 2 different simulation experiments
that compare different dropping algorithms that are themselves based
on random processes may choose to base the traffic generation on one
random number stream and the dropping algorithms on another stream.
However, when using multiple RNG objects in a simulation care should
be taken to insure that they are seeded in such a way as to guarantee
that they produce independent streams of random numbers.  Features
will be added in the future to help seed multiple random number
generators properly.

Most users will be satisfied with a single instance of the RNG.
Hence, a default RNG, created at simulator initialization time, is
provided.

\paragraph{C++ Support}
This random number generator is
implemented by the RNG class, defined in rng.h:
\begin{small}
\begin{verbatim}
class RNG : public TclObject {
enum RNGSources { RAW_SEED_SOURCE, PREDEF_SEED_SOURCE, HEURISTIC_SEED_SOURCE };
	...
        // These are primitive but maybe useful.
        inline int uniform_positive_int() {  // range [0, MAXINT]
                return (int)(stream_.next());
        }
        inline double uniform_double() { // range [0.0, 1.0)
                return stream_.next_double();
        }

        inline int uniform(int k)
                { return (uniform_positive_int() % (unsigned)k); }
        inline double uniform(double r) 
                { return (r * uniform_double());}
        inline double uniform(double a, double b)
                { return (a + uniform(b - a)); }
        inline double exponential()
                { return (-log(uniform_double())); }
        inline double exponential(double r)
                { return (r * exponential());}
        inline double pareto(double scale, double shape)
                { return (scale * (1.0/pow(uniform_double(), 1.0/shape)));}
	...
};


The \code{uniform\_positive\_int} method generates random integers in the
range $[0,2^{31}-1]$.
In particular,
Additional member functions provide the following random variate
generation:
\begin{itemize}
        \item {\tt uniform(double r)} - generate floating-point number uniformly distributed on $[0,r]$
        \item {\tt uniform(double a, double b)} - generate floating-point number uniformly distributed on $[a,b]$
        \item {\tt exponential()} - generate floating-point number exponentially distributed (with parameter 1) on $[0, \infty)$
        \item {\tt integer(int k)} - generate integer uniformly distributed on $[0, (k-1)]$
\end{itemize}
The \code{Random} class can be used to construct randomized algorithms,
as in this code fragment from RED:
\begin{small}
\begin{verbatim}
        ...
        // drop probability is computed, pick random number and act
        double u = rng_->uniform_double();
        if (u <= edv_.v_prob) {
                edv_.count = 0;
                if (edp_.setbit) 
                        iph->flags() |= IP_ECN; // ip ecn bit
                else
                        return (1);
        }
        ...
\end{verbatim}
\end{small}

The set of random numbers produced by an RNG object depends on its
initial seed.  The initial seed can be set with the {\tt set\_seed} method
which takes 2 parameters, where the second
defaults to 1.  The first parameter, an enumerated type,
determines how
the seed is generated and the second is a seed value.  There are 3
values of this enumerated type used determine the initial seed value:

\begin{itemize}

\item[RAW\_SEED\_SOURCE]:  the value contained in the second parameter is used
as the seed.

\item [PREDEF\_SEED\_SOURCE]:  the value contained in the second parameter is
used as the index into an array of predefined seed values.  These
values have been precomputed and provide independent streams of random
numbers.  There are currently 64 such seeds.

\item [HEURISTIC\_SEED\_SOURCE]: use a heuristic to generate the seed.
The second argument is ignored if provided.
\end{itemize}

The heuristic seeding method
attempts to pick some sort of non-recurring seed values (resulting
in different random numbers from run to run).  The heuristic implementation
is contained in the file \code{rng.cc}.

\paragraph{OTcl support}
The RNG class can be accessed from OTcl.  For example, a new RNG is
created and seeded with:

\begin{program}
set rng [new RNG]
$rng seed 0 \; seeds the RNG heuristically;
$rng seed n \; seeds the RNG with value n;
$rng next-random \;  return the next random number;
$rng uniform a b \; return a number uniformly distributed on [a, b];
$rng integer k \; return an integer uniformly distributed on [0, (k-1)];
$rng exponential \; return a number from an exponential distribution with average 1.;
\end{program}

\subsection{\shdr{Random Variables}{ranvar.h}{sec:ranvar}}

The RandomVariable class provides a thin layer of functionality on top
of the base random number generator that implements random variates.
It is defined in {\tt ranvar.h}:

\subsection{\shdr{Integrals}{integrator.h}{sec:integral}}

To support the approximation of (continuous) integration by (discrete)
sums, the \code{Integrator} class is defined in \code{integrator.h}:
\begin{small}
\begin{verbatim}
From integrator.h:
        class Integrator : public TclObject {
        public:
                Integrator();
                void set(double x, double y);
                void newPoint(double x, double y);
                int command(int argc, const char*const* argv);
        protected:
                double lastx_;
                double lasty_;
                double sum_;
        };
From integrator.cc:
        Integrator::Integrator() : lastx_(0.), lasty_(0.), sum_(0.)
        {
                bind("lastx_", &lastx_);
                bind("lasty_", &lasty_);
                bind("sum_", &sum_);
        }

        void Integrator::set(double x, double y)
        {
                lastx_ = x;
                lasty_ = y;
                sum_ = 0.;
        }

        void Integrator::newPoint(double x, double y)
        {
                sum_ += (x - lastx_) * lasty_;
                lastx_ = x;
                lasty_ = y;
        }

        int Integrator::command(int argc, const char*const* argv)
        {
                if (argc == 4) {
                        if (strcmp(argv[1], "newpoint") == 0) {
                                double x = atof(argv[2]);
                                double y = atof(argv[3]);
                                newPoint(x, y);
                                return (TCL_OK);
                        }
                }
                return (TclObject::command(argc, argv));
        }
\end{verbatim}
\end{small}
This class provides a base class used by other classes such
as \code{QueueMonitor} that keep running sums.
Each new element of the running sum is added by
the \code{newPoint(x,y)} function.
After the $k$th execution of \code{newPoint}, the running sum
is equal to $\sum_{i=1}^{k}y_{i-1}(x_i - x_{i-1})$ where
$x_0 = y_0 = 0$ unless \code{lastx\_}, \code{lasty\_}, or \code{sum\_}
are reset via OTcl.
Note that a new point in the sum can be added either by the
C++ member \code{newPoint} or the OTcl member \code{newpoint}.
The use of integrals to compute certain types of averages
(e.g. mean queue lengths) is given in Jain(1991, pp. 429-430).

\include{tracing}

\part{Routing}
\chapter{Unicast Routing}
\label{chap:unicast}

This section describes the structure of unicast routing in \ns.
We begin by describing
\href{the interface to the user}{Section}{sec:API},
through methods in the \clsref{Simulator}{../ns-2/ns-lib.tcl}
and the \clsref{RouteLogic}{../ns-2/ns-lib.tcl}.
We then describe
\href{configuration mechanisms for specialised routing}{%
        Section}{sec:uni:specroute}
such as asymetric routing, or equal cost multipath routing
The next section describes the
\href{the configuration mechanisms for individual routing strategies
and protocols}{Section}{sec:uni:protconfig}.
We conclude with a comprehensive look at 
\href{the internal architecture}{Section}{sec:rtg-internals}
of routing in \ns.

The procedures and functions described in this chapter can be found in
\nsf{tcl/lib/ns-route.tcl}, \nsf{tcl/rtglib/route-proto.tcl}, 
\nsf{tcl/mcast/McastProto.tcl}, and \nsf{rtProtoDV.\{cc, h\}}.

\section{The Interface to the Simulation Operator (The API)}
\label{sec:API}

The user level simulation script requires one command:
to specify the unicast routing strategy or protocols for the simulation.
A routing strategy is a general mechanism by which \ns\
will compute routes for the simulation.
There are three routing strategies in \ns:
Static, Session, and Dynamic.
Conversely, a routing protocol is a realisation of a specific algorithm.
Currently, Static and Session routing use
the
\fcnref{Dijkstra's all-pairs SPF algorithm \cite{}}{../ns-2/route.cc}{%
        RouteLogic::compute\_routes};
one type of dynamic routing strategy is currently implemented: the
\fcnref{Distributed Bellman-Ford algorithm \cite{}}{../ns-2/route-proto.tcl}{%
        Agent/rtProto/DV::compute\_routes}.
In \ns, we blur the distinction between strategy and protocol for
static and session routing, considering them simply as protocols%
\footnote{The consideration is that static and session routing
  strategies/protocols are implemented as agents derived from
  the \clsref{Agent/rtProto},
  similar to how the different dynamic routing protocols are implemented;
  hence the blurred distinctions.}.

\fcnref{\proc[]{rtproto}}{../ns-2/route-proto.tcl}{Simulator::rtproto}
is the instance procedure in the \clsref{Simulator}{../ns-2/ns-lib.tcl}
that specifies the unicast routing protocol to be used in the simulation.
It takes multiple arguments, the first of which is mandatory;
this first argument identifies the routing protocol to be used.
Subsequent arguments specify the nodes
that will run the instance of this protocol.
The default is to run the same routing protocol
on all the nodes in the topology.
As an example, the following commands illustrate the use of the
\proc[]{rtproto} command.
\begin{program}
        $ns rtproto Static            \; Enable static route strategy for the simulation;
        $ns rtproto Session           \; Enable session routing for this simulation;
        $ns rtproto DV $n1 $n2 $n3    \; Run DV agents on nodes $n1, $n2, and $n3;
\end{program}
If a simulation script does not specify any \proc[]{rtproto} command,
then \ns\ will run Static routing on all the nodes in the topology.

Multiple \proc[]{rtproto} lines for the same or different routing 
procotols can occur in a simulation script.
However, a simulation cannot use both
centralised routing mechanisms such as static or session routing and 
detailed dynamic routing protocols such as DV.

In dynamic routing, each node can be running more than one routing protocol.
In such situations, more than one routing protocol can have a route to the
same destination.
Therefore, each protocol affixes a preference value to each of its routes.
These values are non-negative integers in the range 0\ldots255.
The lower the value, the more preferred the route.
When multiple routing protocol agents have a route to the same destination,
the most preferred route is chosen and
installed in the node's forwarding tables.
If more than one agent has the most preferred routes,
the ones with the lowest metric is chosen.
We call the least cost route from the most preferred protocol the
``candidate'' route.
If there are multiple candidate routes from the same or different protocols,
then, currently,
one of the agent's routes is randomly chosen\footnote{%
This really is undesirable, and may be fixed at some point.
The fix will probably be to favour the agents in class preference order.
A user level simulation relying on this behaviour,
or getting into this situation in specific topologies is
not recommended.}.

\paragraph{Preference Assignment and Control}
Each protocol agent stores an array of route preferences, \code{rtpref_}.
There is one element per destination, indexed by the node handle.
The default preference values used by each protocol are derived from
a class variable, \code{preference_}, for that protocol.
The current defaults are:
\begin{program}
        Agent/rtProto set preference_ 200               \; global default preference;
        Agent/rtProto/Direct\footnote{Direct is a special routing strategy that is used in conjunction with Dynamic routing.  We will describe this in greater detail as part of the route architecture description.} set preference_ 100
        Agent/rtProto/DV set preference_ 120
\end{program}
A simulation script can control routing by altering the preference
for routes in one of three ways:
alter the preference 
for a specific route learned \via\ a particular protocol agent,
alter the preference for all routes learned by the agent, or
alter the class variables for the agent before the agent is created.

\paragraph{Link Cost Assignment and Control}
In the currently implemented route protocols,
the metric of a route to a destination, at a node,
is the cost to reach the destiantion from that node.
It is possible to change the link costs at each of the links.
The instance procedure
\fcnref{\proc[]{cost}}{../ns-2/route-proto.tcl}{Simulator::cost}
%XXX MOVE TO NS-LIB.TCL
is invoked as \code{$ns cost \tup{node1} \tup{node2} \tup{cost}},
and sets the cost of the link from \tup{node1} to \tup{node2}
to \tup{cost}.
\begin{program}
        $ns cost $n1 $n2 10        \; set cost of link \textbf{from} $n1 \textbf{to} $n2 to 10;
        $ns cost $n2 $n1  5        \; set cost of link in reverse direction to 5;
        [$ns link $n1 $n2] cost?   \; query cost of link from $n1 to $n2;
        [$ns link $n2 $n1] cost?   \; query cost of link in reverse direction;
\end{program}
Notice that the procedure sets the cost along one direction only.
Similarly, the procedure
\fcnref{\proc[]{cost?}}{../ns-2/route-proto.tcl}{Link::cost?}
returns the cost of traversing the specified unidirectional link.
The default cost of a link is 1.

\section{Other Configuration Mechanisms for Specialised Routing}
\label{sec:uni:specroute}

It is possible to adjust preference and cost mechanisms to get two
special types of route configurations: 
asymetric routing, and multipath routing.

\paragraph{Asymetric Routing}
Asymetric routing occurs when the path from node $n_1$ to node $n_2$
is different from the path from $n_2$ to $n_1$.
The following shows a simple topology, and cost configuration
that can achieve such a result:

\hfil
\begin{minipage}{1.85in}
Nodes $n_1$ and $n_2$ use different paths to reach each other.
All other pairs of nodes use symetric paths to reach each other.
\end{minipage}
\hfil
\begin{minipage}{1.in}
  \begin{pspicture}(-1,-1)(1,1)
    \cnodeput( 0, 1){r1}{$r_1$}
    \cnodeput( 0,-1){r2}{$r_2$}
    \cnodeput( 1, 0){n2}{$n_2$}
    \cnodeput(-1, 0){n1}{$n_1$}
    \ncline{n1}{r1}\ncline{r1}{n2}\ncline{n2}{r2}\ncline{r2}{n1}
  \end{pspicture}
\end{minipage}
\hfil
\begin{minipage}{1.85in}
  \begin{program}
    $ns cost $n1 $r1 2
    $ns cost $n2 $r2 2
    $ns cost $r2 $n2 3
  \end{program}
\end{minipage}
\hfil

Any routing protocol that uses link costs as the metric can observe
such asymetric routing if the link costs are appropriately configured%
\footnote{Link costs can also be used to favour or disregard
specific links in order to achieve particular topology configurations.}.

\paragraph{MultiPath Routing}
Each node can be individually configured
to use multiple separate paths to a particular destination.
The instance variable \code{multiPath_} determines whether or not
that node will use multiple paths to any destination.
Each node initialises its instance variable from a class variable
of the same name.
If multiple candidate routes to a destination are available,
all of which are learned through the same protocol,
then that node can use
all of the different routes to the destination simultaneously.
A typical configuration is as shown below:
\begin{program}
        Node set multiPath_ 1 \; All new nodes in the simulation use multiPaths where applicable;
{\rm or alternately}
        set n1 [$ns Node] \; only enable $n1 to use multiPaths where applicable;
        $n1 set multiPath_ 1
\end{program}
Currently, only DV routing can generate multipath routes.

\section{Protocol Specific Configuration Parameters}
\label{sec:uni:protconfig}

\paragraph{Static Routing}
The static route computation strategy is
the default route computation mechanism  in \ns.
This strategy uses the
\fcnref{Dijkstra's all-pairs SPF algorithm \cite{}}{../ns-2/route.cc}{%
        RouteLogic::compute\_routes}.
The route computation algorithm is run exactly once
prior to the start of the simulation.
The routes are computed
using an adjacency matrix and link costs of all the links in the topology.

\paragraph{Session Routing}
The static routing strategy described earlier
only computes routes for the topology once in the course of a simulation.
If the topology changes while the simulation is in progress,
Then some sources and destinations may become unreachable from each other.

Session routing strategy is almost identical to static routing,
in that it runs the Dijkstra all-pairs SPF algorithm
prior to the start of the simulation, using the
adjacency matrix and link costs of the links in the topology.
However, it will also run the same algorithm to recompute routes
in the event that the topology changes during the course of a simulation.

Note that session routing leads to complete and instantaneous change
in the routes of the topology, whenever that topology changes.
If the topology is always connected, then there is
end-to-end connectivity at all times during the course of the simulation.
However, the user should note that the instantaneous recompute of the
routes in the topology can lead to temporary violations of causality
around the instant that the topology changes.

\paragraph{DV Routing}
DV routing is the implementation of
Distributed Bellman-Ford (or Distance Vector) routing in \ns.
The implementation sends periodic route updates every \code{advertInterval}.
This variable is a class variable in the \clsref{Agent/rtProto/DV}.
Its default value is 2 seconds.

In addition to periodic updates, each agent also sends triggered updates;
it does this whenever the forwarding tables in the node change.
This occurs either due to changes in the topology, 
or because an agent at the node received a route update,
and recomputed and installed new routes.

Each agent employs the split horizon with poisoned reverse mechanisms
to advertise its routes to adjacent peers.
``Split horizon'' is the mechanism by which an agent will not advertise
the route to a destination out of the interface that it is using to
reach that destination.
In a ``Split horizon with poisoned reverse'' mechanism,
the agent will advertise that route out of that interface with 
a metric of infinity.

Each DV agent uses a default \code{preference_} of 120.
The value is determined by the class variable of the same name.

Each agent uses the class variable \code{INFINITY} (set at 32)
to determine the validity of a route.

\section{Internals and Architecture of Routing}
\label{sec:rtg-internals}

We start with a discussion of the classes associated with
unicast routing, and the code path used to configure and execute
each of the different routing protocols.
We conclude with a description of
the interface netween unicast routing and network dynamics, and
that between unicast and multicast routing.

\subsection{The classes}
There are four main classes,
the class RouteLogic, the class rtObject, the class rtPeer, and the
base class Agent/rtProto for all protocols.
In addition, the routing architecture extends 
the classes Simulator, Link, Node and Classifier.

\paragraph{\clsref{RouteLogic}{../ns-2/route-proto.tcl}}
This class defines two methods to configure unicast routing,
and one method to query it for route information.
It also defines an instance procedure that is applicable when
the topology is dynamic.
We discuss this last procedure in conjunction
with the interface to network dynamics.
\begin{itemize}
\item The instance procedure
\fcnref{\proc[]{register}}{../ns-2/route-proto.tcl}{RouteLogic::register}
is invoked by \proc[]{Simulator::rtproto}.
It takes the protocol and a list of nodes as arguments,
and constructs an instance variable, \code{rtprotos_}, as an array;
the array index is the name of the protocol, and the value is the list
of nodes that will run this protocol.
\item The
\fcnref{\proc[]{configure}}{../ns-2/route-proto.tcl}{RouteLogic::configure}
reads the \code{rtprotos_} instance variable, and 
for each element in the array, 
invokes route protocol methods to perform the appropriate initialisations.
It is invoked by
\fcnref{the simulator run procedure}{../ns-2/ns-lib.tcl}{Simulator::run}.

For each protocol \tup{rt-proto} indexed in the \code{rtprotos_} array,
this routine invokes
\code{Agent/rtProto/\tup{rt-proto} init-all rtprotos_(\tup{rt-proto})}.

If there are no elements in \code{rtprotos_},
the routine invokes Static routing, as
\code{Agent/rtProto/Static init-all}.

\item The instance procedure
\fcnref{\proc[]{lookup}}{../ns-2/route-proto.tcl}{RouteLogic::lookup}
takes two node numbers, $nodeId_1$ and $nodeId_2$, as argument;
it returns the id of the neighbour node that $nodeId_1$ uses to 
reach $nodeId_2$.

The procedure is used by the static route computation procedure to query
the computed routes and populate the routes at each of the nodes.
It is also used by the multicast routing protocols to perform the
appropriate RPF check.

Note that this procedure overloads an instproc-like of the same name.
The procedure queries the appropriate \code{rtObject} entities
if they exist
(which they will if dynamic routing strategies are used in the simulation);
otherwise, the procedure invokes the instproc-like to obtain the relevant
information.
\end{itemize}

\paragraph{\clsref{rtObject}{../ns-2/route-proto.tcl}}
is used in simulations that use dynamic routing.
Each node has a rtObject associated with it, that
acts as a co-ordinator for the different routing protocols that
operate at a node.
At any node, the rtObject at that node 
tracks each of the protocols operating at that node;
it computes and installs the nest route to each destination
available via each of the protocols.
In the event that the routing tables cahnge, or the topology changes,
the rtObject will alert the protocols to take the appropriate action.

The class defines the procedure
\fcnref{\proc[]{init-all}}{../ns-2/route-proto.tcl}{rtObject::init-all};
this procedure takes a list of nodes as arguments,
and creates a rtObject at each of the nodes in its argument list.
It subsequently invokes its \code{compute-routes}.

The assumption is that the constructor for each of the new objects
will instantiate the ``Direct'' route protocol at each of these nodes.
This route protocol si responsible for computing the routes to 
immediately adjacent neighbours.
When \proc[]{compute-routes} is run by the \proc[]{init-all} 
procedure, these direct routes are installed in the node by the
appropriate route object.

The other instance procedures in this class are:
\begin{itemize}
\item \fcnref{\proc[]{init}}{../ns-2/route-proto.tcl}{rtObject::init}
The constructor sets up pointers from itself to the node,
in its instance variable \code{node_}, and from the node to itself,
through the Node instance procedure
\proc[]{init-routing} and the Node instance variable \code{rtObject_}.
It then initialises an array of
\code{nextHop_}, \code{rtpref_}, \code{metric_}, \code{rtVia_}.
The index of each of these arrays is the handle of the destination node.

The \code{nextHop_} contains the link that will be used to reach the
particular destination;
\code{rtpref_} and \code{metric_} are
the preference and metric for the route installed in the node;
\code{rtVia_} is the name of the agent whose route is installed in the node.

The constructor also creates the instance of the Direct route protocol,
and invokes \proc[]{compute-routes} for that protocol.

\item
\fcnref{\proc[]{add-proto}}{../ns-2/route-proto.tcl}{rtObject::add-proto}
creates an instance of the protocol, stores a reference to it
in its array of protocols, \code{rtProtos_}.
The index of the array is the name of the protocol.
It also attaches the protocol object to the node,
and returns the handle of the protocol object.

\item \fcnref{\proc[]{lookup}}{../ns-2/route-proto.tcl}{rtObject::lookup}
takes a destination node handle, and returns the id of the neighbour node
that is used to reach the destination.

If multiple paths are in use, then it returns a list of the
neighbour nodes that will be used.

If the node does not have a route to the destination,
the procedure will return -1.

\item
\fcnref{\proc[]{compute-routes}}{../ns-2/route-proto.tcl}{rtObject::compute-routes}
is the core procedure in this class.
It first checks to see if any of the routing protocols
at the node have computed any new routes.
If they have,
it will determine the best route to each destination
from among all the protocols.
If any routes have changed,
the procedure will notify each of the protocols of the number of
such changes, in case any of these protocols wants to send a fresh update.
Finally, it will also notify any multicast protocol that new unicast route
tables have been computed.

The routine checks the protocol agent's instance variable,
\code{rtsChanged_} to see if any of the routes in that protocol
have changed since the protocol was last examined.
It then uses the protocol's instance variable arrays,
\code{nextHop_}, \code{rtpref_}, and \code{metric_}
to compute its own arrays.
The rtObject will install or modify any of the routes as the changes are found.

If any of the routes at the node have changed,
the rtObject will invoke the protocol agent's instance procedures,
\proc[]{send-updates} with the number of changes as argument.
It will then invoke the multicast route object, if it exists.
\end{itemize}

The next set of routines are used to query the rtObject for various state
information.
\begin{itemize}
\item
\fcnref{\proc[]{dump-routes}}{../ns-2/route-proto.tcl}{rtObject::dump-routes}
takes a output file descriptor as arugment, and writes out the
routing table at that node on that file descriptor.

A typical dump output is:
{\small
\begin{verbatim}
\end{verbatim}
}

\item
\fcnref{\proc[]{rtProto?}}{../ns-2/route-proto.tcl}{rtObject::rtProto?}
takes a route protocol as argument, and returns a handle to the instance
of the protocol running at the  node.

\item
\fcnref{\proc[]{nextHop?}}{../ns-2/route-proto.tcl}{rtObject::nextHop?}
takes a destination node handle, and returns the link that is used to reach
that destination.

\item
Similarly,
\fcnref{\proc[]{rtpref?}}{../ns-2/route-proto.tcl}{rtObject::rtpref?} and
\fcnref{\proc[]{metric?}}{../ns-2/route-proto.tcl}{rtObject::metric?}
take a destination node handle as argument, and return the prefernce
and metric of the route to the destination installed at the node.
\end{itemize}

\paragraph{The \clsref{rtPeer}{../ns-2/route-proto.tcl}}
is a container class used by the protocol agents.
Each object stores the address of the peer agent, and the 
metric and preference for each route advertised by that peer.
A protocol agent will store one object per peer.
The class maintains the instance variable \code{addr_}, and the
instance variable arrays, \code{metric_} and \code{rtpref_};
the array indices are the destination node handles.

The class instance procedures,
\fcnref{\proc[]{metric}}{../ns-2/route-proto.tcl}{rtPeer::metric} and
\fcnref{\proc[]{preference}}{../ns-2/route-proto.tcl}{rtPeer::preference},
take one destination and value, and set the respective array variable.
The procedures,
\fcnref{\proc[]{metric?}}{../ns-2/route-proto.tcl}{rtPeer::metric?} and
\fcnref{\proc[]{preference?}}{../ns-2/route-proto.tcl}{rtPeer::preference?},
take a destination and return the current value for that destination.
The instance procedure
\fcnref{\proc[]{addr?}}{../ns-2/route-proto.tcl}{rtPeer::addr?}
returns the address of the peer agent.

\paragraph{\clsref{Agent/rtProto}{../ns-2/route-proto.tcl}}
This class is the base class from
which all routing protocol agents are derived.
Each protocol agent must define the procedure\proc[]{init-all}
to initialise the complete protocol,
and possibly instance procedures \proc[]{init}, \proc[]{compute-routes}, and
\proc[]{send-updates}.
In addition, if the topology is dynamic, and the protocol supports 
route computation to react to changes in the topology,
then the protocol should define the procedure \proc[]{compute-all}, and
possibly the instance procedure \proc[]{intf-changed}.
In this section, we will briefly describe the interface for the basic
procedures.
We will defer the description of \proc[]{compute-all} and
\proc[]{intf-changed}
to the section on network dynamics.
We also defer the description of the details of each of the protocols
to their separate section at the end of the chapter.
\begin{list}{---}{}
\item
The procedure
\fcnref{\proc[]{init-all}}{../ns-2/route-proto.tcl}{Agent/rtProto::init-all}
is a global initialisation procedure for the class.
It may be given a list fo the nodes as an argument.
This the list of nodes that should run this routing protocol.
However, centralised routing protocols such as static and session routing
will ignore this argument;
detailed dynamic routing protocols such as DV will use this argument
list to instantiate protocols agents at each of the nodes specified.

Note that derived classes in OTcl do not inherit the procedures
defined in the base class. 
Therefore, every derived routing protocol class must define its own
procedures explicitly.

\item
The instance procedure
\fcnref{\proc[]{init}}{../ns-2/route-proto.tcl}{Agent/rtProto::init}
is the constructor for protocol agents that are created.
The base class constructor initialises the default preference 
for objects in this class,
identifies the interfaces incident on the node and their current status.
The interfaces are idexed by the neighbour handle and stored in the instance
variable array, \code{ifs_};
the corresponding status instance variable array is \code{ifstat_}.

Centralised routing protocols such as static and session routing do not
create separate agents per node, and therefore do not access any of these
instance procedures.

\item
The instance procedure
\fcnref{\proc[]{compute-routes}}{../ns-2/route-proto.tcl}{Agent/rtProto::compute-routes}
computes the actual routes for the protocol.
The computation is based on the routes learned by the protocol, and
varies from protocol to protocol.

This routine is invoked by the rtObject whenever the topology changes.
It is also invoked when the node receives an update for the protocol.

If the routine computes new routes, 
\proc[]{rtObject::compute-routes} needs to be invoked
to recompute and possibly install new routes at the node.
The actual invoking of the rtObject is done by the procedure
that invoked this routine in the first place.

\item
The instance procedure
\fcnref{\proc[]{send-updates}}{../ns-2/route-proto.tcl}{Agent/rtProto::send-updates}
is invoked by the rtObject whenever the node routing tables have changed,
and fresh updates have to be sent to all peers.
The rtObject passes as argument the number of changes that were done.
This procedure may also be invoked when there are no changes to the routes,
but the topology incident on the node changes state.
The number of changes is used to determine the list of peers to which
a route update must be sent.
\end{list}
Other procedures relate to responding to topology changes and
\href{are described later}{Section}{sec:rtglibAPI}.

\paragraph{Other Extensions to the Simulator, Node, Link, and Classifier}
\begin{list}{---}{}
\item   % class Simulator
  We have discussed the methods \proc[]{rtproto} and \proc[]{cost}
  in the class Simulator \href{earlier}{Section}{sec:API}.
  The one other method used internally is
  \fcnref{\proc[]{get-routelogic}}{../ns-2/route-proto.tcl}{Simulator::get-routelogic};
  this procedure returns the instance of routelogic in the simulation.

  The method is used by the class Simulator, and unicast and multicast routing.

\item   % class Node
   The class Node contains these additional instance procedures
   to support dynamic unicast routing:
\fcnref{\proc[]{init-routing}}{../ns-2/route-proto.tcl}{Node::init-routing},
\fcnref{\proc[]{add-routes}}{../ns-2/route-proto.tcl}{Node::add-routes},
\fcnref{\proc[]{delete-routes}}{../ns-2/route-proto.tcl}{Node::delete-routes},
and
\fcnref{\proc[]{rtObject?}}{../ns-2/route-proto.tcl}{Node::rtObject?}.

The instance procedure \proc[]{init-routing}
is invoked by the \code{rtObject} at the node.
It stoeres a pointer to the rtObject, in its instance variable
\code{rtObject_}, for later mainpulation or retrieval.
It also checks its class variable to see if it should use multiPath routing,
and sets up an instance variable to that effect.
If multiPath routing could be used,
the instance variable array \code{routes_} stores a count of the number of
paths installed for each destination.
This is the only array in unicast routing that is indexed by the node id,
rather than the node handle.

The instance procedure \proc[]{rtObject?}
returns the rtObject handle for that node.

The instance procedure \proc[]{add-routes}
takes a node id, and a list of links.
It will add the list of links as the routes to reach the destination
identified by the node id.
The realisation of multiPath routing is done by using a separate
Classifier/multiPath.
For any given destination id $d$, if this node has multiple paths to $d$,
then the main classifier points to this multipath classifier instead of 
the link to reach the desitnaion.
Each of the multiple paths identified by the interfaces being used is
installed in the multipath classifier.
The multipath classifier will use each of the links installed in it for
succeeding packets forwarded to it.

The instance procedure \proc[]{delete-routes}
takes a node id, a list of itnerfaces, and a nullAgent.
It removes each of the interfaces in the list from the installed list of
itnerfaces.
If the entry did not previously use a multipath classifier,
then it must have had only one route, and the route entry is set to point
to the nullagent specified.

Q:  WHY DOES IT NOT POINT TO NULLAGENT IF THE ENTRIES IN THE MPATHCLASSIFIER
GOES TO ZERO?

\item   % class Link
  The main extension to the class Link for unicast routing is
  to support the notion of link costs.
  The instance variable \code{cost_}
  contains the cost of the unidirectional link.
  The instance procedures
  \fcnref{\proc[]{cost}}{../ns-2/route-proto.tcl}{Link::cost}
  and
  \fcnref{\proc[]{cost?}}{../ns-2/route-proto.tcl}{Link::cost?}
  set and get the cost on the link.

  Note that \proc[]{cost} takes the cost as argument.
  It is preferable to use the simulator method to set the cost variable,
  similar to the simulator instance procedures to set the queue or delay
  on a link.
  
\item   % class Classifier
The \clsref{Classifier}{../ns-2/ns-lib.tcl}
contains three new procedures, two of which overloads an existing
instproc-like, and the other two provide new functionality.

The instance procedure 
\fcnref{\proc[]{install}}{../ns-2/route-proto.tcl}{Classifier::install}
overloads the existing instproc-like of the same name.
The procedure stores the entry being installed in the instance
variable array, \code{elements_}, and then invokes the instproc-like.

The instance procedure 
\fcnref{\proc[]{installNext}}{../ns-2/route-proto.tcl}{Classifier::installNext}
also overloads the existing instproc-like of the same name.
This instproc-like simply installs the entry into the next available slot.

The instance procedure 
\fcnref{\proc[]{adjacents}}{../ns-2/route-proto.tcl}{Classifier::adjacents}
returns a list of \tup{key, value} pairs of all elements installed in the
classifier.
\end{list}

\subsection{Interface to Network Dynamics and Multicast}
\label{sec:rtglibAPI}
This section describes the methods applied in unicast routing to respond
to changes in the topology.
The complete sequence of actions that cause the changes in the topology,
and fire the appropriate actions is described in a different section.
% NEED XREF
The response to topology changes falls into two categories:
actions taken by individual agents at each of the nodes, and
actions to be taken globally for the entire protocol.

Detailed routing protocols such as the DV implementation
require actions to be performed by individual protocol agents at the
affected nodes.
Centralised routing protocols such as static and session routing fall into
the latter category exclusively.
Detailed routing protocols could use such techniques to gather statistics
related to the operation of the routing protocol;
however, no such code is currently implemented in \ns.

\paragraph{Actions at the individual nodes}
Following any change in the topology,
the network dynamics models will first invoke
\fcnref{\proc[]{rtObject::intf-changed}}{../ns-2/route-proto.tcl}{rtObject:;intf-changed}
at each of the affected nodes.
For each of the unicast routing protocols operating at that node,
\proc[]{rtObject::intf-changed} will invoke 
each individual protocol's instance procedure,  \proc[]{intf-changed},
followed by that protocol's \proc[]{compute-routes}.

After each protocol has computed its individual routes
\proc[]{rtObject::intf-changed} invokes \proc[]{compute-routes}
to possibly install new routes.
If new routes were installed in the node,
\proc[]{rtObject::compute-routes} will invoke
\proc[]{send-updates} for each of the protocols operating at the node.
The procedure will also
\fcnref{flag the multicast route
        object}{../ns-2/route-proto.tcl}{rtObject::flag-multicast}
of the route changes at the node, indicating the number of changes 
that have been executed.
\proc[]{rtObject::flag-multicast} will, in turn, notify
the multicast route object to take appropriate action.

The one exception
to the interface between unicast and multicast routing is the interaction
between dynamic dense mode multicast and detailed unicast routing.
This dynamicDM implementation in \ns\ assumes neighbour nodes
will send an implicit update whenever their routes change,
without actually sending the update.  
It then uses this implicit information to compute
appropriate parent-child relationships for the multicast spanning trees.
Therefore, detailed unicast routing will invoke
\code{rtObject_ flag-multicast 1} whenever it receives a route update as well,
even if that update does not result in any change in its own routing tables.

\paragraph{Global Actions}
Once the detailed actions at each of the affected nodes is completed,
the network dynamics models will
\fcnref{notify the RouteLogic instance (\proc[]{RouteLogic::notify})}{%
        ../ns-2/route-proto.tcl}{RouteLogic::notify} 
of changes to topology.
This procedure invokes the procedure \proc[]{compute-all}
for each of the protocols that were ever installed at any of the nodes.
Centralised routing protocols such as session routing use this signal to
recompute the routes to the topology.
Finally, the \proc[]{RouteLogic::notify} procedure notifies 
any instances of centralised multicast that are operating at the node.

\section{Protocol Internals}
\label{sec:protocol-internals}

In this section, we describe any leftover details of each of the routing
protocol agents.
Note that this is the only place where we describe the
internal route protocol agent, ``Direct'' routing.

\paragraph{Direct Routing}
This protocol tracks the state of the incident links,
and maintains routes to immediately adjacent neighbours only.
As with the other protocols, it maintains instance variable arrays
of \code{nextHop_}, \code{rtpref_}, and \code{metric_}, indexed by 
the handle of each of the possible destiantaions in the topology.

The instance procedure
\fcnref{\proc[]{compute-routes}}{../ns-2/route-proto.tcl}{Agent/rtProto/Direct::compute-routes}
computes routes based on the current state of the link, and the previously
known state of the incident links.

No other procedures or instance procedures are defined for this protocol.

\paragraph{Static Routing}
The procedure
\fcnref{\proc[]{compute-routes}}{../ns-2/ns-lib.tcl}{RouteLogic::compute-routes}
in the \clsref{RouteLogic}{../ns-2/ns-lib.tcl}
first creates the adjancency matrix, and then
invokes the C++ method, \fcn[]{compute\_routes} of the shadow object.
Finally, the procedure retrieves the result of the route computation,
and inserts the appropriate routes at each of the nodes in the topology.

The class only defines the procedure
\fcnref{\proc[]{init-all}}{../ns-2/route-proto.tcl}{Agent/rtProto/Static::init-all}
that invokes \proc[]{compute-routes}.

\paragraph{Session Routing}
The class defines the procedure
\fcnref{\proc[]{init-all}}{../ns-2/route-proto.tcl}{Agent/rtProto/Session::init-all}
to compute the routes at the start of the simulation.
It also defines the procedure
\fcnref{\proc[]{compute-all}}{../ns-2/route-proto.tcl}{Agent/rtProto/Session::compute-all}
to compute the routes when the topology changes.
Each of these procedures directly invokes \proc[]{compute-routes}.

\paragraph{DV Routing}
In a dynamic routing strategy, nodes send and receive messages,
and compute the rouets in the topology based on the messages exchanged.
The procedure
\fcnref{\proc[]{init-all}}{../ns-2/route-proto.tcl}{Agent/rtProto/DV::init-all}
takes a list of nodes as the argument;
the default is the list of nodes in the topology.
At each of the nodes in the argument, the procedure starts the
\clsref{rtObject}{../ns-2/route-proto.tcl} and a 
\clsref{Agent/rtProto/DV}{../ns-2/route-proto.tcl} agents.
It then determines the DV peers for each of the newly created DV agents,
and creates the relevant \code{rtPeer} objects.

The
\fcnref{constructor for the DV agent}{%
        ../ns-2/route-proto.tcl}{Agent/rtProto/DV::init}
initialises a number of instance variables;
each agent stores an array, indexed by the destiantion node handle,
of the preference and metric, the interface (or link) to the next hop,
and the remote peer incident on the interface,
for the best route to each destination computed by the agent.
The agent creates these instance variables, and then
schedules sending its first update within the first
0.5 seconds of simulation start.

Each agent stores the list of its peers indexed by the handle
of the peer node.
Each peer is a separate peer structure that holds
the address of the peer agent, the metric and preference
of the route to each destination advertised by that peer.
We discuss the rtPeer structure later
when discuss the route architecture.
The peer structures are initialised by the procedure
\fcnref{\proc[]{add-peer}}{../ns-2/route-proto.tcl}{Agent/rtProto/DV::add-peer}
invoked by \proc[]{init-all}.

The routine 
\fcnref{\proc[]{send-periodic-update}}{../ns-2/route-proto.tcl}{Agent/rtProto/DV::send-periodic-update}
invokes \proc[]{send-updates} to send the actual updates.
It then reschedules sending the next periodic update
after \code{advertInterval} jitterred slightly to avoid
possible synchronisation effects.

\fcnref{\proc[]{send-updates}}{../ns-2/route-proto.tcl}{Agent/rtProto/DV::send-updates}
will send updates to a select set of peers.
If any of the routes at that node have changed, or for periodic updates,
the procedure will send updates to all peers.
Otherwise, if some incident links have jsut recovered,
the procedure will send updates to the adjacent peers on those incident
links only.

\proc[]{send-updates} uses the procedure
\fcnref{\proc[]{send-to-peer}}{../ns-2/route-proto.tcl}{Agent/rtProto/DV::send-to-peer}
to send the actual updates.
This procedure packages the update, taking the
split-horizon and poison reverse mechanisms into account.
It invokes the instproc-like,
\fcnref{\proc[]{send-update} (Note the singular case)}{%
        ../ns-2/rtProto.cc}{rtProtoDV::command}
to send the actual update.
The actual route update is stored in the class variable
\code{msg_} indexed by a non-decreasing integer as index.
The instproc-like only sends the index to \code{msg_} to the remote peer.
This eliminates the need to convert from OTcl strings to alternate formats
and back.

When 
\fcnref{a peer receives a route update}{../ns-2/route-proto.tcl}{%
        Agent/rtProto/DV::recv-update}
it first checks to determine if the update from differs from the previous
ones.
The agent will compute new routes if the update contains new information.


\endinput

### Local Variables:
### mode: latex
### comment-column: 60
### backup-by-copying-when-linked: t
### file-precious-flag: nil
### End:

\chapter{Multicast Routing}
\label{chap:multicast}

This section describes the usage and the internals of multicast
routing implementation in \ns.
We first describe 
\href{the user interface to enable multicast routing}{Section}{sec:mcast-api},
specify the multicast routing protocol to be used and the
various methods and configuration parameters specific to the
protocols currently supported in \ns.
We then describe in detail 
\href{the internals and the architecture of the
multicast routing implementation in \ns}{Section}{sec:mcast-internals}.

The procedures and functions described in this chapter can be found in
various files in the directories \nsf{tcl/mcast}, \nsf{tcl/ctr-mcast};
additional support routines
are in \nsf{mcast\_ctrl.\{cc,h\}},
\nsf{tcl/lib/ns-lib.tcl}, and \nsf{tcl/lib/ns-node.tcl}.

\section{Multicast API}
\label{sec:mcast-api}

Multicast forwarding requires enhancements
to the nodes and links in the topology.
Therefore, the user must specify multicast requirements
to the Simulator class before creating the topology.
This is done in one of two ways:
\begin{program}
        set ns [new Simulator -multicast on]
    {\rm or}
        set ns [new Simulator]
        $ns multicast
\end{program}                   %$
When multicast extensions are thus enabled, nodes will be created with
additional classifiers and replicators for multicast forwarding, and
links will contain elements to assign incoming interface labels to all
packets entering a node.

A multicast routing strategy is the mechanism by which
the multicast distribution tree is computed in the simulation.
\ns\ supports three multiast route computation strategies:
        centralised, dense mode(DM) or shared tree mode(ST).

The method \proc[]{mrtproto} in the Class Simulator specifies either
the route computation strategy, for centralised multicast routing, or
the specific detailed multicast routing protocol that should be used.

%%For detailed multicast routing, \proc[]{mrtproto} will accept, as
%%additional arguments, a list of nodes that will run an instance of
%%that routing protocol.  
%%Polly Huang Wed Oct 13 09:58:40 EDT 199: the above statement
%%is no longer supported.

The following are examples of valid
invocations of multicast routing in \ns:
\begin{program}
        set cmc [$ns mrtproto CtrMcast]    \; specify centralized multicast for all nodes;
        \; cmc is the handle for multicast protocol object;
        $ns mrtproto DM                   \; specify dense mode multicast for all nodes;
        $ns mrtproto ST                  \; specify shared tree mode to run on all nodes;
\end{program}
Notice in the above examples that CtrMcast returns a handle that can
be used for additional configuration of centralised multicast routing.
The other routing protocols will return a null string.  All the
nodes in the topology will run instances of the same protocol.

Multiple multicast routing protocols can be run at a node, but in this
case the user must specify which protocol owns which incoming
interface.  For this finer control \proc[]{mrtproto-iifs} is used.

New/unused multicast address are allocated using the procedure
\proc[]{allocaddr}.
%%The default configuration in \ns\ provides 32 bits each for node address and port address space.
%%The procedure
%%\proc[]{expandaddr} is now obsoleted.

The agents use the instance procedures
\proc[]{join-group} and \proc[]{leave-group}, in
the class Node to join and leave multicast groups. These procedures
take two mandatory arguments. The first argument identifies the
corresponding agent and second argument specifies the group address.

An example of a relatively simple multicast configuration is:
\begin{program}
        set ns [new Simulator {\bfseries{}-multicast on}] \; enable multicast routing;
        set group [{\bfseries{}Node allocaddr}]   \; allocate a multicast address;
        set node0 [$ns node]         \; create multicast capable nodes;
        set node1 [$ns node]
        $ns duplex-link $node0 $node1 1.5Mb 10ms DropTail

        set mproto DM          \; configure multicast protocol;
        set mrthandle [{\bfseries{}$ns mrtproto $mproto}] \; all nodes will contain multicast protocol agents;
        set udp [new Agent/UDP]         \; create a source agent at node 0;
        $ns attach-agent $node0 $udp 
        set src [new Application/Traffic/CBR]        
        $src attach-agent $udp
        {\bfseries{}$udp set dst_addr_ $group}
        {\bfseries{}$udp set dst_port_ 0}

        set rcvr [new Agent/LossMonitor]  \; create a receiver agent at node 1;
        $ns attach-agent $node1 $rcvr
        $ns at 0.3 "{\bfseries{}$node1 join-group $rcvr $group}" \; join the group at simulation time 0.3 (sec);
\end{program} %$

\subsection{Multicast Behavior Monitor Configuration}
\ns\ supports a multicast monitor module that can trace
user-defined packet activity.
The module counts the number of packets in transit periodically
and prints the results to specified files. \proc[]{attach} enables a 
monitor module to print output to a file. 
\proc[]{trace-topo} insets monitor modules into all links. 
\proc[]{filter} allows accounting on specified packet header, 
field in the header), and value for the field).  Calling \proc[]{filter}
repeatedly will result in an AND effect on the filtering condition.
\proc[]{print-trace} notifies the monitor module to begin dumping data.
\code{ptype()} is a global arrary that takes a packet type name (as seen in
\proc[]{trace-all} output) and maps it into the corresponding value.  
A simple configuration to filter cbr packets on a particular group is:

\begin{program}
        set mcastmonitor [new McastMonitor]
        set chan [open cbr.tr w] \; open trace file;
        $mmonitor attach $chan1  \; attach trace file to McastMoniotor object;
        $mcastmonitor set period_ 0.02         \; default 0.03 (sec);
        $mmonitor trace-topo   \; trace entire topology;
        $mmonitor filter Common ptype_ $ptype(cbr) \; filter on ptype_ in Common header;
        $mmonitor filter IP dst_ $group \; AND filter on dst_ address in IP header;
        $mmonitor print-trace  \; begin dumping periodic traces to specified files;

\end{program} %$

% SAMPLE OUTPUT?
The following sample output illustrates the output file format (time, count):
{\small
\begin{verbatim}
0.16 0
0.17999999999999999 0
0.19999999999999998 0
0.21999999999999997 6
0.23999999999999996 11
0.25999999999999995 12
0.27999999999999997 12
\end{verbatim}
}

\subsection{Protocol Specific configuration}

In this section, we briefly illustrate the
protocol specific configuration mechanisms
for all the protocols implemented in \ns.

\paragraph{Centralized Multicast}
The centralized multicast is a sparse mode implementation of multicast
similar to PIM-SM \cite{Deer94a:Architecture}.
A Rendezvous Point (RP) rooted shared tree is built
for a multicast group.  The actual sending of prune, join messages
etc. to set up state at the nodes is not simulated.  A centralized
computation agent is used to compute the forwarding trees and set up
multicast forwarding state, \tup{S, G} at the relevant nodes as new
receivers join a group.  Data packets from the senders to a group are
unicast to the RP.  Note that data packets from the senders are
unicast to the RP even if there are no receivers for the group.

The method of enabling centralised multicast routing in a simulation is:
\begin{program}
        set mproto CtrMcast    \; set multicast protocol;
        set mrthandle [$ns mrtproto $mproto] 
\end{program}
The command procedure \proc[]{mrtproto}
returns a handle to the multicast protocol object.
This handle can be used to control the RP and the boot-strap-router (BSR),
switch tree-types for a particular group,
from shared trees to source specific trees, and
recompute multicast routes.
\begin{program}
        $mrthandle set_c_rp $node0 $node1          \; set the RPs;
        $mrthandle set_c_bsr $node0:0 $node1:1     \; set the BSR, specified as list of node:priority;

        $mrthandle get_c_rp $node0 $group          \; get the current RP ???;
        $mrthandle get_c_bsr $node0                \; get the current BSR;

        $mrthandle switch-treetype $group         \; to source specific or shared tree;

        $mrthandle compute-mroutes       \; recompute routes. usually invoked automatically as needed;
\end{program}

Note that whenever network dynamics occur or unicast routing changes,
\proc[]{compute-mroutes} could be invoked to recompute the multicast routes.
The instantaneous re-computation feature of centralised algorithms
may result in causality violations during the transient
periods.

\paragraph{Dense Mode}
The Dense Mode protocol (\code{DM.tcl}) is an implementation of a
dense--mode--like protocol.  Depending on the value of DM class
variable \code{CacheMissMode} it can run in one of two modes.  If
\code{CacheMissMode} is set to \code{pimdm} (default), PIM-DM-like
forwarding rules will be used.  Alternatively, \code{CacheMissMode}
can be set to \code{dvmrp} (loosely based on DVMRP \cite{rfc1075}).
The main difference between these two modes is that DVMRP maintains
parent--child relationships among nodes to reduce the number of links
over which data packets are broadcast.  The implementation works on
point-to-point links as well as LANs and adapts to the network
dynamics (links going up and down).

Any node that receives data for a particular group for which it has no
downstream receivers, send a prune upstream.  A prune message causes
the upstream node to initiate prune state at that node.  The prune
state prevents that node from sending data for that group downstream
to the node that sent the original prune message while the state is
active.  The time duration for which a prune state is active is
configured through the DM class variable, \code{PruneTimeout}.  A
typical DM configuration is shown below:
\begin{program}
        DM set PruneTimeout 0.3           \; default 0.5 (sec);
        DM set CacheMissMode dvmrp        \; default pimdm;
        $ns mrtproto DM
\end{program} %$

\paragraph{Shared Tree Mode}
Simplified sparse mode \code{ST.tcl} is a version of a shared--tree
multicast protocol.  Class variable array \code{RP\_} indexed by group
determines which node is the RP for a particular group.  For example:
\begin{program}
        ST set RP_($group) $node0
        $ns mrtproto ST
\end{program}
At the time the multicast simulation is started, the protocol will
create and install encapsulator objects at nodes that have multicast
senders, decapsulator objects at RPs and connect them.  To join a
group, a node sends a graft message towards the RP of the group.  To
leave a group, it sends a prune message.  The protocol currently does
not support dynamic changes and LANs.

\paragraph{Bi-directional Shared Tree Mode}
\code{BST.tcl} is an experimental version of a bi--directional shared
tree protocol.  As in shared tree mode, RPs must be configured
manually by using the class array \code{RP\_}.  The protocol currently
does not support dynamic changes and LANs.

\section{Internals of Multicast Routing}
\label{sec:mcast-internals}

We describe the internals in three parts: first the classes to
implement and support multicast routing; second, the specific protocol
implementation details; and finally, provide a list of the variables
that are used in the implementations.

\subsection{The classes}
The main classes in the implementation are the
\clsref{mrtObject}{../ns-2/tcl/mcast/McastProto.tcl} and the
\clsref{McastProtocol}{../ns-2/tcl/mcast/McastProto.tcl}.  There are
also extensions to the base classes: Simulator, Node, Classifier,
\etc.  We describe these classes and extensions in this subsection.
The specific protocol implementations also use adjunct data structures
for specific tasks, such as timer mechanisms by detailed dense mode,
encapsulation/decapsulation agents for centralised multicast \etc.; we
defer the description of these objects to the section on the
description of the particular protocol itself.

\paragraph{mrtObject class}
There is one mrtObject (aka Arbiter) object per multicast capable
node.  This object supports the ability for a node to run multiple
multicast routing protocols by maintaining an array of multicast
protocols indexed by the incoming interface.  Thus, if there are
several multicast protocols at a node, each interface is owned by just
one protocol.  Therefore, this object supports the ability for a node
to run multiple multicast routing protocols.  The node uses the
arbiter to perform protocol actions, either to a specific protocol
instance active at that node, or to all protocol instances at that
node.
\begin{alist}
{\let\[=[\let\]=]
\proc[instance, \[iiflist\]]{addproto}} &
        adds the handle for a protocol instance to its array of
        protocols.  The second optional argument is the incoming
        interface list controlled by the protocol.  If this argument
        is an empty list or not specified, the protocol is assumed to
        run on all interfaces (just one protocol). \\
\proc[protocol]{getType} &
        returns the handle to the protocol instance active at that
        node that matches the specified type (first and only
        argument).  This function is often used to locate a protocol's
        peer at another node.  An empty string is returned if the
        protocol of the given type could not be found. \\
\proc[op, args]{all-mprotos} &
        internal routine to execute ``\code{op}'' with ``\code{args}''
        on all protocol instances active at that node. \\
\proc[]{start} & \\
\proc[]{stop} &
        start/stop execution of all protocols. \\
\proc[dummy]{notify} &
        is called when a topology change occurs. The dummy argument is
        currently not used.\\
{\let\[=[\let\]=]
\proc[file-handle, \[grp\], \[src\]]{dump-mroutes}} &
        dump multicast routes to specified file-handle. \\
\proc[G, S]{join-group} &
        signals all protocol instances to join \tup{S, G}. \\
\proc[G, S]{leave-group} &
        signals all protocol instances to leave \tup{S, G}. \\
\proc[code, s, g, iif]{upcall} &
        signalled by node on forwarding errors in classifier;
        this routine in turn signals the protocol instance that owns
        the incoming interface (\code{iif}) by invoking the
        appropriate handle function for that particular code.\\ 
\proc[rep, s, g, iif]{drop} &
        Called on packet drop, possibly to prune an interface. \\
\end{alist}

In addition, the mrtObject class supports the concept of well known
groups, \ie, those groups that do not require explicit protocol support.
Two well known groups, \code{ALL_ROUTERS} and \code{ALL_PIM_ROUTERS}
are predefined in \ns.

The \clsref{mrtObject}{../ns-2/tcl/mcast/McastProto.tcl} defines
two class procedures to set and get information about these well known groups.
\begin{alist}
\proc[name]{registerWellKnownGroups} & 
        assigns \code{name} a well known group address. \\
\proc[name]{getWellKnownGroup} &
        returns the address allocated to well known group, \code{name}.
        If \code{name} is not registered as a well known group,
        then it returns the address for \code{ALL\_ROUTERS}.
\end{alist}

\paragraph{McastProtocol class}
This is the base class for the implementation of all the multicast protocols.
It contains basic multicast functions:
\begin{alist}
\proc[]{start}, \proc[]{stop} &
        Set the \code{status\_} of execution of this protocol instance. \\
\proc[]{getStatus} &
        return the status of execution of this protocol instance. \\
\proc[]{getType} &
        return the type of protocol executed by this instance. \\
\proc[code args]{upcall} &
        invoked when the node classifier signals an error, either due to 
        a cache-miss or a wrong-iif for incoming packet.  This routine
        invokes the protocol specific handle, \proc{handle-\tup{code}} with
        specified \code{args} to handle the signal. \\
\end{alist}

A few words about interfaces.  Multicast implementation in \ns\
assumes duplex links i.e. if there is a simplex link from node~1 to
node~2, there must be a reverse simplex link from node~2 to node~1.
To be able to tell from which link a packet was received, multicast
simulator configures links with an interface labeller at the end of
each link, which labels packets with a particular and unique label
(id).  Thus, ``incoming interface'' is referred to this label and is a
number greater or equal to zero.  Incoming interface value can be
negative (-1) for a special case when the packet was sent by a local
to the given node agent.

In contrast, an ``outgoing interface'' refers to an object handler,
usually a head of a link which can be installed at a replicator.  This
destinction is important: \textit{incoming interface is a numeric label to
a packet, while outgoing interface is a handler to an object that is
able to receive packets, e.g. head of a link.}
 
\subsection{Extensions to other classes in \ns}
In \href{the earlier chapter describing nodes in
\ns}{Chapter}{chap:nodes}, we described the internal structure of the
node in \ns.  To briefly recap that description, the node entry for a
multicast node is the
\code{switch\_}.  It looks at the highest bit to decide if the
destination is a multicast or unicast packet.  Multicast packets are
forwarded to a multicast classifier which maintains a list of
replicators; there is one replicator per \tup{source, group} tuple.
Replicators copy the incoming packet and forward to all outgoing
interfaces.

\paragraph{Class Node}
Node support for multicast is realized in two primary ways: it serves
as a focal point for access to the multicast protocols, in the areas
of address allocation, control and management, and group membership
dynamics; and secondly, it provides primitives to access and control
interfaces on links incident on that node.
\begin{alist}
\proc[]{expandaddr}, & \\
\proc[]{allocaddr} &
        Class procedures for address management.
        \proc[]{expandaddr} is now obsoleted.
        \proc[]{allocaddr} allocates the next available multicast
        address.\\[2ex]
\proc[]{start-mcast}, & \\
\proc[]{stop-mcast} &
        To start and stop multicast routing at that node. \\
\proc[]{notify-mcast} &
        \proc[]{notify-mcast} signals the mrtObject at that node to
        recompute multicastroutes following a topology change or
        unicast route update from a neighbour.  \\[2ex]
\proc[]{getArbiter} &
        returns a handle to mrtObject operating at that node. \\
\proc[file-handle]{dump-routes} &
        to dump the multicast forwarding tables at that node. \\[2ex]
\proc[s g iif code]{new-group} &
        When a multicast data packet is received, and the multicast
        classifier cannot find the slot corresponding to that data
        packet, it invokes \proc[]{Node~nstproc~new-group} to
        establish the appropriate entry.  The code indicates the
        reason for not finding the slot.  Currently there are two
        possibilities, cache-miss and wrong-iif.  This procedure
        notifies the arbiter instance to establish the new group. \\
\proc[a g]{join-group} &
        An \code{agent} at a node that joins a particular group invokes
        ``\code{node join-group <agent> <group>}''.  The
        node signals the mrtObject to join the particular \code{group},
        and adds \code{agent} to its list of agents at that
        \code{group}.  It then adds \code{agent} to all replicators
        associated with \code{group}. \\
\proc[a g]{leave-group} &
        \code{Node~instproc~leave-group} reverses the process
        described earlier.  It disables the outgoing interfaces to the
        receiver agents for all the replicators of the group, deletes
        the receiver agents from the local \code{Agents\_} list; it
        then invokes the arbiter instance's
        \proc[]{leave-group}.\\[2ex]
\proc[s g iif oiflist]{add-mfc} &
        \code{Node~instproc~add-mfc} adds a \textit{multicast forwarding cache}
        entry for a particular \tup{source, group, iif}.
        The mechanism is:
        \begin{itemize}
        \item create a new replicator (if one does not already exist),
        \item update the \code{replicator\_} instance variable array at the node,
        \item add all outgoing interfaces and local agents to the
            appropriate replicator,
        \item invoke the multicast classifier's \proc[]{add-rep}
            to create a slot for the replicator in the multicast
            classifier.
        \end{itemize} \\
\proc[s g oiflist]{del-mfc} &
        disables each oif in \code{oiflist} from the replicator for \tup{s, g}.\\
\end{alist}

The list of primitives accessible at the node to control its interfaces are listed below.
\begin{alist}
\proc[ifid link]{add-iif}, & \\
\proc[link if]{add-oif} &
        Invoked during link creation to prep the node about its 
        incoming interface label and outgoing interface object. \\

\proc[]{get-all-oifs} &
        Returns all oifs for this node. \\
\proc[]{get-all-iifs} &
        Returns all iifs for this node. \\

\proc[ifid]{iif2link} &
        Returns the link object labelled with given interface
        label. \\
\proc[link]{link2iif} &
        Returns the incoming interface label for the given
        \code{link}. \\

\proc[oif]{oif2link} &
        Returns the link object corresponding to the given outgoing
        interface. \\
\proc[link]{link2oif} &
        Returns the outgoing interface for the \code{link} (\ns\
        object that is incident to the node).\\

\proc[src]{rpf-nbr} &
        Returns a handle to the neighbour node that is its next hop to the 
        specified \code{src}.\\

\proc[s g]{getReps} &
        Returns a handle to the replicator that matches \tup{s, g}.
        Either argument can be a wildcard (*). \\
\proc[s g]{getReps-raw} &
        As above, but returns a list of \tup{key, handle} pairs. \\
\proc[s g]{clearReps} &
        Removes all replicators associated with \tup{s, g}. \\[2ex]
\end{alist}

\paragraph{Class Link and SimpleLink}
This class provides methods to check the type of link, and the label it 
affixes on individual packets that traverse it.
There is one additional method to actually place the interface objects on this link.
These methods are:
\begin{alist}
\proc[]{if-label?} & 
        returns the interface label affixed by this link to packets
        that traverse it. \\
% \proc[]{enable-mcast} & 
%       Internal procedure called by the SimpleLink constructor to add
%       appropriate objects and state for multicast.  By default, (and
%       for the point-to-point link case) it places a NetworkInterface
%       object at the end of the link, and signals the nodes on
%       incident on the link about this link.\\
\end{alist}

\paragraph{Class NetworkInterface}
This is a simple connector that is placed on each link.  It affixes
its label id to each packet that traverses it.  The packet id is used
by the destination node incident on that link to identify the link by
which the packet reached it.  The label id is configured by the Link
constructor.  This object is an internal object, and is not designed
to be manipulated by user level simulation scripts.  The object only
supports two methods:
\begin{alist}
\proc[ifid]{label} & 
        assigns \code{ifid} that this object will affix to each packet. \\
\proc[]{label} & 
        returns the label that this object affixes to each packet.\\
\end{alist}
The global class variable, \code{ifacenum\_}, specifies the next
available \code{ifid} number.

\paragraph{Class Multicast Classifier}
\code{Classifier/Multicast} maintains a \emph{multicast forwarding
cache}.  There is one multicast classifier per node. The node stores a
reference to this classifier in its instance variable
\code{multiclassifier\_}. When this classifier receives a packet, it
looks at the \tup{source, group} information in the packet headers,
and the interface that the packet arrived from (the incoming interface
or iif); does a lookup in the MFC and identifies the slot that should
be used to forward that packet.  The slot will point to the
appropriate replicator.

However, if the classifier does not have an entry for this
\tup{source, group}, or the iif for this entry is different, it will
invoke an upcall \proc[]{new-group} for the classifier, with one of
two codes to identify the problem:

\begin{itemize}
        \item \code{cache-miss} indicates that the classifier did not
        find any \tup{source, group} entries;

        \item \code{wrong-iif} indicates that the classifier found
        \tup{source, group} entries, but none matching the interface
        that this packet arrived on.
\end{itemize}
These upcalls to TCL give it a chance to correct the situation:
install an appropriate MFC--entry (for \code{cache-miss}) or change
the incoming interface for the existing MFC--entry (for
\code{wrong-iif}).  The \emph{return value} of the upcall determines
what classifier will do with the packet.  If the return value is
``1'', it will assume that TCL upcall has appropriately modified MFC
will try to classify packet (lookup MFC) for the second time.  If the
return value is ``0'', no further lookups will be done, and the packet
will be thus dropped.

\proc[]{add-rep} creates a slot in the classifier
and adds a replicator for \tup{source, group, iif} to that slot.

\paragraph{Class Replicator}
When a replicator receives a packet, it copies the packet to all of
its slots.  Each slot points to an outgoing interface for a particular
\tup{source, group}.

If no slot is found, the C++ replicator invokes the class instance
procedure \proc[]{drop} to trigger protocol specific actions.  We will
describe the protocol specific actions in the next section, when we
describe the internal procedures of each of the multicast routing
protocols.

There are instance procedures to control the elements in each slot:
\begin{alist}
\proc[oif]{insert} & inserting a new outgoing interface
                        to the next available slot.\\
\proc[oif]{disable} & disable the slot pointing to the specified oif.\\
\proc[oif]{enable} &  enable the slot pointing to the specified oif.\\
\proc[]{is-active} & returns true if the replicator has at least one active slot.\\
\proc[oif]{exists} & returns true if the slot pointing to the specified oif is active.\\
\proc[source, group, oldiif, newiif]{change-iface} & modified the iif entry for the particular replicator.\\
\end{alist}

\subsection{Protocol Internals}
\label{sec:mcastproto-internals}

We now describe the implementation of the different multicast routing
protocol agents.

\subsubsection{Centralized Multicast}
\code{CtrMcast} is inherits from \code{McastProtocol}.
One CtrMcast agent needs to be created for each node.  There is a
central CtrMcastComp agent to compute and install multicast routes for
the entire topology.  Each CtrMcast agent processes membership dynamic
commands, and redirects the CtrMcastComp agent to recompute the
appropriate routes.
\begin{alist}
\proc[]{join-group} &
        registers the new member with the \code{CtrMCastComp} agent, and
        invokes that agent to recompute routes for that member. \\
\proc[]{leave-group} & is the inverse of \proc[]{join-group}. \\
\proc[]{handle-cache-miss} &
         called when no proper forwarding entry is found
         for a particular packet source.
        In case of centralized multicast,
        it means a new source has started sending data packets.
        Thus, the CtrMcast agent registers this new source with the
        \code{CtrMcastComp} agent.
        If there are any members in that group, compute the new multicast tree.
        If the group is in RPT (shared tree) mode, then
        \begin{enumerate} 
        \item create an encapsulation agent at the source;
        \item a corresponding decapsulation agent is created at the RP;
        \item the two agents are connected by unicast; and
        \item the \tup{S,G} entry points its outgoing interface to the
              encapsulation agent.
        \end{enumerate}
\end{alist}

\code{CtrMcastComp} is the centralised multicast route computation agent.
\begin{alist}
\proc[]{reset-mroutes} & 
        resets all multicast forwarding entries.\\
\proc[]{compute-mroutes} & 
        (re)computes all multicast forwarding entries.\\
\proc[source, group]{compute-tree} & 
        computes a multicast tree for one source to reach all the
        receivers in a specific group.\\ 
\proc[source, group, member]{compute-branch} & 
        is executed when a receiver joins a multicast group.  It could
        also be invoked by \proc[]{compute-tree} when it itself is
        recomputing the multicast tree, and has to reparent all
        receivers.  The algorithm starts at the receiver, recursively
        finding successive next hops, until it either reaches the
        source or RP, or it reaches a node that is already a part of
        the relevant multicast tree.  During the process, several new
        replicators and an outgoing interface will be installed.\\
\proc[source, group, member]{prune-branch} & 
        is similar to \proc[]{compute-branch} except the outgoing
        interface is disabled; if the outgoing interface list is empty
        at that node, it will walk up the multicast tree, pruning at
        each of the intermediate nodes, until it reaches a node that
        has a non-empty outgoing interface list for the particular
        multicast tree.
\end{alist}

\subsubsection{Dense Mode}
\begin{alist}
\proc[group]{join-group} & 
        sends graft messages upstream if \tup{S,G} does not contain
        any active outgoing slots (\ie, no downstream receivers).
        If the next hop towards the source is a LAN, icrements a
        counter of receivers for a particular group for the LAN\\
\proc[group]{leave-group} & 
        decrements LAN counters. \\
\proc[srcID group iface]{handle-cache-miss} & 
        depending on the value of \code{CacheMissMode} calls either
        \code{handle-cache-miss-pimdm} or
        \code{handle-cache-miss-dvmrp}. \\
\proc[srcID group iface]{handle-cache-miss-pimdm} &   
        if the packet was received on the correct iif (from the node
        that is the next hop towards the source), fan out the packet
        on all oifs except the oif that leads back to the
        next--hop--neighbor and oifs that are LANs for which this node
        is not a forwarder. Otherwise, if the interface was incorrect,
        send a prune back.\\
\proc[srcID group iface]{handle-cache-miss-dvmrp} &
        fans out the packet only to nodes for which this node is a
        next hop towards the source (parent).\\
\proc[replicator source group iface]{drop} & 
        sends a prune message back to the previous hop.\\
\proc[from source group iface]{recv-prune} & 
        resets the prune timer if the interface had been pruned
        previously; otherwise, it starts the prune timer and disables
        the interface; furthermore, if the outgoing interface list
        becomes empty, it propagates the prune message upstream.\\
\proc[from source group iface]{recv-graft} & 
        cancels any existing prune timer, andre-enables the pruned
        interface.  If the outgoing interface list was previously
        empty, it forwards the graft upstream.\\
\proc[srcID group iface]{handle-wrong-iif} & 
        This is invoked when the multicast classifier drops a packet
        because it arrived on the wrong interface, and invoked
        \proc[]{new-group}.  This routine is invoked by
        \proc[]{mrtObject~instproc~new-group}.  When invoked, it sends
        a prune message back to the source.\\
\end{alist}

\subsection{The internal variables}
\begin{alist}
\textbf{Class mrtObject}\hfill & \\
\code{protocols\_} &
        An array of handles of protocol instances active at the node
        at which this protocol operates indexed by incoming
        interface. \\
\code{mask-wkgroups} &
        Class variable---defines the mask used to identify well known
        groups. \\
\code{wkgroups} &
        Class array variable---array of allocated well known groups
        addresses, indexed by the group name.  \code{wkgroups}(Allocd)
        is a special variable indicating the highest currently
        allocated well known group. \\[3ex]

\textbf{McastProtocol}\hfill & \\
\code{status\_} &
        takes values ``up'' or ``down'', to indicate the status of
        execution of the protocol instance. \\
\code{type\_} &
        contains the type (class name) of protocol executed by this
        instance, \eg, DM, or ST. \\

\textbf{Simulator}\hfill & \\
\code{multiSim\_} &
        1 if multicast simulation is enabled, 0 otherwise.\\
\code{MrtHandle\_} &
        handle to the centralised multicast simulation object.\\[3ex]

\textbf{Node}\hfill & \\
\code{switch\_} & 
        handle for classifier that looks at the high bit of the
        destination address in each packet to determine whether it is
        a multicast packet (bit = 1) or a unicast packet (bit = 0).\\
\code{multiclassifier\_} & 
        handle to classifier that performs the \tup{s, g, iif} match. \\
\code{replicator\_} & 
        array indexed by \tup{s, g} of handles that replicate a
        multicast packet on to the required links. \\
\code{Agents\_} & 
        array indexed by multicast group of the list of agents at the
        local node that listen to the specific group. \\
\code{outLink\_} & 
        Cached list of outgoing interfaces at this node.\\
\code{inLink\_} &
        Cached list of incoming interfaces at this node.\\

\textbf{Link} and \textbf{SimpleLink}\hfill & \\
\code{iif\_} & 
        handle for the NetworkInterface object placed on this link.\\
\code{head\_} & 
        first object on the link, a no-op connector.  However, this
        object contains the instance variable, \code{link\_}, that
        points to the container Link object.\\

\textbf{NetworkInterface}\hfill & \\
\code{ifacenum\_} & 
        Class variable---holds the next available interface
        number.\\
\end{alist}


\section{Commands at a glance}
\label{sec:mcastcommand}

Following is a list of commands used for multicast simulations:
\begin{flushleft}
\code{set ns [new Simulator -mcast on]}\\
This turns the multicast flag on for the the given simulation, at the time of
creation of the simulator object.


\code{ns_ multicast}\\
This like the command above turns the multicast flag on.


\code{ns_ multicast?}\\
This returns true if multicast flag has been turned on for the simulation
and returns false if multicast is not turned on.


\code{$ns_ mrtproto <mproto> <optional:nodelist>}\\
This command specifies the type of multicast protocol <mproto> to be used
like DM, CtrMcast etc. As an additional argument, the list of nodes <nodelist>
that will run an instance of detailed routing protocol (other than
centralised mcast) can also be passed.


\code{$ns_ mrtproto-iifs <mproto> <node> <iifs>}\\
This command allows a finer control than mrtproto. Since multiple mcast
protocols can be run at a node, this command specifies which mcast protocol
<mproto> to run at which of the incoming interfaces given by <iifs> in the <node>.


\code{Node allocaddr}\\
This returns a new/unused multicast address that may be used to assign a multicast
address to a group.


\code{Node expandaddr}\\
THIS COMMAND IS OBSOLETE NOW
This command expands the address space from 16 bits to 30 bits. However this
command has been replaced by \code{"ns_ set-address-format-expanded"}.


\code{$node_ join-group <agent> <grp>}\\
This command is used when the <agent> at the node joins a particular group <grp>.


\code{$node_ leave-group <agent> <grp>}\\
This is used when the <agent> at the nodes decides to leave the group <grp>.

Internal methods:\\

\code{$ns_ run-mcast}\\
This command starts multicast routing at all nodes. 


\code{$ns_ clear-mcast}\\
This stopd mcast routing at all nodes.


\code{$node_ enable-mcast <sim>}\\
This allows special mcast supporting mechanisms (like a mcast classifier) to
be added to the mcast-enabled node. <sim> is the a handle to the simulator
object.

In addition to the internal methods listed here there are other methods specific to
protocols like centralized mcast (CtrMcast), dense mode (DM), shared tree
mode (ST) or bi-directional shared tree mode (BST), Node and Link class
methods and NetworkInterface and Multicast classifier methods specific to
multicast routing. All mcast related files may be found under
\ns/tcl/mcast directory. 
\begin{description}

\item[Centralised Multicast] A handle to the CtrMcastComp object is
returned when the protocol is specified as `CtrMcast' in mrtproto. 
Ctrmcast methods are: \\

\code{$ctrmcastcomp switch-treetype group-addr}\\
Switch from the Rendezvous Point rooted shared tree to source-specific
trees for the group specified by group-addr. Note that this method cannot
be used to switch from source-specific trees back to a shared tree for a
multicast group. 

\code{$ctrmcastcomp set_c_rp <node-list>}\\
This sets the RPs.

\code{$ctrmcastcomp set_c_bsr <node0:0> <node1:1>}\\
This sets the BSR, specified as list of node:priority.

\code{$ctrmcastcomp get_c_rp <node> <group>}\\
Returns the RP for the group as seen by the node node for the multicast
group with address group-addr. Note that different nodes may see different
RPs for the group if the network is partitioned as the nodes might be in
different partitions. 

\code{$ctrmcastcomp get_c_bsr <node>}\\
Returns the current BSR for the group.

\code{$ctrmcastcomp compute-mroutes}\\
This recomputes multicast routes in the event of network dynamics or a
change in unicast routes.


\item[Dense Mode]
The dense mode (DM) protocol can be run as PIM-DM (default) or DVMRP
depending on the class variable \code{CacheMissMode}. There are no methods
specific to this mcast protocol object. Class variables are:
 \begin{description}
   \item[PruneTimeout] Timeout value for prune state at nodes. defaults to
0.5sec.
   \item[CacheMissMode] Used to set PIM-DM or DVMRP type forwarding rules.
 \end{description}


\item[Shared Tree]
There are no methods for this class. Variables are:
\begin{description}
\item[RP\_] RP\_ indexed by group determines which node is the RP for a
particular group.
\end{description}


\item[Bidirectional Shared Tree]
There are no methods for this class. Variable is same as that of Shared
Tree described above.

\end{description}

\end{flushleft}

\endinput

\chapter{Network Dynamics}
\label{chap:net-dynamics}

This chapter describes the capabilities in \ns\
to make the simulation topologies dynamic.
We start with the instance procedures to the class Simulator
that are \href{useful to a simulation script}{Section}{sec:userAPI}.
The next section describes
\href{the internal architecture}{Section}{sec:nd-internal-arch},
including the different classes and instance variables and procedures;
the following section describes
\href{the interaction with unicast routing}{Section}{sec:unicast-int}.
This aspect of network dynamics is still somewhat experimental in \ns.
The last section of this chapter outlines some of
\href{the deficiencies in the current realization}{Section}{sec:deficiencies}
of network dynamics, some one or which
may be fixed in the future.

The procedures and functions described in this chapter can be found in
\nsf{tcl/rtglib/dynamics.tcl} and \nsf{tcl/lib/route-proto.tcl}.

\section{The user level API}
\label{sec:userAPI}

The user level interface to network dynamics is a collection 
of instance procedures in the class Simulator,
and one procedure to trace and log the dynamics activity.
Reflecting a rather poor choice of names,
these procedures are
\code{rtmodel}, \code{rtmodel-delete}, and \code{rtmodel-at}.
There is one other procedure, \code{rtmodel-configure},
that is used internally by the class Simulator to configure
the rtmodels just prior to simulation start.
We describe this method \href{later}{Section}{sec:nd-internal-arch}.
\begin{list}{---}{}
\item The instance procedure
\fcnref{\proc[]{rtmodel}}{../ns-2/dynamics.tcl}{Simulator::rtmodel}
defines a model to be applied to the nodes and links in the topology.
Some examples of this command as it would be used in a simulation script are:
\begin{program}
        $ns rtmodel Exponential {0.8 1.0 1.0} $n1
        $ns rtmodel Trace dynamics.trc  $n2 $n3
        $ns rtmodel Deterministic {20.0 20.0} $node(1) $node(5)
\end{program}
The procedure requires at least three arguments:
\begin{itemize}
\item % the model definition
The first two arguments define the model that will be used, and the
parameters to configure the model.

The currently implemented models in \ns\ are
Exponential (On/Off), Deterministic (On/Off), Trace (driven), or
Manual (one-shot) models.

\item % the parameters
The number, format, and interpretation of the configuration parameters
is specific to the particular model.
\begin{enumerate}\itemsep0pt
\item The exponential on/off model takes four parameters:
\tup{[start time], up interval, down interval, [finish time]}.
\tup{start time} defaults to $0.5s.$ from the start of the simulation,
\tup{finish time} defaults to the end of the simulation.
\tup{up interval} and \tup{down interval} specify
the mean of the exponential distribution defining the time
that the node or link will be up and down respectively.
The default up and down interval values are $10s.$ and $1s.$ respectively.
Any of these values can be specified as ``$-$'' to default to the
original value.

The following are example specifications of parameters to this model:
\begin{program}
      0.8 1.0 1.0       \; start at \(0.8s.\), up/down = \(1.0s.\), finish is default;
      5.0 0.5           \; start is default, up/down = \(5.0s, 0.5s.\), finish is default;
      - 0.7             \; start, up interval are default, down = \(0.7s.\), finish is default;
      - - - 10          \; start, up, down are default, finish at \(10s.\);
\end{program}

\item The deterministic on/off model
is similar to the exponential model above, and  takes four parameters:
\tup{[start time], up interval, down interval, [finish time]}.
\tup{start time} defaults to the start of the simulation,
\tup{finish time} defaults to the end of the simulation.
Only the interpretation of the up and down interval is different;
\tup{up interval} and \tup{down interval} specify the exact duration
that the node or link will be up and down respectively.
The default values for these parameters are:
\tup{start time} is $0.5s.$ from start of simulation,
\tup{up interval} is $2.0s.$,
\tup{down interval} is $1.0s.$, and
\tup{finish time} is the duration of the simulation.
\item The trace driven model takes one parameter:
the name of the trace file.
The format of the input trace file is identical to that 
output by the dynamics trace modules, \viz,
\code{v \tup{time} link-\tup{operation} \tup{node1} \tup{node2}}.
Lines that do not correspond to the node or link specified are ignored.
{\small
\begin{verbatim}
        v 0.8123 link-up 3 5
        v 3.5124 link-down 3 5
\end{verbatim}
}
\item The manual one-shot model takes two parameters:
the operation to be performed, and the time that it is to be
performed.
\end{enumerate}

\item % the elements
The rest of the arguments to the \proc[]{rtmodel} procedure
define the node or link that the model will be applied to.
If only one node is specified,
it is assumed that the node will fail.
This is modeled by making the links incident on the node fail.
If two nodes are specified, then the command assumes that
the two are adjacent to each other, and the model is applied to the
link incident on the two nodes.
If more than two nodes are specified, only the first is considered,
the subsequent arguments are ignored.

\item % \proc[]{rtmodel} will also enable tracing if the Simulator
  instance variable, \code{traceAllFile_} is set.
\end{itemize}
The command returns the handle to the model that was created in this call.

Internally, \proc[]{rtmodel} stores the list of route models created
in the class Simulator instance variable, \code{rtModel_}.

\item The instance procedure
\fcnref{\proc[]{rtmodel-delete}}{../ns-2/dynamics.tcl}{Simulator::rtmodel-delete}
takes the handle of a route model as argument, removes it from the
\code{rtModel_} list, and deletes the route model.

\item The instance procedure
\fcnref{\proc[]{rtmodel-at}}{../ns-2/dynamics.tcl}{Simulator::rtmodel-at}
is a special interface to the Manual model of network dynamics.

The command takes the time, operation, and node or link as arguments,
and applies the operation to the node or link at the specified time.
Example uses of this command are:
\begin{program}
        $ns rtmodel-at 3.5 up $n0
        $ns rtmodel-at 3.9 up $n(3) $n(5)
        $ns rtmodel-at 40  down  $n4
\end{program}
\end{list}

Finally, the instance procedure \proc[]{trace-dynamics} of the class rtModel
enables tracing of the dynamics effected by this model.
It is used as:
\begin{program}
        set fh [open "dyn.tr" w]
        $rtmodel1 trace-dynamics $fh
        $rtmodel2 trace-dynamics $fh
        $rtmodel1 trace-dynamics stdout
\end{program}
In this example, \code{$rtmodel1} writes out trace entries to both
dyn.tr and stdout; \code{$rtmodel2} only writes out trace entries to dyn.tr.
A typical sequence of trace entries written out by either model might be:
{\small
\begin{verbatim}
        v 0.8123 link-up 3 5
        v 0.8123 link-up 5 3
        v 3.5124 link-down 3 5
        v 3.5124 link-down 5 3
\end{verbatim}
}
These lines above indicate that Link~\tup{3, 5} failed at $0.8123s.$,
and recovered at time $3.5124s.$

\section{The Internal Architecture}
\label{sec:nd-internal-arch}

Each model of network dynamics is implemented as a separate class,
derived from the base \clsref{rtModel}{../ns-2/dynamics.tcl}.
We begin by describing
\href{the base class rtModel and the derived classes}{Section}{sec:rtmodel}.
The network dynamics models use an internal queuing structure
to ensure that simultaneous events are correctly handled,
the \clsref{rtQueue}{../ns-2/dynamics.tcl}.
\href{The next subsection}{Section}{sec:rtqueue}
describes the internals of this structure.
Finally, we describe
\href{the extensions to the existing classes}{Section}{sec:nd-extensions}:
the Node, Link, and others.

\subsection{The class rtModel}
\label{sec:rtmodel}

To use a new route model, the routine \proc[]{rtmodel}
creates an instance of the appropriate type,
defines the node or link that the model will operate upon,
configures the model,
and possibly enables tracing;
The individual instance procedures that accomplish this in pieces are:
\begin{list}{}{}
\item The 
  \fcnref{constructor for the base class}{../ns-2/dynamics.tcl}{rtModel::init}
  stores a reference to the Simulator in its instance variable, \code{ns_}.
  It also initializes the \code{startTime_} and \code{finishTime_}
  from the class variables of the same name.
\item The instance procedure 
  \fcnref{set-elements}{../ns-2/dynamics.tcl}{rtModel::set-elements}
  identifies the node or link that the model will operate upon.
  The command stores two arrays: \code{links_}, of the links that the
  model will act upon; \code{nodes_}, of the incident nodes
  that will be affected by the link failure or recovery caused by the model.
\item The default procedure in the base class
   to set the model configuration parameters is
  \fcnref{set-parms}{../ns-2/dynamics.tcl}{rtModel::set-parms}.
  It assumes a well defined
  start time, up interval, down interval, and a finish time,
  and sets up configuration parameters for some class of models.
  It stores these values in the instance variables:
  \code{startTime_}, \code{upInterval_}, \code{downInterval_},
  \code{finishTime_}.
    The exponential and deterministic models use this default routine,
  the trace based and manual models define their own procedures.
\item % trace
  The instance procedure
  \fcnref{\proc[]{trace}}{../ns-2/dynamics.tcl}{rtModel::trace}
  enables \proc[]{trace-dynamics} on each of the links that it affects.
  Additional details on \proc[]{trace-dynamics} is discussed in the
  \href{section on extensions to the class Link}{Section}{sec:nd-extensions}.
\end{list}
The next sequence of configuration steps are taken just prior to
the start of the simulator.
\ns\ invokes 
\fcnref{\proc[]{rtmodel-configure}}{../ns-2/dynamics.tcl}{Simulator::rtmodel-configure}
just before starting the simulation.
This instance procedure first acquires an instance of the class rtQueue,
and then invokes \proc[]{configure} for each route model in its list,
\code{rtModel_}.
\begin{list}{}{}
\item The instance procedure
  \fcnref{\proc[]{configure}}{../ns-2/dynamics.tcl}{rtModel::configure}
  makes each link that is is applied to dynamic;
  this is the set of links stored in its instance variable array,
  \code{links_}.
  Then the procedure schedules its first event.
\item The default instance procedure
  \fcnref{\proc[]{set-first-event}}{../ns-w/dynamics.tcl}{rtModel::set-first-event}
  schedules the first event to take all the links ``down'' at \\
  \code{$startTime_} + \code{upInterval_}.
  Individual types of route models derived from this base class should
  redefine tihs function.
\item Two instance procedures in the base class ,
  \fcnref{\proc[]{set-event}}{../ns-2/dynamics.tcl}{rtModel::set-event} and
  \fcnref{\proc[]{set-event-exact}}{../ns-2/dynamics.tcl}{rtModel::set-event-exact},
  can be used to schedule events in the route queue.

  \proc[interval, operation]{set-event} schedules \code{operation}
  after \code{interval} seconds from the current time; it uses the
  procedure \proc[]{set-event-exact} below.

  \proc[fireTime, operation]{set-event-exact} schedules \code{operation}
  to execute at \code{fireTime}.

  If the time for execution is greater than the \code{finishTime_},
  then the only possible action is to take a failed link ``up''.

\item  Finally, the base class provides the methods to take the links
  \fcnref{\proc[]{up}}{../ns-2/dynamics.tcl}{rtModel::up} or
  \fcnref{\proc[]{down}}{../ns-2/dynamics.tcl}{rtModel::down}.
  Each method invokes the appropriate procedure on each of the links
  in the instance variable, \code{links_}.
\end{list}

\paragraph{Exponential}
The model schedules its first event to take the links down
at \code{startTime_} + E(\code{upInterval_});

It also defines the procedures, \proc[]{up} and \proc[]{down};
each procedure invokes the base class procedure to perform the actual operation.
This routine then reschedules the next event at
E(\code{upInterval}) or E(\code{downInterval_}) respectively.

\paragraph{Deterministic}
The model defines the procedures, \proc[]{up} and \proc[]{down};
each procedure invokes the base class procedure to perform the actual operation.
This routine then reschedules the next event at
\code{upInterval} or \code{downInterval_} respectively.

\paragraph{Trace}
The model redefines the instance procedure
\fcnref{\proc[]{set-parms}}{../ns-2/dynamics.tcl}{rtModel/Trace::set-parms}
to operan a trace file, and set events based on that input.

The instance procedure
\fcnref{\proc[]{get-next-event}}{../ns-2/dynamics.tcl}{rtModel/Trace::get-next-event}
returns the next valid event from the trace file.
A valid event is an event that is applicable to one of the links 
in this object's \code{links_} variable.

The instance procedure
\fcnref{\proc[]{set-trace-events}}{../ns-2/dynamics.tcl}{rtModel/Trace::set-trace-events}
uses \proc[]{get-next-event}
to schedule the next valid event.

The model redefines
\fcnref{\proc[]{set-first-event}}{../ns-2/dynamics.tcl}{rtModel/Trace::set-first-event},
\fcnref{\proc[]{up}}{../ns-2/dynamics.tcl}{rtModel/Trace::up}, and
\fcnref{\proc[]{down}}{../ns-2/dynamics.tcl}{rtModel/Trace::down}
to use \proc[]{set-trace-events}.

\paragraph{Manual}
The model is designed to fire exactly once.
The instance procedure
\fcnref{\proc[]{set-parms}}{../ns-2/dynamics.tcl}{rtModel/Manual::set-parms}
takes an operation and the time to execute that operation as arguments.
\fcnref{\proc[]{set-first-event}}{../ns-2/dynamics.tcl}{rtModel/Manual::set-first-event}
will schedule the event at the appropriate moment.

This routine also redefines
\fcnref{\proc[]{notify}}{../ns-2/dynamics.tcl}{rtModel/Manual::notify}
to delete the object instance when the operation is completed.
This notion of the object deleting itself is fragile code.

Since the object only fires once and does nto have to be rescheduled,
it does not overload the procedures \proc[]{up} or \proc[]{down}.

\subsection{\protect\clsref{rtQueue}{../ns-2/dynamics.tcl}}
\label{sec:rtqueue}

The simulator needs to co-ordinate multiple simultaneous network
dynamics events, especially to ensure the right coherent behaviour.
Hence, the network dynamics models use their own internal 
route queue to schedule dynamics events.
There is one instance of this object in the simulator, in the
class Simulator instance variable \code{rtq_}.

The queue object stores an array of queued operations
in its instance variable, \code{rtq_}.
The index is the time at which the event will execute.
Each element is the list of operations that will execute at that time.

The instance procedures
\fcnref{\proc[]{insq}}{../ns-2/dynamics.tcl}{rtQueue::insq} and
\fcnref{\proc[]{insq-i}}{../ns-2/dynamics.tcl}{rtQueue::insq-i}
can insert an element into the queue.
\begin{list}{}{}
\item The first argument is the time at which this operation will execute.
  \proc[]{insq} takes the exact time as argument;
  \proc[]{insq-i} takes the interval as argument, and schedules the
  operation \code{interval} seconds after the current time.
\item The following arguments specify the object, \code{$obj},
  the instance procedure of that object, \code{$iproc},
  and the arguments to that procedure, \code{$args}.

  These arguments are placed into the route queue
  for execution at the appropriate time.
\end{list}

The instance procedure
\fcnref{\proc[]{runq}}{../ns-2/dynamics.tcl}{rtQueue::runq}
executes \code{eval $obj $iproc $args} at the appropriate instant.
After all the events for that instance are executed,
\proc[]{runq} will \proc[]{notify} each object about the execution.

Finally, the instance procedure
\fcnref{\proc[]{delq}}{../ns-2/dynamics.tcl}{rtQueue::delq}
can remove a queued action with the time and the name of the object.

\section{Interaction with Unicast Routing}
\label{sec:unicast-int}

In an earlier section,
we had described how
\href{unicast routing reacts}{Section}{sec:rtglibAPI}
to changes to the topology.
This section details the steps by which 
the network dynamics code will notify the nodes and routing
about the changes to the topology.
\begin{enumerate}
\item \proc[]{rtQueue::runq} will invoke the procedures
  specified by each of the route model instances.
  After all of the actions are completed,
  \proc[]{runq} will notify each of the models.
\item
  \fcnref{\proc[]{notify}}{../ns-2/dynamics.tcl}{rtModel::notify}
  will then invoke instance procedures at all of the nodes
  that were incident to the affected links.
  Each route model stores the list of nodes in its instance variable
  array, \code{nodes_}.

  It will then notify the RouteLogic instance of topology changes.
\item
  The rtModel object invokes the class Node instance procedure
  \fcnref{\proc[]{intf-changed}}{../ns-2/dynamics.tcl}{Node::intf-changed}
  for each of the affected nodes.
\item
  \proc[]{Node::intf-changed} will notify any \code{rtObject}
  at the node of the possible changes to the topology.

  Recall that these route objects are created when the simulation uses
  detailed dynamic unicast routing.
\end{enumerate}

\subsection{Extensions to Other Classes}
\label{sec:nd-extensions}

The existing classes assume that the topology is static by default.
In this section, we document the necessary changes to these
classes to support dynamic topologies.

We have already described the instance procedures
in the \clsref{Simulator}{../ns-2/ns-lib.tcl} to create or manipulate
route models, \ie,
\proc[]{rtmodel}, \proc[]{rtmodel-at}, \proc[]{rtmodel-delete}, and
\proc[]{rtmodel-configure} \href{in earlier sections}{Section}{sec:rtmodel}.
Similarly, the \clsref{Node}{../ns-2/ns-node.tcl}
contains the instance procedure \proc[]{intf-changed}
that we described in \href{the previous section}{Section}{sec:unicast-int}.

The network dynamics code operates on individual links.
Each model currently translates its specification into
operations on the appropriate links.
The following paragraphs describe the class Link and related classes.

\paragraph{\protect\clsref{DynamicLink}{../ns-2/dynalink.cc.tcl}}
This class is the only TclObject in the network dynamics code.
The shadow class is called \clsref{DynaLink}{../ns-2/dynalink.h}.
The class supports one bound variable, \code{status_}.
\code{status_} is 1 when the link is up, and 0 when the link is down.
The shadow object's \fcnref{\fcn[]{recv}}{../ns-2/dynalink.cc}{DynaLink::recv}
method checks the \code{status_} variable, to decide whether or not
a packet should be forwarded.

\paragraph{\protect\clsref{Link}{../ns-2/ns-link.tcl}}
This class supports the primitives:
up and down, and up? to set and query \code{status_}.
These primitives are instance procedures of the class.
\begin{list}{}{}
\item  The instance procedures
  \fcnref{\proc[]{up}}{../ns-2/dynamics.tcl}{Link::up} and
  \fcnref{\proc[]{down}}{../ns-2/dynamics.tcl}{Link::down}
  set \code{status_} to 1 and 0 respectively.

  In addition, when the link fails, \proc[]{down}
  will reset all connectors that make up the link.
  Each connector, including all queues and the delay object
  will flush and drop any packets that it currently stores.
  This emulates the packet drop due to link failure.

  Both procedures then write trace entries to each file handle
  in the list, \code{dynT_}.

\item The instance procedure
  \fcnref{\proc[]{up?}}{../ns-2/dynamics.tcl}{Link::up?}
  returns the current value of \code{status_}.
\end{list}
In addition, the class contains the instance procedure
\fcnref{\proc[]{all-connectors}}{../ns-2/dynamics.tcl}{Link::all-connectors}.
This procedure takes an operation as argument, and applies
the operation uniformly to all of the class instance variables
that are handles for TclObjects.

\paragraph{\protect\clsref{SimpleLink}{../ns-2/ns-link.tcl}}
The class supports two instance procedures
\fcnref{\proc[]{dynamic}}{../ns-2/dynamics.tcl}{SimpleLink::dynamic} and
\fcnref{\proc[]{trace-dynamics}}{../ns-2/dynamics.tcl}{SimpleLink::trace-dynamics}.
We have already described the latter procedure when describing the
\proc[]{trace} procedure in the class rtModel.

The instance procedure \proc[]{dynamic} inserts a 
\href{DynamicLink object}{Section}{sec:links:connectors}
at the head of the queue.
It points the down-target of the object to the 
drop target of the link, \code{drpT_}, if the object is defined,
or to the \code{nullAgent_} in the simulator.
It also signals each connector in the link that the link is now
dynamic.

Most connectors ignore this signal to be become dynamic;
the exception is \code{DelayLink} object.
This object will normally schedule each packet it receives
for reception by the destination node at the appropriate time.
When the link is dynamic, the object will queue each packet 
internally; it schedules only one event for the next packet
that will be delivered, instead of one event per packet normally.
If the link fails, the route model will signal a \code{reset},
at which point, the shadow object will execute its
\fcnref{reset instproc-like}{../ns-2/dynalink.cc}{DynaLink::command},
and flush all packets in its internal queue.
Additional details about the DelayLink can be found
\href{in another chapter}{Chapter}{chap:delays}.

\section{Deficencies in the Current Network Dynamics API}
\label{sec:deficiencies}

There are a number of deficencies in the current API that should be
changed in the next iteration:
\begin{enumerate}
\item  There is no way to specify a cluster of nodes or links that
behave in lock-step dynamic synchrony.
\item  Node failure should be dealt with as its own mechanism,
rather than a second grade citizen of link failure.
This shows up in a number of situations, such as:
\begin{enumerate}
\item  The method of emulating node failure as the failure of the
incident links is broken.  Ideally, node failure should cause all
agents incident on the node to be reset.
\item  There is no tracing associated with node failure.
\end{enumerate}
\item  If two distinct route models are applied to two separate links
incident on a common node, and the two links experience a topology change
at the same instant, then the node will be notified more than once.
\end{enumerate}



\section{Commands at a glance}
\label{sec:dynamicscommand}

Following is a list of commands used to simulate dynamic scenarios in \ns:

\begin{flushleft}
\code{$ns_ rtmodel <model> <model-params> <args>}\\
This command defines the dynamic model (currently implemented models are:
Deterministic, Exponential, Manual or Trace) to be applied to nodes and
links in the topology. The first two arguments consists of the rtmodel and
the parameter to configure the model. <args> stands for different type of
arguments expected with different dynamic model types. This returns a
handle to a model object corresponding to the specified model. 
\begin{itemize}
\item In the Deterministic model <model-params> is <start-time>, 
<up-interval>, <down-interval>, <finish-time>. Starting from start-time
the link is made up for up-interval and down for down-interval till
finish-time is reached. The default values for start-time, up-interval,
downinterval are 0.5s, 2.0s, 1.0s respectively. finishtime defaults to the
end of the simulation. The start-time defaults to 0.5s in order to let the
routing protocol computation quiesce. 

\item If the Exponential model is used model-params is of the form
<up-interval>, <down-interval> where the link up-time is an exponential
distribution around the mean upinterval and the link down-time is an
exponential distribution around the mean down-interval. Default values for
up-interval and down-interval are 10s and 1s respectively. 

\item If the Manual distribution is used model-params is <at> <op> where
at
specifies the time at which the operation op should occur. op is one of
up, down. The Manual distribution could be specified alternately using the
rtmodel-at method described later in the section. 

\item If Trace is specified as the model the link/node dynamics is read
from a
Tracefile. The model-params argument would in this case be the file-handle
of the Tracefile that has the dynamics information. The tracefile format
is identical to the trace output generated by the trace-dynamics link
method (see TRACE AND MONITORING METHODS SECTION). 
\end{itemize}


\code{$ns_ rtmodel-delete <model>}\\
This command takes the handle of the routemodel <model> as an argument,
removes  it from the list of rtmodels maintained by simulator and deletes
the model.


\code{$ns_ rtmodel-at  <at> <op> <args>}\\
This command is a special interface to the Manual model of network dynamics.
It takes the time <at>, type of operation <op> and node or link on which
to apply the operation <args> as the arguments. At time <at>, the operation <op>
which maybe up or down is applied to a node or link.

\code{$rtmodel trace <ns> <f> <optional:op>}\\
This enables tracing of dynamics effected by this model in the links. <ns>
is an instance of the simulator, <f> the output file to write the traces to
and <op> is an optional argument that may be used to define a type of
operation (like nam). This is a wrapper for the class Link procedure
\code{trace-dynamics}.


\code{$link trace-dynamics <ns> <f> <optional:op>}\\
This is a class link instance procedure that is used to setup tracing of
dynamics in that particular link. The arguments are same as that of class
rtModel's procedure \code{trace} described above.


\code{$link dynamic}\\
This command inserts a DynamicLink object at the head of the queue and signals
to all connectors in the link that the link is now dynamic.


Internal procedures:\\

\code{$ns_ rtmodel-configure}\\
This is an internal procedure that configures all dynamic models that are
present in the list of models maintained by the simulator.

\end{flushleft}

\endinput

### Local Variables:
### mode: latex
### comment-column: 60
### backup-by-copying-when-linked: t
### file-precious-flag: nil
### End:


\part{Transport}
\chapter{TCP Agents}
\label{sec:tcpAgents}

This section describes the operation of the TCP agents in \ns.
There are two major types of TCP agents: one-way agents
and a two-way agent.
One-way agents are further subdivided into a set of TCP senders
(which obey different congestion and error control techniques)
and receivers (``sinks'').
The two-way agent is symmetric in the sense that it represents
both a sender and receiver.
It is still under development.

The files described in this section are too numerous to enumerate here.
Basically it covers most files matching the regular expression
\nsf{tcp*.\{cc, h\}}.

The one-way TCP sending agents currently supported are:
\begin{itemize}\itemsep0pt
        \item Agent/TCP - a ``tahoe'' TCP sender
        \item Agent/TCP/Reno - a ``Reno'' TCP sender
        \item Agent/TCP/NewReno - Reno with a modification
        \item Agent/TCP/Sack1 - TCP with selective repeat (follows RFC2018)
        \item Agent/TCP/Vegas - TCP Vegas
        \item Agent/TCP/Fack - Reno TCP with ``forward acknowledgment''
\end{itemize}
The one-way TCP receiving agents currently supported are:
\begin{itemize}\itemsep0pt
        \item Agent/TCPSink - TCP sink with one ACK per packet
        \item Agent/TCPSink/DelAck - TCP sink with configurable delay per ACK
        \item Agent/TCPSink/Sack1 - selective ACK sink (follows RFC2018)
        \item Agent/TCPSink/Sack1/DelAck - Sack1 with DelAck
\end{itemize}
The two-way experimental sender currently supports only a Reno form of TCP:
\begin{itemize}
        \item Agent/TCP/FullTcp
\end{itemize}

The section comprises three parts:
the first part is a simple overview and example of configuring
the base TCP send/sink agents (the sink requires no configuration).
The second part describes the internals of the base send agent,
and last part is a description of the extensions
for the other types of agents that have been included in the
simulator.

\section{One-Way TCP Senders}
\label{sec:oneWayTcp}

The simulator supports several versions of an abstracted TCP sender.
These objects attempt to capture the essence of the TCP congestion
and error control behaviors, but are not intended to be faithful
replicas of real-world TCP implementations.
They do not contain a dynamic window advertisement, they do segment
number and ACK number computations entirely in packet units,
there is no SYN/FIN connection establishment/teardown, and no
data is ever transferred (e.g. no checksums or urgent data).

\subsection{The Base TCP Sender (Tahoe TCP)}
\label{sec:tahoetcp}

The ``Tahoe'' TCP agent \code{Agent/TCP} performs congestion
control and round-trip-time estimation
in a way similar to the version of TCP released with the
4.3BSD ``Tahoe'' UN'X system release from UC Berkeley.
The congestion window is increased by one packet per new ACK received
during slow-start (when $cwnd\_ < ssthresh\_$) and is increased
by $\frac{1}{cwnd\_}$ for each new ACK received during congestion avoidance
(when $cwnd\_ \geq ssthresh\_$).

\paragraph{Responses to Congestion}
Tahoe TCP assumes a packet has been lost (due to congestion)
when it observes {\tt NUMDUPACKS} (defined in \code{tcp.h}, currently 3)
duplicate ACKs, or when a retransmission timer expires.
In either case, Tahoe TCP reacts by setting {\tt ssthresh\_} to half
of the current window size (the minimum of {\tt cwnd\_} and {\tt window\_})
or 2, whichever is larger.
It then initializes {\tt cwnd\_} back to the value of
{\tt windowInit\_}.  This will typically cause the TCP to
enter slow-start.

\paragraph{Round-Trip Time Estimation and RTO Timeout Selection}
Four variables are used to estimate the round-trip time and
set the retransmission timer: {\tt rtt\_, srtt\_, rttvar\_, tcpTick\_,
and backoff\_}.
TCP initializes rttvar to $3/tcpTick\_$ and backoff to 1.
When any future retransmission timer is set, it's timeout
is set to the current time plus $\max(bt(a+4v+1), 64)$ seconds,
where $b$ is the current backoff value, $t$ is the value of tcpTick,
$a$ is the value of srtt, and $v$ is the value of rttvar.

Round-trip time samples arrive with new ACKs.
The RTT sample is computed as the difference between the current
time and a ``time echo'' field in the ACK packet.
When the first sample is taken, its value is used as the initial
value for {\tt srtt\_}.  Half the first sample is used as the initial
value for {\tt rttvar\_}.
For subsequent samples, the values are updated as follows:

\[ srtt = \frac{7}{8} \times srtt + \frac{1}{8} \times sample \]
\[ rttvar = \frac{3}{4} \times rttvar + \frac{1}{4} \times |sample-srtt| \]

\subsection{Configuration}
\label{sec:tcp-config}

Running an TCP simulation requires
creating and configuring the agent,
attaching an application-level data source (a traffic generator), and
starting the agent and the traffic generator.

\subsection{Simple Configuration}

\paragraph{Creating the Agent}
\begin{program}
set ns [new Simulator]                  \; preamble initialization;
set node1 [$ns node]                     \; agent to reside on this node;
set node2 [$ns node]                     \; agent to reside on this node;

{\bfseries{}set tcp1 [$ns create-connection TCP $node1 TCPSink $node2 42]}
$tcp  set window_ 50                   \; configure the TCP agent;

{\bfseries{}set ftp1 [new Application/FTP]}
{\bfseries{}$ftp1 attach-agent $tcp1}

$ns at 0.0 "$ftp start"
\end{program}
This example illustrates the use of the simulator built-in
function {\tt create-connection}.
The arguments to this function are: the source agent to create,
the source node, the target agent to create, the target node, and
the flow ID to be used on the connection.
The function operates by creating the two agents, setting the
flow ID fields in the agents, attaching the source and target agents
to their respective nodes, and finally connecting the agents
(i.e. setting appropriate source and destination addresses and ports).
The return value of the function is the name of the source agent created.

\paragraph{TCP Data Source}
The TCP agent does not generate any application data on its own;
instead, the simulation user can connect any traffic generation
module to the TCP agent to generate data.
Two applications are commonly used for TCP: FTP and Telnet.
FTP represents a bulk data transfer of large size, and telnet chooses
its transfer sizes randomly from tcplib (see the file
\code{tcplib-telnet.cc}.
Details on configuring these application source objects are in
Section~\ref{sec:simapps}.

\subsection{Other Configuration Parameters}

In addition to the \code{window_} parameter listed above,
the TCP agent supports additional configuration variables.
Each of the variables described in this subsection is
both a class variable and an instance variable.
Changing the class variable changes the default value
for all agents that are created subsequently.
Changing the instance variable of a particular agent
only affects the values used by that agent.
For example,
\begin{program}
  Agent/TCP set window_ 100     \; Changes the class variable;
  $tcp set window_ 2.0          \; Changes window_ for the $tcp object only;
\end{program}

The default parameters for each TCP agent are:
\begin{program}
Agent/TCP set window_   20              \; max bound on window size;
Agent/TCP set windowInit_ 1             \; initial/reset value of cwnd;
Agent/TCP set windowOption_ 1           \; cong avoid algorithm (1: standard);
Agent/TCP set windowConstant_ 4         \; used only when windowOption != 1;
Agent/TCP set windowThresh_ 0.002       \; used in computing averaged window;
Agent/TCP set overhead_ 0               \; !=0 adds random time between sends;
Agent/TCP set ecn_ 0                    \; TCP should react to ecn bit ;
Agent/TCP set packetSize_ 1000          \; packet size used by sender (bytes);
Agent/TCP set bugFix_ true              \; see explanation;
Agent/TCP set slow_start_restart_ true  \; see explanation;
Agent/TCP set tcpTick_ 0.1              \; timer granulatiry in sec (.1 is NONSTANDARD);
Agent/TCP set maxrto_ 64                \; bound on RTO (seconds);
Agent/TCP set dupacks_ 0                \; duplicate ACK counter;
Agent/TCP set ack_ 0                    \; highest ACK received;
Agent/TCP set cwnd_ 0                   \; congestion window (packets);
Agent/TCP set awnd_ 0                   \; averaged cwnd (experimental);
Agent/TCP set ssthresh_ 0               \; slow-stat threshold (packets);
Agent/TCP set rtt_ 0                    \; rtt sample;
Agent/TCP set srtt_ 0                   \; smoothed (averaged) rtt;
Agent/TCP set rttvar_ 0                 \; mean deviation of rtt samples;
Agent/TCP set backoff_ 0                \; current RTO backoff factor;
Agent/TCP set maxseq_ 0                 \; max (packet) seq number sent;
\end{program}

For many simulations, few of the configuration parameters are likely
to require modification.
The more commonly modified parameters include: {\tt window\_} and
{\tt packetSize\_}.
The first of these bounds the window TCP uses, and is considered
to play the role of the receiver's advertised window in real-world
TCP (although it remains constant).
The packet size essentially functions like the MSS size in real-world
TCP.
Changes to these parameters can have a profound effect on the behavior
of TCP.
Generally, those TCPs with larger packet sizes, bigger windows, and
smaller round trip times (a result of the topology and congestion) are
more agressive in acquiring network bandwidth.

\subsection{Other One-Way TCP Senders}

\paragraph{Reno TCP}
The Reno TCP agent is very similar to the Tahoe TCP agent,
except it also includes {\em fast recovery}, where the current
congestion window is ``inflated'' by the number of duplicate ACKs
the TCP sender has received before receiving a new ACK.
A ``new ACK'' refers to any ACK with a value higher than the higest
seen so far.
In addition, the Reno TCP agent does not return to slow-start during
a fast retransmit.
Rather, it reduces sets the congestion window to half the current
window and resets {\tt ssthresh\_} to match this value.

\paragraph{NewReno TCP}
This agent is based on the Reno TCP agent, but which modifies the
action taken when receiving new ACKS.
In order to exit fast recovery, the sender must receive an ACK for the
highest sequence number sent.
Thus, new ``partial ACKs'' (those which represent new ACKs but do not
represent an ACK for all outstanding data) do not deflate the window
(and possibly lead to a stall, characteristic of Reno).

\paragraph{Vegas TCP}
This agent implements ``Vegas'' TCP (\cite{Brak94:TCP,Brak94a:TCP}).
It was contributed by Ted Kuo.

\paragraph{Sack TCP}
This agent implements selective repeat, based on selective ACKs provided
by the receiver.
It follows the ACK scheme described in \cite{rfc2018}, and was developed
with Matt Mathis and Jamshid Mahdavi.

\paragraph{Fack TCP}
This agent implements ``forward ACK'' TCP, a modification of Sack
TCP described in \cite{Math96:Forward}.

\section{TCP Receivers (sinks)}

The TCP senders described above represent one-way data senders.
They must peer with a ``TCP sink'' object.

\subsection{The Base TCP Sink}

The base TCP sink object ({\tt Agent/TCPSink})
is responsible for returning ACKs to
a peer TCP source object.
It generates one ACK per packet received.
The size of the ACKs may be configured.
The creation and configuration of the TCP sink object
is generally performed automatically by a library
call (see {\tt create-connection} above).

\paragraph{configuration parameters}
\begin{program}
        Agent/TCPSink set packetSize_ 40
\end{program}

\subsection{Delayed-ACK TCP Sink}

A delayed-ACK sink object ({\tt Agent/Agent/TCPSink/DelAck}) is available
for simulating a TCP receiver that ACKs less than once per packet received.
This object contains a bound variable {\tt interval\_} which gives the
number of seconds to wait between ACKs.
The delayed ACK sink implements an agressive ACK policy whereby
only ACKs for in-order packets are delayed.
Out-of-order packets cause immediate ACK generation.

\paragraph{configuration parameters}
\begin{program}
        Agent/TCPSink/DelAck set interval_ 100ms
\end{program}

\subsection{Sack TCP Sink}

The selective-acknowledgment TCP sink ({\tt Agent/TCPSink/Sack1}) implements
SACK generation modeled after the description of SACK in RFC 2018.
This object includes a bound variable {\tt maxSackBlocks\_} which gives
the maximum number of blocks of information in an ACK available for
holding SACK information.
The default value for this variable is 3, in accordance with the expected
use of SACK with RTTM (see RFC 2018, section 3).
Delayed and selective ACKs together are implemented by
an object of type {\tt Agent/TCPSink/Sack1/DelAck}.

\paragraph{configuration parameters}
\begin{program}
        Agent/TCPSink set maxSackBlocks_ 3
\end{program}

\section{Two-Way TCP Agents (FullTcp)}
\label{sec:fulltcp}

The {\tt Agent/TCP/FullTcp} object is a new addition to the suite of
TCP agents supported in the simulator and is still under development.
It is different from (and incompatible with) the other agents, but
does use some of the same architecture.
It differs from these agents in the following ways:
following ways:
\begin{itemize}\itemsep0pt
\item connections may be establised and town down
(SYN/FIN packets are exchanged)
\item bidirectional data transfer is supported
\item sequence numbers are in bytes rather than packets
\end{itemize}

The generation of SYN packets (and their ACKs) can be
of critical importance in trying to model real-world behavior
when using many very short data transfers.
This version of TCP currently defaults to sending
data on the 3rd segment of an initial 3-way handshake, a behavior
somewhat different than common real-world TCP implementations.
A ``typical'' TCP connection proceeds with an active opener
sending a SYN, the passive opener responding with a SYN+ACK,
the active opener responding with an ACK, and then some time later
sending the first segment with data (corresponding to the first
application write).
Thus, this version of TCP sends data at a time somewhat earlier
than typical implementations.
This TCP can also be configured to send data on the initial SYN
segment.
Future changes to FullTCP may include a modification to send the
first data segment later, and possibly to implement T/TCP functionality.

Currently FullTCP is only implemented with Reno congestion control,
but ultimately it should be available with the full range of
congestion control algorithms (e.g., Tahoe, SACK, Vegas, etc.).


\subsection{Simple Configuration}
Running an Full TCP simulation requires
creating and configuring the agent,
attaching an application-level data source (a traffic generator), and
starting the agent and the traffic generator.

\paragraph{Creating the Agent}
\begin{program}
# {\cf set up connection (do not use "create-connection" method because }
# {\cf we need a handle on the sink object)}
set src [new Agent/TCP/FullTcp] \; create agent;
set sink [new Agent/TCP/FullTcp] \; create agent;
$ns_ attach-agent $node_(s1) $src \; bind src to node;
$ns_ attach-agent $node_(k1) $sink \; bind sink to node;
$src set fid_ 0   \; set flow ID field;
$sink set fid_ 0  \; set flow ID field;
$ns_ connect $src $sink \; active connection src to sink;

# {\cf set up TCP-level connections}
$sink listen \; will figure out who its peer is;
$src set window_ 100;
\end{program}

The creation of the FullTcp agent is similar to the other agents,
but the sink is placed in a listening state by the {\tt listen} method.
Because a handle to the receiving side is required in order to make
this call, the {\tt create-connection} call used above cannot be used.

\paragraph{Configuration Parameters}
The following configuration parameters are available through Tcl
for the FullTcp agent:
\begin{program}
Agent/TCP/FullTcp set segsperack_ 1 \; segs received before generating ACK;
Agent/TCP/FullTcp set segsize_ 536  \; segment size (MSS size for bulk xfers);
Agent/TCP/FullTcp set tcprexmtthresh_ 3 \; dupACKs thresh to trigger fast rexmt;
Agent/TCP/FullTcp set iss_ 0 \; initial send sequence number;
Agent/TCP/FullTcp set nodelay_ false \; disable sender-side Nagle algorithm;
Agent/TCP/FullTcp set data_on_syn_ false \; send data on initial SYN?;
Agent/TCP/FullTcp set dupseg_fix_ true \; avoid fast rxt due to dup segs+acks;
Agent/TCP/FullTcp set dupack_reset_ false \; reset dupACK ctr on !0 len data segs containing dup ACKs;
Agent/TCP/FullTcp set interval_ 0.1 \; as in TCP above, (100ms is non-std);
\end{program}

\section{Architecture and Internals}
\label{sec:tcparchitecture}

The base TCP agent (class {\tt Agent/TCP}) is constructed
as a collection of routines for sending packets, processing ACKs,
managing the send window, and handling timeouts.
Generally, each of these routines may be over-ridden by a
function with the same name in a derived class (this is how
many of the TCP sender variants are implemented).

\paragraph{The TCP header}
The TCP header is defined by the {\tt hdr\_tcp} structure
in the file \nsf{tcp.h}.
The base agent only makes use of the following subset of the fields:
\begin{program}
ts_     \* current time packet was sent from source */
ts_echo_ \* for ACKs: timestamp field from packet associated with this ACK */
seqno_ \* sequence number for this data segment or ACK (Note: overloading!) */
reason_ \* set by sender when (re)transmitting to trace reason for send */
\end{program}

\paragraph{Functions for Sending Data}
Note that generally the sending TCP never actually sends
data (it only sets the packet size).

{\bf send\_much(force, reason, maxburst)} - this function
attempts to send as many packets as the current sent window allows.
It also keeps track of how many packets it has sent, and limits to the
total to {\em maxburst}. \\
The function {\tt output(seqno, reason)} sends one packet
with the given sequence number and updates the maximum sent sequence
number variable ({\tt maxseq\_}) to hold the given sequence number if
it is the greatest sent so far.
This function also assigns the various fields in the TCP
header (sequence number, timestamp, reason for transmission).
This function also sets a retransmission timer if one is not already
pending.

\paragraph{Functions for Window Management}

The usable send window at any time is given by the function {\bf window()}.
It returns the minimum of the congestion window and the variable {\tt wnd\_},
which represents the receiver's advertised window.

{\bf opencwnd()} - this function opens the congestion window.  It is invoked
when a new ACK arrives.
When in slow-start, the function merely increments {\tt cwnd\_} by each
ACK received.
When in congestion avoidance, the standard configuration increments {\tt cwnd\_}
by its reciprocal.
Other window growth options are supported during congestion avoidance,
but they are experimental (and not documented; contact Sally Floyd for
details).

{\bf closecwnd(int how)} - this function reduces the congestion window. It
may be invoked in several ways: when entering fast retransmit, due to
a timer expiration, or due to a congestion notification (ECN bit set).
Its argument {\tt how} indicates how the congestion window should
be reduced.  The value {\bf 0} is used for retransmission timeouts and
fast retransmit in Tahoe TCP.  It typically causes the TCP to enter
slow-start and reduce {\tt ssthresh\_} to half the current window.
The value {\bf 1} is used by Reno TCP for implementing fast recovery
(which avoids returning to slow-start).
The value {\bf 2} is used for reducing the window due to an ECN indication.
It resets the congestion window to its initial value (usually causing
slow-start), but does not alter {\tt ssthresh\_}.

\paragraph{Functions for Processing ACKs}

{\bf recv()} - this function is the main reception path for ACKs.
Note that because only one direction of data flow is in use, this function
should only ever be invoked with a pure ACK packet (i.e. no data).
The function stores the timestamp from the ACK in {\tt ts\_peer\_}, and
checks for the presence of the ECN bit (reducing the send window if
appropriate).
If the ACK is a new ACK, it calls {\bf newack()}, and otherwise
checks to see if it is a duplicate of the last ACK seen.
If so, it enters fast retransmit by closing the window, resetting the
retransmission timer, and sending a packet by calling {\tt send\_much}.

{\bf newack()} - this function processes a ``new'' ACK (one that contains
an ACK number higher than any seen so far).
The function sets a new retransmission timer by calling {\bf newtimer()},
updates the RTT estimation by calling {\bf rtt\_update}, and updates
the highest and last ACK variables.

\paragraph{Functions for Managing the Retransmission Timer}

These functions serve two purposes: estimating the round-trip time
and setting the actual retransmission timer.
{\bf rtt\_init} - this function initializes {\tt srtt\_} and {\tt rtt\_}
to zero, sets {\tt rttvar\_} to $3/tcp\_tick\_$, and sets the backoff
multiplier to one.

{\bf rtt\_timeout} - this function gives the timeout value in seconds that
should be used to schedule the next retransmission timer.
It computes this based on the current estimates of the mean and deviation
of the round-trip time.  In addition, it implements Karn's
exponential timer backoff for multiple consecutive retransmission timeouts.

{\bf rtt\_update} - this function takes as argument the measured RTT
and averages it in to the running mean and deviation estimators
according to the description above.
Note that {\tt t\_srtt\_} and {\tt t\_rttvar} are both
stored in fixed-point (integers).
They have 3 and 2 bits, respectively, to the right of the binary
point.

{\bf reset\_rtx\_timer} -  This function is invoked during fast retransmit
or during a timeout.
It sets a retransmission timer
by calling {\tt set\_rtx\_timer} and if invoked by a timeout also calls
{\tt rtt\_backoff}.

{\bf rtt\_backoff} - this function backs off the retransmission timer
(by doubling it).

{\bf newtimer} - this function called only when a new ACK arrives.
If the sender's left window edge is beyond the ACK, then
{\tt set\_rtx\_timer} is called, otherwise if a retransmission timer
is pending it is cancelled.

\section{Tracing TCP Dynamics}
\label{sec:traceTcpdyn}

The behavior of TCP is often observed by constructing a
sequence number-vs-time plot.
Typically, a trace is performed by enabling tracing on a link
over which the TCP packets will pass.
Two trace methods are supported: the default one (used for tracing
TCP agents), and an extension used only for FullTcP.

\section{One-Way Trace TCP Trace Dynamics}
\label{sec:trace1WayTcpdyn}

TCP packets generated by one of the one-way TCP agents and destined for
a TCP sink agent
passing over a traced link (see section~\ref{chap:trace})
will generate a trace file lines of the form:
\begin{verbatim}
+ 0.94176 2 3 tcp 1000 ------ 0 0.0 3.0 25 40
+ 0.94276 2 3 tcp 1000 ------ 0 0.0 3.0 26 41
d 0.94276 2 3 tcp 1000 ------ 0 0.0 3.0 26 41
+ 0.95072 2 0 ack 40 ------ 0 3.0 0.0 14 29
- 0.95072 2 0 ack 40 ------ 0 3.0 0.0 14 29
- 0.95176 2 3 tcp 1000 ------ 0 0.0 3.0 21 36
+ 0.95176 2 3 tcp 1000 ------ 0 0.0 3.0 27 42
\end{verbatim}
The exact format of this trace file is given in section~\ref{sec:traceformat}.
When tracing TCP, packets of type {\sf tcp} or {\sf ack} are relevant.
Their type, size, sequence number (ack number for ack packets),
and arrival/depart/drop time are given by field positions
5, 6, 11, and 2, respectively.
The {\sf +} indicates a packet arrival, {\sf d} a drop, and {\sf -} a
departure.
A number of scripts process this file to produce graphical output or
statistical summaries (see,  for example, \nsf{test-suite.tcl}, the
{\tt finish} procedure.

\section{One-Way Trace TCP Trace Dynamics}
\label{sec:tcpdyn}

TCP packets generated by FullTcp and
passing over a traced link contain additional information not displayed
by default using the regular trace object.
By enabling the flag {\tt show\_tcphdr\_} on the trace object
(see section~ref{sec:traceformat}), 3 additional header fields are
written to the trace file: ack number, tcp-specific flags, and header length.

\endinput

\documentclass{article}

\usepackage{times}
\usepackage[T1]{fontenc}

\PassOptionsToPackage{draft}{nsDoc}
\usepackage{nsDoc}

\begin{document}

\title{\nsTcl\ internals documentation}
\author{%
  Various members of the VINT project \tup{vint@catarina.usc.edu}\\
  Kevin Fall \tup{kfall@ee.lbl.gov}, Editor,\\
  Kannan Varadhan \tup{kannan@catarina.usc.edu}, Editor.}
\date{\today}

\def\c#1{\ensuremath{C_{#1}}}
\def\d#1{\ensuremath{D_{#1}}}

% \maketitle

\section{Agent/SRM}
\label{sec:agent/srm}

This section describes the internals of the SRM implementation in \ns.
The section is in three parts:
the first part is an overview of a minimal SRM configuration,
and a ``complete'' description of the configuration parameters 
of the base SRM agent.
The second part describes the architecture, internals, and the code path
of the base SRM agent.
The last part of the section is a description of the extensions
for other types of SRM agents that have been attempted to date.

\subsection{Configuration}
\label{sec:srm-config}

Running an SRM simulation requires
creating and configuring the agent,
attaching an application-level data source (a traffic generator), and
starting the agent and the traffic generator.

\subsubsection{Trivial Configuration}

\paragraph{Creating the Agent}
\begin{program}
        set ns [new Simulator]          \; preamble initialization;
        $ns enableMcast
        set node [$ns node]                \; agent to reside on this node;
        set group [$ns allocaddr]           \; multicast group for this agent;

        {\bfseries{}set srm [new Agent/SRM]}
        $srm  set dst_ $group            \; configure the SRM agent;
        {\bfseries{}$ns attach-agent $node $srm}

        $srm set fid_ 1                \; optional configuration;
        $srm log [open srmStats.tr w]   \; log statistics in this file;
        $srm trace [open srmEvents.tr w]  \; trace events for this agent;
\end{program}
The key steps in configuring a virgin SRM agent are to assign
its multicast group, and attach it to a node.

Other useful configuration parameters are
to assign a separate flow id to traffic originating from this agent,
to open a log file for statistics, and
a trace file for trace data%
\footnote{%
Note that the trace data can also be used
to gather certain kinds of trace data.
We will illustrate this later.}.

The file
\fcnref{\|tcl/mcast/srm-nam.tcl|}{../ns-2/srm-nam.tcl}{Agent/SRM::send}
contains definitions that overload the agent's \|send| methods;
this separates control traffic originating from the agent by type.
Each type is allocated a separate flowID.
The traffic is separated into session messages (flowid = 40),
requests (flowid = 41), and repair messages (flowid = 42).
The base flowid can be changed by setting global variable \|ctrlFid|
to one less than the desired flowid before sourcing \|srm-nam.tcl|.
To do this, the simulation script must source \|srm-nam.tcl|
before creating any SRM agents.
This is useful for analysis of traffic traces, or
for visualization in nam.

\paragraph{Application Data Handling}
The agent does not generate any application data on its own;
instead, the simulation user can connect any traffic generation
module to any SRM agent to generate data.
The following code demonstrates
how a traffic generation agent can be attached to an SRM agent:
\begin{program}
        set packetSize 210
        set exp0 [new Traffic/Expoo]    \; configure traffic generator;
        $exp0 set packet-size $packetSize
        $exp0 set burst-time 500ms 
        $exp0 set idle-time 500ms
        $exp0 set rate 100k 

        set s0 [new Agent/CBR/UDP]   \; attach traffic generator to application;
        $s0 set fid_ 0
        $s0 attach-traffic $exp0

        {\bfseries{}$srm(0) traffic-source $s0} \; attach application to SRM agent;
        {\bfseries{}$srm(0) set packetSize_ $packetSize} \; to generate repair packets of appropriate size;
\end{program}
The instproc \texttt{\textbf{traffic-source}} specifies the application agent
that will produce data for the SRM agent.
The user can attach any agent;
the only distinguishing criteria is that the destination address must be zero.
The SRM agent will add the SRM headers, 
set the destination address to the multicast group, and
deliver the packet to its target.
The SRM header contains the type of the message,
the identity of the sender,
the sequence number of the message,
and (for control messages), the round for which this message is being sent.
Each data unit in SRM is identified as
\tup{sender's id, message sequence number}.

The SRM agent does not generate its own data;
it does not also keep track of the data sent,
except to record the sequence numbers of messages received
in the event that it has to do error recovery.
Since the agent has no actual record of past data,
it needs to know what packet size to use for each repair message.
Hence, the instance variable \|packetSize_| specifies the size
of repair messages generated by the agent.

\paragraph{Starting the Agent and Traffic Generator}
The agent and the traffic generator must be started separately.
\begin{program}
        {\bfseries{}\fcnref{$srm start}{../ns-2/srm.tcl}{Agent/SRM::start}}
        {\bfseries{}\fcnref{$srm start-source}{../ns-2/srm.tcl}{Agent/SRM::start-source}}
\end{program}
At \|start|, the agent joins the multicast group, and 
starts generating session messages.
The \|start-source| triggers the traffic generator to start sending
data.

\subsubsection{Other Configuration Parameters}
\label{sec:config-param}

In addition to the above parameters,
the SRM agent supports additional configuration variables.
Each of the variables described in this subsection is
both an OTcl class variable and an OTcl object's instance variable.
Changing the class variable changes the default value
for all agents that are created subsequently.
Changing the instance variable of a particular agent
only affects the values used by that agent.
For example,
\begin{program}
                Agent/SRM set D1_ 2.0 \; Changes the class variable;
                $srm set D1_ 2.0        \; Changes D1_ for the particular $srm object only;
\end{program}

The default request and repair timer parameters \cite{Floy95:Reliable}
for each SRM agent are:
\begin{program}
        Agent/SRM set C1_       2.0 \; request parameters;
        Agent/SRM set C2_       2.0
        Agent/SRM set D1_       1.0 \; repair parameters;
        Agent/SRM set D2_       1.0
\end{program}
It is thus possible to trivially obtain two flavours of SRM agents
based on whether the agents use probabilistic or deterministic
suppression by using the following definitions:
\begin{program}
        Class Agent/SRM/Deterministic -superclass Agent/SRM
        Agent/SRM/Deterministic set C2_ 0.0
        Agent/SRM/Deterministic set D2_ 0.0

        Class Agent/SRM/Probabilistic -superclass Agent/SRM
        Agent/SRM/Probabilistic set C1_ 0.0
        Agent/SRM/Probabilistic set D1_ 0.0
\end{program}
In \href{a later section}{Section}{sec:extensions},
we will discuss other ways of extending the SRM agent.

Timer related functions are handled by separate objects
belonging to the class  SRM.
Timers are required for loss recovery and sending periodic session messages.
There are loss recovery objects to send request and repair messages.
The agent creates a separate request or repair object to handle each loss.
In contrast, the agent only creates one session object to send
periodic session messages.
The default classes the express each of these functions are:
\begin{program}
        Agent/SRM set requestFunction_  "SRM/request"
        Agent/SRM set repairFunction_   "SRM/repair"
        Agent/SRM set sessionFunction_  "SRM/session"

        Agent/SRM set requestBackoffLimit_      5       \; parameter to requestFunction_;
        Agent/SRM set sessionDelay_             1.0     \; parameter to sessionFunction_;
\end{program}
The instance procedures
\fcnref{\proc[]{requestFunction}}{../ns-/srm.tcl}{Agent/SRM::requestFunction},
\fcnref{\proc[]{repairFunction}}{../ns-/srm.tcl}{Agent/SRM::repairFunction},
and
\fcnref{\proc[]{sessionFunction}}{../ns-/srm.tcl}{Agent/SRM::sessionFunction}
can be used to change the default function for individual agents.
The last two lines are specific parameters used by the request 
and session objects.
The \href{following section}{Section}{sec:architecture}
describes the implementation of theses objects in greater detail.

\subsubsection{Statistics}
Each agent tracks two sets of statistics:
statistics to measure the response to data loss,
and overall statistics for each request/repair.
In addition, there are methods to access other
information from the agent.

\paragraph{Data Loss}
The statistics to measure the response to data losses
tracks the duplicate requests (and repairs),
and the average request (and repair) delay.
The algorithm used is documented in Floyd \etal \cite{Floy95:Reliable}.
In this algorithm,
each new request (or repair) starts a new request (or repair) period.
During the request (or repair) period, the agent measures
the number of first round duplicate requests (or repairs)
until the round terminates either due to receiving a request (or
repair), or due to the agent sending one.
% These statistics are used by the adaptive timer algorithms;
% we will describe our implementation of these algorithms
% in the following subsections.
The following code illustrates how the user can simple retrieve the
current values in an agent:
\begin{program}
                set statsList [$srm array get statistics_]
                array set statsArray [$srm array get statistics_]
\end{program}
The first form returns a list of key-value pairs.
The second form loads the list into the \|statsArray| for further manipulation.
The keys of the array are
\|dup-req|, \|ave-dup-req|, \|req-delay|, \|ave-req-delay|,
\|dup-rep|, \|ave-dup-rep|, \|rep-delay|, and \|ave-rep-delay|.

\paragraph{Overall Statistics}
In addition, each loss recovery and session object keeps track of
times and statistics.
In particular, each object records its
\|startTime|, \|serviceTime|, \|distance|, as are relevant to that object;
startTime is the time that this object was created,
serviceTime is the time for this object to complete its task, and the
distance is the one-way time to reach the remote peer.

For request objects, startTime is the time a packet loss is detected,
serviceTime is the time to finally receive that packet,
and distance is the distance to the original sender of the packet.
For repair objects, startTime is the time that a request for
retransmission is received, serviceTime is the time send a repair,
and the distance is the distance to the original requester.
For both types of objects, the serviceTime is normalised by the
distance.
For  the session object,
startTime is the time that the agent joins the multicast group.
serviceTime and distance are not relevant.

Each object also maintains statistics particular to that type of object.
Request objects track the number of duplicate requests and repairs received,
the number of requests sent, and the number of times this object
had to backoff before finally receiving the data.
Repair objects track the number of duplicate requests and repairs,
as well as whether or not this object for this agent sent the repair.
Session objects simply record the number of session messages sent.

The values of the timers and the statistics for each object are written
to the log file every time an object completes the error recovery function
it was tasked to do.
The format of this trace file is:
\begin{program}
                \tup{prefix} \tup{id} \tup{times} \tup{stats}
{\itshape{}where}
\tup{prefix} is         \tup{time} n \tup{node id} m \tup{msg id} r \tup{round}
                \tup{msg id} is expressed as \tup{source id:sequence number}
\tup{id} is             type \tup{of object}
\tup{times} is          list of key-value pairs of startTime, serviceTime, distance
\tup{stats} is          list of key-value pairs of per object statistics
                \|dupRQST|, \|dupREPR|, \|#sent|, \|backoff|             {\itshape for request objects}
                \|dupRQST|, \|dupREPR|, \|#sent|                      {\itshape for repair objects}
                \|#sent|                                        {\itshape for session objects}
\end{program}
The following sample output illustrates the output file format (the lines
have been folded to fit on the page):
{\small
\begin{verbatim}
 3.6274 n 0 m <1:1> r 1 type repair serviceTime 0.500222 \
        startTime 3.5853553333333332 distance 0.0105 #sent 1 dupREPR 0 dupRQST 0
 3.6417 n 1 m <1:1> r 2 type request serviceTime 2.66406 \
        startTime 3.5542666666666665 distance 0.0105 backoff 1 #sent 1 dupREPR 0 dupRQST 0
 3.6876 n 2 m <1:1> r 2 type request serviceTime 1.33406 \
        startTime 3.5685333333333333 distance 0.021 backoff 1 #sent 0 dupREPR 0 dupRQST 0
 3.7349 n 3 m <1:1> r 2 type request serviceTime 0.876812 \
        startTime 3.5828000000000002 distance 0.032 backoff 1 #sent 0 dupREPR 0 dupRQST 0
 3.7793 n 5 m <1:1> r 2 type request serviceTime 0.669063 \
        startTime 3.5970666666666671 distance 0.042 backoff 1 #sent 0 dupREPR 0 dupRQST 0
 3.7808 n 4 m <1:1> r 2 type request serviceTime 0.661192 \
        startTime 3.5970666666666671 distance 0.0425 backoff 1 #sent 0 dupREPR 0 dupRQST 0
\end{verbatim}
}

\paragraph{Miscellaneous Information}
Finally, the user can use the following methods to gather
additional information about the agent:
\begin{list}{\textbullet}{}
\item
  \fcnref{\proc[]{groupSize?}}{../ns-2/srm.tcl.html}{Agent/SRM::groupSize?} 
  returns the agent's current estimate of the multicast group size.
\item
  \fcnref{\proc[]{distances?}}{../ns-2/srm.cc.html}{SRMAgent::command}
  returns a list of key-value pairs of distances;
  the key is the address of the agent, 
  the value is the estimate of the distance to that agent.
  The first element is the address of this agent, and the distance of 0.
\item
  \fcnref{\proc[]{distance?}}{../ns-2/srm.cc.html}{SRMAgent::command}
  returns the distance to the particular agent specified as argument.

  The default distance at the start of any simulation is 1.
\end{list}
\begin{program}
        $srm(i) groupSize?    \; returns $srm(i)'s estimate of the group size;
        $srm(i) distances?    \; returns list of \tup{address, distance} tuples;
        $srm(i) distance? 257 \; returns the distance to agent at address 257;
\end{program}

\subsubsection{Tracing}
Each object writes out trace information that can be used to track the
progress of the object in its error recovery.
Each trace entry is of the form:
\begin{program}
\tup{prefix} \tup{tag} \tup{type of entry} \tup{values}
\end{program}
The prefix is as describe in the previous subsection for statistics.
The tag is {\bf Q} for request objects, {\bf P} for repair objects, and
{\bf S} for session objects.
The following types of trace entries and parameters are written by each
object:

\centerline{\small\renewcommand{\arraystretch}{1.3}
\begin{tabular}{rclp{2in}}\hline
      & Type of &              & \\
  Tag & Object  & Other values & Comments\\ \hline
  Q & DETECT & & \\
  Q & INTERVALS & C1 \tup{C1\_} C2 \tup{C2\_} dist \tup{distance} i \tup{backoff\_} & \\
  Q & NTIMER & at \tup{time} & Time the request timer will fire \\
  Q & SENDNACK & & \\
  Q & NACK & IGNORE-BACKOFF \tup{time} & Receive NACK, ignore other NACKs until
  \tup{time} \\
  Q & REPAIR & IGNORES \tup{time} & Receive REPAIR, ignore NACKs until \tup{time}  \\
  Q & DATA & & Agent receives data instead of repair.  Possibly indicates out of order arrival of data. \\ \hline
  P & NACK & from \tup{requester} & Receive NACK, initiate repair \\
  P & INTERVALS & D1 \tup{D1\_} D2 \tup{D2\_} dist \tup{distance} & \\
  P & RTIMER & at \tup{time} & Time the repair timer will fire \\
  P & SENDREP & \\
  P & REPAIR & IGNORES \tup{time} & Receive REPAIR, ignore NACKs until \tup{time} \\
  P & DATA & & Agent receives data instead of repair.  Indicates premature request by an agent. \\ \hline
  S & SESSION & & logs session message sent \\ \hline
\end{tabular}}
The following illustrates a typical trace for a single loss and recovery.
{\small
\begin{verbatim}
 3.5543 n 1 m <1:1> r 0 Q DETECT
 3.5543 n 1 m <1:1> r 1 Q INTERVALS C1 2.0 C2 0.0 d 0.0105 i 1
 3.5543 n 1 m <1:1> r 1 Q NTIMER at 3.57527
 3.5685 n 2 m <1:1> r 0 Q DETECT
 3.5685 n 2 m <1:1> r 1 Q INTERVALS C1 2.0 C2 0.0 d 0.021 i 1
 3.5685 n 2 m <1:1> r 1 Q NTIMER at 3.61053
 3.5753 n 1 m <1:1> r 1 Q SENDNACK
 3.5753 n 1 m <1:1> r 2 Q INTERVALS C1 2.0 C2 0.0 d 0.0105 i 2
 3.5753 n 1 m <1:1> r 2 Q NTIMER at 3.61727
 3.5753 n 1 m <1:1> r 2 Q NACK IGNORE-BACKOFF 3.59627
 3.5828 n 3 m <1:1> r 0 Q DETECT
 3.5828 n 3 m <1:1> r 1 Q INTERVALS C1 2.0 C2 0.0 d 0.032 i 1
 3.5828 n 3 m <1:1> r 1 Q NTIMER at 3.6468
 3.5854 n 0 m <1:1> r 0 P NACK from 257
 3.5854 n 0 m <1:1> r 1 P INTERVALS D1 1.0 D2 0.0 d 0.0105
 3.5854 n 0 m <1:1> r 1 P RTIMER at 3.59586
 3.5886 n 2 m <1:1> r 2 Q INTERVALS C1 2.0 C2 0.0 d 0.021 i 2
 3.5886 n 2 m <1:1> r 2 Q NTIMER at 3.67262
 3.5886 n 2 m <1:1> r 2 Q NACK IGNORE-BACKOFF 3.63062
 3.5959 n 0 m <1:1> r 1 P SENDREP
 3.5959 n 0 m <1:1> r 1 P REPAIR IGNORES 3.62736
 3.5971 n 4 m <1:1> r 0 Q DETECT
 3.5971 n 4 m <1:1> r 1 Q INTERVALS C1 2.0 C2 0.0 d 0.0425 i 1
 3.5971 n 4 m <1:1> r 1 Q NTIMER at 3.68207
 3.5971 n 5 m <1:1> r 0 Q DETECT
 3.5971 n 5 m <1:1> r 1 Q INTERVALS C1 2.0 C2 0.0 d 0.042 i 1
 3.5971 n 5 m <1:1> r 1 Q NTIMER at 3.68107
 3.6029 n 3 m <1:1> r 2 Q INTERVALS C1 2.0 C2 0.0 d 0.032 i 2
 3.6029 n 3 m <1:1> r 2 Q NTIMER at 3.73089
 3.6029 n 3 m <1:1> r 2 Q NACK IGNORE-BACKOFF 3.66689
 3.6102 n 1 m <1:1> r 2 Q REPAIR IGNORES 3.64171
 3.6172 n 4 m <1:1> r 2 Q INTERVALS C1 2.0 C2 0.0 d 0.0425 i 2
 3.6172 n 4 m <1:1> r 2 Q NTIMER at 3.78715
 3.6172 n 4 m <1:1> r 2 Q NACK IGNORE-BACKOFF 3.70215
 3.6172 n 5 m <1:1> r 2 Q INTERVALS C1 2.0 C2 0.0 d 0.042 i 2
 3.6172 n 5 m <1:1> r 2 Q NTIMER at 3.78515
 3.6172 n 5 m <1:1> r 2 Q NACK IGNORE-BACKOFF 3.70115
 3.6246 n 2 m <1:1> r 2 Q REPAIR IGNORES 3.68756
 3.6389 n 3 m <1:1> r 2 Q REPAIR IGNORES 3.73492
 3.6533 n 4 m <1:1> r 2 Q REPAIR IGNORES 3.78077
 3.6533 n 5 m <1:1> r 2 Q REPAIR IGNORES 3.77927
\end{verbatim}
The logging of request and repair traces is done by
\fcnref{\proc[]{SRM::evTrace}}{../ns-2/srm.tcl}{SRM::evTrace}.
However, the routine
\fcnref{\proc[]{SRM/Session::evTrace}}{../ns-2/srm.tcl}{SRM/Session::evTrace},
overrides the base class definition of \proc[]{srm::evTrace},
and writes out nothing.
Individual simulation scripts can override these methods
for greater flexibility in logging options.
One possible reason to override these methods might to
reduce the amount of data generated;
the new procedure could then generate compressed and processed output.

Notice that the trace filoe contains sufficient information and details
to derive most of the statistics written out in the log file, or
is stored in the statistics arrays.

\subsection{Architecture and Internals}
\label{sec:architecture}

The SRM agent implementation splits the protocol functions
into packet handling, loss recovery, and session message activity.
\begin{list}{}{}
\item  Packet handling consists of forwarding application data messages,
  sending and receipt of control messages.
  These activities are executed by C++ methods.
\item  Error detection is done in C++ due to receipt of messages.
  However, the loss recovery is entirely done through 
  instance procedures in OTcl.
\item  The sending and processing of messages is accomplished in C++;
  the policy about when these messages should be sent is decided
  by instance procedures in OTcl.
\end{list}
We first describe the C++
\href{processing due to receipt of messages}{Section}{sec:reciept}.
Loss recovery and the sending of session messages involves
timer based processing.
The agent uses a separate \clsref{SRM}{../ns-2/srm.tcl}
to perform the timer based functions.
For each loss, an agent may do either request or repair processing.
Each agent will instantiate a separate loss recovery object
for every loss, as is appropriate for the processing that it has to do.
In the following section
\href{we describe the basic timer based functions and
the loss recovery mechanisms}{Section}{sec:recovery}.
Finally, each agent uses one timer based function
for \href{sending periodic session messages}{Section}{sec:session}.

\subsection{Packet Handling: Processing received messages}
\label{sec:reciept}

The
\fcnref{\fcn[]{recv}}{../ns-2/srm.cc}{SRMAgent::recv}
method can receive four type of messages:
data, request, repair, and session messages.

\paragraph{Data Packets}
The agent does not generate any data messages.
The user has to specify an external agent to generate traffic.
The \fcn[]{recv} method must distinguish between
locally originated data that must be sent to the multicast group,
and data received from multicast group that must be processed.
Therefore, the application agent must
set the packet's destination address to zero.

For locally originated data, 
the agent adds the appropriate SRM headers,
sets the destination address to the multicast group, 
and forwards the packet to its target.

On receiving a data message from the group,
\fcnref{\fcn[sender, msgid]{recv\_data}}{../ns-2/srm.cc}{SRMAgent::recv\_data}
will update its state marking message \tup{sender, msgid} received,
and possibly trigger requests if it detects losses.
In addition, if the message was an older message received out of order,
then there must be a pending request or repair that must be cleared.
In that case, the compiled object invokes the OTcl instance procedure,
\fcnref{\proc[sender, msgid]{recv-data}}{%
  ../ns-2/srm.tcl}{Agent/SRM::recv-data}%
\footnote{Technically,
  \fcn[]{recv\_data} invokes the instance procedure
  \|recv data \tup{sender} \tup{msgid}|,
  that then invokes \proc[]{recv-data}.
  The indirection allows individual simulation scripts to override the
  \proc[]{recv} as needed.}.

Currently, there is no provision for the receivers
to actually receive any application data.
The agent does not also store any of the user data.
It only generates repair messages of the appropriate size,
defined by the instance variable \|packetSize_|.
However, the agent assumes that any application data
is placed in the data portion of the packet,
pointed to by \|packet->accessdata()|.

\paragraph{Request Packets}
On receiving a request, 
\fcnref{\fcn[sender, msgid]{recv\_rqst}}{../ns-2/srm.cc}{SRMAgent::recv\_rqst}
will check whether it needs to schedule requests for other missing data.
If it has received this request
before it was aware that the source had generated this data message
(\ie, the sequence number of the request is higher than 
the last known sequence number of data from this source),
then the agent can infer that it is missing this, as well as data
from the last known sequence number onwards;
it schedules requests for all of the missing data and returns.
On the other hand, if the sequence number of the request is less
than the last known sequence number from the source,
then the agent can be in one of three states:
(1) it does not have this data, and has a request pending for it,
(2) it has the data, and has seen an earlier request,
    upon which it has a repair pending for it, or
(3) it has the data, and it should instantiate a repair.
All of these error recovery mechanisms are done in OTcl;
\fcn[]{recv\_rqst} invokes the instance procedure
\fcnref{\proc[sender, msgid,
  requester]{recv-rqst}}{../ns-2/srm.tcl}{Agent/SRM::recv-rqst}
for further processing.

\paragraph{Repair Packets}
On receiving a repair, 
\fcnref{\fcn[sender, msgid]{recv\_repr}}{../ns-2/srm.cc}{SRMAgent::recv\_repr}
will check whether it needs to schedule requests for other missing data.
If it has received this repair
before it was aware that the source had generated this data message
(\ie, the sequence number of the repair is higher than 
the last known sequence number of data from this source),
then the agent can infer that it is missing all
data between the last known sequence number and that on the repair;
it schedules requests for all of this data,
 marks this message as received, and returns.
On the other hand, if the sequence number of the request is less
than the last known sequence number from the source,
then the agent can be in one of three states:
(1) it does not have this data, and has a request pending for it,
(2) it has the data, and has seen an earlier request,
    upon which it has a repair pending for it, or
(3) it has the data, and probably scheduled a repair for it at some time;
    after error recovery, its holddown timer (equal to three times its
    distance to some requestor) expired, at which time the pending object
    was cleared.  In this last situation, the agent will simply ignore
    the repair, for lack of being able to do anything meaningful.
All of these error recovery mechanisms are done in OTcl;
\fcn[]{recv\_repr} invokes the instance procedure
\fcnref{\proc[sender, msgid]{recv-repr}}{%
  ../ns-2/srm.tcl}{Agent/SRM::recv-rqst}
to complete the loss recovery phase for the particular message.
  
\paragraph{Session Packets}
On receiving a session message,
the agent updates its sequence numbers for all active sources,
and computes its instantaneous distance to the sending agent if possible.
The agent will ignore earlier session messages from a group member,
if it has received a later one out of order.
  
Session message processing is done in
\fcnref{\fcn[]{recv\_sess}}{../ns-2/srm.cc}{SRMAgent::recv\_sess}.
The format of the session message is:
\tup{count of tuples in this message, list of tuples},
where each tuple indicates the
\tup{sender id, last sequence number from the source, time the last
  session message was received from this sender, time that that message
  was sent}.
The first tuple is the information about the local agent%
\footnote{Note that this implementation of session message handling
  is subtly different from that used in \emph{wb} or described in
  \cite{Floy95:Reliable}.
  In principle, an agent disseminates a list of the data it has
  actually received.
  Our implementation, on the other hand, only disseminates
  a count of the last message sequence number per source that the
  agent knows that that the source has sent.
  This is a constraint when studying aspects of loss recovery
  during partition and healing.
  It is reasonable to expect that the maintainer of this code will fix
  this problem during one of his numerous intervals of copious spare time.}.

\subsection{Loss Recovery Objects}
\label{sec:recovery}

In the last section,
we described the agent behaviour when it receives a message.
Timers are used to control when any particular control message is to be sent.
The SRM agent uses a separate
\clsref{SRM}{../ns-2/srm.tcl}
to do the timer based processing.
In this section, we describe the basecs if the class SRM,
and the loss recovery objects.
The following section will describe how the class SRM is used 
for sending periodic session messages.
An SRM agent will instantiate one object to recover from one lost data packet.
Agents that detect the loss will instantiate an object in the
\clsref{SRM/request}{../ns-2/srm.tcl};
agents that receive a request and have the required data will
instantiate an object in the \clsref{SRM/repair}{../ns-2/srm.tcl}.

\paragraph{Request Mechanisms}
SRM agents detect loss when they receive a message, and
infer the loss based on the sequence number on the message received.
Since packet reception is handled entirely by the compiled object,
loss detection occurs in the C++ methods.
Loss recovery, however, is handled entirely by instance procedures
of the corresponding interpreted object in OTcl.

When any of the methods detects new losses, it invokes
\fcnref{\proc[]{Agent/SRM::request}}{../ns-2/srm.tcl}{Agent/SRM::request}
with a list of the message sequence numbers that are missing.
\proc[]{request} will create a new \|requestFunction_|
object for each message that is missing.
The agent stores the object handle in its array of \|pending_| objects.
The key to the array is the message identifier \tup{sender}:\tup{msgid}.
\begin{list}{}{}
\item 
  The default \|requestFunction_| is \clsref{SRM/request}.
  The constructor for the class SRM/request
  calls the base class constructor to initialise 
  the simulator instance (\|ns_|), the SRM agent (\|agent_|),
  trace file (\|trace_|), and the \|times_| array.
  It then initialises its \|statistics_| array with the pertinent elements.

\item
  A separate call to
  \fcnref{\proc[]{set-params}}{../ns-2/srm.tcl}{SRM::set-params}
  sets the \|sender_|, \|msgid_|, \|round_| instance variables for
  the request object.
  The object determines \|C1_| and \|C2_| by querying its \|agent_|.
  It sets its distance to the sender (\|times_(distance)|)
  and fixes other scheduling parameters:
  the backoff constant (\|backoff_|),
  the current number of backoffs (\|backoffCtr_|),
  and the limit (\|backoffLimit_|) fixed by the agent.
  \proc[]{set-params} writes the trace entry ``\textsc{q detect}''.

\item
  The final step in \proc[]{request} is to schedule the timer
  to send the actual request at the appropriate moment.
  \fcnref{\proc[]{SRM/request::schedule}}{../ns-2/srm.tcl}{%
    SRM/request::schedule}
  uses 
  \fcnref{\proc[]{compute-delay}}{%
    ../ns-2/srm.tcl}{SRM/request::compute-delay}
  and its current backoff constant to determine the delay.
  The object schedules
  \fcnref{\proc[]{send-request}}{../ns-2/srm.tcl}{SRM/request::send-request}
  to be executed after \|delay_| seconds.
  The instance variable \|eventID_| stores a handle to the scheduled event.
  The default \proc[]{compute-delay} function returns a value
  uniformly distributed in the interval $[C_1 d_s, (C_1 + C_2) d_s]$,
  where $d_s$ is twice \|$times_(distance)|.
  The \proc[]{schedule} schedules an event to send a request
  after the computed delay. 
  The routine writes a trace entry ``\textsc{q ntimer } at \tup{time}''.
\end{list}

When the scheduled timer fires, the routine
\fcnref{\proc[]{send-request}}{../ns-2/srm.tcl}{SRM/request::send-request}
sends the appropriate message.
It invokes ``\|$agent_| send request \tup{args}'' to send the request.
Note that \proc[]{send} is an instproc-like,
executed by the \fcn[]{command} method of the compiled object.
However, it is possible to overload the instproc-like
with a specific instance procedure \proc[]{send}
for specific configurations.
As an example, recall that the file \|tcl/mcast/srm-nam.tcl|
overloads the \proc[]{send} command
to set the flowid based on type of message that is sent.
\proc[]{send-request} updates the statistics, and writes the trace entry
``\textsc{q sendnack}''.

When the agent receives a control message for a packet
for which a pending object exists,
the agent will hand the message off to the object for processing.
\begin{list}{}{}
\item When a 
  \fcnref{request for a particular packet is received}{../ns-2/srm.tcl}{%
        SRM/request::recv-request},
  the request object can be in one of two states:
  it is ignoring requests, considering them to be duplicates, or
  it will cancel its send event and re-schedule another one,
  after having backed off its timer.
  If ignoring requests it will update its statistics,
  and write the trace entry ``\textsc{q nack } dup''.
  Otherwise, set a time based on its current estimate of the \|delay_|,
  until which to ignore further requests.
  This interval is marked by the instance variable \|ignore_|.
  If the object reschedules its timer, it will write the trace entry
  ``\textsc{ q nack ignore-backoff } \tup{ignore}''.
  Note that this re-scheduling relies on the fact that
  the agent has joined the multicast group, and will therefore
  receive a copy of every message it sends out.
  
\item When the
  \fcnref{request object receives a repair for the particular packet}{%
    ../ns-2/srm.tcl}{SRM/request::recv-repair},
  it can be in one of two states:
  either it is still waiting for the repair,
  or it has already received an earlier repair.
  If it is the former, there will be an event pending
  to send a request, and \|eventID_| will point to that event.
  The object will compute its serviceTime, cancel that event,
  and set a holddown period during which it will ignore 
  other requests.
  At the end of the holddown period, the object will ask its
  agent to clear it.
  It will write the trace entry ``\textsc{q repair ignores } \tup{ignore}''.
  On the other hand, if this is a duplicate repair,
  the object will update its statistics, and write the trace entry
  ``\textsc{q repair } dup''.
\end{list}

When the loss recovery phase is completed by the object,
\fcnref{\proc[]{Agent/SRM::clear}}{../ns-2/srm.tcl}{Agent/SRM::clear}
will remove the object from its array of \|pending_| objects,
and place it in its list of \|done_| objects.
Periodically, the agent will cleanup and delete the \|done_| objects.

\paragraph{Repair Mechanisms}
The agent will initiate a repair if it receives a request for a packet,
and it does not have a request object \|pending_| for that packet.
The default repair object belongs to the
\clsref{SRM/repair}{../ns-2/srm.tcl}.
Barring minor differences,
the sequence of events and the instance procedures in this class
are identical to those for SRM/request.
Rather than outline every single procedure, we only outline
the differences from those described earlier for a request object.

The repair object uses the repair parameters, \|D1_|, \|D2_|.
A repair object does not repeatedly reschedule is timers;
therefore, it does not use any of the backoff variables
such as that used by a request object.
The repair object ignores all requests for the same packet.
The repair objet does not use the \|ignore_| variable that
request objects use.
The trace entries written by repair objects are marginally different;
they are ``\textsc{p nack } from \tup{requester}'',
``\textsc{p rtimer } at \tup{fireTime}'',
``\textsc{p sendrep}'', ``\textsc{p repair ignores } \tup{holddown}''.

Apart from these differences,
the calling sequence for events in a repair object is similar to that
of a request object.

\paragraph{Mechanisms for Statistics}
The agent, in concert with the request and repair objects, 
collect statistics about their response to data loss \cite{Floy95:Reliable}.
Each call to the agent \proc[]{request} procedure marks a new period.
At the start of a new period,
\fcnref{\proc[]{mark-period}}{../ns-2/srm.tcl}{Agent/SRM::mark-period}
computes the moving average of the number of duplicates in the last period.
Whenever the agent receives a first round request from another agent,
and it had sent a request in that round, then it considers the request
as a duplicate request, and increments the appropriate counters.
A request object does not consider duplicate requests if it did not
itself send a request in the first round. 
If the agent has a repair object pending, then it does not consider
the arrival of duplicate requests for that packet.
The object methods
\fcnref{\proc[]{SRM/request::dup-request?}}{../ns-2/srm.tcl}{%
        SRM/request::dup-request?} and
\fcnref{\proc[]{SRM/repair::dup-request?}}{../ns-2/srm.tcl}{%
        SRM/repair::dup-request?} 
encode these policies, and return 0 or 1 as required.

A request object also computes the elapsed time between 
when the loss is detected to when it receives the first request.
The agent computes a moving average of this elapsed time.
The object computes the elapsed time (or delay) when it
\fcnref{cancels}{../ns-2/srm.tcl}{SRM/request::cancel}
its scheduled event for the first round.
The object invokes
\fcnref{Agent/SRM::update-ave}{../ns-2/srm.tcl}{Agent/SRM::update-ave}
to compute the moving average of the delay.

The agent keeps similar statistics of the duplicate repairs,
and the repair delay.

The agent stores the number of rounds taken for one loss recovery,
to ensure that subsequent loss recovery phases for that packet
that are not definitely not due to data loss
do not account for these statistics.
The agent stores the number of routes taken for a phase in
the array \|old_|.
When a new loss recovery object is instantiated,
the object will use the agent's instance procedure
\fcnref{\proc[]{round?}}{../ns-2/srm.tcl}{Agent/SRM::round?}
to determine the number of rounds in a previous loss recovery phase
for that packet.

\subsection{Session Objects}
\label{sec:session}

Session objects,
\href{like the loss recovery objects}{Section}{sec:recovery},
are derived from the base \clsref{SRM}.
Unlike the loss recovery objects though,
the agent only creates one session object for the lifetime of the agent.
The constructor invokes the base class constructor as before;
it then sets its instance variable \|sessionDelay_|.
The agent creates the session object when it \proc[]{start}s.
At that time, it also invokes
\fcnref{SRM/session::schedule}{../ns-2/srm.tcl}{SRM/session::schedule},
to send a session message after \|sessionDelay_| seconds.

When the object sends a session message,
it will schedule to send the next one after some interval.
It will also update its statistics.
\fcnref{\proc[]{send-session}}{../ns-2/srm.tcl}{SRM/session::send-session}
writes out the trace entry ``\textsc{s session}''.

The class overrides the
\proc[]{evTrace} routine that writes out the trace entries.
\fcnref{SRM/session::evTrace}{../ns-2/srm.tcl}{SRM/sesion::evTrace}
disable writing out the trace entry for session messages.

Two types of session message scheduling strategies are currently
available:
The function in the base class schedules sending session messages at
fixed intervals of \|sessionDelay_| jittered around a small value
to avoid synchronization among all the agents at all the nodes.
\clsref{SRM/session/logScaled} schedules sending messages
at intervals of \|sessionDelay| times $\log_2$(\|groupSize_|)
so that the frequency of session messages is inversely proportional to 
the size of the group.

The base class that sends messages at fixed intervals
is the default \|sessionFunction_| for the agent.

\subsection{Extending the Base Class Agent}
\label{sec:extensions}

In
\href{the earlier section on configuration parameters}{Section}{sec:config-param},
we had shown how to trivially extend the agent to
get deterministic and probabilistic protocol behaviour.
In this section, we describe how to derive more complex
extensions to the protocol for fixed and adaptive timer mechanisms.

\subsubsection{Fixed Timers}

The fixed timer mechanism are done in
the derived \clsref{Agent/SRM/Fixed}.
The main difference with fixed timers is that
the repair parameters are set to $\log$(\|groupSize_|).
Therefore, 
\fcnref{the repair procedure of a fixed timer agent}{../ns-2/srm.tcl}{%
        Agent/SRM/Fixed::repair}
will set \d1 and \d2 to be proportional to the group size
before scheduling the repair object.

\subsubsection{Adaptive Timers}

Agents using adaptive timer mechanisms
modify their request and repair parameters under three conditions
(1) every time a new loss object is created;
(2) when sending a message; and
(3) when they receive a duplicate, if their relative distance to the loss
    is less than that of the agent that sends the duplicate.
All three changes require extensions to the agent and the loss objects.
The \clsref{Agent/SRM/Adaptive}{../ns-2/srm-adaptive.tcl}
uses \clsref{SRM/request/Adaptive}{../ns-2/srm-adaptive.tcl} and
\clsref{SRM/repair/Adaptive}{../ns-2/srm-adaptive.tcl}
as the request and repair functions respectively.
In addition, the last item requires extending the packet headers,
to advertise their distances to the loss.
The corresponding compiled class for the agent is the
\clsref{ASRMAgent}{../ns-2/srm.h}.

\paragraph{Recompute for Each New Loss Object}
Each time a new request object is created,
\fcnref{SRM/request/Adaptive::set-params}{../ns-2/srm-adaptive.tcl}{%
        SRM/request/Adaptive::set-params}
invokes \|$agent_ recompute-request-params|.
The agent method
\fcnref{\fcn[]{recompute-request-params}}{../ns-2/srm-adaptive.tcl}{%
        Agent/SRM/Adaptive::recompute-request-params}.
uses the statistics about duplicates and delay
to modify \c1 and \c2 for the current and future requests.

Similarly,
\fcnref{SRM/request/Adaptive::set-params}{../ns-2/srm-adaptive.tcl}{%
        SRM/request/Adaptive::set-params}
for a new repair object
invokes \|$agent_ recompute-repair-params|.
The agent method
\fcnref{\fcn[]{recompute-repair-params}}{../ns-2/srm-adaptive.tcl}{%
        Agent/SRM/Adaptive::recompute-repair-params}.
uses the statistics objects to modify \d1 and \d2
for the current and future repairs.

\paragraph{Sending a Message}
If a loss object 
\fcnref{sends a request}{../ns-2/srm-adaptive.tcl}{%
        SRM/request/Adaptive::send-request}
in its first \|round_|,
then the agent, in the instance procedure
\fcnref{\proc[]{sending-request}}{../ns-2/srm-adaptive.tcl}{%
        Agent/SRM/Adaptive::sending-request},
will lower \c1,
and set its instance variable \|closest_(requestor)| to 1.

Similarly,
a loss object that
\fcnref{sends a repair}{../ns-2/srm-adaptive.tcl}{%
        SRM/repair/Adaptive::send-repair}
in its first \|round_|
will invoke the agent's instance procedure,
\fcnref{\proc[]{sending-repair}}{../ns-2/srm-adaptive.tcl}{%
        Agent/SRM/Adaptive::sending-repair},
to lower \d1 and set \|closest_(repairor)| to 1.

\paragraph{Advertising the Distance}
Each agent must add additional information to each request/repair
that it sends out.
The base \clsref{SRMAgent}{../ns-2/srm.cc}
invokes the virtual method
\fcnref{\fcn[]{addExtendedHeaders}}{../ns-2/srm.h}{%
        SRMAgent::addExtendedHeaders}
for each SRM packet that it sends out.
The method is invoked after adding the SRM packet headers, and
before the packet is transmitted.
The adaptive SRM agent overloads the method
\fcnref{\fcn[]{addExtendedHeaders}}{../ns-2/srm.h}{%
        ASRMAgent::addExtendedHeaders}
to specify its distances in the additional headers.
When sending a request, that agent unequivocally knows the
identity of the sender.
As an example, the definition of
\fcn[]{addExtendedHeaders} for the adaptive SRM agent is:
\begin{program}
        void addExtendedHeaders(Packet* p) \{
                SRMinfo* sp;
                hdr_srm*  sh = (hdr_srm*) p->access(off_srm_);
                hdr_asrm* seh = (hdr_asrm*) p->access(off_asrm_);
                switch (sh->type()) \{
                case SRM_RQST:
                        sp = get_state(sh->sender());
                        seh->distance() = sp->distance_;
                        break;
                \ldots
                \}
        \}
\end{program}


Sinilarly, the method
\fcnref{\fcn[]{parseExtendedHeaders}}{../ns-2/srm.h}{%
        ASRMAgent::parseExtendedHeaders}
is invoked everytime an SRM paket is received.
It sets the agent member variable \|pdistance_|
to the distance advertised by the peer that sent the message.
The member variable is bound to an instance variable of the same name,
so that the peer distance can be accessed
by the appropriate instance procedures.
The corresponding \fcn[]{parseExtendedHeaders} method for the
Adaptive SRM agent is simply:
\begin{program}
        void parseExtendedHeaders(Packet* p) \{
                hdr_asrm* seh = (hdr_asrm*) p->access(off_asrm_);
                pdistance_ = seh->distance();
        \}
\end{program}


Finally, the adaptive SRM agent's extended headers are defined as
\structref{hdr\_asrm}{../ns-2/srm.h}.
The header declaration is identical to declaring other packet headers in \ns.
% xref external documentation here.
Unlike most other packet headers, 
these are not automatically available in the packet.
The
\fcnref{interpreted constructor}{../ns-2/srm-adaptive.tcl}{%
        Agent/SRM/Adaptive::init}
for the first adaptive agent
will add the header to the packet format.
For example, the start of the constructor for the
\code{Agent/SRM/Adaptive} agent is:
\begin{program}
        Agent/SRM/Adaptive set done_ 0
        Agent/SRM/Adaptive instproc init args \{
            if ![$class set done_] \{
                set pm [[Simulator instance] set packetManager_]
                TclObject set off_asrm_ [$pm allochdr aSRM]
                $class set done_ 1
            \}

            eval $self next $args
            \ldots
        \}
\end{program}


\end{document}

### Local Variables:
### mode: latex
### comment-column: 60
### backup-by-copying-when-linked: t
### file-precious-flag: nil
### End:


\part{Scale}
\chapter{Session-level Packet Distribution}
\label{chap:session}

This section describes the internals of the Session-level Packet Distribution
implementation in \ns.
The section is in two parts:
the first part is an overview of 
a basic Session configuration,
and a ``complete'' description of the configuration parameters 
of a Session.
The second part describes the architecture, internals, and the code path
of the Session-level Packet distribution.

Session-level Packet Distribution enables simulations with large-scale 
topologies.  A 2048 node and 8 connectivity degree topology takes roughly 
40 MB in memory, 2049-4096 node topology takes about 167 MB, and 4097-
8194 node topology takes about 671 MB.  However, the queuing delays that
may occur in routers are ignored.  Therefore, if simulations are involved 
with high source rate or multiple sources merging at some point resulting
a high aggregated rate, please avoid using Session-level Packet Distribution.

\section{Configuration}

\subsection{Basic Configuration}
\label{sec:basic-config}

Each Session (i.e., a multicast tree) must be configured strictly in
this order:
creating(obtaining) the session source,
assigning the destination address,
creating the session helper, 
attaching to session source, and
the session members joining the group.


\begin{program}
        set ns [new SessionSim]          \; preamble initialization;
        set node [$ns node]              \; source and receiver to reside on this node;
        set group [$ns allocaddr]        \; multicast group for this session;

        set src [new Agent/CBR]
        $src set dst_ $group            \; configure the source;
        $ns attach-agent $node $src

        $ns create-session $node $src   \; creating the session helper and attaching to the source;

        set rcvr [new Agent/NULL]        \; configure the receiver;
        $ns attach-agent $node $rcvr
        $ns at 0.0 "$node join-group $rcvr $group" \; joining the session;

        $ns at 0.1 "$src start"          \; start the source;

\end{program}

\subsection{Inserting a Loss Module}
\label{sec:loss-config}

When simulating mechanism robustness(e.g., SRM error recovery mechanism), 
modules like lossy links are desired to create error senarios.  This 
subsection is describe how to create a lossy link, meaning inserting 
a loss module for a 'virtual link' (a link directly connecting source
and receiver with accumulative bandwidth and delay).

Please note that packets dropped at a particular link in a
multicast tree will not be received by
the receivers in the particular downstream subtree. We have worked 
on this dependency problem and now the loss modules for the downstream 
receivers will be installed automatically when a lossy link is created.


\paragraph{Creating a Loss Module}
Before we can insert a loss module in between a source-receiver pair,
we have to create the loss module.  Basically,
a loss module compares two values to decide whether to drop a packet.
The first value is obtained every time when the loss module receives 
a packet from a random variable.  The second value
is fixed and configured when the loss module is created.

The following code gives an example to create a uniform 
0.1 loss rate.

\begin{program}
        # creating the uniform distribution random variable
        set loss_random_variable [new RandomVariable/Uniform] 
        # setting the range of random variable
        $loss_random_variable set min_ 0
        $loss_random_variable set max_ 100

        # creating an error module;
        set loss_module [new ErrorModel]
        # set target for dropped packets;
        $loss_module drop-target [new Agent/Null]
        # setting error rate to 0.1, 10/(100-0);
        $loss_module set rate_ 10
        # attaching the random variable to the loss module;
        $loss_module ranvar $loss_random_variable 

\end{program}

Several random variable distributions are available.
%%% Need xref to ranvar pages
Please refer to tcl/ex/ranvar.tcl.

\paragraph{Inserting a Loss Module}

If it is intended to insert a loss module for a receiver, keep a handle to the 
loss module when created.  Loss modules can only be inserted after the
corresponding receivers finish joining the group.

\begin{program}
        # keep a handle to the loss module;
        set sessionhelper [$ns create-session $node $src] 
        # insert the loss module;
        $ns at 0.1 "$sessionhelper insert-depended-loss $loss_module $rcvr" 
\end{program}

\section{Architecture}
\label{sec:session-arch}
The purpose of Session-level packet distribution is to
speed up simulations and reduce memory consumption while 
maintaining reasonable accuracy(if no queuing involved).  The first
bottleneck observed is the memory consumption by heavy-weight
links and nodes.  Therefore, in SessionSim (Simulator for Session-level
packet distribution), we keep only minimal amount of 
states for links and nodes, and connect the higher level source and 
receiver applications with appropriate delay and loss modules.  When
a connection is a multicast group, we attach a replicator 
to the source application, so the replicator replicates packets
to all loss or delay modules attached to the receiver applications.

In short, almost the entire network layer(routing and queuing)
is abstract out.  Packets in SessionSim do not get routed.  
They only follow the established Session.

\section{Internals}
In this section, we explain the internals of Session-level Packet 
Distribution.  The implementation is split into two parts:
\begin{list}{}{}
\item  Linkage of objects to make a Session in OTcl 
\item  Packet forwarding activities are executed by C++ methods.  
\end{list}

\subsection{Object Linkage}
\label{sec:session-objlink}

\begin{list}{}{}
\item  Simplified links and nodes.
\item  Replicator
\item  Delay and loss modules
\end{list}

\paragraph{Nodes and Links}
A link only contains the values of
its bandwidth and delay, and a node contains only its id and port number
for next agent.

\begin{program}
SessionSim instproc simplex-link \{ n1 n2 bw delay type \} \{
    $self instvar bw_ delay_
    set sid [$n1 id]
    set did [$n2 id]

    set bw_($sid:$did) [expr [string trimright $bw Mb] * 1000000]
    set delay_($sid:$did) [expr [string trimright $delay ms] * 0.001]
\}

SessionNode instproc init \{\} \{
    $self instvar id_ np_
    set id_ [Node getid]
    set np_ 0
\}
\end{program}

\paragraph{Replicator}
One replicator is required per source.  While the source is configured,
a replicator (session helper) need to be attached to the source.  By
calling \proc[]{create-session}, a replicator is:
created,
attached to the source application, and 
kept in a SessionSim instance variable \code{session_} array with 
its source and destination addresses as the index.

Note that the destination of source agent must be set before
calling \proc[]{create-session}.

\begin{program}
SessionSim instproc create-session \{ node agent \} \{
    $self instvar session_

    set nid [$node id]                           \; get source address;
    set dst [$agent set dst_]                    \; get destination address;
    set session_($nid:$dst) [new Classifier/Replicator/Demuxer]  \; creating the replicator;
    $agent target $session_($nid:$dst)           \; attach the replicator to the source;
    return $session_($nid:$dst) \; keep the replicator in the SessionSim instance variable session_ array;
\}
\end{program}

\paragraph{Delay and Loss Modules}

At least one delay module is required per receiver.
See Section~\ref{sec:loss-config} for inserting a loss module for a receiver.
When a receiver joins a group, 
the \proc[]{join-group} method goes through
all replicators (session helpers) maintained in \code{session_}.
If the destination index matches the group address
the receiver are joining, then the following actions are performed.

1. A new slot of the replicator (session helper) is created and assigned to the receiver.

2. An accumulated bandwidth and delay between the source and receiver are obtained by SessionSim instance procedure \proc[]{get-bw} and \proc[]{get-delay}.

3. A constant random variable is created and assigned with the
accumulative delay.

4. A delay module is created and assigned with the constant random 
variable and the accumulative bandwidth.

5. The delay module in inserted into the replicator slot in
front of the receiver.

\begin{program}
SessionSim instproc join-group \{ agent group \} \{
    $self instvar session_

    foreach index [array names session_] \{
        set pair [split $index :]
        if \{[lindex $pair 1] == $group\} \{
            # Note: must insert the chain of loss, delay, 
            # and destination agent in this order:

            #1. insert destination agent into session replicator
            $session_($index) insert $agent

            #2. find accumulative bandwidth and delay
            set src [lindex $pair 0]
            set dst [[$agent set node_] id]
            set accu_bw [$self get-bw $dst $src]
            set delay [$self get-delay $dst $src]

            #3. set up a constant delay random variable
            set random_variable [new RandomVariable/Constant]
            $random_variable set avg_ $delay

            #4. set up the delay module
            set delay_module [new DelayModel]
            $delay_module bandwidth $accu_bw
            $delay_module ranvar $random_variable

            #5. insert the delay module in front of the dest agent
            $session_($index) insert-module $delay_module $agent
        \}
    \}
\}
\end{program}


\subsection{Packet Forwarding}
\label{sec:session-pktforward}
Packet forwarding activities are executed in C++.  A source application 
generates a packet and forwards to its target which must be a replicator 
(session helper).  The replicator copies the packet and forwards 
to targets in the active slots which are either delay modules or loss modules. If loss modules, a decision is made whether to drop the packet.
If yes, the packet is forwarded to the loss modules drop target.  If not,
the loss module forwards it to its target which must be a delay module.
The delay module will forward the packet with a delay to its target which
must be a receiver application.

%% PH: not sure this will come out right
%% PH: make .eps picture but not sure how to import that

\begin{program}
                    / Loss module - Delay module - Receiver 1
Source - Replicator --------------- Delay module - Receiver 2
    (Session Helper)\bs Loss module - Delay module - Receiver 3

\end{program}

\endinput

### Local Variables:
### mode: latex
### comment-column: 60
### backup-by-copying-when-linked: t
### file-precious-flag: nil
### End:


\part{Other}
\include{debugging}

\end{document}
