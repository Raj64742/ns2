\documentclass{report}

\usepackage{times}
\usepackage[T1]{fontenc}

\PassOptionsToPackage{draft}{nsdoc}
\usepackage{nsdoc}

\title{\ns\ Notes and Documentation}
\author{Multiple Members of the VINT project\\
  Kevin Fall \tup{kfall@ee.lbl.gov}, Editor\\
  Kannan Varadhan \tup{kannan@catarina.usc.edu}, Editor}
\date{\today}

\begin{document}
\begin{titlepage}
  \begin{center}\makeatletter%
    {\Large\@title\par}%
    \vskip 1.5em%
    {\normalsize
      \begin{tabular}[t]{c}%
        \@author
      \end{tabular}\par}%
    \vskip 1em%
    {\large\@date}
  \makeatother\end{center}
  \par
  \vfil
  \begin{quote}\small
    \xref{\ns}{http://www-nrg.ee.lbl.gov/ns/}
    \xref{\copyright}{../copyright.html}
    is LBNL's \underline{N}etwork \underline{S}imulator \cite{ns}.
    The simulator is written in C++;
    it uses OTcl as a command and configuration interface.
    \ns~v2 has three substantial changes from \ns~v1:
    (1) the more complex objects in \ns~v1
        have been decomposed into simpler components
        for greater flexibility and composability;
    (2) the configuration interface is now OTcl, 
        an object oriented version of Tcl; and
    (3) the interface code to the OTcl interpreter is
        separate from the main simulator.
  \end{quote}
  \vfil
\end{titlepage}
\clearpage

\tableofcontents
\bibliographystyle{plain}

\part{Interface to the Interpreter}
\include{otcl}

\part{Simulator Basics}
\include{simulator}
\chapter{Nodes and Packet Forwarding}
\label{chap:nodes}

This chapter describes one aspect of creating a topology in \ns,
\ie, creating the nodes.
In 
\href{the next chapter}{Chapter}{chap:links},
we will describe second aspect of creating the topology,
\ie, connecting the nodes to form links.
Recall that each simulation requires a single instance of the
\clsref{Simulator}{../ns-/ns-lib.tcl} to control and operate that simulation.
The class provides instance procedures to create and manage the topology,
and internally stores references to each element of the topology.
We begin 
\href{by describing the procedures in the class Simulator}{%
        Section}{sec:node:simulator}.
We then
\href{describe the instance procedures in the class Node}{%
        Section}{sec:node:node}
to access and operate on individual nodes.
We conclude
\href{with detailed descriptions of the Classifier}{%
        Section}{sec:node:classifiers}
from which the more complex node objects are formed.

The procedures and functions described in this chapter can be found in
\nsf{tcl/lib/ns-lib.tcl}, \nsf{tcl/lib/ns-nodes.tcl}, \\
\nsf{classifier.\{cc, h\}}, \nsf{classifier-addr.cc},
\nsf{classifier-mcast.cc}, \nsf{classifier-mpath.cc}, 
and, \nsf{replicator.cc}.

\section{Simulator Methods: Creating the Topology}
\label{sec:node:simulator}

The basic primitive for acquiring a node is
\begin{program}
    set ns [new Simulator]
    $ns \fcnref{\textbf{node}}{../ns-2/ns-lib.tcl}{Simulator::node}
\end{program}
The instance procedure \code{node} constructs
a node out of more simple
\href{classifier objects}{Section}{sec:node:classifiers}.
The Node itself is a standalone class in OTcl.
However, most of the components of the node are themselves TclObjects.
The typical structure of a node is as shown in Figure~\ref{fig:node:unicast}.
\begin{figure}[tb]
  \centerline{\includegraphics{node}}
  \caption{Structure of an Unicast Node}
  \label{fig:node:unicast}
\end{figure}

The address of an agent in a node is 16 bits wide:
the higher 8 bits define the node \code{id_},
the lower 8 bits identify the individual agent at the node.
This limits the number of nodes in a simulation to 256 nodes.
If the user needs to create a topology larger than 256 nodes,
then they should first expand the address space before creating any
nodes, as
\begin{program}
        $ns set-address-format expanded
\end{program}
This expands the address space to 30 bits, the higher 22 of which are 
used to assign node numbers.
{\bf NOTE}: Starting from release 2.1b6, the above is no longer necessary;
\ns is using 32-bit integers for both address and port.
Therefore, the above restriction of 256 nodes is no longer applicable.

By default, nodes in \ns\ are constructed for unicast simulations.
%In order to create nodes for multicast simulation, the
%class variable, \code{EnableMcast_}, should be set to 1, as:
In order to enable multicast simulation, the simulation should be created 
with an option ``-multicast on'', e.g.:
\begin{program}
        set ns [new Simulator -multicast on]
\end{program}
{\bf NOTE}: the old way of enabling multicast, e.g., \code{Simulator set 
EnableMcast_ 1}, is no longer needed.
The internal structure of a typical multicast node is shown in
Figure~\ref{sec:node:multicast}.
\begin{figure}
  \centerline{\includegraphics{mcastNode}}
  \caption{Internal Structure of a Multicast Node}
  \label{sec:node:multicast}
\end{figure}

When a simulation uses multicast routing,
the highest bit of the address indicates whether the particular
address is a multicast address or an unicast address.
If the bit is 0, the address represents a unicast address,
else the address represents a multicast address.
This implies that, by default, 
a multicast simulation is restricted to 128 nodes.

\section{Node Methods: Configuring the Node}
\label{sec:node:node}

Procedures to configure an individual node can be classified into:
\begin{list}{---}{\itemsep0pt}
\item Control functions
\item Address and Port number management, unicast routing functions
\item Agent management
\item Adding neighbors
\end{list}
We describe each of the functions in the following paragraphs.

\paragraph{Control functions}
\begin{enumerate}
\item \code{$node entry} returns the entry point for a node.
This is the first element which will handle packets arriving at that node.

The Node instance variable, \code{entry_}, stores the reference this element.
For unicast nodes, this is the address classifier that looks at the higher
bits of the destination address.
The instance variable, \code{classifier_} contains the reference to this
classifier.
However, for multicast nodes, the entry point is the
\code{switch_} which looks at the first bit to decide whether
it should forward the packet to the unicast classifier, or the multicast
classifier as appropriate.

\item \code{$node reset} will reset all agents at the node.

\item \code{$node enable-mcast} is an internal procedure used
  by the class simulator when creating a node.
  This procedure installs the additional classifiers needed
  to convert a unicast node to a multicast node.
\end{enumerate}


\paragraph{Address and Port number management}
The procedure \code{$node id} returns the node number of the node.
This number is automatically incremented and assigned to each node at
creation by the class Simulator method, \code{$ns node}.
The class Simulator also stores an instance variable array\footnote{%
  \ie, an instance variable of a class that is also an array variable},
  \code{Node_}, indexed by the node id, and contains a reference to the
  node with that id.

The procedure \code{$node agent \tup{port}} returns the handle of the
agent at the specified port.
If no agent at the specified port number is available, the procedure returns
the null string.

The procedure \code{alloc-port} returns the next available port number.
It uses an instance variable, \code{np_},
to track the next unallocated port number.

The procedures, \code{add-route} and \code{add-routes},
are used by \href{unicast routing}{Chapter}{chap:unicast}
to add routes to populate the \code{classifier_}
The usage syntax is
\code{$node add-route \tup{destination id} \tup{TclObject}};
\code{TclObject} is the entry of \code{dmux_}, the port demultiplexer
at the node, if the destination id is the same as this node's id,
it is often the head of a link to send packets for that destination to,
but could also be the the entry for other classifiers or types of classifiers.

\code{$node add-routes \tup{destination id} \tup{TclObjects}}
is used to add multiple routes to the same destination that must be used
simultaneously in round robin manner to spread the bandwidth used to reach
that destination across all links equally.
It is used only if the instance variable \code{multiPath_} is set to 1,
and detailed dynamic routing strategies are in effect,
and requires the use of a multiPath classifier.
We describe the implementation of the multiPath classifier
\href{later in this chapter}{Section}{sec:node:classifiers};
however, \href{we defer the discussion of multipath
routing}{Chapter}{chap:unicast} to the chapter on unicast routing.

The dual of \proc[]{add-routes} is \proc[]{delete-routes}.
It takes the id, a list of \code{TclObjects}, and a reference to
the simulator's \code{nullagent}.
It removes the TclObjects in the list from the installed routes in the
multipath classifier.
If the route entry in the classifier does not point to a multipath
classifier,
the routine simply clears the entry from \code{classifier_}, and
installs the \code{nullagent} in its place.

Detailed dynamic routing also uses two additional methods:
the instance procedure \proc[]{init-routing} sets the instance variable
\code{multiPath_} to be equal to the class variable of the same name.
It also adds a reference to the route controller object at that node
in the instance variable, \code{rtObject_}.
The procedure \proc[]{rtObject?} returns the handle for the route object 
at the node.

Finally, the procedure \proc[]{intf-changed} is invoked by
the network dynamics code if a link incident on the node
changes state.  Additional details on how this procedure
is used are discussed later
\href{in the chapter on network dynamics}{Chapter}{chap:net-dynamics}.

\paragraph{Agent management}
Given an \tup{agent}, the procedure \proc[]{attach} will
add the agent to its list of \code{agents_},
assign a port number the agent and set its source address,
set the target of the agent to be its (\ie, the node's) \proc[]{entry},
and add a pointer to the port demultiplexer at the node (\code{dmux_})
to the agent at the corresponding slot in the \code{dmux_} classifier.

Conversely, \proc[]{detach}will remove the agent from \code{agents_},
and point the agent's target, and the entry in the node \code{dmux_}
to \code{nullagent}.

\paragraph{Tracking Neighbors}
Each node keeps a list of its adjacent neighbors in its instance variable,
\code{neighbor_}.  The procedure \proc[]{add-neighbor} adds a neighbor to the list.  The procedure \proc[]{neighbors} returns this list.

\section{The Classifier}
\label{sec:node:classifiers}

The function of a node when it receives a packet is to examine
the packet's fields, usually its destination address, and
on occasion, its source address.
It should then map the values to an outgoing interface object
that is the next downstream recipient of this packet.

In \ns, this task is performed by a simple \emph{classifier} object.
Multiple classifier objects,
each looking at a specific portion of the packet
forward the packet through the node.
A node in \ns\ uses many different types of classifiers for different purposes.
This section describes some of the more common, or simpler,
classifier objects in \ns.

We begin with a description of the base class in this section.
The next subsections describe
\href{the address classifier}{Section}{sec:node:addr-classifier},
\href{the multicast classifier}{Section}{sec:node:mcast-classifier},
\href{the multipath classifier}{Section}{sec:node:mpath-classifier}, and
finally, \href{the replicator}{Section}{sec:node:replicator}.

A classifier provides a way to match a packet against some
logical criteria and retrieve a reference to another simulation
object based on the match results.
Each classifier contains a table of simulation objects indexed
by {\em slot number}.
The job of a classifier is to determine the slot number associated
with a received packet and forward that packet to the
object referenced by that particular slot.
The C++ \clsref{Classifier}{../ns-2/classifier.h}
(defined in \nsf{classifier.h})
provides a base class from which other classifiers are derived.
\begin{program}
        class Classifier : public NsObject \{
        public:
                ~Classifier();
                void recv(Packet*, Handler* h = 0);
         protected:
                Classifier();
                void install(int slot, NsObject*);
                void clear(int slot);
                virtual int command(int argc, const char*const* argv);
                virtual int classify(Packet *const) = 0;
                void alloc(int);
                NsObject** slot_;       \* table that maps slot number to a NsObject */
                int nslot_;
                int maxslot_;
        \};
\end{program}
The \fcn[]{classify} method is pure virtual, indicating the
class \code{Classifier} is to be used only as a base class.
The \fcn[]{alloc} method dynamically allocates enough space
in the table to hold the specified number of slots.
The \fcn[]{install} and \fcn[]{clear} methods
add or remove objects from the table.
The \fcn[]{recv} method and the OTcl interface are implemented
as follows in \nsf{classifier.cc}:
\begin{program}
        /*
         *{\cf objects only ever see "packet" events, which come either}
         *{\cf from an incoming link or a local agent (i.e., packet source).}
         */
        void Classifier::recv(Packet* p, Handler*)
        \{
                NsObject* node;
                int cl = classify(p);
                if (cl < 0 || cl >= nslot_ || (node = slot_[cl]) == 0) \{
                        Tcl::instance().evalf("%s no-slot %d", name(), cl);
                        Packet::free(p);
                        return;
                \}
                node->recv(p);
        \}

        int Classifier::command(int argc, const char*const* argv)
        \{
                Tcl& tcl = Tcl::instance();
                if (argc == 3) \{
                        /*
                         * $classifier clear $slot
                         */
                        if (strcmp(argv[1], "clear") == 0) \{
                                int slot = atoi(argv[2]);
                                clear(slot);
                                return (TCL_OK);
                        \}
                        /*
                         * $classifier installNext $node
                         */
                        if (strcmp(argv[1], "installNext") == 0) \{
                                int slot = maxslot_ + 1;
                                NsObject* node = (NsObject*)TclObject::lookup(argv[2]);
                                install(slot, node);
                                tcl.resultf("%u", slot);
                                return TCL_OK;
                        \}
                        if (strcmp(argv[1], "slot") == 0) \{
                                int slot = atoi(argv[2]);
                                if ((slot >= 0) || (slot < nslot_)) \{
                                        tcl.resultf("%s", slot_[slot]->name());
                                        return TCL_OK;
                                \}
                                tcl.resultf("Classifier: no object at slot %d", slot);
                                return (TCL_ERROR);
                        \}
                \} else if (argc == 4) \{
                        /*
                         * $classifier install $slot $node
                         */
                        if (strcmp(argv[1], "install") == 0) \{
                                int slot = atoi(argv[2]);
                                NsObject* node = (NsObject*)TclObject::lookup(argv[3]);
                                install(slot, node);
                                return (TCL_OK);
                        \}
                \}
                return (NsObject::command(argc, argv));
        \}
\end{program}
When a classifier \fcn[]{recv}'s a packet,
it hands it to the \fcn[]{classify} method.
This is defined differently in each type of classifier
derived from the base class.
The usual format is for the \fcn[]{classify} method to
determine and return a slot index into the table of slots.
If the index is valid, and points to a valid TclObject,
the classifier will hand the packet to that object using 
that object's \fcn[]{recv} method.
If the index is not valid, the classifier will invoke
the instance procedure \proc[]{no-slot} to attempt to 
populate the table correctly.
However, in the base class \proc[]{Classifier::no-slot} prints
and error message and terminates execution.

The \fcnref{\fcn[]{command} method}{../ns-2/classifier.cc}{Classifier::command}
provides the following instproc-likes to the interpreter:
\begin{itemize}\itemsep0pt
\item \proc[\tup{slot}]{clear} clears the entry in a particular slot.
\item \proc[\tup{object}]{installNext} installs the object
        in the next available slot, and returns the slot number.

        Note that this instproc-like is 
        \fcnref{overloaded by an instance procedure of the same name}{%
                ../ns-2/ns-lib.tcl}{Classifier::installNext}
        that stores a reference to the object stored.
        This then helps quick query of the objects
        installed in the classifier from OTcl.
\item \proc[\tup{index}]{slot} returns the object stored in the specified slot.
\item \proc[\tup{index}, \tup{object}]{install} installs the specified
        \tup{object} at the slot \tup{index}.

        Note that this instproc-like too is 
        \fcnref{overloaded by an instance procedure of the same name}{%
                ../ns-2/ns-lib.tcl}{Classifier::install}
        that stores a reference to the object stored.
        This is also to quickly query of the objects
        installed in the classifier from OTcl.
\end{itemize}

\subsection{Address Classifiers}
\label{sec:node:addr-classifier}

An address classifier is used in supporting unicast packet forwarding.
It applies a bitwise shift and mask operation to a packet's destination
address to produce a slot number.
The slot number is returned from the \fcn[]{classify} method.
The \clsref{AddressClassifier}{../ns-2/classifier-addr.cc}
(defined in \nsf{classifier-addr.cc}) ide defined as follows:
\begin{program}
        class AddressClassifier : public Classifier \{
        public:
                AddressClassifier() : mask_(~0), shift_(0) \{
                        bind("mask_", (int*)&mask_);
                        bind("shift_", &shift_);
                \}
        protected:
                int classify(Packet *const p) \{
                        IPHeader *h = IPHeader::access(p->bits());
                        return ((h->dst() >> shift_) & mask_);
                \}
                nsaddr_t mask_;
                int shift_;
        \};
\end{program}
The class imposes no direct semantic meaning
on a packet's destination address field.
Rather, it returns some number of bits from the packet's
\code{dst_} field as the slot number used
in the \fcnref{\fcn[]{Classifier::recv}}{../ns-2/classifier.cc}{Classifier::recv} method.
The \code{mask_} and \code{shift_} values are set through OTcl.

\subsection{Multicast Classifiers}
\label{sec:node:mcast-classifier}

The multicast classifier classifies packets
according to both source and destination (group) addresses.
It maintains a (chained hash) table mapping source/group pairs to slot numbers.
When a packet arrives containing a source/group unknown to the classifier,
it invokes an Otcl procedure \proc[]{Node::new-group}
to add an entry to its table.
This OTcl procedure may use the method \code{set-hash} to add
new (source, group, slot) 3-tuples to the classifier's table.
The multicast classifier is defined in \nsf{classifier-mcast.cc}
as follows:
\begin{program}
        static class MCastClassifierClass : public TclClass \{
        public:
                MCastClassifierClass() : TclClass("Classifier/Multicast") \{\}
                TclObject* create(int argc, const char*const* argv) \{
                        return (new MCastClassifier());
                \}
        \} class_mcast_classifier;

        class MCastClassifier : public Classifier \{
        public:
                MCastClassifier();
                ~MCastClassifier();
        protected:
                int command(int argc, const char*const* argv);
                int classify(Packet *const p);
                int findslot();
                void set_hash(nsaddr_t src, nsaddr_t dst, int slot);
                int hash(nsaddr_t src, nsaddr_t dst) const \{
                        u_int32_t s = src ^ dst;
                        s ^= s >> 16;
                        s ^= s >> 8;
                        return (s & 0xff);
                \}
                struct hashnode \{
                        int slot;
                        nsaddr_t src;
                        nsaddr_t dst;
                        hashnode* next;
                \};
                hashnode* ht_[256];
                const hashnode* lookup(nsaddr_t src, nsaddr_t dst) const;
        \};

        int MCastClassifier::classify(Packet *const pkt)
        \{
                IPHeader *h = IPHeader::access(pkt->bits());
                nsaddr_t src = h->src() >> 8; /*XXX*/
                nsaddr_t dst = h->dst();
                const hashnode* p = lookup(src, dst);
                if (p == 0) \{
                        /*
                         * Didn't find an entry.
                         * Call tcl exactly once to install one.
                         * If tcl doesn't come through then fail.
                         */
                        Tcl::instance().evalf("%s new-group %u %u", name(), src, dst);
                        p = lookup(src, dst);
                        if (p == 0)
                                return (-1);
                \}
                return (p->slot);
        \}
\end{program}
The \clsref{MCastClassifier}  implements a chained hash table
and applies a hash function on both the packet source and
destination addresses.
The hash function returns the slot number
to index the \code{slot_} table in the underlying object.
A hash miss implies packet delivery to a previously-unknown group;
OTcl is called to handle the situation.
The OTcl code is expected to insert an appropriate entry into the hash table.

\subsection{MultiPath Classifier}
\label{sec:node:mpath-classifier}

This object is devised to support equal cost multipath
forwarding, where the node has multiple equal cost routes
to the same destination, and would like to use all of them
simultaneously.
This object does not look at any field in the packet.
With every succeeding packet, 
it simply returns the next filled slot in round robin fashion.
The definitions for this classifier are in \nsf{classifier-mpath.cc},
and are shown below:
\begin{program}
class MultiPathForwarder : public Classifier \{
public:
        MultiPathForwarder() : ns_(0), Classifier() \{\} 
        virtual int classify(Packet* const) \{
                int cl;
                int fail = ns_;
                do \{
                        cl = ns_++;
                        ns_ %= (maxslot_ + 1);
                \} while (slot_[cl] == 0 && ns_ != fail);
                return cl;
        \}
private:
        int ns_;     \* next slot to be used.  Probably a misnomer? */
\};
\end{program}

\subsection{Hash Classifier}
\label{sec:node:hash-classifier}

This object is used to classify a packet as a member of a
particular {\em flow}.
As their name indicates,
hash classifiers use a hash table internally to assign
packets to flows.
These objects are used where flow-level information is
required (e.g. in flow-specific queuing disciplines and statistics
collection).
Several ``flow granularities'' are available.  In particular,
packets may be assigned to flows based on flow ID, destination address,
source/destination addresses, or the combination of source/destination
addresses plus flow ID.
The fields accessed by the hash classifier are limited to
the {\tt ip} header: {\tt src(), dst(), flowid()} (see {\tt ip.h}).

The hash classifier is created with an integer argument specifying
the initial size of its hash table.  The current hash table size may
be subsequently altered with the {\tt resize} method (see below).
When created, the instance variables \code{shift_} and \code{mask_}
are initialized with the simulator's current {\sf NodeShift} and
{\sf NodeMask} values, respectively.  These values are retrieved
from the {\tt AddrParams} object when the hash classifier is
instantiated.  The hash classifier will fail to operate properly if
the {\tt AddrParams} structure is not initialized.
The following constructors are used for the various hash classifiers:
\begin{program}
        Classifier/Hash/SrcDest
        Classifier/Hash/Dest
        Classifier/Hash/Fid
        Classifier/Hash/SrcDestFid
\end{program}

The hash classifier receives packets, classifies them according
to their flow criteria, and retrieves the classifier {\em slot}
indicating the next node that should receive the packet.
In several circumstances with hash classifiers, most packets should
be associated with a single slot, while only a few flows should
be directed elsewhere. 
The hash classifier includes a \code{default_} instance variable
indicating which slot is to be used for packets that do not match
any of the per-flow criteria.
The \code{default_} may be set optionally.

The methods for a hash classifier are as follows:
\begin{program}
        $hashcl set-hash buck src dst fid slot
        $hashcl lookup buck src dst fid
        $hashcl del-hash src dst fid
        $hashcl resize nbuck
\end{program}

The \fcn[]{set-hash} method inserts a new entry into the hash
table within the hash classifier.
The {\tt buck} argument specifies the hash table bucket number
to use for the insertion of this entry.
When the bucket number is not known, {\tt buck} may be specified
as {\tt auto}. 
The {\tt src, dst} and {\tt fid} arguments specify the IP source,
destination, and flow IDs to be matched for flow classification.
Fields not used by a particular classifier (e.g. specifying {\tt src}
flor a flow-id classifier) is ignored.
The {\tt slot} argument indicates the index into the underlying
slot table in the base {\tt Classifier} object from which
the hash classifier is derived.
The {\tt lookup} function returns the name of the object
associated with the given {\tt buck/src/dst/fid} tuple.
The {\tt buck} argument may be {\tt auto}, as for {\tt set-hash}.
The {\tt del-hash} function removes the specified entry from
the hash table.
Currently, this is done by simply marking the entry as inactive,
so it is possible to populate the hash table with unused entries.
The {\tt resize} function resizes the hash table to include
the number of buckets specified by the argument {\tt nbuck}.

Provided no default is defined, a hash classifier will
perform a call into OTcl when it
receives a packet which matches no flow criteria.
The call takes the following form:
\begin{program}
        \$obj unknown-flow src dst flowid buck
\end{program} 
Thus, when a packet matching no flow criteria is received,
the method {\tt unknown-flow} of the instantiated hash classifier
object is invoked with the source, destination, and flow id
fields from the packet.
In addition, the {\tt buck} field indicates the hash bucket
which should contain this flow if it were inserted using
{\tt set-hash}.  This arrangement avoids another hash
lookup when performing insertions into the classifier when the
bucket is already known.

\subsection{Replicator}
\label{sec:node:replicator}

The replicator is different from the other classifiers
we have described earlier,
in that it does not use the classify function.
Rather, it simply uses the classifier as a table of $n$ slots;
it overloads the \fcn[]{recv} method to produce $n$ copies
of a packet, that are delivered to all $n$ objects referenced in the table.

To support multicast packet forwarding, a classifier receiving a
multicast packet from source $S$
destined for group $G$ computes a hash function $h(S,G)$ giving
a ``slot number'' in the classifier's object table.
%Thus, the maximum size of the table is $O(|S|\times|G|)$.
In multicast delivery, the packet must be copied once for
each link leading to nodes subscribed to $G$ minus one.
Production of additional copies of the packet is performed
by a \code{Replicator} class, defined in \code{replicator.cc}:
\begin{program}
        /*
         * {\cf A replicator is not really a packet classifier but}
         * {\cf we simply find convenience in leveraging its slot table.}
         * {\cf (this object used to implement fan-out on a multicast}
         * {\cf router as well as broadcast LANs)}
         */
        class Replicator : public Classifier \{
        public:
                Replicator();
                void recv(Packet*, Handler* h = 0);
                virtual int classify(Packet* const) \{\};
        protected:
                int ignore_;
        \};

        void Replicator::recv(Packet* p, Handler*)
        \{
                IPHeader *iph = IPHeader::access(p->bits());
                if (maxslot_ < 0) \{
                        if (!ignore_)
                                Tcl::instance().evalf("%s drop %u %u", name(), 
                                        iph->src(), iph->dst());
                        Packet::free(p);
                        return;
                \}
                for (int i = 0; i < maxslot_; ++i) \{
                        NsObject* o = slot_[i];
                        if (o != 0)
                                o->recv(p->copy());
                \}
                /* {\cf we know that maxslot is non-null} */
                slot_[maxslot_]->recv(p);
        \}
\end{program}
As we can see from the code,
this class  does not really classify packets.
Rather, it replicates a packet, one for each entry in its table,
and delivers the copies to each of the nodes listed in the table.
The last entry in the table gets the ``original'' packet.
Since the \fcn[]{classify} method is pure virtual in the base class,
the replicator defines an empty \fcn[]{classify} method.

\clearpage

\section{Commands at a glance}
\label{sec:nodescommand}
\begin{flushleft}
Following is a list of common node commands used in simulation scripts:

\code{$ns_ node}\\
Command to create a simple node. This returns a handle to the node instance
created.


\code{$ns_ node <hierarchical address>}\\
Command to create a node with hierarchical addressing/routing. This too returns
a handle to the hier-node instance.


\code{$ns_ node-config <-option> <value>}\\
This command is a part of the new Node APIs and sets up configuration for a
given node type. Note that this command has to be called before creation of
the nodes. The default value (i.e if node-config is not called) sets up
configuration for a simple node with flat addressing/routing. The details
on different <options> and their available <values> that can be used to
configure a node can be found in chapter titled "Restructuring ns node
and new Node APIs" in ns Notes and Documentation. 


\code{$node id}\\
Returns the id number of the node.


\code{$node node-addr}\\
Returns the address of the node. In case of flat addressing, the node address
is same as its node-id. In case of hierarchical addressing, the node address
in the form of a string (viz. "1.4.3") is returned.


\code{$node reset}\\
Resets all agent attached to this node.


\code{$node agent <port_num>}\\
Returns the handle of the agent at the specified port. If no agent is found
at the given port, a null string is returned.


\code{$node entry}\\
Returns the entry point for the node. This is first object that handles packet
receiving at this node.


\code{$node attach <agent> <optional:port_num>}\\
Attaches the <agent> to this node. Incase no specific port number is passed,
the node allocates a port number and binds the agent to this port. Thus once
the agent is attached, it receives packets destined for this host (node) and port.


\code{$node detach <agent> <null_agent>}\\
This is the dual of "attach" described above. It detaches the agent from this node
and installs a null-agent to the port this agent was attached. This is done to
handle transit packets that may be destined to the detached agent. These on-the-fly
pkts are then sinked  at the null-agent.


\code{$node neighbors}\\
This returns the list of neighbors for the node.


\code{$node add-neighbor <neighbor_node>}\\
This is a command to add \code{<neighbor_node>} to the list of neighbors 
maintained by the node.


Following is a list of internal node methods:

\code{$node enable-mcast}\\
This is an internal procedure used by Class Simulator while creating a 
node and is used to install additional multicast classifiers within the node
that converts an unicast node into a multicast one.


\code{$node add-route <destination_id> <target>}\\
This is used in unicast routing to populate the classifier. The target is a
Tcl object, which may be the entry of \code{dmux_} (port demultiplexer in
the node) incase the \code{<destination_id>} is same as this node-id.
Otherwise it is usually the head of the link for that destination. It
could also be the entry for other classifiers.


\code{$node alloc-port <null_agent>}\\
This returns the next available port number. Uses instance variable
\code{np_} to track the next unallocated port number.


\code{$node incr-rtgtable-size}\\
The instance variable \code{rtsize_} is used to keep track of size of
routing-table in each node. This command is used to increase the
routing-table size every time an routing-entry is added to the
classifiers.


\code{$node getNode}\\
Returns the instance of self (the node).


\code{$node attachInterfaces <ifaces>}\\
Attaches the node to each of the <ifaces>.


\code{$node addInterface <iface>}\\
Adds the <iface> to the list of interfaces for the node.


\code{$node createInterface <num>}\\
Creates a new network interface and labels it with <num>.


\code{$node getInterfaces}\\
Returns the list of interfaces for the node.


There are other node commands that supports hierarchical
routing, PIM, IntTCP, detailed dynamic routing, equal cost multipath
routing, PGM router, manual routing and power model for mobilenodes. These
and other methods described earlier can be found in 
\ns/tcl/lib/ns-node.tcl.

\end{flushleft}
\endinput

\chapter{Links: Simple Links}
\label{chap:links}

This is the second aspect of defining the topology.
In \href{the previous chapter}{Chapter}{chap:nodes},
we had described how to create the nodes in the topology in \ns.
We now describe how to create the links to connect the nodes and complete
the topology.
In this chapter, we restrict ourselves to describing the simple
point to point links.
\ns\ supports a variety of other media, including
an emulation of a multi-access LAN using a mesh of simple links,
and other true simulation of wireless and broadcast media.
They will be described in a separate chapter.
The CBQlink is derived from simple links and is a considerably more
complex form of link that is also not described in this chapter.

We begin by describing the commands to create a link in this section.
As with the node being composed of classifiers, 
a simple link is built up from a sequence of connectors.
We also briefly describe some of the connectors in a simple link.
We then describe
\href{the instance procedures that operate on the various components of
defined by some of these connectors}{Section}{sec:links:components}.
We conclude the chapter
\href{with a description the connector object}{Section}{sec:links:connectors},
including brief
descriptions of the common link connectors.

The \clsref{Link}{../ns-2/ns-link.tcl}
is a standalone class in OTcl,
that provides a few simple primitives.
The \clsref{SimpleLink}{../ns-2/ns-link.tcl}
provides the ability to connect two nodes with a point to point link.
\ns\ provides the instance procedure
\fcnref{\proc[]{simplex-link}}{../ns-2/ns-link.tcl}{Simulator::simplex-link}
to form a unidirectional link from one node to another.
The link is in the class SimpleLink.
The following describes the syntax of the simplex link:
\begin{program}
    set ns [new Simulator]
    $ns simplex-link \tup{node0} \tup{node1} \tup{bandwidth} \tup{delay} \tup{queue_type}
\end{program}
The command creates a link from \code{\tup{node0}} to \code{\tup{node1}},
with specified \code{\tup{bandwidth}} and \code{\tup{delay}} characteristics.
The link uses a queue of type \code{\tup{queue_type}}.
The procedure also adds a TTL checker to the link.
Five instance variables define the link:

\centerline{\begin{tabular}{rp{4in}}
	\code{head\_} &	Entry point to the link, it points
			to the first object in the link. \\
	\code{queue\_} & Reference to the main queue element of the link.
			Simple links usually have one queue per link.
			Other more complex types of links may have multiple
			queue elements in the link. \\
	\code{link\_} &	A reference to the element
			that actually models the link,
			in terms of the delay and bandwidth characteristics
			of the link. \\
	\code{ttl\_} &	Reference to the element that manipulates the
			ttl in every packet. \\
	\code{drophead\_} & Reference to an object that is the head of a
			queue of elements that process link drops. \\
	    \end{tabular}}

In addition, if the simulator instance variable, 
\code{$traceAllFile_}, is defined, 
the procedure will add trace elements that track when a packet is
enqueued and dequeued from \code{queue_}.
Furthermore, tracing interposes a drop trace element after the
\code{drophead_}.
\begin{figure}[tb]
  \centerline{\includegraphics{link}}
  \caption{Composite Construction of a Unidirectional Link}
  \label{fig:link}
\end{figure}
The following instance variables track the trace elements:

\centerline{\begin{tabular}{rp{4in}}
	\code{enqT\_} &	Reference to the element that traces
			packets entering \code{queue\_}.\\
	\code{deqT\_} &	Reference to the element that traces
			packets leaving \code{queue\_}.\\
	\code{drpT\_} &	Reference to the element that traces
			packets dropped from \code{queue\_}.\\
	    \end{tabular}}

Note however, that if the user enable tracing multiple times on the link,
these instance variables will only store a reference to the
last elements inserted.

Other configuration mechanisms that add components to a simple link
are network interfaces (used in multicast routing), 
link dynamics models, and tracing and monitors.
We give 
\href{a brief overview of the related objects at the end of this chapter}{%
		Section}{sec:links:connectors},
and discuss their functionality/implementation in other chapters.

The instance procedure
\fcnref{\proc[]{duplex-link}}{../ns-2/ns-link.tcl}{Simulator::duplex-link}
constructs a bi-directional link from two simplex links.

\section{Instance Procedures for Links and SimpleLinks}
\label{sec:links:components}

\paragraph{Link procedures}
The \clsref{Link}{../ns-2/ns-link.tcl} is implemented entirely in Otcl.
The OTcl \code{SimpleLink} class uses the C++ \code{LinkDelay} class
to simulate packet delivery delays.
The instance procedures in the class Link are:

\begin{tabularx}{\linewidth}{rX}
\fcnref{\proc[]{head}}{../ns-2/ns-link.tcl}{Link::head} &
		returns the handle for \code{head\_}. \\
\fcnref{\proc[]{queue}}{../ns-2/ns-link.tcl}{Link::queue} &
		returns the handle for \code{queue\_}. \\
\fcnref{\proc[]{link}}{../ns-2/ns-link.tcl}{Link::link} &
		returns the handle for the delay element, \code{link\_}. \\
\fcnref{\proc[]{up}}{../ns-2/ns-link.tcl}{Link::up} &
		set link status to ``up'' in the \code{dynamics\_} element.
		Also, writes out a trace line to each file specified through
		the procedure \proc[]{trace-dynamics}.\\
\fcnref{\proc[]{down}}{../ns-2/ns-link.tcl}{Link::down} &
		As with \proc[]{up},
		set link status to ``down'' in the \code{dynamics\_} element.
		Also, writes out a trace line to each file specified through
		the procedure \proc[]{trace-dynamics}.\\
\fcnref{\proc[]{up?}}{../ns-2/ns-link.tcl}{Link::up?} &
		returns status of the link.   Status is ``up'' or ``down'';
		status is ``up'' if link dynamics is not enabled.\\
\fcnref{\proc[]{all-connectors}}{../ns-2/ns-link.tcl}{Link::all-connectors} &
		Apply specified operation to all connectors on the link.p
		An example of such usage is \code{\$link all-connectors reset}.\\
\fcnref{\proc[]{cost}}{../ns-2/ns-link.tcl}{Link::cost} &
		set link cost to value specified.\\
\fcnref{\proc[]{cost?}}{../ns-2/ns-link.tcl}{Link::cost?} &
		returns the cost of the link.  Default cost of link is 1, 
		if no cost has been specified earlier.\\
\end{tabularx}

\paragraph{SimpleLink Procedures}
The Otcl \clsref{SimpleLink}{../ns-2/ns-link.tcl}
implements a simple point-to-point
link with an associated queue and delay\footnote{The current
version also includes an object to examine the
network layer ``ttl'' field and discard packets if the
field reaches zero.}.
It is derived from the base Otcl class Link as follows:
\begin{program}
        Class SimpleLink -superclass Link
        SimpleLink instproc init \{ src dst bw delay q \{ lltype "DelayLink" \} \} \{
                $self next $src $dst
                $self instvar link_ queue_ head_ toNode_ ttl_
                ...
                set queue_ $q
                set link_ [new Delay/Link]
                $link_ set bandwidth_ $bw
                $link_ set delay_ $delay

                $queue_ target $link_
                $link_ target [$toNode_ entry]

                ...
                # XXX
                # put the ttl checker after the delay
                # so we don't have to worry about accounting
                # for ttl-drops within the trace and/or monitor
                # fabric
                #
                set ttl_ [new TTLChecker]
                $ttl_ target [$link_ target]
                $link_ target $ttl_
        \}
\end{program}
Notice that when a \code{SimpleLink} object is created,
new \code{Delay/Link} and \code{TTLChecker} objects are
also created.
Note also that,
the \code{Queue} object must have already been created.

There are two additional methods implemented (in OTcl) as part
of the \code{SimpleLink} class: \code{trace} and \code{init-monitor}.
These functions are described in further detail
\href{in the section on tracing}{Chapter}{chap:trace}. 

\section{Connectors}
\label{sec:links:connectors}

Connectors, unlink  classifiers, ony generate data for one recipient;
either the packe is delivered to the \code{target_} neighbour, or it
is sent ot he \code{drop-target_}.

A connector will receive a packet, perform some function,
and deliver the packet to its neighbour, or drop the packet.
There are a number of differnt types of connectors in \ns.
Each connector performs a different function.

\begin{tabularx}{\linewidth}{rX}
networkinterface & labels packets with incoming interface identifier---it 
			is used by some multicast routing protocols.
			The class variable ``Simulator NumberInterfaces\_ 1''
			tells \ns\ to add these interfaces, and then, it is
			added to either end of the simplex link.
			\href{Multicast routing protocols are discussed in
				a separate chapter}{Chapter}{chap:multicast}.\\
DynaLink &	Object that gates traffic depending on whether the link 
		is up or down.  It expects to be at the head of the link,
		and is inserted on the link just prior to simulation start.
		It's \code{status\_} variable control whether the link is
		up or down.
		\href{The description of how the DynaLink object is used
		is in a separate chapter}{Chapter}{chap:net-dynamics}.\\
DelayLink &	Object that models the link's
		delay and bandwidth characteristics.
		If the link is not dynamic, then this object simply
		schedules receive events for the downstream object
		for each packet it receives at the appropriate time
		for that packet.  However, if the link is dynamic,
		then it queues the packets internally, and schedules
		one receives event for itself for the next packet that must
		be delivered.
		Thus, if the link goes down at some point, this object's
	 \fcnref{\fcn[]{reset} method}{../ns-2/delay.cc}{DelayLink::reset}
		is invoked, and the object will drop all packets in transit
		at the instant of link failure.
		We discuss the
		\href{specifics of this class in another chapter}{Chapter}{%
			chap:delays}.\\
Queues &	model the output buffers attached
		to a link in a ``real'' router in a network.
		In \ns, they are attached to, and 
		are considered as part of the link.
		We discuss the
		\href{details of queues and different types of queues in \ns
			in another chapter}{Chapter}{chap:qmgmt}.\\
TTLChecker &	will decrement the ttl in each packet that it receives.
		If that ttl then has a positive value, the packet is forwarded
		to the next element on the link.  In the simple links,
		TTLCheckers are automatically added, and are placed
		as the last element on the link, between the delay element,
		and the entry for the next node.\\
\end{tabularx}

\include{packets}
%       This draft written by Tom Henderson (8/29/97) based on John Heidemann's
%   code comments.
%
%
% If you get conflicts, here's what you need to keep:  The chapter heading
% in the first entry is essential.  The \endinput at end is useful.
% Other mods are to promote each sub*section one level up.
%
\chapter{\shdr{Timers}{timer-handler.h}{sec:timers}}

Timers may be implemented in C++ or OTcl.  In C++, timers are based on an 
abstract base class defined in \code{timer-handler.h}.  They are most often 
used in agents, but the 
framework is general enough to be used by other objects.  The discussion
below is oriented towards the use of timers in agents.

In OTcl, a simple timer class is defined in \code{tcl/ex/timer.tcl}.  
Subclasses can be derived to provide a simple mechanism for scheduling events 
at the OTcl level.

\section{\shdr{C++ abstract base class TimerHandler}{timer-handler.h}{sec:abstractbaseclass}}

The abstract base class \code{TimerHandler} contains the following public member functions:
\begin{tt}
\begin{quote}
\begin{itemize}
\item[void sched(double delay)] - schedule a timer to expire delay seconds in the future
\item[void resched(double delay)] - reschedule a timer (similar to sched(), but
timer may be pending)
\item[void cancel()] - cancel a pending timer
\item[int status()] - returns timer status (either IDLE, PENDING, or HANDLING)
\end{itemize}
\end{quote}
\end{tt}

The abstract base class \code{TimerHandler} contains the following protected members:
\begin{tt}
\begin{quote}
\begin{itemize}
\item[virtual void expire(Event *e) = 0] - this method must be filled in by the timer client
\item[virtual void handle(Event *e) = 0] - consumes an event 
\item[int status\_] - keeps track of the current timer status
\item[Event event\_] - event to be consumed upon timer expiry 
\end{itemize}
\end{quote}
\end{tt}

The pure virtual functions must be defined by the timer classes deriving
from this abstract base class.

Finally, two private inline functions are defined:
\begin{small}
\begin{verbatim}
        inline void _sched(double delay) {
            (void)Scheduler::instance().schedule(this, &event_, delay);
        }
        inline void _cancel() {
            (void)Scheduler::instance().cancel(&event_);
        }
\end{verbatim}
\end{small}

From this code we can see that timers make use of methods of the 
\code{Scheduler} class.

\subsection{\shdr{Definition of a new timer}{timer-handler.h}{sec:definition}}

To define a new timer, subclass this function and define handle() if needed 
(handle() is not always required):

\begin{small}
\begin{verbatim}

        class MyTimer : public TimerHandler {
        public:
          MyTimer(MyAgentClass *a) : TimerHandler() { a_ = a; }
          virtual double expire(Event *e);
        protected:
          MyAgentClass *a_;
        };

\end{verbatim}
\end{small}

Then define expire:

\begin{small}
\begin{verbatim}

        double
        MyTimer::expire(Event *e)
        {
          // do the work
          // return TIMER_HANDLED;    // => do not reschedule timer
          // return delay;            // => reschedule timer after delay
        }

\end{verbatim}
\end{small}

Note that \code{expire()} can return either the flag TIMER\_HANDLED or a
delay value, depending on the requirements for this timer.

Often \code{MyTimer} will be a friend of \code{MyAgentClass}, or 
\code{expire()} will only call a public function of \code{MyAgentClass}.

Timers are not directly accessible from the OTcl level, although users are
free to establish method bindings if they so desire.

\subsection{\shdr{Example: Tcp retransmission timer}{tcp.cc}{sec:timerexample}}

TCP is an example of an agent which requires timers.  There are three timers
defined in the basic Tahoe TCP agent defined in \code{tcp.cc}:

\begin{small}
\begin{verbatim}
        rtx_timer_;      //  Retransmission timer
        delsnd_timer_;   //  Delays sending of packets by a small random
                             amount of time, to avoid phase effects
        burstsnd_timer_;   // Helps TCP to stagger the transmission of a large
                              window into several smaller bursts
\end{verbatim}
\end{small}

In \code{tcp.h}, three classes are derived from the base class 
\code{TimerHandler}:
\begin{small}
\begin{verbatim}

class RtxTimer : public TimerHandler {
public:
    RtxTimer(TcpAgent *a) : TimerHandler() { a_ = a; }
protected:                   
    virtual void expire(Event *e);
    TcpAgent *a_;
};  
    
class DelSndTimer : public TimerHandler {
public:
    DelSndTimer(TcpAgent *a) : TimerHandler() { a_ = a; }
protected:
    virtual void expire(Event *e);
    TcpAgent *a_;
};      
    
class BurstSndTimer : public TimerHandler {
public: 
    BurstSndTimer(TcpAgent *a) : TimerHandler() { a_ = a; }
protected:
    virtual void expire(Event *e); 
    TcpAgent *a_;
};  

\end{verbatim}
\end{small}

In the constructor for \code{TcpAgent} in \code{tcp.cc}, each of these timers
is initialized with the \code{this} pointer, which is assigned to the pointer
\code{a_}.

\begin{small}
\begin{verbatim}

TcpAgent::TcpAgent() : Agent(PT_TCP), rtt_active_(0), rtt_seq_(-1), 
    ...
    rtx_timer_(this), delsnd_timer_(this), burstsnd_timer_(this)
{
    ...
}

\end{verbatim}
\end{small}

In the following, we will focus only on the retransmission timer.  Various
helper methods may be defined to schedule timer events; \eg,

\begin{small}
\begin{verbatim}

/*
 * Set retransmit timer using current rtt estimate.  By calling resched(),
 * it does not matter whether the timer was already running.
 */
void TcpAgent::set_rtx_timer()
{
    rtx_timer_.resched(rtt_timeout());
}

/*
 * Set new retransmission timer if not all outstanding
 * data has been acked.  Otherwise, if a timer is still
 * outstanding, cancel it.
 */
void TcpAgent::newtimer(Packet* pkt)
{
    hdr_tcp *tcph = (hdr_tcp*)pkt->access(off_tcp_);
    if (t_seqno_ > tcph->seqno())
        set_rtx_timer();
    else if (rtx_timer_.status() == TIMER_PENDING)
        rtx_timer_.cancel();
}

\end{verbatim}
\end{small}

In the above code, the \code{set_rtx_timer()} method reschedules the 
retransmission timer by calling \code{rtx_timer_.resched()}.  Note that if
it is unclear whether or not the timer is already running, calling
\code{resched()} eliminates the need to explicitly cancel the timer.  In
the second function, examples are given of the use of the \code{status()}
and \code{cancel()} methods.

Finally, the \code{expire()} method for class \code{RtxTimer} must be 
defined.  In this case, \code{expire()} calls the \code{timeout()} method
for \code{TcpAgent}.  This is possible because \code{timeout()} is a 
public member function; if it were not, then \code{RtxTimer} would have
had to have been declared a friend class of \code{TcpAgent}.

\begin{small}
\begin{verbatim}

void TcpAgent::timeout(int tno)
{                     
    /* retransmit timer */
    if (tno == TCP_TIMER_RTX) {
        if (highest_ack_ == maxseq_ && !slow_start_restart_) {
            /*
             * TCP option:
             * If no outstanding data, then don't do anything.
             */
            return;  
        };
        recover_ = maxseq_;
        recover_cause_ = 2;
        closecwnd(0);
        reset_rtx_timer(0,1);
        send_much(0, TCP_REASON_TIMEOUT, maxburst_); 
    }       
    else {  
        /*  
         * delayed-send timer, with random overhead
         * to avoid phase effects  
         */     
        send_much(1, TCP_REASON_TIMEOUT, maxburst_);
    }           
}           
            
void RtxTimer::expire(Event *e) {
    a_->timeout(TCP_TIMER_RTX);
}

\end{verbatim}
\end{small}

The various TCP agents contain additional examples of timers.

\section{\shdr{OTcl Timer class}{timer.tcl}{sec:otcltimer}}

A simple timer class is defined in \code{tcl/ex/timer.tcl}.  Subclasses of
\code{Timer} can be defined as needed.  Unlike the C++ timer API, where a 
\code{sched()} aborts if the timer is already set, \code{sched()} and
\code{resched()} are the same; i.e., no state is kept for the OTcl timers.
The following methods are defined in the \code{Timer} base class:
\begin{program}

    $self sched $delay   \; causes "$self timeout" to be called $delay seconds in the future;
    $self resched $delay \; same as "$self sched $delay" ;
    $self cancel         \; cancels any pending scheduled callback;
    $self destroy        \; same as "$self cancel";
    $self expire         \; calls "$self timeout" immediately;

\end{program}

\endinput

\include{events}
%
% personal commentary:
%        handlers and how they are used are confusing
%        Connector::send is needed, but so is just send()... confusing
%        default handler in Connector::recv is confusing
%        this is a DRAFT DRAFT DRAFT
%        - KFALL
%
\chapter{\shdr{Agents}{agent.h}{sec:agents}}

Agents represent endpoints where network-layer
packets are constructed or consumed, and provide
some functions helpful in developing transport-layer and other
protocols.
Generally, a user wishing to create a new
source or sink for network-layer packets
will create a class derived from {\tt Agent}.
The class \code{Agent} has an implementation partly in
OTcl and partly in C++.
The C++ implementation is contained in \code{agent.cc} and
\code{agent.h}, and the OTcl support is in
\code{tcl/lib/ns-agent.tcl}.

\section{\shdr{Agent state}{agent.h}{sec:agentstate}}

The C++ class \code{Agent} includes enough internal state
to assign various fields to a simulated packet before
it is sent.
This state includes the following:

\begin{tabularx}{\linewidth}{rX}
\code{addr\_} & node address of myself (source address in packets) \\
\code{dst\_} & where I am sending packets to \\
\code{size\_} & packet size in bytes (placed into the common packet header) \\
\code{type\_} & type of packet (in the common header, see packet.h) \\
\code{fid\_} & the IP flow identifier (formerly {\em class} in ns-1) \\
\code{prio\_} & the IP priority field \\
\code{flags\_} & packet flags (similar to ns-1) \\
\code{defttl\_} & default IP ttl value \\
\end{tabularx}

These variables may be modified by any class derived from \code{Agent},
although not all of them may be needed by any particular agent.

\section{\shdr{Agent methods}{agent.h}{sec:agentmethods}}

The \code{Agent} class supports packet generation and reception.
The following member functions are implemented by the C++ Agent class, and are
generally {\em not} over-ridden by derived classes:

\begin{tabularx}{\linewidth}{rX}
\fcn[]{Packet* allocpkt} & allocate new packet and assign its fields \\
\fcn[int]{Packet* allocpkt} & allocate new packet with a data payload of n bytes and assign its fields \\
\end{tabularx}

The following member functions are also defined by the \code{Agent} class,
but {\em are} intended to be over-ridden by classes deriving from Agent:

\begin{tabularx}{\linewidth}{rX}
  \fcn[timeout number]{void timeout} & subclass-specific time out method \\
  \fcn[Packet*, Handler*]{void recv} & receiving agent main receive path \\
\end{tabularx}

The \code{allocpkt} function is used by derived classes to create
packets to send.
The function fills in the following fields in the common packet
header (see \ref{sec:pformat}): {\tt uid, ptype, size}, and the
following fields in the IP header: {\tt src, dst, flowid, prio, ttl}.
It also zero-fills in the following fields of the Flags header:
{\tt ecn, pri, usr1, usr2}.
Any packet header information not included in these lists must
be must be handled in the classes derived from \code{Agent}.

The \code{recv} function is the main entrypoint for an
Agent which receives packets, and
is invoked by upstream nodes when sending a packet.
In most cases, Agents make no use of the second argument (the handler
defined by upstream nodes).

\section{\shdr{Protocol Agents}{cbr.h}{sec:protoagents}}

There are several agents supported in the simulator.
These are their names in OTcl:

\begin{longtable}{rl}
  TCP & a ``Tahoe'' TCP sender (cwnd = 1 on any loss)	\\
  TCP/Reno & a ``Reno'' TCP sender  (with fast recovery)	\\
  TCP/NewReno & a modified Reno TCP sender (changes fast recovery)	\\
  TCP/Sack1 & a SACK TCP sender	\\
  TCP/Fack & a ``forward'' SACK sender TCP 	\\
  TCP/FullTcp & a more full-functioned TCP with 2-way traffic	\\
  TCP/Vegas & a ``Vegas'' TCP sender	\\
  TCP/Vegas/RBP & a Vegas TCP with ``rate based pacing''	\\
  TCP/Vegas/RBP & a Reno TCP with ``rate based pacing''	\\
  TCP/Asym & an experimental Tahoe TCP for asymmetric links	\\
  TCP/Reno/Asym & an experimental Reno TCP for asymmetric links	\\
  TCP/Newreno/Asym & an experimental NewReno TCP for asymmetric links	\\
  TCPSink & a Reno or Tahoe TCP receiver (not used for FullTcp)	\\
  TCPSink/DelAck & a TCP delayed-ACK receiver	\\
  TCPSink/Asym & an experimental  TCP sink for asymmetric links	\\
  TCPSink/Sack1 & a SACK TCP receiver	\\
  TCPSink/Sack1/DelAck & a delayed-ACK SACK TCP receiver	\\
	\\
  CBR & connectioness protocol with sequence numbers	\\
  CBR/RTP & an RTP sender and receiver	\\
  CBR/UDP & UDP with sequence numbers and traffic sources	\\
  RTCP & an RTCP sender and receiver	\\
	\\
  LossMonitor & a packet sink which checks for losses	\\
	\\
  IVS/Source & an IVS source	\\
  IVS/Receiver & an IVS receiver	\\
	\\
  CtrMcast/Encap & a ``centralised multicast'' encapsulator	\\
  CtrMcast/Decap & a ``centralised multicast'' de-encapsulator	\\
  Message & a protocol to carry textual messages	\\
  Message/Prune & processes multicast routing prune messages	\\
	\\
  SRM & an SRM agent with non-adaptive timers	\\
  SRM/Adaptive & an SRM agent with adaptive timers	\\
	\\
  Tap & interfaces the simulator to a live network	\\
	\\
  Null & a degenerate agent which discards packets	\\
	\\
  rtProto/DV & distance-vector routing protocol agent	\\
\end{longtable}

Agents are used in the implementation of protocols at various layers.
Thus, for some transport protocols (e.g.~UDP) the distribution
of packet sizes and/or inter-departure times
may be dictated by some separate
object representing the demands of an application.
For agents used in the implementation of lower-layer protocols
(e.g. routing agents), size and departure timing is generally dictated
by the agent's own processing of protocol messages.

\section{\shdr{OTcl Linkage}{../ns/ns-default.tcl}{sec:agentotcl}}

Agents may be created within OTcl and an agent's internal
state can be modified by use of Tcl's \code{set} function and
any Tcl functions an Agent (or its base classes) implements.
Note that some of an Agent's internal state may exist
only within OTcl, and is thus is not directly accessible from C++.

\subsection{\shdr{creating and manipulating agents}{../ns/ns-lib.tcl}{sec:agentcreateotcl}}

The following example illustrates the creation and modification
of an Agent in OTcl:
\begin{program}
        set newtcp [new Agent/TCP] \; create new object (and C++ shadow object);
        $newtcp set window_ 20 \; sets the tcp agent's window to 20;
        $newtcp target $dest \; target is implemented in Connector class;
        $newtcp set portID_ 1 \; exists only in OTcl, not in C++;
\end{program}

\subsection{\shdr{default values}{../ns/ns-default.tcl}{sec:agentdefaults}}

Default values for member variables, those visible in OTcl only and those
linked between OTcl and C++ with \code{bind} are initialized
in the \nsf{tcl/lib/ns-default.tcl} file.  For example,
\code{Agent} and \code{CBR_Agent}
are initialized as follows:
\begin{program}
        Agent set fid_ 0
        Agent set prio_ 0
        Agent set addr_ 0
        Agent set dst_ 0
        Agent set flags_ 0

        Agent/CBR set interval_ 3.75ms
        Agent/CBR set random_ 0
        Agent/CBR set packetSize_ 210
	\ldots
\end{program}

Generally these initializations are placed in the OTcl namespace
before any objects of these types are created.
Thus, when an \code{Agent} or \code{Agent/CBR} object
is created, the calls to \code{bind}
in the objects' constructors will causes the corresponding member variables
to be set to these specified defaults.

\subsection{\shdr{OTcl methods}{../ns/ns-agent.tcl}{sec:agentmethodsotcl}}

The instance procedures defined for the OTcl \code{Agent} class are
currently found in \nsf{tcl/lib/ns-agent.tcl}.
They are as follows:
\begin{tabularx}{\linewidth}{rX}
\code{port} & the agent's port identifier \\
\code{dst-port} & the destination's port identifier \\
\code{attach-source \tup{stype}} & create and attach a Source object to an agent \\
\end{tabularx}

\section{\shdr{Examples: Tcp, TCP Sink Agents}{tcp-simple.cc}{sec:agentexample}}

The class \code{TCP} represents a simplified TCP sender.
It sends data to a \code{TCPSink} agent and processes its acknowledgments.
It has a separate object associated with it which represents
an application's demand.
By looking at the \code{TCPAgent} and \code{TCPSinkAgent} classes
we may see how relatively complex agents are constructed.
An example from the Tahoe TCP agent \code{TCPAgent} is also given
to illustrate the use of timers.

\subsection{\shdr{creating the agent}{tcp-simple.cc}{sec:createtcpsimple}}

The following OTcl code fragment creates a \code{TCP} agent
and sets it up:
\begin{small}
\begin{verbatim}
        set tcp [new Agent/TCP] ; # create sender agent
        $tcp set fid_ 2 ; # set IP-layer flow ID
        set sink [new Agent/TCPSink] ; # create receiver agent
        $ns attach-agent $n0 $tcp ; # put sender on node $n0
        $ns attach-agent $n3 $sink ; # put receiver on node $n3
        $ns connect $tcp $sink ; # establish TCP connection
        set ftp [new Source/FTP] ; # create an FTP source "application"
        $ftp set agent_ $tcp ; # associate FTP with the TCP sender
        $ns at 1.2 "$ftp start" ; # arrange for FTP to start at time 1.2 secs
\end{verbatim}
\end{small}
The OTcl instruction \code{new Agent/TCP} results in the
creation of a C++ \code{TcpAgent} class object.
It's constructor performs first invokes the constructor of the
\code{Agent} base class and then performs its own bindings.
These two constructors appear as follows:
\begin{small}
\begin{verbatim}

The TcpSimpleAgent constructor (tcp.cc):

        TcpAgent::TcpAgent() : Agent(PT_TCP), rtt_active_(0), rtt_seq_(-1),
			rtx_timer_(this), delsnd_timer_(this)
        {
                bind("window_", &wnd_);
                bind("windowInit_", &wnd_init_);
                bind("windowOption_", &wnd_option_);
                bind("windowConstant_", &wnd_const_);
                ...
                bind("off_ip_", &off_ip_);
                bind("off_tcp_", &off_tcp_);
                ...

The Agent constructor (agent.cc):

        Agent::Agent(int pkttype) : 
                addr_(-1), dst_(-1), size_(0), type_(pkttype), fid_(-1),
                prio_(-1), flags_(0)
        {
                memset(pending_, 0, sizeof(pending_)); // timers
                // this is really an IP agent, so set up
                // for generating the appropriate IP fields...
                bind("addr_", (int*)&addr_);
                bind("dst_", (int*)&dst_);
                bind("fid_", (int*)&fid_);
                bind("prio_", (int*)&prio_);
                bind("flags_", (int*)&flags_);
                ...
\end{verbatim}
\end{small}
These code fragments illustrate the common case where an agent's
constructor passes a packet type identifier to the \code{Agent}
constructor.
The values for the various packet types are used by the packet tracing
facility (see \ref{sec:trace}) and are defined in \code{trace.h}.
The variables which are bound in the \code{TcpAgent} constructor
are ordinary instance/member variables for the class
with the exception of the special integer values \code{off_tcp_}
and \code{off_ip_}.
These are needed in order to access a TCP header and IP header,
respectively.
For details on how to access packet headers, see \ref{sec:ppackethdr}.

Note that the \code{TcpAgent} constructor contains initializations for
two timers, \code{rtx_timer_} and \code{delsnd_timer_}.  \code{TimerHandler} 
objects are initialized by providing a pointer (the \code{this} pointer) to
the relevant agent.

\subsection{\shdr{starting the agent}{tcp.cc}{sec:starttcp}}

The \code{TcpAgent} agent is started in the example when its
FTP source receives the \code{start} directive at time 1.2.
The \code{start} operation is an instance procedure defined on the
\code{Source/FTP} class (and described in \ref{sec:sources}).
It is defined in \code{tcl/lib/ns-source.tcl} as follows:
\begin{small}
\begin{verbatim}
        Source/FTP instproc start {} {
                $self instvar agent_ maxpkts_
                $agent_ advance $maxpkts_
        }
\end{verbatim}
\end{small}
In this case, \code{agent_} refers to our simple TCP agent and
\code{maxpkts_} defaults to a large value (2147483647).

The call to \code{advance} eventually results in the simple TCP sender
generating some packets.  The following function \code{output}
performs this:
\begin{small}
\begin{verbatim}
        void TcpAgent::output(int seqno, int reason)
        {
                Packet* p = allocpkt();
                hdr_tcp *tcph = (hdr_tcp*)p->access(off_tcp_);
                double now = Scheduler::instance().clock();
                tcph->seqno() = seqno;
                tcph->ts() = now;
                tcph->reason() = reason;
                Connector::send(p, 0);
                ...
                if (!(rtx_timer_.status() == TIMER_PENDING))
                        /* No timer pending.  Schedule one. */
                        set_rtx_timer();
        }
\end{verbatim}
\end{small}
Here we see an illustration of the use of the \code{Agent::allocpkt()}
function.
This output routine first allocates a new packet
(with its common and IP headers already filled in), but then must fill
in the appropriate TCP-layer header fields.
To find the TCP header in a packet (assuming it has been enabled, see
\ref{sec:packethdrmgr}) the \code{off_tcp_} must be properly initialized,
as illustrated in the constructor.
The packet \code{access()} function returns a pointer to the TCP
header, its sequence number and time stamp fields are filled in,
and the \code{send} function of the \code{Connector} class is called
to send the packet downstream one hop.
Note that the C++ \code{::} scoping operator is used here to avoid
calling \code{TcpSimpleAgent::send()} (which is also defined).
The check for a pending timer uses the timer method \code{status()} which
is defined in the \code{TimerHandler} base class.
It is used here to set a retransmission timer if one is not already set
(a TCP sender only sets one timer per window of packets on each connection).

\subsection{\shdr{processing input at receiver}{tcp-sink.cc}{sec:tcpsink}}

Many of the TCP agents can be used with the \code{TCPSink} class
as the peer.
This class defines the \code{recv} and \code{ack} functions as follows:
\begin{small}
\begin{verbatim}
        void TcpSink::recv(Packet* pkt, Handler*)
        {
                hdr_tcp *th = (hdr_tcp*)pkt->access(off_tcp_);
                acker_->update(th->seqno());
                ack(pkt);
                Packet::free(pkt);
        }
        void TcpSink::ack(Packet* opkt)
        {
                Packet* npkt = allocpkt();
        
                hdr_tcp *otcp = (hdr_tcp*)opkt->access(off_tcp_);
                hdr_tcp *ntcp = (hdr_tcp*)npkt->access(off_tcp_);
                ntcp->seqno() = acker_->Seqno();
                ntcp->ts() = otcp->ts();
        
                hdr_ip* oip = (hdr_ip*)opkt->access(off_ip_);
                hdr_ip* nip = (hdr_ip*)npkt->access(off_ip_);
                nip->flowid() = oip->flowid();
        
                hdr_flags* of = (hdr_flags*)opkt->access(off_flags_);
                hdr_flags* nf = (hdr_flags*)npkt->access(off_flags_);
                nf->ecn_ = of->ecn_;
        
                acker_->append_ack((hdr_cmn*)npkt->access(off_cmn_),
                                   ntcp, otcp->seqno());
                send(npkt, 0);
        }
\end{verbatim}
\end{small}
The \code{recv} function overrides the \code{Agent::recv} function
(which merely discards the received packet).
It updates some internal state with the sequence number of the
received packet (and therefore requires the \code{off_tcp_} variable
to be properly initialized.
It then generates an acknowledgment for the received packet.
The \code{ack} function makes liberal use of access to packet header
fields including separate accesses to the TCP header, IP header,
Flags header, and common header.
The call to \code{send} invokes the \code{Connector::send} function.

\subsection{\shdr{processing responses at the sender}{tcp-simple.cc}{sec:tcpsimpleack}}

Once the simple TCP's peer receives data and generates an ACK, the
sender must (usually) process the ACK.
In the \code{TcpAgent} agent, this is done as follows:
\begin{small}
\begin{verbatim}
        /*
         * main reception path - should only see acks, otherwise the
         * network connections are misconfigured
         */
        void TcpAgent::recv(Packet *pkt, Handler*)
        {
                hdr_tcp *tcph = (hdr_tcp*)pkt->access(off_tcp_);
                hdr_ip* iph = (hdr_ip*)pkt->access(off_ip_);
                ...
                if (((hdr_flags*)pkt->access(off_flags_))->ecn_)
                        quench(1);
                if (tcph->seqno() > last_ack_) {
                        newack(pkt);
                        opencwnd();
                } else if (tcph->seqno() == last_ack_) {
                        if (++dupacks_ == NUMDUPACKS) {
                                ...
                        }
                }
                Packet::free(pkt);
                send(0, 0, maxburst_);
       }
\end{verbatim}
\end{small}
This routine is invoked when an ACK arrives at the sender.
In this case, once the information in the ACK is processed (by \code{newack})
the packet is no longer needed and is returned to the packet memory
allocator.
In addition, the receipt of the ACK indicates the possibility of sending
additional data, so the \code{TcpSimpleAgent::send} routine is
invoked which attempts to send more data if the TCP window allows.

\subsection{\shdr{implementing timers}{tcp.cc}{sec:tcptimer}}

As described in the following section (Section \ref{sec:timers}), specific
timer classes must be derived from an abstract base class 
\code{TimerHandler} defined in \code{timer-handler.h}.  Instances of these
subclasses can then be used as various agent timers.
An agent may wish to override the \code{Agent::timeout} function (which does 
nothing).  In the case of the Tahoe TCP agent, two timers are used: a delayed
send timer \code{delsnd_timer_} and a retransmission timer \code{rtx_timer_}.
Section \ref{sec:timerexample} describes the retransmission timer in TCP
as an example of timer usage.  

\section{\shdr{Creating a New Agent}{agent.cc}{sec:createagent}}

To create a new agent, one has to do the following:
\begin{enumerate}
        \item decide its inheritance structure (section \ref{sec:pingexample})
        \item create the class, \code{recv}, and \code{timeout} functions (section \ref{})
		\item define any necessary timer classes,
        \item define OTcl linkage functions (section \ref{})
        \item write the necessary OTcl code to access your agent (section \ref{})
\end{enumerate}

The action required to create and agent can be illustrated
by means of a very simple example.
Suppose we wish to construct an agent which performs
the ICMP ECHO REQUEST/REPLY (or ``ping'') operations.

\subsection{\shdr{Example: A ``ping'' requestor (Inheritance Structure)}{agent.h}{sec:pingexample}}

Deciding on the inheritance structure is a matter of personal choice, but is
likely to be related to the layer at which the agent will operate
and its assumptions on lower layer functionality.
For protocol endpoints wishing to use a connectionless
datagram-oriented transport layer (like UDP) with sequencing,
the \code{CBR_Agent} base class is likely to be appropriate.
For protocols wishing to use a connection-oriented stream transport
(like TCP), the various TCP Agents could be used.
Finally, if a new transport or ``sub-transport'' protocol
is to be developed, using \code{Agent}
as the base class would likely be the best choice.
In our example, we'll use Agent as the base class, given that
we are constructing an agent logically belonging to the IP layer
(or just above it).

We may use the following class definitions:
\begin{small}
\begin{verbatim}
	
        class ECHO_Timer;
 
        class ECHO_Agent : public Agent {
         public:
                ECHO_Agent();
                int command(int argc, const char*const* argv);
         protected:
                void timeout(int);
                void sendit();
                double interval_;
                ECHO_Timer echo_timer_;
        };

        class ECHO_Timer : public TimerHandler {
        public:
                ECHO_Timer(ECHO_Agent *a) : TimerHandler() {a_ = a; }
        protected:
                virtual void expire(Event *e);
                ECHO_Agent *a_;
        }; 
\end{verbatim}
\end{small}

The implementation of the member functions will look similar to
the \code{CBR_Agent} described above.

\subsection{The \code{recv} and \code{timeout} functions}

The \code{recv} function is not defined here, as this agent
represents a request function and will generally not be receiving
events or packets\footnote{This is perhaps unrealistically simple.
An ICMP ECHO REQUEST agent would likely wish to process
ECHO REPLY messages.}
By not defining the \code{recv} function, the base class version
of \code{recv} (i.e. \code{Connector::recv}) is used.
The \code{timeout} function is used to periodically send request packets.
The following \code{timeout} function is used, along with a helper
function \code{sendit}:

\begin{small}
\begin{verbatim}
        void ECHO_Agent::timeout(int)
        {
                sendit();
                echo_timer_.resched(interval_);
        }
        void ECHO_Agent::sendit()
        {
                Packet* p = allocpkt();
                ECHOHeader *eh = ECHOHeader::access(p->bits());
                eh->timestamp() = Scheduler::instance().clock();
                send(p, 0);     // Connector::send()
        }

        void ECHO_Timer::expire(Event *e) {
                a_->timeout(0);
        }
\end{verbatim}
\end{small}

The \code{timeout} function simply arranges for \code{sendit} to be
executed every \code{interval_} seconds.
The \code{sendit} function create a new packet with most of its
header fields already set up by the \code{allocpkt} function.
The packet is only lacking the current time stamp. 
The call to \code{access} provides for a structured interface to the
packet header fields, and is used to set the timestamp field.
Note that this agent uses its own special header (``ECHOHeader'').
The creation and use of packet headers is described in (see \ref{pheader});
To send the packet to the next downstream node, \code{Connector::send()}
is invoked without a handler.

\subsection{Linking the ``ping'' agent with OTcl}

There are three items we must handle to properly link our agent
with Otcl.
First we need to establish a mapping between the OTcl name
for our class and the actual object created when an
instantiation of the class is requested in OTcl.
This is done as follows:
\begin{small}
\begin{verbatim}
        static class ECHOClass : public TclClass {
        public:
                ECHOClass() : TclClass("Agent/ECHO") {}
                TclObject* create(int argc, const char*const* argv) {
                        return (new ECHO_Agent());
                }
        } class_echo;
\end{verbatim}
\end{small}

Here, a {\em static} object ``class\_echo'' is created. It's constructor
(executed immediately when the simulator is executed) places the class name
``Agent/ECHO'' into the OTcl name space.  The mixing of case is
by convention, although the ``/'' character is interpreted by the
\code{Tcl} library as a hierarchy delimiter and is required.
The definition of the \code{create} function specifies how a C++
shadow object should be created when
the OTcl interpreter is instructed to create an
object of class ``Agent/ECHO''.  In this case, a dynamically-allocated
object is returned.  This is the normal way new C++ shadow objects
are created.
Note that arguments could have been passed to our constructor
via OTcl through the conventional \code{argc/argv} pairs of the
\code{create} function, although this is rare.

Once we have the object creation set up, we will want to link
C++ member variables with corresponding variables in the OTcl
name space, so that accesses to OTcl variables are actually
backed by member variables in C++.
Assume we would like OTcl to be able to adjust the sending
interval and the packet size.
This is accomplished in the class's constructor:
\begin{small}
\begin{verbatim}

        ECHO_Agent::ECHO_Agent() : Agent(PT_ECHO)
        {
                bind_time("interval_", &interval_);
                bind("packetSize_", &size_);
        }
\end{verbatim}
\end{small}

Here, the C++ variables \code{interval_} and \code{size_} are
linked to the OTcl instance variables \code{interval_} and
\code{packetSize_}, respectively.
Any read or modify operation to the Otcl variables will result
in a corresponding access to the underlying C++ variables.
See \ref{c++linkage} for more detail on how the various \code{bind}
functions work.
The defined constant \code{PT_ECHO} is passed to the \code{Agent}
constuctor so that the \code{Agent::allocpkt} function may set
the packet type field used by the trace support
(see section \ref{traceptype}).  In this case, \code{PT_ECHO} represents
a new packet type and must be defined in \code{trace.h}.
(see section \ref{sec:traceformat}).

Once object creation and variable binding is set up, we may
want to create methods implemented in C++ but which can
be invoked from OTcl.
These are often control functions that initiate, terminate or
modify behavior.
In our present example, we may wish to be able to start the
ping query agent from OTcl using a ``start'' directive.
This may be implemented as follows (very similar to the CBR Agent):
\begin{small}
\begin{verbatim}

        int ECHO_Agent::command(int argc, const char*const* argv)
        {
                if (argc == 2) {
                        if (strcmp(argv[1], "start") == 0) {
                                timeout(0);
                                return (TCL_OK);
                        }
                }
                return (Agent::command(argc, argv));
        }
\end{verbatim}
\end{small}
Here, the \code{start} method available to OTcl simply calls
the C++ member function \code{timeout} which initiates the
first packet generation and schedules the next.
Note this class is so simple it does not even include a
way to be stopped.

\section{Using the agent through OTcl}

The agent we have created will have to be instantiated and attached
to a node.
Note that a node and simulator object is assumed to have
already been
created (Section \ref{tcllink} describes how this is done).
The following OTcl code performs these functions:
\begin{small}
\begin{verbatim}
        set echoagent [new Agent/ECHO]
        $simulator attach-agent $node $echoagent
\end{verbatim}
\end{small}

To set the interval and packet size, and start packet generation,
the following OTcl code is executed:
\begin{small}
\begin{verbatim}

        $echoagent set dst_ $dest
        $echoagent set fid_ 0
        $echoagent set prio_ 0
        $echoagent set flags_ 0
        $echoagent set interval_ 1.5
        $echoagent set packetSize_ 1024
        $echoagent start
\end{verbatim}
\end{small}

This will cause our agent to generate one 1024-byte packet destined for
node \code{$dest} every 1.5 seconds.

\endinput


\part{Support}
%
% personal commentary:
%        DRAFT DRAFT DRAFT
%        - KFALL
%
\section{\shdr{Mathematical Support}{random.h}{sec:math}}

The simulator includes a small collection of mathematical
functions used to implement random variate generation and integration.
It is anticipated the support for random variates will
change in the near future.

\subsection{\shdr{Random Numbers}{random.h}{sec:random}}

There is limited support for generating
random\footnote{Any mention of {\em random} in this context is
understood to mean {\em pseudo-random}.}
variates according to a probability distribution.
The \code{Random} class supports uniformly and exponentially
distributed random variates.
It is defined in \code{random.h}:
\begin{small}
\begin{verbatim}
        class Random {
        public:
                static void seed(int);
                static int seed_heuristically();
                static inline int random() {
        #if defined(__svr4__) || defined(__SVR4)
                        return (::lrand48() & 0x7fffffff);
        #else
                        return (::random());
        #endif
                }
                static inline double uniform() {
                        /* random returns numbers in the range [0,2^31-1] */
        #if defined(__svr4__) || defined(__SVR4)
                        return (drand48());
        #else
                        return ((double)random() / 0x7fffffff);
        #endif
                }
                static inline double uniform(double r) {
                        return (r * uniform());
                }
                static inline double uniform(double a, double b) {
                        return (a + uniform(b - a));
                }
                static inline double exponential() {
                        return (-log(uniform()));
                }
                static inline int integer(int k) {
                        return (random() % (unsigned)k);
                }
        };
\end{verbatim}
\end{small}
The \code{random} method generates random integers in the
range $[0,2^{31}-1]$.
It currently uses the built-in UNIX random number generating
functions to achieve this.
In particular,
the \code{random} or \code{lrand48} functions are used, which generate
integers uniformly
on the interval $[0,2^{31}-1]$ directly.
The generation of random floating-point numbers depends on
the \code{random} method just defined or the built-in \code{drand48}
function.
The SysV library function \code{drand48} returns a random floating
point number, uniformly distributed on the interval $[0.0, 1.0)$
Additional member functions provide the following random variate
generation:
\begin{itemize}
        \item {\tt uniform(double r)} - generate floating-point number uniformly distributed on $[0,r]$
        \item {\tt uniform(double a, double b)} - generate floating-point number uniformly distributed on $[a,b]$
        \item {\tt exponential()} - generate floating-point number exponentially distributed (with parameter 1) on $[0, \infty)$
        \item {\tt integer(int k)} - generate integer uniformly distributed on $[0, (k-1)]$
\end{itemize}
The \code{Random} class can be used to construct randomized algorithms,
as in this code fragment from RED:
\begin{small}
\begin{verbatim}
        ...
        // drop probability is computed, pick random number and act
        double u = Random::uniform();
        if (u <= edv_.v_prob) {
                edv_.count = 0;
                if (edp_.setbit) 
                        iph->flags() |= IP_ECN; // ip ecn bit
                else
                        return (1);
        }
        ...
\end{verbatim}
\end{small}

The \code{seed} member allows for specification of an initialization
seed to the underlying random number generator (an integer
for the \code{srandom} function and a long integer for the
\code{srand48} function).
This can be useful to ensure the generation of the same random numbers
from run to run.
It is assumed that specification of the same seed will result in the
same sequence of random numbers.
To use the \code{Random} class without specifying a seed,
the \code{seed\_heuristically} member may be used, which
attempts to pick some sort of non-recurring seed values (resulting
in different random numbers from run to run).  It is defined
in \code{random.cc}:
\begin{small}
\begin{verbatim}
        int Random::seed_heuristically()
        {
                timeval tv;
                gettimeofday(&tv, 0);
                int s = (tv.tv_sec ^ tv.tv_usec) & 0x7fffffff;
                seed(s);
                return (s);
        }
\end{verbatim}
\end{small}
This function just obtains the number of elapsed seconds and microseconds
since the UNIX {\em epoch} (00:00 GMT, January 1, 1970) and
takes the low-order 31 bits of the {\sf XOR} of these values as a seed.
Clearly, this value is likely to be different from run to run.

\subsection{\shdr{Integrals}{integrator.h}{sec:integral}}

To support the approximation of (continuous) integration by (discrete)
sums, the \code{Integrator} class is defined in \code{integrator.h}:
\begin{small}
\begin{verbatim}
From integrator.h:
        class Integrator : public TclObject {
        public:
                Integrator();
                void set(double x, double y);
                void newPoint(double x, double y);
                int command(int argc, const char*const* argv);
        protected:
                double lastx_;
                double lasty_;
                double sum_;
        };
From integrator.cc:
        Integrator::Integrator() : lastx_(0.), lasty_(0.), sum_(0.)
        {
                bind("lastx_", &lastx_);
                bind("lasty_", &lasty_);
                bind("sum_", &sum_);
        }

        void Integrator::set(double x, double y)
        {
                lastx_ = x;
                lasty_ = y;
                sum_ = 0.;
        }

        void Integrator::newPoint(double x, double y)
        {
                sum_ += (x - lastx_) * lasty_;
                lastx_ = x;
                lasty_ = y;
        }

        int Integrator::command(int argc, const char*const* argv)
        {
                if (argc == 4) {
                        if (strcmp(argv[1], "newpoint") == 0) {
                                double x = atof(argv[2]);
                                double y = atof(argv[3]);
                                newPoint(x, y);
                                return (TCL_OK);
                        }
                }
                return (TclObject::command(argc, argv));
        }
\end{verbatim}
\end{small}
This class provides a base class used by other classes such
as \code{QueueMonitor} that keep running sums.
Each new element of the running sum is added by
the \code{newPoint(x,y)} function.
After the $k$th execution of \code{newPoint}, the running sum
is equal to $\sum_{i=1}^{k}y_{i-1}(x_i - x_{i-1})$ where
$x_0 = y_0 = 0$ unless \code{lastx\_}, \code{lasty\_}, or \code{sum\_}
are reset via OTcl.
Note that a new point in the sum can be added either by the
C++ member \code{newPoint} or the OTcl member \code{newpoint}.
The use of integrals to compute certain types of averages
(e.g. mean queue lengths) is given in Jain(1991, pp. 429-430).

%
% personal commentary:
%        DRAFT DRAFT DRAFT
%        - KFALL
%
\section{\shdr{Trace and Monitoring Support}{trace.h}{sec:trace}}

There are a number of ways of collecting output or
trace data on a simulation.
Generally all data is stored in a file to be analyzed
after completion of a simulation.
Trace objects are configured into a simulation as nodes in the
network topology, usually with a Tcl ``Channel'' object
hooked to them, representing the destination of collected data
(typically a trace file in the current directory).
There are really two parts to the tracing support: individual
packet traces and derived statistics.

To support packet traces, there is a special trace header
included in each packet (this format is defined in \code{trace.h}).
It presently includes a unique identifier on each packet and a
packet type field (set by agents when they generate packets).
This packet type field is not currently otherwise used by the simulator.
This trace facility has been inherited from an earlier version of the
simulator and produces output traces compatible with \code{ns} version 1.

To support the collection of aggregated statistics, a separate
set of objects is created and inserted into the network topology
around queues.
They provide a place where
arrival statistics and times are gathered and make
use of the \code{Integrator} class (see \ref{sec:integclass}) to
compute statistics over time intervals.

\subsection{\shdr{Otcl Trace and Monitor Support}{ns-lib.tcl}{sec:otcltrace}}

The trace support in OTcl consists of a number of specialized
classes visible in OTcl but implemented in C++, combined
with a set of Tcl helper procedures and classes defined in the ns library.

\subsubsection{\shdr{OTcl trace classes}{ns-lib.tcl}{sec:traceclass}}
The following OTcl classes are supported by underlying C++
classes defined in \code{trace.cc}:
\begin{itemize}
\begin{quote}
	\item [Trace/Hop] - trace a ``hop'' (XXX what does this mean exactly; it is not really used XXX)
	\item [Trace/Enque] - packet arrival at a queue
	\item [Trace/Deque] - packet departure at a queue
	\item [Trace/Drop] - packet drop at a queue (delivered to drop-target)
	\item [SnoopQueue/In] - on input, collect a time/size sample (pass packet on)
	\item [SnoopQueue/Out] - on output, collect a time/size sample (pass packet on)
	\item [QueueMonitor] - receive and aggregate collected samples from snoopers
\end{quote}
\end{itemize}

\subsubsection{\shdr{otcl helper functions}{ns-lib.tcl}{sec:helptrace}}
The following helper functions may be used within simulation
scripts to help in attaching trace elements (see \code{ns-lib.tcl}):
\begin{small}
\begin{verbatim}
NOT USED?  Steve uses for debugging..
        #
        # A simple method to wrap any object around
        # a trace object that dumps to stdout
        #
        Simulator instproc dumper obj {
                set t [$self alloc-trace hop stdout]
                $t target $obj
                return $t
        }
NOT USED?
        Simulator instproc flush-trace {} {
                $self instvar alltrace_
                if [info exists alltrace_] {
                        foreach trace $alltrace_ {
                                $trace flush
                        }
                }
        }
NOT USED?
        Simulator instproc trace-all file {
                $self instvar traceAllFile_
                set traceAllFile_ $file
        }

        Simulator instproc create-trace { type file src dst } {
                $self instvar alltrace_
                set p [new Trace/$type]
                $p set src_ [$src id]
                $p set dst_ [$dst id]
                lappend alltrace_ $p
                $p attach $file
                return $p
        }

        Simulator instproc trace-queue { n1 n2 file } {
                $self instvar link_
                $link_([$n1 id]:[$n2 id]) trace $self $file
        }
        #
        # arrange for queue length of link between nodes n1 and n2
        # to be tracked and return object that can be queried
        # to learn average q size etc.  XXX this API still rough
        #
        Simulator instproc monitor-queue { n1 n2 } {
                $self instvar link_
                return [$link_([$n1 id]:[$n2 id]) init-monitor $self]
        }

        Simulator instproc drop-trace { n1 n2 trace } {
                $self instvar link_
                [$link_([$n1 id]:[$n2 id]) queue] drop-target $trace
        }
\end{verbatim}
\end{small}
XXX this is confusing XXXX
The \code{create-trace} procedure is used to create a new \code{Trace}
object of the appropriate kind and attach an I/O channel to it
(typically a file handle).
The \code{src\_} and \code{dst\_} fields are bound to underlying C++
variables so that trace output can include the node addresses between
which the tracing is taking place.
Note that they are not used for {\em matching}.  Specifically, these
values in no way relate to the packet header \code{src} and \code{dst}
fields, which are also displayed when tracing.
See the description of the \code{Trace}
class below (\ref{sec:traceclass}).

The \code{trace-queue} function enables
\code{Enque}, \code{Deque}, and \code{Drop} tracing on the link
between nodes \code{n1} and \code{n2}.
The reference \verb!$link_([$n1 id]:[$n2 id])! inspects the master
table of links stored in the associative array \code{link\_} and provides
a reference to the link connecting the nodes with the IDs associated
with \code{n1} and \code{n2}.
The Link \code{trace} procedure is described below (\ref{sec:libexam}).

The \code{monitor-queue} function is constructed similarly to
\code{trace-queue}.
By calling the link's \code{init-monitor} procedure, it arranges
for the creation of objects (\code{SnoopQueue} and \code{QueueMonitor}
objects) which can, in turn, be used to ascertain time-aggregated
queue statistics.

The \code{drop-trace} function provides a way to specify a
\code{Queue}'s drop target without having a direct handle of
the queue.

\subsubsection{\shdr{Library support and examples}{ns-lib.tcl}{sec:libexam}}

The \code{Simulator} procedures described above require the \code{trace}
and \code{init-monitor} methods associated with the OTcl \code{Link} class.
At the moment, only one subclass of link is defined, namely
\code{SimpleLink}.  Thus, the \code{trace} and \code{init-monitor}
methods are actually part of the \code{SimpleLink} class rather than
the \code{Link} base class.
The \code{trace} function is defined as follows (in \code{ns-link.tcl}):
\begin{small}
\begin{verbatim}
        #
        # Build trace objects for this link and
        # update the object linkage
        #
        SimpleLink instproc trace { ns f } {
                $self instvar enqT_ deqT_ drpT_ queue_ link_ head_ fromNode_ toNode_
                set enqT_ [$ns create-trace Enque $f $fromNode_ $toNode_]
                set deqT_ [$ns create-trace Deque $f $fromNode_ $toNode_]
                set drpT_ [$ns create-trace Drop $f $fromNode_ $toNode_]

                $drpT_ target [$queue_ drop-target]
                $queue_ drop-target $drpT_

                $deqT_ target [$queue_ target]
                $queue_ target $deqT_

                $enqT_ target $head_
                set head_ $enqT_
        }
\end{verbatim}
\end{small}

This function establishes \code{Enque}, \code{Deque}, and \code{Drop}
traces in the simulator \code{\$ns} and directs their
output to I/O handle \code{\$f}.
The function assumes a queue has been associated with the link.
It operates by first creating three new trace objects
and inserting the \code{Enque} object before the queue, the
\code{Deque} object after the queue, and the \code{Drop} object
between the queue and its previous drop target.
Note that all trace output is directed to the same I/O handle.

The \code{init-monitor} function is used to create a set of
objects used to monitor queue sizes of a queue associated
with a link.
It is defined as follows:
\begin{small}
\begin{verbatim}
        #
        # Insert objects that allow us to monitor the queue size
        # of this link.  Return the name of the object that
        # can be queried to determine the average queue size.
        #
        SimpleLink instproc init-monitor ns {
                $self instvar drpT_ queue_ head_ \
                        snoopIn_ snoopOut_ snoopDrop_ qMonitor_

                set snoopIn_ [new SnoopQueue/In]
                set snoopOut_ [new SnoopQueue/Out]
                set snoopDrop_ [new SnoopQueue/Out]

                $snoopIn_ target $head_
                set head_ $snoopIn_

                $snoopOut_ target [$queue_ target]
                $queue_ target $snoopOut_

                if [info exists drpT_] {
                        $snoopDrop_ target [$drpT_ target]
                        $drpT_ target $snoopDrop_
                } else {
                        $snoopDrop_ target [$ns set nullAgent_]
                }
                $queue_ drop-target $snoopDrop_

                set qMonitor_ [new QueueMonitor]
                $snoopIn_ set-monitor $qMonitor_
                $snoopOut_ set-monitor $qMonitor_
                $snoopDrop_ set-monitor $qMonitor_

                return $qMonitor_
        }
\end{verbatim}
\end{small}

This function establishes queue monitoring on the \code{SimpleLink} object
in the simulator \code{ns}.
Queue monitoring is implemented by constructing three \code{SnoopQueue}
objects and one \code{QueueMonitor} object.
The \code{SnoopQueue} objects are linked in around a \code{Queue} in a way
similar to \code{Trace} objects.
The \code{SnoopQueue/In(Out)} object monitors packet arrivals(departures)
and reports them to an associated \code{QueueMonitor} agent.
In addition, a \code{SnoopQueue/Out} object is also used to accumulate
packet drop statistics to an associated \code{QueueMonitor} object.
For \code{init-monitor} the same \code{QueueMonitor} object is used
in all cases.
The C++ definitions of the \code{SnoopQueue} and \code{QueueMonitor}
classes are described below.

\subsection{\shdr{C++ Trace and Monitor Support}{trace.cc}{sec:tracemoncplus}}

Underlying C++ objects are created in support of the interface specified
in Section\ref{sec:traceclass} and are linked into the network topology
as network elements.

\subsubsection{\shdr{the C++ Trace class}{trace.cc}{sec:tracecplus}}

The single C++ \code{Trace} class is used to implement the OTcl
classes \code{Trace/Hop}, \code{Trace/Enque}, \code{Trace/Deque},
and \code{Trace/Drop}.
The \code{type\_} field is used to differentiate among the
various types of
traces any particular \code{Trace} object might implement.
Currently, this field may contain one of the following symbolic characters:
{\bf +} for enque, {\bf -} for deque, {\bf h} for hop, and
{\bf d} for drop.
The overall class is defined as follows in \code{trace.cc}:
\begin{small}
\begin{verbatim}
        class Trace : public Connector {
         protected:
                int type_;
                nsaddr_t src_;
                nsaddr_t dst_;
                Tcl_Channel channel_;
                int callback_;
                char wrk_[256];
                void format(int tt, int s, int d, Packet* p);
         public:
                Trace(int type);
                ~Trace();
                int command(int argc, const char*const* argv);
                void recv(Packet* p, Handler*);
                void dump();
                inline char* buffer() { return (wrk_); }
        };
\end{verbatim}
\end{small}

The \code{src\_}, and \code{dst\_} internal state is used
to label trace output and is independent of the corresponding field
names in packet headers.
The main \code{recv} method is defined as follows:
\begin{small}
\begin{verbatim}
        void Trace::recv(Packet* p, Handler* h)
        {
                format(type_, src_, dst_, p);
                dump();
                /* hack: if trace object not attached to anything free packet */
                if (target_ == 0)
                        Packet::free(p);
                else
                        send(p, h); /* Connector::send() */
        }
\end{verbatim}
\end{small}
The function merely formats a trace entry using the source, destination,
and particular trace type character.
The \code{dump} function writes the formatted entry out to the
I/O handle associated with \code{channel\_}.
The \code{format} function, in effect, dictates the trace file format.

\subsubsection{\shdr{trace file format}{trace.cc}{sec:traceformat}}

The \code{Trace::format} function defines the trace file format used
in trace files produced by the \code{Trace} class.
It is constructed to maintain backward compatibility with output files
in earlier versions of the simulator (i.e. {\em ns-1}) so that ns-1
post-processing scripts continue to operate.
The important pieces of its implementation are as follows:
\begin{small}
\begin{verbatim}
        char* pt_names[] = {
                PT_NAMES
        };

        // this function should retain some backward-compatibility, so that
        // scripts don't break.
        void Trace::format(int tt, int s, int d, Packet* p)
        {
                TraceHeader *th = TraceHeader::access(p->bits());
                IPHeader *iph = IPHeader::access(p->bits());
                TCPHeader *tcph = TCPHeader::access(p->bits());
                const char* name = pt_names[th->ptype()];
                int seqno;
                /* XXX */
                if (strncmp(name, "tcp", 3) == 0)
                        seqno = tcph->seqno();
                else
                        seqno = -1;


                char flags[5];
                flags[0] = flags[1] = flags[2] = flags[3] = '-';
                flags[4] = 0;
                flags[0] = (iph->flags() & IP_ECN) ? 'C' : '-';
                sprintf(wrk_, "%c %g %d %d %s %d %s %d %d %d %d %d",
                        tt,
                        Scheduler::instance().clock(),
                        s,
                        d,
                        name,
                        iph->size(),
                        flags,
                        iph->flowid() /* was p->class_ */,
                        iph->src(),
                        iph->dst(),
                        seqno,
                        th->uid() /* was p->uid_ */);
        }
\end{verbatim}
\end{small}
This function is somewhat unelegant, primarily due to the desire
to maintain backward compatibility.
It formats the source, destination, and type fields defined in the
trace object ({\em not in the packet headers}), the current time,
along with various packet header fields including,
type of packet (as a name), size, flags (symbolically),
flow identifier, source and destination packet header fields,
sequence number (if present), and unique identifier.
Thus, a fragment of a trace file might appear as follows:
\begin{small}
\begin{verbatim}
        + 1.45176 2 3 tcp 1000 ---- 1 256 769 27 48
        + 1.45276 2 3 tcp 1000 ---- 1 256 769 28 49
        - 1.46176 2 3 tcp 1000 ---- 1 256 769 22 43
        + 1.46176 2 3 tcp 1000 ---- 1 256 769 29 50
        + 1.46276 2 3 tcp 1000 ---- 1 256 769 30 51
        d 1.46276 2 3 tcp 1000 ---- 1 256 769 30 51
        - 1.47176 2 3 tcp 1000 ---- 1 256 769 23 44
        + 1.47176 2 3 tcp 1000 ---- 0 0 768 3 52
        + 1.47276 2 3 tcp 1000 ---- 0 0 768 4 53
        d 1.47276 2 3 tcp 1000 ---- 0 0 768 4 53
\end{verbatim}
\end{small}
Here we see ten trace entries, 6 enque operations (indicated by ``+''
in the first column), 2 deque operations (indicated by ``-''),
and 2 packet drops (indicated by ``d'').
(this had better be a trace fragment, or 2 packets would have just vanished!).
The simulated time (in seconds) at which each event occurred is listed
in the second column.
The next two fields indicate between which two nodes tracing is happening.
The next field is a descriptive name for the the type of packet seen
(see \ref{sec:traceptype} below).
The next field is the packet's size, as encoded in its IP header.
The next four characters represent special flag bits which may be
enabled.  Presently only one such bit exists (explicit congestion
notification, or {\sf ECN}).  In this example, {\sf ECN} is not used.
The next field gives the IP {\em flow identifier} field as defined
for IP version 6.\footnote{In ns-1, each packet included a \code{class}
field, which was used by CBQ to classify packets.
It then found additional use to differentiate between
``flows'' at one trace point.  In ns-2, the flow ID field is available
for this purpose, but any additional information (which was commonly overloaded
into the class field in ns-1) should be placed in its own separate field,
possibly in some other header}.
The subsequent two fields indicate the packet's source and destination
node addresses, respectively.
The following field indicates the sequence number.\footnote{In ns-1,
all packets contained a sequence number, whereas in ns-2 only those
Agents interested in providing sequencing will generate sequence numbers.
Thus, this field may not be useful in ns-2 for packets generated by
agents that have not filled in a sequence number.  It is used here
to remain backward compatible with ns-1.}
The last field is a unique packet identifier.  Each new packet
created in the simulation is assigned a new, unique identifier.

\subsubsection{\shdr{packet types}{trace.h}{sec:traceptype}}

Each packet contains a packet type field used by \code{Trace::format}
to print out the type of packet encountered.
The type field is defined in the \code{TraceHeader} class, and is considered
to be part of the trace support; it is not interpreted
elsewhere in the simulator.
Initialization of the type field in packets is performed by the
\code{Agent::allocpkt()} function.
The type field is set to integer values associated with the
definition passed to the \code{Agent} constructor.
See Section~\ref{sec:agentmethodsotcl} for more details.
The currently-supported definitions, their values, and their
associated symblic names are as follows
(defined in \code{trace.h}):
\begin{small}
\begin{verbatim}
        #define PT_TCP          0
        #define PT_TELNET       1
        #define PT_CBR          2
        #define PT_AUDIO        3
        #define PT_VIDEO        4
        #define PT_ACK          5
        #define PT_START        6
        #define PT_STOP         7
        #define PT_PRUNE        8
        #define PT_GRAFT        9
        #define PT_MESSAGE      10
        #define PT_RTCP         11
        #define PT_RTP          12
        #define PT_NTYPE        13

        #define PT_NAMES "tcp", "telnet", "cbr", "audio", "video", "ack", \
                "start", "stop", "prune", "graft", "message", "rtcp", "rtp"
\end{verbatim}
\end{small}
The definition of \code{PT\_NAMES} is used to initialize the
\code{pt\_names} array as indicated above;
\code{PT\_NTYPE} is not presently used.

\subsubsection{\shdr{queue monitoring}{queue-monitor.cc}{sec:qmonitor}}

Objects of three different classes are used to support monitoring
of queue occupancy.
A \code{QueueMonitor} object keeps running statistics, and
\code{SnoopQueue} objects receive packets and provide
a copy to a corresponding \code{QueueMonitor} object before forwarding
them downstream.
A \code{QueueMonitor} is defined as follows (\code{queue-monitor.cc}):
\begin{small}
\begin{verbatim}
        class QueueMonitor : public Integrator {
         public:
                QueueMonitor() : size_(0) {
                        bind("size_", &size_);
                }
                void in(Packet*);
                void out(Packet*);
                //      int command(int argc, const char*const* argv);
        protected:
                int size_;
        };
        void QueueMonitor::in(Packet* p)
        {
                IPHeader *iph = IPHeader::access(p->bits());
                size_ += iph->size();
                double now = Scheduler::instance().clock();
                newPoint(now, double(size_));
        }

        void QueueMonitor::out(Packet* p)
        {
                IPHeader *iph = IPHeader::access(p->bits());
                size_ -= iph->size();
                double now = Scheduler::instance().clock();
                newPoint(now, double(size_));
        }
\end{verbatim}
\end{small}
The \code{QueueMonitor} class is derived from \code{Integrator}, which
is defined in Section\ref{sec:mathinteg}.
The \code{Integrator} class provides a simple implementation of
integral approximation by discrete sums.
In this case, a \code{QueueMonitor} computes the approximate integral of the
queue size (in bytes)
with respect to time over the interval $[t_0, now]$, where
$t_0$ is either the start of the simulation or the last time the
\code{sum\_} field of the underlying \code{Integrator} class was reset.
In addition, the instantaneous queue size is recorded and made available
to OTcl in the \code{size\_} member variable.

The \code{QueueMonitor} class is not derived from \code{Connector}, and
is not linked directly into the network topology.
Rather, objects of the \code{SnoopQueue} class (or its derived classes)
are used instead.
They are defined as follows (in \code{queue-monitor.cc}):
\begin{small}
\begin{verbatim}
        class SnoopQueue : public Connector {
         public:
                SnoopQueue() : qm_(0) {}
                int command(int argc, const char*const* argv) {
                        if (argc == 3) {
                                if (strcmp(argv[1], "set-monitor") == 0) {
                                        qm_ = (QueueMonitor*)
                                                TclObject::lookup(argv[2]);
                                        return (TCL_OK);
                                }
                        }
                        return (Connector::command(argc, argv));
                }
         protected:
                QueueMonitor* qm_;
        };

        class SnoopQueueIn : public SnoopQueue {
         public:
                void recv(Packet* p, Handler* h) {
                        qm_->in(p);
                        send(p, h); /* Connector::send() */
                }
        };

        class SnoopQueueOut : public SnoopQueue {
         public:
                void recv(Packet* p, Handler* h) {
                        qm_->out(p);
                        send(p, h); /* Connector::send() */
                }
        };
\end{verbatim}
\end{small}
Objects constructed out of these classes are linked in the simulation
topology as described above and call \code{QueueManager}
\code{out} or \code{in} procedures,
depending on the particular type of snoopy queue.


\part{Routing}
% SEPARATE STRATEGY FROM PROTOCOL FROM PROTOCOL INSTANCE

\documentclass{article}

\usepackage{times}
\usepackage[T1]{fontenc}

\PassOptionsToPackage{draft}{nsDoc}
\usepackage{nsDoc}

\begin{document}

\title{\ns\ internals documentation}
\author{%
  Various members of the VINT project \tup{vint@catarina.usc.edu}\\
  Kevin Fall \tup{kfall@ee.lbl.gov}, Editor,\\
  Kannan Varadhan \tup{kannan@catarina.usc.edu}, Editor.}
\date{\today}

\section{Unicast Routing}
\label{sec:unicast}

This section describes the structure of unicast routing in \ns.
We begin by describing
\href{the interface to the user}{Section}{sec:API},
through methods in the \clsref{Simulator}{../ns-2/ns-lib.tcl}
and the \clsref{RouteLogic}{../ns-2/ns-lib.tcl}.
We then describe the
\href{static routing}{Section}{sec:static}, which is the default in \ns.
In our next section, we discuss the interface between 
\href{unicast routing and network dynamics}{Section}{sec:rtglibAPI},
and that between
\href{unicast routing and multicast routing}{Secion}{sec:mcastAPI}.
We then continue with our descriptions of
\href{session routing}{Section}{sec:session}, and
\href{dynamic DV routing}{Section}{sec:dynamicDV}.
We conclude with a sketch of how the current mechanism would be
\href{adapted for dynamic LS routing}{Section}{sec:dynamicLS}.

\subsection{The Interface to the Simulation Operator (The API)}
\label{sec:API}

The user level simulation script requires one command:
to specify the unicast routing strategy or protocols for the simulation.
A routing strategy is a general mechanism by which \ns\
will compute routes for the simulation.
There are three routing strategies in \ns:
Static, Session, and Dynamic.
Conversely, a routing protocol is a realisation of a specific algorithm.
Currently, Static and Session routing use
the
\fcnref{Dijkstra's all-pairs SPF algorithm \cite{}}{../ns-2/route.cc}{%
	RouteLogic::compute\_routes};
There is one type of dynamic 
\fcnref{\proc[]{rtproto}}{../ns-2/route-proto.tcl}{Simulator::rtproto}.
This command is an instance procedure in the
\clsref{Simulator}{../ns-2/ns-lib.tcl}.
It takes multiple arguments:
the first argument is mandatory, and
specifies the routing protocol to be used in the simulation;
currently defined routing protocols in \ns\ are: Static, Session, and DV.
Subsequent arguments specify the nodes that will run 
an instance of this particular routing protocol.
The default is to run the same routing protocol
on all the nodes in the topology.
As an example, the following commands illustrate the use of the
\proc[]{rtproto} command.
\begin{program}
        $ns rtproto Static            \; Enable static route strategy for the simulation;
        $ns rtproto Session           \; Enable session routing for this simulation;
        $ns rtproto DV $n1 $n2 $n3    \; Run DV agents on nodes $n1, $n2, and $n3;
\end{program}
If a simulation script does not specify any \proc[]{rtproto} command,
then \ns\ will run Static routing on all the nodes in the topology.

Multiple \proc[]{rtproto} lines for the same or different routing 
procotols can occur in a simulation script.
However, a simulation cannot use both centralised routing 
mechanisms such as static or session routing and
detailed routing protocols such as DV.

In a simulation that uses more than one routing protocol,
there can be nodes that run more than one of these protocols simultaneously.
These ``border'' routers may learn of routes to a particular destination
through many of the routing protocols it runs.
In such an event, each protocol affixes a preference value to the route
to that destination.

Each node has a controlling \clsref{rtObject}{../ns-2/route-proto.tcl}.
For each destination and protocol,
the rtObject determines the preference and metric of the route to that
destination using that protocol.
It selects the route through the protocol that is most preferred.
If two routes through different protocols have the same preference,
the route with the lower metric is chosen.

As we said earlier, each route in each routing protocol is assigned
a preference value that is used in route selection by the \code{rtObject}.
These values are non-negative integers in the range 0\ldots256.
The lower the value, the more preferred the route.
Each routing protocol instance at a node
has a default preference value that is derived from the class variable
for that protocol.
The current preference values are:
\begin{program}
        Agent/rtProto set preference_ 200		\; global default preference;
        Agent/rtProto/Direct set preference_ 100
        Agent/rtProto/DV set preference_ 120
\end{program}
A simulation script can alter the class variables,
or the preferences for individual route protocol objects
at a specific node.
Finally, each protocol object stores an array of route preferences,
\code{rtpref_}.
There is one element per destination, indexed by the node handle.
This provides the ability for a simulation to alter
the preference of a route at a particular node.

In the currently implemented route protocols,
the metric of a route to a destination, at a node,
is the cost to reach the destiantion from that node.
It is possible to change the link costs at each of the links.
The instance procedure
\fcnref{\proc[]{cost}}{../ns-2/route-proto.tcl}{Simulator::cost}
%XXX MOVE TO NS-LIB.TCL
is invoked as \code{$ns cost \tup{node1} \tup{node2} \tup{cost}},
and sets the cost of the link from \tup{node1} to \tup{node2}
to \tup{cost}.
\begin{program}
        $ns cost $n1 $n2 10        \; set cost of link \textbf{from} $n1 \textbf{to} $n2 to 10;
        $ns cost $n2 $n1  5        \; set cost of link in reverse direction to 5;
        [$ns link $n1 $n2] cost?   \; query cost of link from $n1 to $n2;
        [$ns link $n2 $n1] cost?   \; query cost of link in reverse direction;
\end{program}
Notice that the procedure sets the cost along one direction only.
Similarly, the procedure
\fcnref{\proc[]{cost?}}{../ns-2/route-proto.tcl}{Link::cost?}
returns the cost of traversing the specified unidirectional link.
The default cost of a link is 1.

Finally, each node can be individually configured
to use multiple separate paths to a particular destination.
The instance variable \code{multiPath_} determines whether or not
that node will use multiple paths to any destination.
Each node initialises its instance variable from a class variable
of the same name.
If multiple routes with different paths are available at a node, and
the routes to that destination are all learned via the same protocol,
and,
either are the most preferred routes (\ie, have the lowest preference value),
or have the lowest metrics,
then the \code{rtObject} at that node can install all of the routes
into that node's classifiers.
\begin{program}
        Node set multiPath_ 1 \; All new nodes in the simulation will use multiPaths where applicable;
 {\rm or alternately}
        set n1 [$ns Node]
        $n1 set multiPath_ 1         \; only enable $n1 to use multiPaths where applicable;
\end{program}

In the following subsections, we discuss the implementation details
of each of the routing strategies currently implemented in \ns.

\subsection{Static Routing Strategy}
\label{sec:static}

The static route computation strategy is
the default route computation mechanism  in \ns.
This strategy uses the 
\fcnref{Dijkstra's all-pairs SPF algorithm \cite{}}{../ns-2/route.cc}{%
	RouteLogic::compute\_routes}.
The function takes as input an adjacency matrix.
The procedure
\fcnref{\proc[]{compute-routes}}{../ns-2/ns-lib.tcl}{RouteLogic::compute-routes}
in the \clsref{RouteLogic}{../ns-2/ns-lib.tcl}
first creates the adjancency matrix, and then
invokes \fcn[]{compute\_routes}.
Finally, the procedure retrieves the result of the route computation,
and inserts the appropriate routes at each of the nodes in the topology.

The class defines the procedure
\fcnref{\proc[]{init-all}}{../ns-2/route-proto.tcl}{Agent/rtProto/Static::init-all}
that invokes \proc[]{compute-routes}.
We discuss additional details of the internal route architecture in \ns\ later.

\subsection{Session Routing Strategy}
\label{sec:session}

The static routing strategy described in the previous section
only computes routes for the topology once at the start of the simulation.
If the topology changes during the course of the simulation, 
the topology will be partitioned.
Session routing strategy will use the procedure
\fcnref{\proc[]{compute-routes}}{../ns-2/ns-lib.tcl}{RouteLogic::compute-routes}
in the \clsref{RouteLogic}{../ns-2/ns-lib.tcl}
to recompute new routes when the topology changes.

Session routing leads to complete and instantaneous change
in the routes of the topology, whenever that topology changes.
If the topology is always connected, then there is
end-to-end connectivity at all times during the course of the simulation.

The class defines the procedure
\fcnref{\proc[]{init-all}}{../ns-2/route-proto.tcl}{Agent/rtProto/Session::init-all}
to compute the routes at the start of the simulation.
It also defines the procedure
\fcnref{\proc[]{compute-all}}{../ns-2/route-proto.tcl}{Agent/rtProto/Session::compute-all}
to compute the routes when the topology changes.
Each of these procedures directly invokes \proc[]{compute-routes}.

\subsection{Dynamic Routing Strategy: DV}
\label{sec:DV}

In a dynamic routing strategy, nodes send and receive messages,
and compute the rouets in the topology based on the messages exchanged.
DV routing is the implementation of Distributed Bellman-Ford (or
Distance Vector) routing in \ns.

The implementation sends periodic route updates every \code{advertInterval}.
This variable is a class variable in the \clsref{Agent/rtProto/DV}.
Its default value is 2 seconds.
In addition to periodic updates, each agent also sends triggered updates;
it does this whenever the forwarding tables in the node change.
This occurs either due to changes in the topology, 
or because an agent at the node received a route update,
and recomputed and installed new routes.
One final point about route advertisements by DV agents:
each agent employs split horizon with poisoned reverse mechanisms
to advertise its routes to adjacent peers.
``Split horizon'' is the mechanism by which an agent will not advertise
the route to a destination out of the interface that it is using to
reach that destination.
In a ``Split horizon with poisoned reverse'' mechanism,
the agent will advertise that route out of that interface with 
a metric of infinity.

Each DV agent uses a default \code{preference_} of 120.
The value is determined by the class variable of the same name.

Each agent uses the class variable \code{INFINITY} (set at 32)
to determine the validity of a route.

The procedure
\fcnref{\proc[]{init-all}}{../ns-2/route-proto.tcl}{Agent/rtProto/DV::init-all}
takes a list of nodes as the argument;
the default is the list of nodes in the topology.
At each of the nodes in the argument, the procedure starts the
\clsref{rtObject}{../ns-2/route-proto.tcl} and a 
\clsref{Agent/rtProto/DV}{../ns-2/route-proto.tcl} agents.
It then determines the DV peers for each of the newly created DV agents.

The
\fcnref{constructor for the DV
agent}{../ns-2/route-proto.tcl}{Agent/rtProto/DV::init}
initialises a number of instance variables;
each agent stores an array, indexed by the destiantion node handle,
of the preference and metric, the interface (or link) to the next hop,
and the remote peer incident on the interface,
for the best route to each destination computed by the agent.
The agent creates these instance variables, and then
schedules sending its first update within the first
0.5 seconds of simulation start.

Each agent stores the list of its peers indexed by the handle
of the peer node.
Each peer is a separate peer structure that holds
the address of the peer agent, the metric and preference
of the route to each destination advertised by that peer.
We discuss the rtPeer structure later
when discuss the route architecture.
The peer strucutres are initialised by the procedure
\fcnref{\proc[]{add-peer}}{../ns-2/route-proto.tcl}{Agent/rtProto/DV::add-peer}
invoked by \proc[]{init-all}.

The routine 
\fcnref{\proc[]{send-periodic-update}}{../ns-2/route-proto.tcl}{Agent/rtProto/DV::send-periodic-update}
invokes \proc[]{send-updates} to send the actual updates.
It then reschedules sending the next periodic update
after \code{adverInterval} jitterred slightly to avoid
possible synchronisation effects.

\fcnref{\proc[]{send-updates}}{../ns-2/route-proto.tcl}{Agent/rtProto/DV::send-updates}
will send updates to a select set of peers.
If any of the routes at that node have changed, or for periodic updates,
the procedure will send updates to all peers.
Otherwise, if some incident links have jsut recovered,
the procedure will send updates to the adjacent peers on those incident
links only.

\proc[]{send-updates} uses the procedure
\fcnref{\proc[]{send-to-peer}}{../ns-2/route-proto.tcl}{Agent/rtProto/DV::send-to-peer}
to send the actual updates.
his procedure packages the update, taking the
split-horizon and poison reverse mechanisms into account.
It invokes the instproc-like,
\fcnref{\proc[]{send-update} (Note the singular case)}{%
	../ns-2/rtProto.cc}{rtProtoDV::command}
to send the actual update.
The actual route update is stored in the class variable
\code{msg_} indexed by a non-decreasing integer as index.
The instproc-like only sends the index to \code{msg_} to the remote peer.
This eliminates the need to convert from OTcl strings to alternate formats
and back.

When 
\fcnref{a peer receives a route update}{../ns-2/route-proto.tcl}{%
	Agent/rtProto/DV::recv-update}
it first checks to determine if the update from differs from the previous
ones.
The agent will compute new routes if the update contains new information.


\subsection{Internals and Architecture of Routing}
\label{sec:rtg-internals}

In the earlier sections,
we have already discussed the implementation architecture
of the different routing strategies in sufficient detail.
We now discuss the meta classes associated with routing
that co-ordinate the different routing protocols.
There are three classes,
the class RouteLogic, the class rtObject, and the class rtPeer.
The routing architecture also requires interfaces to
the classes Simulator, Link, Node and Classifier.
We conclude this section with a description of the
interface of the \ns\  routing architecture with
the network dynamics architecture, and the multicast architecture.


Node init-routing
Node rtObject?
Node add-routes
Node add-route
Node delete-routes
Classifier install
Classifier installNext
Classifier adjacents

\paragraph{Class RouteLogic}

register

configure

lookup


\paragraph{Class rtObject}
\paragraph{Class rtPeer}

\subsubsection{Interface to \rtglib}
\label{sec:rtglibAPI}
The hooks to recompute routes whenever the topology changes.
falls into two categories of actions:
those to be taken at each node, and tose to be taken globally.

\subsubsection{Interface to multicast}
\label{sec:mcastAPI}

call whenever routes change.
The densemode dynamic DM specification ,

\end{document}

### Local Variables:
### mode: latex
### comment-column: 60
### backup-by-copying-when-linked: t
### file-precious-flag: nil
### End:

\chapter{Multicast Routing}
\label{chap:multicast}

This section describes the usage and the internals of multicast
routing implementation in \ns.
We first describe 
\href{the user interface to enable multicast routing}{Section}{sec:mcast-api},
specify the multicast routing protocol to be used and the
various methods and configuration parameters specific to the
protocols currently supported in \ns.
We then describe in detail 
\href{the internals and the architecture of the
multicast routing implementation in \ns}{Section}{sec:mcast-internals}.

The procedures and functions described in this chapter can be found in
various files in the directories \nsf{tcl/mcast}, \nsf{tcl/ctr-mcast},
in the files \nsf{ctrmcast.\{cc, h\}}, and \nsf{prune.\{cc, h\}};
additional support routines
are in \nsf{tcl/lib/ns-lib.tcl}, and \nsf{tcl/lib/ns-node.tcl}.

\section{Multicast API}
\label{sec:mcast-api}

Multicast routing is enabled in the simulation by setting 
the class Simulator class variable \code{EnableMcast_} to 1
before any node, link or agent objects are created.
This is so that the subsequently created 
node, link and agent objects are appropriately
configured during creation to support multicast routing.
For example the link objects are created with interface labels that
are required by some multicast routing protocols, 
node objects are created with the 
appropriate multicast classifier objects
and agent objects are made to point to the 
appropriate classifier at that node.

A multicast routing strategy is the mechanism by which the multicast
distribution tree is computed in the simulation.  The multicast
routing strategy or protocol to be used is specified through the
instance procedure \proc[]{mrtproto} in the class Simulator.
A handle is returned to an object that has methods
and configuration parameters specific to a particular multicast
routing strategy or protocol.  A null string is returned otherwise.
There are currently 4 multicast routing strategies in \ns: Centralized
Multicast, static Dense Mode, dynamic Dense Mode (\ie, adapts to
network changes), Protocol Independent Multicast---Dense Mode.
Currently only Centralized Multicast returns an object that has
methods and configuration parameters. Subsequent arguments to command
procedure \proc[]{mrtproto} specify the nodes that will run the
instance of multicast routing strategy.
The default is to run the same multicast routing
protocol on all the nodes in the topology. As an example, the
following commands illustrate the use of procedure \proc[]{mrtproto} command.
\begin{program}
        set cmc [$ns mrtproto CtrMcast \{\}]    \; specify centralized multicast for all nodes;
        \; cmc is the handle for multicast protocol object;
        $ns mrtproto DM $n1 $n2 $n3             \; specify dense mode multicast for nodes n1, n2 and n3;
        $ns mrtproto dynamicDM                  \; specify dynamic dense mode;
\end{program}

New/unused multicast address can be allocated using the procedure
\proc[]{allocaddr}. With default configuration ns can provide multicast
support for only 128 nodes. This bound on number of nodes is due to
the addressing scheme. The procedure \proc[]{expandaddr} can be used to
expand the address space if the number of nodes in simulation is more
than 128. \proc[]{allocaddr} and \proc[]{expandaddr} are class
procedures in the class Node.

The agents use the instance procedures
\proc[]{join-group} and \proc[]{leave-group}, in
the class Node to join and leave multicast groups. These procedures
take two mandatory arguments. The first argument identifies the
corresponding agent and second argument specifies the group address.

An example of a relatively simple multicast configuration is:
\begin{program}
        set ns [new Simulator]
        {\bfseries{}Simulator set EnableMcast_ 1} \; enable multicast routing;
        Node expandaddr  \; expand address space, required if more than 128 nodes;
        set group [{\bfseries{}Node allocaddr}]   \; allocate a multicast address;
        set node0 [$ns node]         \; create multicast capable nodes;
        set node1 [$ns node]
        {\bfseries{}Simulator set NumberInterfaces_ 1}  \; number interfaces for all links;
        $ns duplex-link $n0 $n1 1.5Mb 10ms DropTail \; create links with interfaces;

        set mproto dynamicDM          \; configure multicast protocol;
        set mrthandle [{\bfseries{}$ns mrtproto $mproto \{\}}] \; if an empty list is give,;
                \; all nodes will contain multicast protocol agents;
        set src [new Agent/CBR]        \; create a source agent at node 0;
        $ns attach-agent $node0 $src 
        {\bfseries{}$src set dst_ $group}

        set rcvr [new Agent/LossMonitor]  \; create a receiver agent at node 1;
        $ns attach-agent $node1 $rcvr
        $ns at 0.3 "{\bfseries{}$node1 join-group $rcvr $group}" \; join the group at simulation time 0.3 (sec);
\end{program}

\subsection{Multicast Behavior Monitor Configuration}
\ns\ supports a multicast monitor module that can trace
some types of packet activity about a specific source group pair.
The module counts the number of packets in transit periodically
and prints the results to stdout. \proc[]{trace-topo} counts
all links.  \proc[]{trace-tree} counts only links on the tree.
Currently, it only traces prune, join, register, and data packets.

\begin{program}
        set mcastmonitor [$ns McastMonitor]
        $mcastmonitor set period_ 0.02         \; default 0.03 (sec);
        $mcastmonitor trace-topo $src $group   \; trace the entire topology;
\end{program}

% SAMPLE OUTPUT?
The following sample output illustrates the output file format (time, prune, join, register, data, source, group):
{\small
\begin{verbatim}
0.14999999999999999 0 0 4 0 0 32770
0.17999999999999999 0 0 4 0 0 32770
0.20999999999999999 0 0 4 2 0 32770
0.23999999999999999 0 0 4 7 0 32770
0.27000000000000002 0 0 4 8 0 32770
\end{verbatim}
}

\subsection{Protocol Specific configuration}

In this section, we briefly illustrate the
protocol specific configuration mechanisms
for all the protocols implemented in \ns.

\paragraph{Centralized Multicast}
The centralized multicast is a sparse mode implementation of multicast
similar to PIM-SM \cite{Deer94a:Architecture}.
A Rendezvous Point (RP) rooted shared tree is built
for a multicast group.  The actual sending of prune, join messages
etc. to set up state at the nodes is not simulated.  A centralized
computation agent is used to compute the forwarding trees and set up
multicast forwarding state, \tup{S, G} at the relevant nodes as new
receivers join a group.  Data packets from the senders to a group are
unicast to the RP.  Note that data packets from the senders are
unicast to the RP even if there are no receivers for the group.

Note that whenever network dynamics occur or unicast routing changes,
\proc[]{compute-mroutes} could be invoked to recompute the multicast routes.
The instantaneous re-computation feature of centralised algorithms
may result in causality violations during the transient
periods.  Thus, centralised multicast routing strategies are not
ideal for studying transient behaviors.

The method of enabling centralised multicast routing in a simulation is:
\begin{program}
        set mproto CtrMcast    \; set multicast protocol;
        set mrthandle [$ns mrtproto $mproto \{\}]
\end{program}
The command procedure \proc[]{mrtproto}
returns a handle to the multicast protocol object.
This handle can be used to control the RP and the boot-strap-router (BSR),
switch tree-types for a particular group,
from shared trees to source specific trees or vice-versa, and
recompute multicast routes.
\begin{program}
        $mrthandle set_c_rp \{$node0 $node1\}      \; set the RPs;
        $mrthandle set_c_bsr \{$node0:0 $node1:1\} \; set the BSR, specified as list of node:priority;

        $mrthandle get_c_rp $node0 $group          \; get the current RP ???;
        $mrthandle get_c_bsr $node0                \; get the current BSR;

        $mrthandle switch-treetype $group         \; to source specific or shared tree;

        $mrthandle compute-mroutes       \; recompute routes; usually invoked automatically as needed;
\end{program}

\paragraph{Static Dense Mode}
The Static Dense Mode protocol is based on DVMRP \cite{rfc1075}
with the exception
that it does not adapt to network dynamics.  It uses parent-child
lists as in DVMRP to reduce the number of links over which the data
packets are broadcast.
At each node,
the implementation
 infers poison-reverse information
and
 computes the parent-child relationships 
 for that node by looking
at the unicast routing tables at each of the adjacent nodes.

  Prune messages for a particular group are sent
upstream by nodes in case they do not lead to any group members.
These prune messages instantiate prune state in the appropriate
upstream nodes to prevent multicast packets from being forwarded down
links that do not lead to any group members.  The prune state at the
nodes times out after a prune timeout value.  The prune timeout is a
class variable in DM and its value can be changed as shown below.
\begin{program}
        DM set PruneTimeout 0.3           \; default 0.5 (sec);
        set mproto DM
        set mrthandle [$ns mrtproto $mproto \{\}]
\end{program}
This multicast protocol computes parent-child relationships prior to
the start of simulation and does not respond to the topology
changes. It models group membership dynamics very well.

\paragraph{Dynamic dense mode}
This protocol implements a DVMRP-like dense mode protocol,
similar to DM described earlier.
However, this implementation is also capable of adapting to changes
in the topology.

As in the case of DM, a node in this implementation uses the
unicast routing tables at all of its adjacent neighbors to
compute its parent-child relationships%
\footnote{This is somewhat different from classic DVMRP, in that
the implementation does not actually send out route updates;
however, it infers the arrival of these updates from other events
and attempts to reduce memory usage through shared routing table information.}.
The multicast routes are computed periodically every
\code{ReportRouteTimeout} seconds.
They are also computed when network dynamics triggers a notification,
or the node receives a unicast route update.
The last event is used as a signal of a neighbor sending a 
DVMRP-like unicast route update to notify multicast route changes at that
neighbor.
The class dynamicDM is
inherited from the class DM.
\begin{program}
        dynamicDM set PruneTimeout 0.3           \; default 0.5 (sec);
        dynamicDM set ReportRouteTimeout 0.5  \; default 1 (sec);
        set mproto dynamicDM
        set mrthandle [$ns mrtproto $mproto \{\}]
\end{program}

\paragraph{PIM dense mode}
The Protocol Independent Multicast Dense Mode implementation
does not compute any parent-child relationships.
Therefore, incoming  data packets at a node are broadcast
to all outgoing links except the link that the packet came over.
Neighbors send prune messages if they have no listeners
at that node or downstream of them.
The prune timeout value is 0.5s by default.
This class pimDM is inherited from the class DM.
\begin{program}
        pimDM set PruneTimeout  0.3     \; default 0.5 (sec);
        set mproto pimDM
        set mrthandle [$ns mrtproto $mproto \{\}]
\end{program}

\section{Internals of multicast routing}
\label{sec:mcast-internals}

We first describe the main classes that are used to implement multicast
routing and then describe how each multicast routing strategy
or protocol is implemented.

\subsection{The classes}
The main classes in the implementation are
the \clsref{McastProtoArbiter}{../ns-2/tcl/mcast/McastProto.tcl} and
the \clsref{McastProtocol}{../ns-2/tcl/mcast/McastProto.tcl}.
The class McastProtocol is the base
class for the various multicast routing strategy and protocol objects.
In addition some methods and configuration parameters have been defined in
the Simulator, Node, Classifier, and Replicator objects for multicast
routing.

\paragraph{McastProtoArbiter class}
There is one McastProtoArbiter object per multicast capable node.
Each  McastProtoArbiter object maintain a list of multicast protocols.
This arbiter supports the ability for a node to run multiple multicast
routing protocols.
However, most simulations in \ns\ appear to be
configured with one multicast routing protocol for the entire simulations.

Whenever there is a multicast related action in a node, the
node will access its multicast protocols through this McastProtoArbiter.

Basic functions are \proc[]{join-group}, \proc[]{leave-group}, and
\proc[]{upcall}.
These functions translate into calls to corresponding functions of the
same name in each of the multicast protocols running at that node.

\paragraph{McastProtocol class}
This is the base class for the implementation of all the multicast protocols.
It contains basic multicast functions:
\proc[]{join-group}, \proc[]{leave-group},
\proc[]{handle-cache-miss}, \proc[]{handle-wrong-iif}, \proc[]{drop}.
Protocol specific actions are appropriately redefined
by the individual multicast protocols that inherit these functions.

\subsection{Extensions to other classes in \ns}
We have already described the simulator class variable
\code{EnableMcast_} to enable multicast simulations.
In 
\href{the earlier chapter describing nodes in \ns}{Chapter}{chap:nodes},
we described the internal structure of the node in \ns.
To briefly recap that description, the node entry for a multicast node is
the \code{switch_}.  It looks at the highest bit to decide
 if the destination is a multicast or unicast packet.
 Multicast packets are forwarded to a multicast
classifier which maintains a list of replicators;
there is one replicator per \tup{source, group, incoming interface} tuple.
Replicators copy the incoming packet and forward to all outgoing interfaces.

When a node receives a join-group message,
\proc[]{Node::join-group} is invoked, that then invokes
the arbiter instance's
\proc[]{join-group}.
The node stores a reference to the McastProtoArbiter at that node in
its instance variable \code{mcastproto_}.
After the arbiter has returned, the node
adds the appropriate receiver agents to its list of \code{Agents_},
and adds the outgoing interfaces to all the replicators for that group.

\proc[]{Node::leave-group} reverses the process described earlier.
It disables the outgoing interfaces to the receiver agents
for all the replicators of the group,
deletes the receiver agents from the local \code{Agents_} list; it then
invokes the arbiter instance's 
\proc[]{leave-group}.

When a multicast data packet is received, and the multicast
classifier cannot find the slot corresponding to that data packet,
it calls \proc[]{Node::new-group}.
This procedure notifies the arbiter instance to establish the new group.

\proc[]{Node::add-mfc} adds a multicast forwarding cache entry for
a particular \tup{source, group, iif}.
The mechanism is:
(1) create a new replicator (if one does not already exist),
(2) update the \code{replicator_} and \code{repByGroup_}
 instance variable arrays at the node,
(3) adds all outgoing interfaces and local agents
to the appropriate replicator,
and finally,
(4) invoke the multicast classifier's \proc[]{add-rep}
 to create a slot for the replicator in the multicast classifier.

\paragraph{Class Classifier}
There is once multicast classifier per node.
The node stores a reference to this classifier in its instance variable
\code{multiclassifier_}.
When this classifier receives a packet,
it looks at the \tup{source, group} information in the packet headers,
and the interface that the packet arrived from (the incoming interface or iif);
using that information, the classifier will identify the slot
that should be used to forward that packet.  The slot will point
to the appropriate replicator.

However, if the slot is invalid, or the classifier does not have an
entry for this \tup{source, group, iif},
then it will invoke \proc[]{new-group} for the classifier,
with one of two codes to identify the problem.
\code{CACHE_MISS} indicates that the classifier did not find any
\tup{source, group} entries;
\code{WRONG_IIF} indicates that the classifier found \tup{source, group}
entries, but none matching the interface that this packet arrived over.
In this situation, this particular data is dropped.

The classifier \proc[]{new-group} invokes \proc[]{Node::new-group}
to create the group entries.

\proc[]{add-rep} creates a slot in the classifier
and adds a replicator for \tup{source, group, iif} to that slot.

\paragraph{Class Replicator}
When a replicator receives a packet,
it copies the packet to all of its slots.
Each slot points to an outgoing interface for a particular 
\tup{source, group, iif}.
If no slot is found, the C++ replicator invokes the class 
instance procedure \proc[]{drop} to
trigger protocol specific actions.
We will describe the protocol specific actions in the next section,
when we describe the internal procedures of each of the 
multicast routing protocols.

There are instance procedures to control the elements in each slot:
\begin{alist}
\proc[oif]{insert} & inserting a new outgoing interface
                        to the next available slot.\\
\proc[oif]{disable} & disable the slot pointing to the specified oif.\\
\proc[oif]{enable} &  enable the slot pointing to the specified oif.\\
\proc[]{is-active} & returns true if the replicator has at least one active slot.\\
\proc[oif]{exists} & returns true if the slot pointing to the specified oif is active.\\
\proc[source, group, oldiif, newiif]{change-iface} & modified the iif entry for the particular replicator.\\
\end{alist}

\subsection{Protocol Internals}
\label{sec:mcastproto-internals}

We now describe the implementation of
the different multicast routing protocol agents.

\subsubsection{Centralized Multicast}
\begin{list}{}{}
\item
\code{CtrMcast} is inherited from \code{McastProtocol}.
One CtrMcast agent needs to be created for each node.
This CtrMcast agent handles multicast commands for the node it is
attached to;
\eg, \proc[]{join-group}, 
\proc[]{leave-group}, and \proc[]{handle-cache-miss}.
When multicast forwarding entries need to be updated, it calls a 
global, unique CtrMcastComp
(centralized multicast computation) agent to compute and install the new
entries.
We will describe the centralised multicast computation details
in the following paragraphs.

--- 
\proc[]{join-group} adds the node to the instance variable
\code{Mlist} of the instance of \code{CtrMcastComp}.
\proc[]{join-group}
then calls the CtrMcastComp agent, \code{Agent}, to compute the branches
for this node to reach all existing sources.

---
\proc[]{leave-group} is the reverse of \proc[]{join-group}

---
\proc[]{handle-cache-miss} is called when no proper forwarding entry is
found for a particular packet source.
In case of centralized multicast,
it means a new source has started sending data packets.
Thus, the new source is added to the instance variable \code{Slist}
of the instance of \code{CtrMcastComp}.
If there are members in \code{Mlist},
the new multicast tree will be computed.
In the case that the particular
group is in RPT(shared tree) mode, an encapsulation agent is created at the
source; 
a corresponding decapsulation agent is created at the RP;
the two agents are connected by unicast, and the \tup{S,G} entry points
its outgoing interface to the encapsulation agent.
\item
\code{CtrMcastComp} is the centralised multicast route computation agent.
\begin{alist}
\proc[]{reset-mroutes} & resets all multicast forwarding entries.\\
\proc[]{compute-mroutes} & (re)computes all multicast forwarding entries.\\
\proc[source, group]{compute-tree} & computes a multicast tree for one source to reach 
                all the receivers in a specific group.\\
\proc[source, group, member]{compute-branch} & is executed when a receiver joins a multicast group.
        It could also be invoked by \proc[]{compute-tree} when it itself
        is recomputing the multicast tree, and has to reparent
        all receivers.
        The algorithm starts at the receiver, recursively
        finding successive next hops,
        until it either reaches the source or RP,
        or it reaches a node that is already 
        a part of the relevant multicast tree.
         During the process, several new replicators and an
        outgoing interface will be installed.\\
\proc[source, group, member]{prune-branch} & is similar to \proc[]{compute-branch} except the
        outgoing interface is disabled;
        if the outgoing interface list is empty at that node,
        it will walk up the multicast tree, pruning at each of the
        intermediate nodes, until it reaches a node that has a
        non-empty outgoing interface list for the particular multicast tree.
\end{alist}
\end{list}

\subsubsection{Dense Mode}
\begin{alist}
\proc[group]{join-group} & sends graft messages upstream if \tup{S,G} does not
        contain any active outgoing slots (\ie, no downstream receivers).\\
\proc[group]{leave-group} & does not do anything.\\
\proc[group]{handle-cache-miss} & creates \tup{S,G} with only
        the outgoing interfaces to each of the child nodes.
        For basic dense mode,
        parent-child relationship is computed only once at the start-up.\\
 &      However, for dynamic dense mode,
        the incoming interface must be set properly so packets from the
        wrong incoming interfaces will be dropped. \\
 &      Finally, in the case of PIM dense mode, 
        \proc[]{handle-cache-miss} is the same as in dynamic dense mode,
        except that the outgoing interface list is set to all neighbors
        excluding the incoming interface.\\
\proc[replicator, source, group]{drop} & sends prune messages upstream.
        Recall that the packet is only dropped when there are
        no downstream receivers for the \tup{S, G} tuple.\\
\proc[from, source, group]{recv-prune} & resets the prune timer
         if the interface had been pruned previously;
        otherwise, it starts the prune timer and disables the interface;
        furthermore,  if the outgoing interface list becomes empty,
        it forwards the prune message upstream.\\
\proc[from, source, group]{recv-graft} & cancels any existing prune timer, and
        re-enables the pruned interface.
        If the outgoing interface list was previously empty,
        it forwards the graft upstream.\\
\proc[]{periodic-check} & This procedure is only used in dynamic dense mode.
        Each node periodically updates its parent-child relationships
        with respect to each of its neighbors.\\
proc[]{handle-wrong-iif} & This procedure is only used by PIM dense mode.
        This is invoked when the multicast classifier drops a packet
        because it arrived on the wrong interface, and
        invoked \proc[]{new-group}.
        This routine is invoked by \proc[]{McastProtoArbiter::new-group}.
        When invoked, the agent checks if the incoming interface in the
        forwarding entry has become outdated.
        If so, this procedure will update the forwarding entry to the
        new incoming interface;
        otherwise, it will send a prune message back towards the
        wrong incoming interface to stop packets
        reaching from the wrong path.
\end{alist}

\endinput

\include{dynamics}

\part{Transport}
%
%
% Q's for sally:
%	timestamps are supplied in each packet, and any new ACK
%	triggers an update to the estimator, but yet there is
%	other (unused?) code which apparently looks like it once
%	did the one measurement per window of data... what's the
%	story on that?
%
\documentclass{article}

\PassOptionsToPackage{draft}{MyPreamble}
%%\usepackage[widen-page,skrunch-figures]{MyPreamble}
%%\usepackage[widen-page,skrunch-figures]
\usepackage{nsDoc}
\include{localhack}
\newcommand{\shdr}[3]{\htmladdnormallink{#1}{#2}\label{#3}}

\begin{document}

\title{\nsTcl\ internals documentation}
\author{%
  Kevin Fall \tup{kfall@ee.lbl.gov}\\
  Kannan Varadhan \tup{kannan@catarina.usc.edu}}
\date{\today}

\def\c#1{\ensuremath{C_{#1}}}
\def\d#1{\ensuremath{D_{#1}}}

% \maketitle

\section{\shdr{TCP}{tcp.cc}{sec:tcp}}

This section describes the operation of the TCP agents in \ns.
There are two major types of TCP agents: one-way agents
and a two-way agent.
One-way agents are further subdivided into a set of TCP senders
(which obey different congestion and error control techniques)
and receivers (``sinks'').
The two-way agent is symmetric in the sense that it represents
both a sender and receiver.
It is still under development.

The one-way TCP sending agents currently supported are:
\begin{itemize}
	\item Agent/TCP - a ``tahoe'' TCP sender
	\item Agent/TCP/Reno - a ``Reno'' TCP sender
	\item Agent/TCP/NewReno - Reno with a modification
	\item Agent/TCP/Sack1 - TCP with selective repeat (follows RFC2018)
	\item Agent/TCP/Vegas - TCP Vegas
	\item Agent/TCP/Fack - Reno TCP with ``forward acknowledgement''
\end{itemize}
The one-way TCP receiving agents currently supported are:
\begin{itemize}
	\item Agent/TCPSink - TCP sink with one ACK per packet
	\item Agent/TCPSink/DelAck - TCP sink with configurable delay per ACK
	\item Agent/TCPSink/Sack1 - selective ACK sink (follows RFC2018)
	\item Agent/TCPSink/Sack1/DelAck - Sack1 with DelAck
\end{itemize}
The two-way experimental sender currently supports only a Reno form of TCP:
\begin{itemize}
	\item Agent/TCP/FullTcp
\end{itemize}

The section comprises three parts:
the first part is a simple overview and example of configuring
the base TCP send/sink agents (the sink requires no configuration).
The second part describes the internals of the base send agent,
and last part is a description of the extensions
for the other types of agents that have been included in the
simulator.

\subsection{\shdr{One-Way TCP Senders}{tcp.h}{sec:tcp}}

The simulator supports several versions of an abstracted TCP sender.
These objects attempt to capture the essence of the TCP congestion
and error control behaviors, but are not intended to be faithful
replicas of real-world TCP implementations.
They do not contain a dynamic window advertisement, they do segment
number and ACK number computations entirely in packet units,
there is no SYN/FIN connection establishment/teardown, and no
data is ever transferred (e.g. no checksums or urgent data).

\subsubsection{\shdr{the base TCP sender (Tahoe TCP)}{tcp.cc}{sec:tahoetcp}}

The ``Tahoe'' TCP agent \code{Agent/TCP} performs congestion
control and round-trip-time estimation
in a way similar to the version of TCP released with the
4.3BSD ``Tahoe'' UN'X system release from UC Berkeley.
The congestion window is increased by one packet per new ACK received
during slow-start (when $cwnd\_ < ssthresh\_$) and is increased
by $\frac{1}{cwnd\_}$ for each new ACK received during congestion avoidance
(when $cwnd\_ \geq ssthresh\_$).

\paragraph{responses to congestion}
Tahoe TCP assumes a packet has been lost (due to congestion)
when it observes {\tt NUMDUPACKS} (defined in \code{tcp.h}, currently 3)
duplicate ACKs, or when a retransmission timer expires.
In either case, Tahoe TCP reacts by setting {\tt ssthresh\_} to half
of the current window size (the minimum of {\tt cwnd\_} and {\tt window\_})
or 2, whichever is larger.
It then initializes {\tt cwnd\_} back to the value of
{\tt windowInit\_}.  This will typically cause the TCP to
enter slow-start.

\paragraph{round-trip time estimation and RTO timeout selection}
Four variables are used to estimate the round-trip time and
set the retransmission timer: {\tt rtt\_, srtt\_, rttvar\_, tcpTick\_,
and backoff\_}.
TCP initializes rttvar to $3/tcpTick\_$ and backoff to 1.
When any future retransmission timer is set, it's timeout
is set to the current time plus $\max(bt(a+4v+1), 64)$ seconds,
where $b$ is the current backoff value, $t$ is the value of tcpTick,
$a$ is the value of srtt, and $v$ is the value of rttvar.

Round-trip time samples arrive with new ACKs.
The RTT sample is computed as the difference between the current
time and a ``time echo'' field in the ACK packet.
When the first sample is taken, its value is used as the initial
value for {\tt srtt\_}.  Half the first sample is used as the initial
value for {\tt rttvar\_}.
For subsequent samples, the values are updated as follows:

\[ srtt = \frac{7}{8} \times srtt + \frac{1}{8} \times sample \]
\[ rttvar = \frac{3}{4} \times rttvar + \frac{1}{4} \times |sample-srtt| \]

\subsubsection{Configuration}
\label{sec:tcp-config}

Running an TCP simulation requires
creating and configuring the agent,
attaching an application-level data source (a traffic generator), and
starting the agent and the traffic generator.

\subsubsection{Simple Configuration}

\paragraph{Creating the Agent}
\begin{program}
set ns [new Simulator]                  \; preamble initialisation;
set node1 [$ns node]                     \; agent to reside on this node;
set node2 [$ns node]                     \; agent to reside on this node;

{\bfseries set tcp1 [$ns create-connection TCP $node1 TCPSink $node2 42]}     \\
$tcp  set window_ 50                   \; configure the TCP agent;
{\bfseries set ftp1 [$tcp1 attach-source FTP]}     \\
$ns at 0.0 "$ftp start"
\end{program}
This example illustrates the use of the simulator built-in
function {\tt create-connection}.
The arguments to this function are: the source agent to create,
the source node, the target agent to create, the target node, and
the flow ID to be used on the connection.
The function operates by creating the two agents, setting the
flow ID fields in the agents, attaching the source and target agents
to their respective nodes, and finally connecting the agents
(i.e. setting appropriate source and destination addresses and ports).
The return value of the function is the name of the source agent created.

\paragraph{TCP Data Source}
The TCP agent does not generate any application data on its own;
instead, the simulation user can connect any traffic generation
module to the TCP agent to generate data.
Two sources are commonly used for TCP: FTP and Telnet.
FTP represents a bulk data transfer of large size, and telnet chooses
its transfer sizes randomly from tcplib (see the file \code{tcplib-telnet.cc}.
Creation and configuration of the source
is accomplished by the {\tt Agent attach-source} {\em stype} function,
which creates a new object of type \code{Source/}{\em stype} and
returns its name.
The returned object may be started at a later time.

\subsubsection{Other Configuration Parameters}

In addition to the \code{window\_} parameter listed above,
the TCP agent supports additional configuration variables.
Each of the variables described in this subsection is
both a class variable and an instance variable.
Changing the class variable changes the default value
for all agents that are created subsequently.
Changing the instance variable of a particular agent
only affects the values used by that agent.
For example,
\begin{program}
  Agent/TCP set window_ 100     \; Changes the class variable;
  $tcp set window_ 2.0          \; Changes window_ for the $tcp object only;
\end{program}

The default parameters for each TCP agent are:
\begin{program}
Agent/TCP set window_   20              \; max bound on window size;
Agent/TCP set windowInit_ 1             \; initial/reset value of cwnd;
Agent/TCP set windowOption_ 1           \; cong avoid algorithm (1: standard);
Agent/TCP set windowConstant_ 4         \; used only when windowOption != 1;
Agent/TCP set windowThresh_ 0.002       \; used in computing averaged window;
Agent/TCP set overhead_ 0               \; !=0 adds random time between sends;
Agent/TCP set ecn_ 0                    \; TCP should react to ecn bit ;
Agent/TCP set packetSize_ 1000          \; packet size used by sender (bytes);
Agent/TCP set bugFix_ true              \; see explanation;
Agent/TCP set slow_start_restart_ true  \; see explanation;
Agent/TCP set tcpTick_ 0.1              \; timer granulatiry in sec (.1 is NONSTANDARD);
Agent/TCP set maxrto_ 64                \; bound on RTO (seconds);
Agent/TCP set dupacks_ 0                \; duplicate ACK counter;
Agent/TCP set ack_ 0                    \; highest ACK received;
Agent/TCP set cwnd_ 0                   \; congestion window (packets);
Agent/TCP set awnd_ 0                   \; averaged cwnd (experimental);
Agent/TCP set ssthresh_ 0               \; slow-stat threshold (packets);
Agent/TCP set rtt_ 0                    \; rtt sample;
Agent/TCP set srtt_ 0                   \; smoothed (averaged) rtt;
Agent/TCP set rttvar_ 0                 \; mean deviation of rtt samples;
Agent/TCP set backoff_ 0                \; current RTO backoff factor;
Agent/TCP set maxseq_ 0                 \; max (packet) seq number sent;

\end{program}

For many simulations, few of the configuration parameters are likely
to require modification.
The more commonly modified parameters include: {\tt window\_} and
{\tt packetSize\_}.
The first of these bounds the window TCP uses, and is considered
to play the role of the receiver's advertised window in real-world
TCP (although it remains constant).
The packet size essentially functions like the MSS size in real-world
TCP.
Changes to these parameters can have a profound effect on the behavior
of TCP.
Generally, those TCPs with larger packet sizes, bigger windows, and
smaller round trip times (a result of the topology and congestion) are
more agressive in acquiring network bandwidth.

\subsubsection{other one-way TCP senders}

\paragraph{Reno TCP}
The Reno TCP agent is very similar to the Tahoe TCP agent,
except it also includes {\em fast recovery}, where the current
congestion window is ``inflated'' by the number of duplicate ACKs
the TCP sender has received before receiving a new ACK.
A ``new ACK'' refers to any ACK with a value higher than the higest
seen so far.
In addition, the Reno TCP agent does not return to slow-start during
a fast retransmit.
Rather, it reduces sets the congestion window to half the current
window and resets {\tt ssthresh\_} to match this value.

\paragraph{NewReno TCP}
This agent is based on the Reno TCP agent, but which modifies the
action taken when receiving new ACKS.
In order to exit fast recovery, the sender must receive an ACK for the
highest sequence number sent.
Thus, new ``partial ACKs'' (those which represent new ACKs but do not
represent an ACK for all outstanding data) do not deflate the window
(and possibly lead to a stall, characteristic of Reno).

\paragraph{Vegas TCP}
This agent implements ``Vegas'' TCP (\cite{vegas}).
It was contributed by Ted Kuo.

\paragraph{Sack TCP}
This agent implements selective repeat, based on selective ACKs provided
by the receiver.
It follows the ACK scheme described in RFC 2018, and was developed
with Matt Mathis and Jamshid Mahdavi.

\paragraph{Fack TCP}
This agent implements ``forward ACK'' TCP, a modification of Sack
TCP described in \cite{matt-jamshid}.

\subsection{TCP Receivers (sinks)}

The TCP senders described above represent one-way data senders.
They must peer with a ``TCP sink'' object.

\subsubsection{the base TCP sink}

The base TCP sink object ({\tt Agent/TCPSink})
is responsible for returning ACKs to
a peer TCP source object.
It generates one ACK per packet received.
The size of the ACKs may be configured.
The creation and configuration of the TCP sink object
is generally performed automatically by a library
call (see {\tt create-connection} above).

\paragraph{configuration parameters}
\begin{program}
Agent/TCPSink set packetSize_ 40
\end{program}

\subsubsection{delayed-ACK TCP sink}

A delayed-ACK sink object ({\tt Agent/Agent/TCPSink/DelAck}) is available
for simulating a TCP receiver that ACKs less than once per packet received.
This object contains a bound variable {\tt interval\_} which gives the
number of seconds to wait between ACKs.
The delayed ACK sink implements an agressive ACK policy whereby
only ACKs for in-order packets are delayed.
Out-of-order packets cause immediate ACK generation.

\paragraph{configuration parameters}
\begin{program}
Agent/TCPSink/DelAck set interval_ 100ms
\end{program}

\subsubsection{Sack TCP Sink}

The selective-acknowledgement TCP sink ({\tt Agent/TCPSink/Sack1}) implements
SACK generation modeled after the description of SACK in RFC 2018.
This object includes a bound variable {\tt maxSackBlocks\_} which gives
the maximum number of blocks of information in an ACK available for
holding SACK information.
The default value for this variable is 3, in accordance with the expected
use of SACK with RTTM (see RFC 2018, section 3).
Delayed and selective ACKs together are implemented by
an object of type {\tt Agent/TCPSink/Sack1/DelAck}.

\paragraph{configuration parameters}
\begin{program}
Agent/TCPSink set maxSackBlocks_ 3
\end{program}

\subsection{Architecture and Internals}
\label{sec:tcparchitecture}

The base TCP agent (class {\tt Agent/TCP}) is constructed
as a collection of routines for sending packets, processing ACKs,
managing the send window, and handling timeouts.
Generally, each of these routines may be over-ridden by a
function with the same name in a derived class (this is how
many of the TCP sender variants are implemented).

\paragraph{the TCP header}
The TCP header is defined by the {\tt hdr\_tcp} structure
in the file {\tt tcp.h}.
The base agent only makes use of the following subset of the fields:
\begin{program}
ts_	\*current time packet was sent from source ;
ts_echo_ \*for ACKs: timestamp field from packet associated with this ACK;
seqno_ \* sequence number for this data segment or ACK (Note: overloading!);
reason_ \* set by sender when (re)transmitting to trace reason for send;
\end{program}

\paragraph{functions for sending data}
Note that generally the sending TCP never actually sends
data (it only sets the packet size).

{\bf send\_much(force, reason, maxburst)} - this function
attempts to send as many packets as the current sent window allows.
It also keeps track of how many packets it has sent, and limits to the
total to {\em maxburst}. \\
The function {\tt output(seqno, reason)} sends one packet
with the given sequence number and updates the maximum sent sequence
number variable ({\tt maxseq\_}) to hold the given sequence number if
it is the greatest sent so far.
This function also assigns the various fields in the TCP
header (sequence number, timestamp, reason for transmission).
This function also sets a retransmission timer if one is not already
pending.

\paragraph{functions for window management}

The usable send window at any time is given by the function {\bf window()}.
It returns the minimum of the congestion window and the variable {\tt wnd\_},
which represents the receiver's advertised window.

{\bf opencwnd()} - this function opens the congestion window.  It is invoked
when a new ACK arrives.
When in slow-start, the function merely increments {\tt cwnd\_} by each
ACK received.
When in congestion avoidance, the standard configuration increments {\tt cwnd\_}
by its reciprocal.
Other window growth options are supported during congestion avoidance,
but they are experimental (and not documented; contact Sally Floyd for
details).

{\bf closecwnd(int how)} - this function reduces the congestion window. It
may be invoked in several ways: when entering fast retransmit, due to
a timer expiration, or due to a congestion notification (ECN bit set).
Its argument {\tt how} indicates how the congestion window should
be reduced.  The value {\bf 0} is used for retransmission timeouts and
fast retransmit in Tahoe TCP.  It typically causes the TCP to enter
slow-start and reduce {\tt ssthresh\_} to half the current window.
The value {\bf 1} is used by Reno TCP for implementing fast recovery
(which avoids returning to slow-start).
The value {\bf 2} is used for reducing the window due to an ECN indication.
It resets the congestion window to its initial value (usually causing
slow-start), but does not alter {\tt ssthresh\_}.

\paragraph{functions for processing ACKs}

{\bf recv()} - this function is the main reception path for ACKs.
Note that because only one direction of data flow is in use, this function
should only ever be invoked with a pure ACK packet (i.e. no data).
The function stores the timestamp from the ACK in {\tt ts\_peer\_}, and
checks for the presence of the ECN bit (reducing the send window if
appropriate).
If the ACK is a new ACK, it calls {\bf newack()}, and otherwise
checks to see if it is a duplicate of the last ACK seen.
If so, it enters fast retransmit by closing the window, resetting the
retransmission timer, and sending a packet by calling {\tt send\_much}.

{\bf newack()} - this function processes a ``new'' ACK (one that contains
an ACK number higher than any seen so far).
The function sets a new retransmission timer by calling {\bf newtimer()},
updates the RTT estimation by calling {\bf rtt\_update}, and updates
the highest and last ACK variables.

\paragraph{functions for managing the retransmission timer}

These functions serve two purposes: estimating the round-trip time
and setting the actual retransmission timer.
{\bf rtt\_init} - this function initializes {\tt srtt\_} and {\tt rtt\_}
to zero, sets {\tt rttvar\_} to $3/tcp\_tick\_$, and sets the backoff
multiplier to one.

{\bf rtt\_timeout} - this function gives the timeout value in seconds that
should be used to schedule the next retransmission timer.
It computes this based on the current estimates of the mean and deviation
of the round-trip time.  In addition, it implements Karn's
exponential timer backoff for multiple consecutive retransmission timeouts.

{\bf rtt\_update} - this function takes as argument the measured RTT
and averages it in to the running mean and deviation estimators
according to the description above.
Note that {\tt t\_srtt\_} and {\tt t\_rttvar} are both
stored in fixed-point (integers).
They have 3 and 2 bits, respectively, to the right of the binary
point.

{\bf reset\_rtx\_timer} -  This function is invoked during fast retransmit
or during a timeout.
It sets a retransmission timer
by calling {\tt set\_rtx\_timer} and if invoked by a timeout also calls
{\tt rtt\_backoff}.

{\bf rtt\_backoff} - this function backs off the retransmission timer
(by doubling it).

{\bf newtimer} - this function called only when a new ACK arrives.
If the sender's left window edge is beyond the ACK, then
{\tt set\_rtx\_timer} is called, otherwise if a retransmission timer
is pending it is cancelled.

\subsection{Extending the Base Class Agent}
\label{sec:extensions}


\end{document}

\documentclass{article}

%\usepackage{times}
%\usepackage[T1]{fontenc}

\PassOptionsToPackage{draft}{MyPreamble}
\usepackage[widen-page,skrunch-figures]{MyPreamble}
\usepackage{nsDoc}

\begin{document}

\title{\nsTcl\ internals documentation}
\author{%
  Kevin Fall \tup{kfall@ee.lbl.gov}\\
  Kannan Varadhan \tup{kannan@catarina.usc.edu}}
\date{\today}

\def\c#1{\ensuremath{C_{#1}}}
\def\d#1{\ensuremath{D_{#1}}}

% \maketitle

\section{Agent/SRM}
\label{sec:agent/srm}

This section describes the internals of the SRM implementation in \ns.
The section is in three parts:
the first part is an overview of a minimal SRM configuration,
and a ``complete'' description of the comfiguation parameters 
of the base SRM agent.
The second part describes the architecture, internals, and the code path
of the base SRM agent.
The last part of the section is a description of the extensions
for other types of SRM agents that have been attempted to date.

\subsection{Configuration}
\label{sec:srm-config}

Running an SRM simulation requires
creating and configuring the agent,
attaching an application-level data source (a traffic generator), and
starting the agent and the traffic generator.

\subsubsection{Trivial Configuration}

\paragraph{Creating the Agent}
\begin{program}
set ns [new Simulator]                  \; preamble initialisation;
$ns enableMcast                         \\
set node [$ns node]                     \; agent to reside on this node;
set group [$ns allocaddr]               \; multicast group for this agent;
\\
{\bfseries set srm [new Agent/SRM]}     \\
$srm  set dst_ $group                   \; configure the SRM agent;
{\bfseries $ns attach-agent $node $srm} \\
\\
$srm  set fid_ 1                        \; optional configuration;
$srm  log [open srmStats.tr w]          \; log statistics in this file;
$srm  trace [open srmEvents.tr w]       \; trace events for this agent;
\end{program}
The key steps in configuring a virgin SRM agent are to assign
its multicast group, and attach it to a node.

Other useful configuration parameters are
to assign a separate flow id to traffic originating from this agent,
to open a log file for statistics, and
a trace file for trace data%
\footnote{%
Note that the trace data can also be used
to gather certain kinds of trace data.
We will illustrate this later.}.

The file
\fcnref{\|tcl/mcast/srm-nam.tcl|}{../ns-2/srm-nam.tcl}{Agent/SRM::send}
contains definitions to further separate control traffic
originating from all SRM agents by type.
Each type is allocated a separate flowID.
This is useful for analysis of traffic traces, or
for visualisation in nam.
To do this, the user must source \|srm-nam.tcl| before
creating any SRM agents.
The traffic is separated into session messages (flowid = 40),
requests (flowid = 41), and repair messages (flowid = 42).
The base flowid can be changed by setting global variable \|ctrlFid|
to one less than the desired flowid before sourcing \|srm-nam.tcl|.

\paragraph{Application Data Handling}
The agent does not generate any application data on its own;
instead, the simulation user can connect any traffic generation
module to any SRM agent to generate data.
The following code demonstrates how the traffic generation
modules can be attached to an SRM agent:
\begin{program}
  set packetSize 210                                                     \\
  set exp0 [new Traffic/Expoo]          \; configure traffic generator;
  $exp0 set packet-size $packetSize                                      \\
  $exp0 set burst-time 500ms                                             \\
  $exp0 set idle-time 500ms                                              \\
  $exp0 set rate 100k                                                    \\
\\
  set s0 [new Agent/CBR/UDP]    \; attach traffic generator to application;
  $s0 set fid_ 0                                                         \\
  $s0 attach-traffic $exp0                                               \\
\\
  {\bfseries $srm(0) traffic-source $s0} \; attach application to SRM agent;
  {\bfseries $srm(0) set packetSize_ $packetSize} \; to generate repair packets of appropriate size;
\end{program}
The instproc \texttt{\textbf{traffic-source}} specifies the application agent
that will produce data for the SRM agent.
The user can attach any agent;
the only distinguishing criteria is that the destination address must be zero.
The SRM agent will add the SRM headers, 
set the destination address to the multicast group, and
deliver the packet to its target.

The SRM agent does not generate its own data;
it does not also keep track of the data sent,
except to record the sequence numbers of messages received
in the event that it has to do error recovery.
Since the agent has no actual record of past data,
it needs to know what packet size to use for each repair message.
Hence, the instance variable \|packetSize\_| specifies the size
of repair messages generated by the agent.

\paragraph{Starting the Agent and Traffic Generator}
The user can separately start the agent and the traffic generator.
\begin{program}
{\bfseries \fcnref{$srm start}{../ns-2/srm.tcl}{Agent/SRM::start}} \\
{\bfseries \fcnref{$srm start-source}{../ns-2/srm.tcl}{Agent/SRM::start-source}}
\end{program}
At \|start|, the agent joins the multicast group, and 
starts generating session messages.
The \|start-source| triggers the traffic generator to start sending
data.

\subsubsection{Other Configuration Parameters}

In addition to the above parameters,
the SRM agent supports additional configuration variables.
Each of the variables described in this subsection is
both a class variable and an instance variable.
Changing the class variable changes the default value
for all agents that are created subsequently.
Changing the instance variable of a particular agent
only affects the values used by that agent.
For example,
\begin{program}
  \>Agent/SRM set D1_ 2.0     \; Changes the class variable;
  \>$srm set D1_ 2.0          \; Changes D1_ for the $srm object only;
\end{program}

The default request and repair timer parameters \cite{Floy95:Reliable}
for each SRM agent are:
\begin{program}
Agent/SRM set C1_       2.0             \; request parameters;
Agent/SRM set C2_       2.0             \\
Agent/SRM set D1_       1.0             \; repair parameters;
Agent/SRM set D2_       1.0             \\
\end{program}
It is thus possible to trivially obtain two flavours of SRM agents
based on whether the agents use probabilistic or deterministic
suppression by using the following definitions:
\begin{program}
Class Agent/SRM/Deterministic -superclass Agent/SRM     \\
Agent/SRM/Deterministic set C2_ 0.0                     \\
Agent/SRM/Deterministic set D2_ 0.0                     \\
                                                        \\
Class Agent/SRM/Probabilistic -superclass Agent/SRM     \\
Agent/SRM/Probabilistic set C1_ 0.0                     \\
Agent/SRM/Probabilistic set D1_ 0.0                     \\
\end{program}

Timer related functions are handled by separate objects
belonging to the class  SRM.
Timers are required for loss recovery and sending periodic session messages.
There are loss recovery objects to send request and repair messages.
The agent creates a separate request or repair object to handle each loss.
In contrast, the agent only creates one session object to send
periodic session messages.
The default classes the express each of these functions are:
\begin{program}
Agent/SRM set requestFunction_  "SRM/request"           \\
Agent/SRM set repairFunction_   "SRM/repair"            \\
Agent/SRM set sessionFunction_  "SRM/session"           \\
\\
Agent/SRM set requestBackoffLimit_      5   \; parameter to requestFunction_;
Agent/SRM set sessionDelay_             1.0 \; parameter to sessionFunction_;
\end{program}
The last two lines are specific parameters used by the request 
and session objects.
The \href{following section}{Section}{sec:architecture}
describes the implementation of theses objects in greater detail.

\subsubsection{Statistics}
Each agent tracks two sets of statistics:
statistics in response to data losses only,
and overall statistics for each request/repair.

\paragraph{Statistics in Response to Data Losses}
The statistics in response to data losses
measures the duplicate requests (and repairs),
and the average request (and repair) delay.
The algorithm used is as documented by Floyd \etal
\cite{Floy95:Reliable}.
In this algorithm,
each new request (or repair) starts a new request (or repair) period.
During the request (or repair) period, the agent measures
the number of first round duplicate requests (or repairs)
until the round terminates either due to receiving a request (or
repair), or due to the agent sending one.
These statistics are used by the adaptive timer algorithms;
we will describe our implementation of these algorithms in the following
subsections.
The following code illustrates how the user can simple retrieve the
current values in an agent:
\begin{program}
  set statsList [$srm array get statistics_]                \\
  array set statsArray [$srm array get statistics_]             \\
\end{program}
The first form simply returns a list of key-value pairs.
The second form loads the current values into the \|statsArray|;
The keys for each element of the array are
\|dup-req|, \|ave-dup-req|, \|req-delay|, \|ave-req-delay|,
\|dup-rep|, \|ave-dup-rep|, \|rep-delay|, and \|ave-rep-delay|.

\paragraph{Overall Statistics}
In addition, each error recovery and session object keeps track of
times and statistics.
In particular, each object records its
\|startTime|, \|serviceTime|, \|distance|, as are relevant to that object;
startTime is the time that this object was created,
serviceTime is the time for this object to complete its task, and the
distance is the one-way time to reach the remote peer.

For request objects, startTime is the time a packet loss is detected,
serviceTime is the time to finally receive that packet,
and distance is the distance to the original sender of the packet.
For repair objects, startTime is the time that a request for
retransmission is received, serviceTime is the time send a repair,
and the distance is the distance to the original requester.
For both types of objects, the serviceTime is normalised by the
distance.
``serviceTime'' and ``distance'' are not relevant to session objects.

Each object also maintains statistics particular to that type of object.
Request objects track the number of duplicate requests and repairs received,
the number of requests sent, and the number of times this object
had to backoff before finally receiving the data.
Repair objects track the number of duplicate requests and repairs,
as well as whether or not this object for this agent sent the repair.
Session objects simply record the number of session messages sent.

The values of the timers and the statistics for each object are written
to the log file every time an object completes the error recovery function
it was tasked to do.
The format of this trace file is:
\begin{program}
  \tup{prefix} \tup{id} \tup{times} \tup{stats} \\[1em]
\< {\itshape where} \\
\< \tup{prefix} is \>\> \tup{time} n \tup{node id} m \tup{msg id} r \tup{round} \\
\>\>\tup{msg id} is expressed as \tup{source id:sequence number}\\
\< \tup{id} is \>\> type \tup{of object} \\
\< \tup{times} is \>\> list of key-value pairs of startTime, serviceTime, distance \\
\< \tup{stats} is \>\> list of key-value pairs of per object statistics \\
\>\>\|dupRQST|, \|dupREPR|, \|#sent|, \|backoff|  \`for request objects \\
\>\>\|dupRQST|, \|dupREPR|, \|#sent|  \`for repair objects \\
\>\>\|#sent| \`for session objects \\
\end{program}
The following sample output illustrates the output file format:
\begin{verbatim}
 3.6274 n 0 m <1:1> r 1 type repair serviceTime 0.500222 startTime 3.5853553333333332 distance 0.0105 #sent 1 dupREPR 0 dupRQST 0
 3.6417 n 1 m <1:1> r 2 type request serviceTime 2.66406 startTime 3.5542666666666665 distance 0.0105 backoff 1 #sent 1 dupREPR 0 dupRQST 0
 3.6876 n 2 m <1:1> r 2 type request serviceTime 1.33406 startTime 3.5685333333333333 distance 0.021 backoff 1 #sent 0 dupREPR 0 dupRQST 0
 3.7349 n 3 m <1:1> r 2 type request serviceTime 0.876812 startTime 3.5828000000000002 distance 0.032 backoff 1 #sent 0 dupREPR 0 dupRQST 0
 3.7793 n 5 m <1:1> r 2 type request serviceTime 0.669063 startTime 3.5970666666666671 distance 0.042 backoff 1 #sent 0 dupREPR 0 dupRQST 0
 3.7808 n 4 m <1:1> r 2 type request serviceTime 0.661192 startTime 3.5970666666666671 distance 0.0425 backoff 1 #sent 0 dupREPR 0 dupRQST 0
\end{verbatim}

\paragraph{Miscellaneous Information}
Finally, the user can use the following methods to gather
additional information about the agent:
\begin{list}{\textbullet}{}
\item
  \fcnref{\proc[]{groupSize?}}{../ns-2/srm.tcl.html}{Agent/SRM::groupSize?} 
  returns the agent's current estimate of the multicast group size.
\item
  \fcnref{\proc[]{distances?}}{../ns-2/srm.cc.html}{SRMAgent::command}
  returns a list of key-value pairs of distances;
  the key is the address of the agent, 
  the value is the estimate of the distance to that agent.
  The first element is the address of this agent, and the distance of 0.
\item
  \fcnref{\proc[]{distance?}}{../ns-2/srm.cc.html}{SRMAgent::command}
  returns the distance to the particular agent specified as argument.
\end{list}
\begin{program}
  $srm(i) groupSize?    \; returns $srm(i)'s estimate of the group size;
  $srm(i) distances?    \; returns list of \tup{address, distance} tuples;
  $srm(i) distance? 257 \; returns the distance to agent at address 257;
\end{program}

\subsubsection{Tracing}
Each object writes out trace information that can be used to track the
progress of the object in its error recovery.
Each trace entry is of the form:
\begin{program}
\tup{prefix} \tup{tag} \tup{type of entry} \tup{values}
\end{program}
The prefix is as describe in the previous subsection for statistics.
The tag is {\bf Q} for request objects, {\bf P} for repair objects, and
{\bf S} for session objects.
The following types of trace entries and parameters are written by each
object:

\centerline{\small\renewcommand{\arraystretch}{1.3}
\begin{tabular}{rclp{2in}}\hline
      & Type of &              & \\
  Tag & Object  & Other values & Comments\\ \hline
  Q & DETECT & & \\
  Q & INTERVALS & C1 \tup{C1\_} C2 \tup{C2\_} dist \tup{distance} i \<backoff\_> & \\
  Q & NTIMER & at \tup{time} & Time the request timer will fire \\
  Q & SENDNACK & & \\
  Q & NACK & IGNORE-BACKOFF \tup{time} & Receive NACK, ignore other NACKs until
  \tup{time} \\
  Q & REPAIR & IGNORES \tup{time} & Receive REPAIR, ignore NACKs until \tup{time}  \\
  Q & DATA & & Agent receives data instead of repair.  Possibly indicates out of order arrival of data. \\ \hline
  P & NACK & from \tup{requester} & Receive NACK, initiate repair \\
  P & INTERVALS & D1 \tup{D1\_} D2 \tup{D2\_} dist \tup{distance} & \\
  P & RTIMER & at \tup{time} & Time the repair timer will fire \\
  P & SENDREP & \\
  P & REPAIR & IGNORES \tup{time} & Receive REPAIR, ignore NACKs until \tup{time} \\
  P & DATA & & Agent receives data instead of repair.  Indicates premature request by an agent. \\ \hline
  S & SESSION & & logs session message sent \\ \hline
\end{tabular}}
The following illustrates a typical trace for a single loss and recovery.
\begin{verbatim}
 3.5543 n 1 m <1:1> r 0 Q DETECT
 3.5543 n 1 m <1:1> r 1 Q INTERVALS C1 2.0 C2 0.0 d 0.0105 i 1
 3.5543 n 1 m <1:1> r 1 Q NTIMER at 3.57527
 3.5685 n 2 m <1:1> r 0 Q DETECT
 3.5685 n 2 m <1:1> r 1 Q INTERVALS C1 2.0 C2 0.0 d 0.021 i 1
 3.5685 n 2 m <1:1> r 1 Q NTIMER at 3.61053
 3.5753 n 1 m <1:1> r 1 Q SENDNACK
 3.5753 n 1 m <1:1> r 2 Q INTERVALS C1 2.0 C2 0.0 d 0.0105 i 2
 3.5753 n 1 m <1:1> r 2 Q NTIMER at 3.61727
 3.5753 n 1 m <1:1> r 2 Q NACK IGNORE-BACKOFF 3.59627
 3.5828 n 3 m <1:1> r 0 Q DETECT
 3.5828 n 3 m <1:1> r 1 Q INTERVALS C1 2.0 C2 0.0 d 0.032 i 1
 3.5828 n 3 m <1:1> r 1 Q NTIMER at 3.6468
 3.5854 n 0 m <1:1> r 0 P NACK from 257
 3.5854 n 0 m <1:1> r 1 P INTERVALS D1 1.0 D2 0.0 d 0.0105
 3.5854 n 0 m <1:1> r 1 P RTIMER at 3.59586
 3.5886 n 2 m <1:1> r 2 Q INTERVALS C1 2.0 C2 0.0 d 0.021 i 2
 3.5886 n 2 m <1:1> r 2 Q NTIMER at 3.67262
 3.5886 n 2 m <1:1> r 2 Q NACK IGNORE-BACKOFF 3.63062
 3.5959 n 0 m <1:1> r 1 P SENDREP
 3.5959 n 0 m <1:1> r 1 P REPAIR IGNORES 3.62736
 3.5971 n 4 m <1:1> r 0 Q DETECT
 3.5971 n 4 m <1:1> r 1 Q INTERVALS C1 2.0 C2 0.0 d 0.0425 i 1
 3.5971 n 4 m <1:1> r 1 Q NTIMER at 3.68207
 3.5971 n 5 m <1:1> r 0 Q DETECT
 3.5971 n 5 m <1:1> r 1 Q INTERVALS C1 2.0 C2 0.0 d 0.042 i 1
 3.5971 n 5 m <1:1> r 1 Q NTIMER at 3.68107
 3.6029 n 3 m <1:1> r 2 Q INTERVALS C1 2.0 C2 0.0 d 0.032 i 2
 3.6029 n 3 m <1:1> r 2 Q NTIMER at 3.73089
 3.6029 n 3 m <1:1> r 2 Q NACK IGNORE-BACKOFF 3.66689
 3.6102 n 1 m <1:1> r 2 Q REPAIR IGNORES 3.64171
 3.6172 n 4 m <1:1> r 2 Q INTERVALS C1 2.0 C2 0.0 d 0.0425 i 2
 3.6172 n 4 m <1:1> r 2 Q NTIMER at 3.78715
 3.6172 n 4 m <1:1> r 2 Q NACK IGNORE-BACKOFF 3.70215
 3.6172 n 5 m <1:1> r 2 Q INTERVALS C1 2.0 C2 0.0 d 0.042 i 2
 3.6172 n 5 m <1:1> r 2 Q NTIMER at 3.78515
 3.6172 n 5 m <1:1> r 2 Q NACK IGNORE-BACKOFF 3.70115
 3.6246 n 2 m <1:1> r 2 Q REPAIR IGNORES 3.68756
 3.6389 n 3 m <1:1> r 2 Q REPAIR IGNORES 3.73492
 3.6533 n 4 m <1:1> r 2 Q REPAIR IGNORES 3.78077
 3.6533 n 5 m <1:1> r 2 Q REPAIR IGNORES 3.77927
\end{verbatim}


The logging of request and repair traces is done by
\fcnref{\proc[]{SRM::evTrace}}{../ns-2/srm.tcl}{SRM::evTrace}.
However, the routine
\fcnref{\proc[]{SRM/Session::evTrace}}{../ns-2/srm.tcl}{SRM/Session::evTrace},
overrides the base class definition of \proc[]{srm::evTrace},
and writes out nothing.
The user can override these methods and achieve greater
flexibility in their logging options.

\subsection{Architecture and Internals}
\label{sec:architecture}

The SRM agent implementation splits the protocol functions
into packet handling, error recovery, and session message activity.

Packet handling consists of forwarding application data messages,
sending and receipt of control messages.
These activities are executed by C++ methods.

Error detection is done in C++ due to reciept of messages.
However, the error recovery is entirely done through 
instance procedures in OTcl.

The sending and processing of messages is accomplished in C++;
the policy about when these messages should be sent is decided
by instance procedures in OTcl.

\paragraph{Packet Handling: Processing received messages}
The
\fcnref{\fcn[]{recv}}{../ns-2/srm.cc}{SRMAgent::recv}
method can receive four type of messages:
data, request, repair, and session messages.
The method assumes that
the agent is a member of the multicast group, and 
therefore will get a copy of each message sent by it to the group.

\subparagraph{Processing Data Packets}
The agent does not generate any data messages.
The user has to specify an external agent to generate traffic.
The \fcn[]{recv} method must distinguish between
locally originated data that must be sent to the multicast group,
and data received from multicast group that must be processed.
Therefore, the application agent must
set the packet's destination address to zero.

For locally originated data, 
the agent adds the appropriate SRM headers,
sets the destination address to the multicast group, 
and forwards the packet to its target.

On receiving a data message from the group,
\fcnref{\fcn[sender, msgid]{recv\_data}}{../ns-2/srm.cc}{SRMAgent::recv\_data}
will update its state marking message
\tup{sender, msgid} received,
and possibly trigger requests if it detects losses.
In addition, if the message was an older message received out of order,
then there must be a pending request or repair that must be cleared.
In that case, the compiled object invokes the OTcl instance procedure,
\fcnref{\proc[sender,
  msgid]{recv-data}}{../ns-2/srm.tcl}{Agent/SRM::recv-data}%
\footnote{Technically,
  \fcn[]{recv\_data} invokes the instance procedure
  \|recv data \tup{sender} \tup{msgid}|,
  that then invokes \proc[]{recv-data}.}.

Currently, there is no provision for the receivers
to actually receive any application data.
The agent does not also store any of the user data.
It only generates repair messages of the appropriate size,
defined by the instance variable \|packetSize\_|.
However, the agent assumes that any application data
is placed in the data portion of the packet,
pointed to by \|packet->accessdata()|.

\subparagraph{Processing Request Packets}
On receiving a request, 
\fcnref{\fcn[sender, msgid]{recv\_rqst}}{../ns-2/srm.cc}{SRMAgent::recv\_rqst}
will check whether it needs to schedule requests for other missing data.
If it has received this request
before it was aware that the source had generated this data message
(\ie, the sequence number of the request is higher than 
the last known sequence number of data from this source),
then the agent can infer that it is missing this, as well as
data between the last known sequence number and that on the request;
it schedules requests for all of this data, and returns.
On the other hand, if the sequence number of the request is less
than the last known sequence number from the source,
then the agent can be in one of three states:
(1) it does not have this data, and has a request pending for it,
(2) it has the data, and has seen an earlier request,
    upon which it has a repair pending for it, or
(3) it has the data, and it should instantiate a repair.
All of these error recovery mechanisms are done in OTcl,
and \fcn[]{recv\_rqst} invokes the instance procedure
\fcnref{\proc[sender, msgid,
  requester]{recv-rqst}}{../ns-2/srm.tcl}{Agent/SRM::recv-rqst}
for further processing.

\subparagraph{Processing Repair Packets}
On receiving a repair, 
\fcnref{\fcn[sender, msgid]{recv\_repr}}{../ns-2/srm.cc}{SRMAgent::recv\_repr}
will check whether it needs to schedule requests for other missing data.
If it has received this repair
before it was aware that the source had generated this data message
(\ie, the sequence number of the repair is higher than 
the last known sequence number of data from this source),
then the agent can infer that it is missing all
data between the last known sequence number and that on the repair;
it schedules requests for all of this data,
marks this message as received, and returns.
On the other hand, if the sequence number of the request is less
than the last known sequence number from the source,
then the agent can be in one of three states:
(1) it does not have this data, and has a request pending for it,
(2) it has the data, and has seen an earlier request,
    upon which it has a repair pending for it, or
(3) it has the data, and probably scheduled a repair for it at some time;
    after error recovery, its holddown timer (equal to three times its
    distance to some requestor) expired, at which time the pending object
    was cleared.  In this last situation, the agent will simply ignore
    the repair, for lack of being able to do anything meaningful.
All of these error recovery mechanisms are done in OTcl,
and \fcn[]{recv\_repr} invokes the instance procedure
\fcnref{\proc[sender,
  msgid]{recv-rqst}}{../ns-2/srm.tcl}{Agent/SRM::recv-rqst}
for further processing.

\subparagraph{Receiving Session Packets}
On receiving a session message,
the agent updates its sequence numbers for all active sources,
and computes its instantaneous distance to the sending agent if possible.
The agent will ignore earlier session messages from a group member,
if it has received a later one out of order.

Session message processing is done in
\fcnref{\fcn[]{recv\_sess}}{../ns-2/srm.cc}{SRMAgent::recv\_sess}.
The format of the session message is:
\tup{count of tuples in this message, list of tuples},
where each tuple indicates the
\tup{sender id, last sequence number from the source, time the last
  session message was received from this sender, time that that message
  was sent}.
The first tuple is the information about the local agent%
\footnote{Note that this implementation of session message handling
  is subtly different from that used in \emph{wb} or described in
  \cite{Floy95:Reliable} in that, in the other schemes, agents
  disseminate a list of the data actually available at that agent,
  whereas we currently disseminate a list of the last message we are
  aware of.
  This will be fixed at some point in time.}.


\subsection{Timer Based Functions}
\label{sec:timers}

In the last section,
we described the agent behaviour when it receives a message.
Recall that the agent does not generate any data.
All control messages are sent based on timer settings.
This section describes the mehanisms to set timers, and send these messages.



\subsection{Extending the Base Class Agent}
\label{sec:extensions}


\subsubsection{Fixed Timers}


\subsubsection{Adaptive Timers}


\end{document}

### Local Variables:
### mode: latex
### comment-column: 60
### backup-by-copying-when-linked: t
### file-precious-flag: nil
### End:


\part{Scale}
\chapter{Session-level Packet Distribution}
\label{chap:session}

This section describes the internals of the Session-level Packet Distribution
implementation in \ns.
The section is in two parts:
the first part is an overview of 
a minimal Session configuration,
and a ``complete'' description of the configuration parameters 
of a Session.
The second part describes the architecture, internals, and the code path
of the Session-level Packet distribution.

\section{Configuration}
\label{sec:session-config}

Each Session (i.e., a multicast tree) must be configured strictly in
this order:
 
1. creating(obtaining) the session source,
2. assigning the destination address,
3. creating the session helper, 
4. attaching to session source, and
5. the session members joining the group.


\subsection{Basic Configuration}
\label{sec:basic-config}
\begin{program}
        set ns [new SessionSim]          \; preamble initialization;
        set node [$ns node]              \; source and receiver to reside on this node;
        set group [$ns allocaddr]        \; multicast group for this session;

        set src [new Agent/CBR]
        $src set dst_ $group            \; configure the source;
        $ns attach-agent $node $src

        $ns create-session $node $src   \; creating the session helper and attaching to the source;

        set rcvr [new Agent/NULL]        \; configure the receiver;
        $ns attach-agent $node $rcvr
        $ns at 0.0 "$node join-group $rcvr $group" \; joining the session;

        $ns at 0.1 "$src start"          \; start the source;

\end{program}

\subsection{Inserting a Loss Module}
\label{sec:loss-config}
\paragraph{Creating a Loss Module}
Before we can insert a loss module in between a source-receiver pair,
we have to create the loss module.  Basically,
a loss module compares two values to decide whether to drop a packet.
The first value is obtained every time when the loss module receives 
a packet from a random variable.  The second value
is fixed and configured when the loss module is created.

The following code gives an example to create a uniform 
0.1 loss rate.

\begin{program}
        # creating the uniform distribution random variable
        set loss_random_variable [new RandomVariable/Uniform] 
        # setting the range of random variable
        $loss_random_variable set min_ 0
        $loss_random_variable set max_ 100

        # creating an error module;
        set loss_module [new ErrorModel]
        # set target for dropped packets;
        $loss_module drop-target [new Agent/Null]
        # setting error rate to 0.1, 10/(100-0);
        $loss_module set rate_ 10
        # attaching the random variable to the loss module;
        $loss_module ranvar $loss_random_variable 

\end{program}

Several random variable distributions are available.
%%% Need xref to ranvar pages
Please refer to tcl/ex/ranvar.tcl.

\paragraph{Inserting a Loss Module}

One loss module is required for each pair source-receiver pair. If it is
intended to insert a loss module for a receiver, keep a handle to the 
loss module when created.  Loss modules can only be inserted after the
corresponding receivers finish joining the group.

\begin{program}
        # keep a handle to the loss module;
        set sessionhelper [$ns create-session $node $src] 
        # insert the loss module;
        $ns at 0.1 "$sessionhelper insert-loss $loss_module $rcvr" 
\end{program}

Please note that packets dropped for a particular receiver should also 
be dropped for its downstream receivers.  We are currently
working on solving this problem error dependency problem.  Before 
we complete the implementation, please carefully calculate and 
place your loss modules.

\section{Architecture}
\label{sec:session-arch}
The purpose of Session-level packet distribution is to
speed up simulations and reduce memory consumption while 
maintaining reasonable accuracy(if no queuing involved).  The first
bottleneck observed is the memory consumption by heavy-weight
links and nodes.  Therefore, in SessionSim (Simulator for Session-level
packet distribution), we keep only minimal amount of 
states for links and nodes, and connect the higher level source and 
receiver applications with appropriate delay and loss modules.  When
a connection is a multicast group, we attach a replicator 
to the source application, so the replicator replicates packets
to all loss or delay modules attached to the receiver applications.

In short, almost the entire network layer(routing and queuing)
is abstract out.  Packets in SessionSim do not get routed.  
They only follow the established Session.

\section{Internals}
In this section, we explain the internals of Session-level Packet 
Distribution.  The implementation is split into two parts:
\begin{list}{}{}
\item  Linkage of objects to make a Session in OTcl 
\item  Packet forwarding activities are executed by C++ methods.  
\end{list}

\subsection{Object Linkage}
\label{sec:session-objlink}

\begin{list}{}{}
\item  Simplified links and nodes.
\item  Replicator
\item  Delay and loss modules
\end{list}

\paragraph{Nodes and Links}
\label{sec:session-nodenlink}
A link only contains the values of
its bandwidth and delay, and a node contains only its id and port number
for next agent.

\begin{program}
SessionSim instproc simplex-link \{ n1 n2 bw delay type \} \{
    $self instvar link_ delay_
    set sid [$n1 id]
    set did [$n2 id]

    set link_($sid:$did) [expr [string trimright $bw Mb] * 1000000]
    set delay_($sid:$did) [expr [string trimright $delay ms] * 0.001]
\}

SessionNode instproc init \{\} \{
    $self instvar id_ np_
    set id_ [Node getid]
    set np_ 0
\}
\end{program}

\paragraph{Replicator}
One replicator is required per source.  While the source is configured,
a replicator (session helper) need to be attached to the source.  By
calling \proc[]{create-session}, a replicator is:
1. created,
2. attached to the source application
3. kept in a SessionSim instance variable \code{session_} array with 
its source and destination addresses as the index.

Note that the destination of source agent must be set before
calling \proc[]{create-session}.

\begin{program}
SessionSim instproc create-session \{ node agent \} \{
    $self instvar session_

    set nid [$node id]                           \; get source address;
    set dst [$agent set dst_]                    \; get destination address;
    set session_($nid:$dst) [new Classifier/Replicator/Demuxer]  \; creating the replicator;
    $agent target $session_($nid:$dst)           \; attach the replicator to the source;
    return $session_($nid:$dst)  \; keep the replicator in the SessionSim instance variable session_ array;
\}
\end{program}

\paragraph{Delay and Loss Modules}

At least one delay module is required per receiver.
See Section~\ref{sec:loss-config} for inserting a loss module for a receiver.
When a receiver joins a group, 
the \proc[]{join-group} method goes through
all replicators (session helpers) maintained in \code{session_}.
If the destination index matches the group address
the receiver intends to join, then the following actions are performed.

1. a new slot of the replicator (session helper) is created and assigned to the receiver.
2. accumulated bandwidth and delay between the source and receiver are obtained by SessionSim instance procedure \proc[]{get-bw} and \proc[]{get-delay}.
3. a constant random variable is created and assigned with the
accumulative delay.
4. a delay module is created and assigned with the constant random 
variable and the accumulative bandwidth.
5. the delay module in inserted into the replicator slot in
front of the receiver.

\begin{program}
SessionSim instproc join-group \{ agent group \} \{
    $self instvar session_

    foreach index [array names session_] \{
        set pair [split $index :]
        if \{[lindex $pair 1] == $group\} \{
            # Note: must insert the chain of loss, delay, 
            # and destination agent in this order:

            #1. insert destination agent into session replicator
            $session_($index) insert $agent

            #2. find accumulative bandwidth and delay
            set src [lindex $pair 0]
            set dst [[$agent set node_] id]
            set accu_bw [$self get-bw $dst $src]
            set delay [$self get-delay $dst $src]

            #3. set up a constant delay random variable
            set random_variable [new RandomVariable/Constant]
            $random_variable set avg_ $delay

            #4. set up the delay module
            set delay_module [new DelayModel]
            $delay_module bandwidth $accu_bw
            $delay_module ranvar $random_variable

            #4. insert the delay module in front of the dest agent
            $session_($index) insert-module $delay_module $agent
        \}
    \}
\}
\end{program}


\subsection{Packet Forwarding}
Packet forwarding activities are executed in C++.  A source application 
generates a packet and forwards to its target which must be a replicator 
(session helper).  The replicator copies the packet and forwards 
to targets in the active slots which are either delay modules or loss modules. If loss modules, a decision is made whether to drop the packet.
If yes, the packet is forwarded to the loss modules drop target.  If not,
the loss module forwards it to its target which must be a delay module.
The delay module will forward the packet with a delay to its target which
must be a receiver application.

%% PH: not sure this will come out right
\begin{program}
                    / Loss module - Delay module - Receiver 1
Source - Replicator --------------- Delay module - Receiver 2
    (Session Helper)\bs Loss module - Delay module - Receiver 3

\end{program}

\endinput

### Local Variables:
### mode: latex
### comment-column: 60
### backup-by-copying-when-linked: t
### file-precious-flag: nil
### End:


\part{Other}
\include{debugging}

\bibliography{ns}

\end{document}
